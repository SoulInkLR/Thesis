\begin{abstract}


Nowadays, real-time systems are ubiquitous in several application domains. 
Such an emergence led to an increasing need of performance (resources, 
availability, concurrency, etc.) and initiated a shift from the
use of single processor based hardware platforms, to large sets 
of interconnected and distributed computing nodes. This trend introduced the birth 
of a new family of systems that are intrinsically distributed, namely
\emph{Networked Embedded Systems}.
Such an evolution stems from the growing complexity of real-time software 
embedded on such platforms (e.g. electronic control in avionics 
and automotive domains), and the need to integrate formerly isolated systems so that 
they can cooperate, as well as, share resources improving thus functionalities 
and reducing costs.
Undoubtedly, the design, implementation and verification of such systems are 
acknowledged to be very hard tasks since they 
are prone to different kinds of factors, such as communication delays, CPU(s) 
speed or even hardware imprecisions, which increases considerably the complexity of 
coordinating parallel activities.

In this thesis, we propose a rigorous design flow intended for 
building distributed real-time applications.
We investigate timed automata based models, with well defined semantics, in order 
to study the behavior of a given system with some imposed timing constraints when deployed 
in a distributed environment. Particularly, we study \emph{(i)} the impact of the communication 
delays by introducing a minimum latency between actions executions and the effective 
date at which actions executions have been decided, 
and \emph{(ii)} the effect of hardware imperfections, more precisely clocks imprecisions, 
on systems execution by breaking the clock synchrony hypothesis, often adopted during 
the modeling phase. Nevertheless, timed automata formalism is intended to describe a high
level abstraction of the behavior of a given application. 
Therefore, we use an intermediate representation of 
the initial application, that besides having \say{equivalent} behavior, explicitly expresses
implementation mechanisms, and thus reduces the gap between the modeling and the concrete
implementation. Additionally,  we contribute in building such systems by \emph{(iii)} 
proposing a knowledge based optimization method that aims to eliminate unnecessary 
computation time or exchange of messages during the execution. 
  
We compare the behavior of each proposed model to the initial high level model and study the
relationships between both. Then, we identify and formally characterize the potential problems 
resulting from these additional constraints. Furthermore, we propose execution strategies that 
allow to preserve some desired properties and reach a \say{similar} execution scenario, 
faithful to the original specifications.   
  
\end{abstract}


\begin{abstractfr}
\begin{otherlanguage}{french}
Aujourd'hui, les systèmes temps réel sont omniprésents dans plusieurs domaines.
Une telle expansion donne lieu à un besoin croissant en terme de performance (ressources,
disponibilité, parallélisme, etc.) et a initié par la même occasion une transition de
l'utilisation de plateformes matérielles à processeur unique, à de grands ensembles
de nœuds de calcul inter-connectés et distribués. Cette tendance a introduit la naissance
d'une nouvelle famille de systèmes connue sous le nom de \emph{Networked Embedded Systems},
qui sont intrinsèquement distribués.
Une telle évolution provient de la complexité croissante des logiciels temps réel
embarqués sur de telles plateformes (par exemple les système de contrôle en avionique
et dans domaines de l'automobile), ainsi que la nécessité d'intégrer des systèmes autrefois 
isolés afin d'accomplir les fonctionnalités requises, améliorant ainsi les performances 
et réduisant les coûts.
Sans surprise, la conception, l'implémentation et la vérification de ces systèmes sont
des tâches très difficiles car elles sont sujettes à différents types de facteurs, tels que les 
délais de communication, la fréquence du CPU ou même les imprécisions matérielles, 
ce qui augmente considérablement la complexité de coordonner les activités parallèles.

Dans cette thèse, nous proposons un workflow rigoureux destiné à la construction d'applications 
distribuées temps réel.
Pour ce faire, nous étudions des modèles basés sur les automates temporisés, avec une sémantique
bien définie, afin d'étudier le comportement d'un système donné avec des contraintes de temps 
imposées lorsqu'il est déployé dans un environnement distribué. En particulier, nous étudions 
\emph{(i)} l'impact des délais de communication en introduisant une latence minimale entre 
les exécutions d'actions et la date à laquelle elles ont été décidées,
et \emph{(ii)} l'effet des imperfections matérielles, plus précisément les imprécisions 
d'horloges, sur l'exécution des systèmes. 
Le paradigme des automates temporisés reste néanmoins destiné à décrire une abstraction 
du comportement d'une application donnée.
Par conséquent, nous utilisons une représentation intermédiaire de
l'application initiale, qui en plus d'avoir un comportement \say{équivalent}, exprime 
explicitement les mécanismes mis en œuvre durant l'implémentation, et donc réduit ainsi 
l'écart entre la modélisation et l'implémentation réelle.
De plus, nous contribuons à la construction de tels systèmes en \emph{(iii)}
proposant une optimisation basée sur la \emph{connaissance}, qui a pour but d'éliminer les 
temps de calcul inutiles et de réduire les échanges de messages pendant l'exécution.
  

  Nous comparons le comportement de chaque modèle proposé au modèle initial et étudions
les relations entre les deux. Ensuite, nous identifions et caractérisons formellement les 
problèmes potentiels résultants de ces contraintes supplémentaires. Aussi, nous proposons 
des stratégies d'exécution qui permettent de préserver certaines propriétés souhaitées 
et d'obtenir des scénarios d'exécution \say{similaires}, et fidèles aux spécifications 
de départs.
\end{otherlanguage}
\end{abstractfr}
