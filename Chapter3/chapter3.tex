\chapter{Modeling Distributed Real-Time Systems}\label{chap:3} 
\minitoc
In the previous chapter, we presented a timed automata based model for representing 
timed systems with multiparty interactions. The semantics of such model is based on 
the notion of \emph{global states}, that is, interactions executions are based not only on the 
state of participating components but on the states of all components of the system.
Moreover, this type of model does not provide any details on how an implementation of 
multiparty interactions can be derived.
Conversely, a distributed system can be seen as a collection of loosely coupled independent
components communicating by explicit messages passing (components state may be known only
through communication). 
In order to reduce the gap between the model presented in the previous chapter and its concrete
implementation, we propose an intermediate model more suited for the distributed real-time
context, and that is obtained by applying transformation rules on the first model. 
The main idea is to explicitly express the ongoing communication mechanism as well as 
allowing interactions executions based only on their participating components.
The key concept of this approach is to structure a given system in two main layers: \emph{(1)}
an application layer that consists of a set of distributed components and \emph{(2)} a 
scheduling layer that is responsible for scheduling the execution of the latter. Additionally, 
a third layer may be needed by the scheduling layer in order to achieve global consistency.
\section{Target Architecture}

In a distributed context, we consider that components communicate through asynchronous 
messages passing. Consequently, each component is able either to send a message, to wait for 
a notification or to execute an internal computation. Our approach proposes an architecture for 
executing multiparty interactions as a two way handshake 
protocol~\cite{ahlem_these,conf_free,jean} involving asynchronous 
exchange of messages between \emph{distributed} components and a second layer responsible for 
triggering interactions, the scheduling layer (see Figure~\ref{fig:ta}). 
In order to evaluate the enabled interactions at a given state, distributed components are 
required to send their current local information (e.g. enabled actions, invariants, 
clock constraints, etc.) to the scheduling layer using \emph{offer} messages. 
As offers are sent asynchronously,  
the scheduling layer may not have a global knowledge of the system. In fact, the latter
is built from partial (local) knowledge. Additionally, in order to increase the parallelism
the scheduling layer may decide to execute an interaction once it accumulates enough offers.
We require that the exchange of messages is sender-triggered and not blocking, that is,
each time a sender is ready to transmit the corresponding receiver is ready to receive.
The class of models satisfying this restriction are called Send/Receive models.
\begin{figure}[h!]
\centering
  \begin{tikzpicture}

  \node [rectnode={1.1,1.1},label={[shift={(0,-1.1)}]$T_1^{SR}$}] (t1) {};
  \node [rectnode={1.1,1.1},right=7.5mm of t1,label={[shift={(0,-1.1)}]$C^{SR}$}] (c) {};
  \node [rectnode={1.1,1.1},right=7.5mm of c,label={[shift={(0,-1.1)}]$R^{SR}$}] (r) {};
  \node [rectnode={1.1,1.1},right=7.5mm of r,label={[shift={(0,-1.1)}]$T_2^{SR}$}] (t2) {};

  \node [rectnode={7,2},above=1cm of c,xshift=8.75mm] (sch) {$Scheduling$ $Layer$};
  
  
  \node [triangle,label=-90:$o_1$] (ot1) at ($(t1.north west)!0.25!(t1.north east)$) {};
  \node [dots,label=-90:$n_1$] (nt1) at ($(t1.north west)!0.75!(t1.north east)$) {};
  
  \node [triangle,label=-90:$o_2$] (oc) at ($(c.north west)!0.25!(c.north east)$) {};
  \node [dots,label=-90:$n_2$] (nc) at ($(c.north west)!0.75!(c.north east)$) {};
  
  \node [triangle,label=-90:$o_3$] (or) at ($(r.north west)!0.25!(r.north east)$) {};
  \node [dots,label=-90:$n_3$] (nr) at ($(r.north west)!0.75!(r.north east)$) {};
  
  \node [triangle,label=-90:$o_4$] (ot2) at ($(t2.north west)!0.25!(t2.north east)$) {};
  \node [dots,label=-90:$n_4$] (nt2) at ($(t2.north west)!0.75!(t2.north east)$) {};


  \node [dots,label=90:$o_1$,above=7.5mm of ot1] (sot1) {};
  \node [triangle,label=-90:$n_1$,above=10mm of nt1,rotate=180] (snt1) {};

  \node [dots,label=90:$o_3$,above=7.5mm of oc] (soc) {};
  \node [triangle,label=-90:$n_3$,above=10mm of nc,rotate=180] (snc) {};

  \node [dots,label=90:$o_3$,above=7.5mm of or] (sor) {};
  \node [triangle,label=-90:$n_3$,above=10mm of nr,rotate=180] (snr) {};

  \node [dots,label=90:$o_4$,above=7.5mm of ot2] (sot2) {};
  \node [triangle,label=-90:$n_4$,above=10mm of nt2,rotate=180] (snt2) {};

  \draw [thick,red,-] (ot1) -- (sot1);
  \draw [thick,dgreen,-] (nt1) -- (snt1);

  \draw [thick,red,-] (oc) -- (soc);
  \draw [thick,dgreen,-] (nc) -- (snc);
 
  \draw [thick,red,-] (or) -- (sor);
  \draw [thick,dgreen,-] (nr) -- (snr);
  
  \draw [thick,red,-] (ot2) -- (sot2);
  \draw [thick,dgreen,-] (nt2) -- (snt2);
\end{tikzpicture} 
\caption{High Level Representation of the Target Architecture}
\label{fig:ta}
\end{figure}

\begin{example}
  Figure~\ref{fig:ta} depicts a high level representation of the timed system~\ref{fig:tm} in a 
  distributed setting. Components $C$, $T_1$, $T_2$ and $R$ are transformed into the distributed
  components $C^{SR}$, $T_1^{SR}$, $T_2^{SR}$ and $R^{SR}$ respectively. Each component sends 
  information about its current state to the scheduling layer through offer messages
  ($\{o_1,\cdots,o_4\}$), and is notified through notification messages ($\{n_1,\cdots,n_4\}$).
  Triangles (respectively dots) indicates the sender (respectively the receiver). 
\end{example}


\subsection{Interface}
In order to express the message passing mechanism, we introduce the notion of 
\emph{communication ports}. A communication port defines the interface of a distributed 
components, that is how it interacts with the rest of the system. For our purpose, we distinguish
three types of ports: send ports, receive ports, and unary ports.
A send port is used to export data outside of the sender when sending offers, whereas
a receive port imports data inside the receiver  
when being notified. Unary ports correspond to independent execution of a distributed 
component, which is formally expressed using a unary interaction (singleton). 
Effectively, each action of a timed component as presented in Definition~\ref{def:tc} 
will correspond to a receive port in a distributed component, which is responsible for triggering
the execution of the underlying action.

\subsection{From Local Time to Global Time}
In the timed systems model of Chapter~\ref{chap:2}, every timed component can define a set of 
local clocks to be used for expressing clock constraints on transitions or the allowed time 
progress on locations, locally. In our intermediate model, we choose to make use of \emph{global 
clocks}.
In fact, a global clock measures the absolute time elapsed since the system startup and is never
reset. This approach allows to have a common timescale between the distributed components
and the scheduling layer, which reduces considerably the effort when keeping track of the
actual time progress since one needs to maintain only the global clock(s). 
Notice that any component clock $x$ can be derived from the global clock simply by shifting its 
value by an amount of time that is constant between successive resets of $x$. 
Consequently, achieving 
the global time mechanism is done by removing local clocks from individual components 
and adding global clock(s) to the scheduling layer. Moreover, given a global clock $g$ 
and for each local clock $x$,
we include a variable $\rho_x^g$ that stores the absolute time of its last reset with respect
to $g$. This variable is updated each time a transition resetting $x$ 
is executed. Then, the value of $x$ can be found by the equality $x=g-\rho_x^g$.  
As a result, any clock constraint $c$ involved in a component can be expressed using clock $g$
as follows:
\begin{equation}\label{eq:g_clk}
  c=\bigwedge_{x\in\mc{X}} l_x\le x\le u_x=\bigwedge_{x\in\mc{X}}l_x+\rho_x^g\le g\le u_x+\rho_x^g
\end{equation}

\subsection{Conflicting Interactions and Interaction Partitioning}
\label{sub:conf}
In a distributed context, interaction executions may occur in parallel. However,
when two interactions share at least a component it is impossible to execute both
interaction concurrently. Particularly, if these interactions are enabled from the same state 
then they are \emph{conflicting} since they will compete on the same resources (shared 
component(s)) at the same time.
In general, conflicts can be very hard to characterize for real life case studies since 
they depend on the reachability of particular states. In~\cite{ahlem_these,conf_free}, 
the computation of
the conflicting interactions relies on over-approximations. It is based on a notion 
of \emph{potential conflicts} that can be detected by simple syntactic pre-checks, as depicted
in Figure~\ref{fig:pconf}, and are used to quickly exclude conflicts since two interactions
that are not potentially conflicting are also not conflicting. 

\begin{definition}[Potential Conflict]\label{def:pconf}
  Two interactions $\alpha_1$ and $\alpha_2$ are potentially conflicting if 
  $\p{\alpha_1}\cap\p{\alpha_2}\neq\emptyset$ and for each component $B_i\in\p{\alpha_1}\cap
  \p{\alpha_2}$ there exists two transitions of $B_i$ 
  $e_1,e_2\in\mc{E}_i$ such that $\source(e_1)=\source(e_2)$ and $\action(e_1)\in\alpha_1$,
  $\action(e_2)\in\alpha_2$.
\end{definition}
\begin{figure}[h]
 \centering
\begin{tikzpicture}[->,node distance=1cm,>=stealth',bend angle=20,auto,
  place/.style={circle,thick,draw=black,minimum size=5mm},
  dots/.style={fill=black,circle,inner sep=1.5pt},
  initial text={}]

  \node [place] (l0) {};
  \node [place,below=1cm of l0,xshift=-5mm] (l1) {};
  \node [place,below=1cm of l0,xshift=5mm] (l2) {};
  \path (l0) edge  node[left]{$a_1$} (l1)
             edge  node[right]{$a_2$} (l2);
  \node [rounded corners,inner xsep=5mm,fit=(l0)(l1)(l2),draw,thick] (rec1) {};
  \node [rectnodes={2.6,2.3}, right=2cm of rec1] (rec2) {};
  
  \node [dots,label=-90:$a_1$] (a) at ($(rec1.north west)!0.25!(rec1.north east)$) {};
  \node [dots,label=-90:$a_2$] (b) at ($(rec1.north west)!0.75!(rec1.north east)$) {};
  
  \node [dots,label=-90:$a$] (c) at ($(rec2.north west)!0.5!(rec2.north east)$) {};

  \path (a) +(0,5mm) coordinate (a2) +(-5mm,5mm) coordinate(a3);
  \path (b) +(0,5mm) coordinate (b2) +(5mm,5mm) coordinate(b3);

  \path (c) +(5mm,5mm) coordinate (c2) +(-5mm,5mm) coordinate(c3) +(10mm,5mm) coordinate(c4)
+(-10mm,5mm) coordinate(c5);

  \draw[thick,-] (a) -- (a2); 
  \draw[thick,-] (a2) -- node[above]{$\alpha_1$}(a3); 
  \draw[thick,-] (b) -- (b2); 
  \draw[thick,-] (b2) -- node[above]{$\alpha_2$}(b3); 
  \draw[thick,-] (c) -- (c2); 
  \draw[thick,-] (c) -- (c3); 
  \draw[thick,-] (c2) -- node [above]{$\alpha_2$}(c4); 
  \draw[thick,-] (c3) -- node [above]{$\alpha_1$} (c5); 

\end{tikzpicture}
\caption{Potential Conflict Between Interactions $\alpha_1$ and $\alpha_2$}
\label{fig:pconf}
\end{figure}


In order to avoid a centralized scheduling and to introduce concurrency between interaction
executions, we propose to decentralize the scheduling layer into several schedulers each one
responsible of scheduling a subset of interactions. The purpose behind this practice is to 
\emph{(i)} spread the workload across concurrent schedulers and
as much as possible independently, and \emph{(ii)} map schedulers as close as possible to 
the components that they concretely handle (with respect to the corresponding subset of 
interactions), which brings back the communication overhead between components to the same 
magnitude. Our work does not address interaction partitioning, nonetheless it is a crucial 
concern for load-balancing and for tuning the system to achieve a desired level of performance. 

\begin{definition}[Interaction Partition]\label{def:inter_part}
  Given an interaction set $\gamma$, a partition of $\gamma$ is a set of subset 
  $\{\gamma_k\}_{k=1}^m$ such that $\gamma=\gamma_1\cup\cdots\cup\gamma_m$ and 
  $\forall i,j\in\{1,\cdots,m\}$ such that $i\neq j$, $\gamma_i\cap\gamma_j=\emptyset$.
\end{definition}

Decentralizing the schedulers generates situational conflict between interactions, that is, 
if two interactions handled in separate schedulers (from two class of the interaction partition)
are potentially conflicting, they cannot execute in parallel. We call such interactions,
\emph{externally conflicting} interactions. 
A simple solution to resolve such conflicts is to enforce
a \emph{conflict-free} partitioning of interactions. In spite of that, this solution will 
restrict the choice for distributing interactions across schedulers. Thus, another method
~\cite{jean,ahlem_these} is to incorporate a third layer that will arbiter the execution of 
potentially 
conflicting interactions. The latter can be represented using an additional component realizing 
a \emph{conflict resolution protocol} (CRP). This component implements an algorithm based on 
the idea of messages counting technique~\cite{msg_cnt}.
This technique counts the number of times that a component participates in an 
interaction. Conflicts are then resolved by ensuring that each participation number is used 
only once, which is achieved by counting the number of the interaction offer for each 
component. Then, conflicts are simply resolved by comparing the offer numbers of participating
components with the numbers of their last execution.
On the other hand, conflicts raised from interactions of the same class, that is, handled by 
the same scheduler, are resolved locally by the latter.
\begin{example}\label{exp:partition}
  Let us consider example of Figure~\ref{fig:tm}. For the interaction set $\gamma=\{\alpha_1,
  \cdots,\alpha_8\}$, let $\gamma_1=\{\alpha_{2\times i-1}\}_{i=1}^4\cup\gamma_2=
  \{\alpha_{2\times i}\}_{i=1}^4$ be an interaction partition. Then from 
  Definition~\ref{def:pconf} the set of potentially conflicting interactions two-by-two is:
  $\{(\alpha_1,\alpha_2);(\alpha_3,\alpha_4);(\alpha_5,\alpha_6);$\\$(\alpha_7,\alpha_8)\}$.
  This mean that the set of conflicting interactions of the whole system is $\gamma$.
\end{example}

\section{3-Layer Send/Receive Model}\label{sec:3.2}
Let $S=\gamma(B_1,\cdots,B_n)$ be a timed system. Given a partition of interactions
$\{\gamma_k\}^m_{k=1}$, the Send/Receive model corresponding to $S$ is based on the three
following layers:
\begin{itemize}
  \item The \emph{Distributed Component Layer} consists of a transformation of timed components
    $B_i$ into Send/Receive components $B_i^{SR}$ that send, asynchronously, offer messages 
    enclosing their current state to the scheduling layer. 
  \item The \emph{Scheduling Layer} is responsible of interactions executions. Based on offers 
    received form the Send/Receive components, it may decide or not to execute an interaction.
    In case of conflicts, the scheduling layer relies on the conflict resolution layer to grant
    or deny the execution of an interaction.
  \item The \emph{Conflict Resolution Layer} resolves conflicts between externally conflicting 
    interactions based on messages counting technique.
\end{itemize}

\begin{figure*}[h!]
\centering
  \captionsetup{justification=centering}
  \shorthandoff{:!}
\begin{tikzpicture}
  \node [rectnode={1.1,1.1},label={[shift={(0,-1.1)}]$T_1^{SR}$}] (t1) {};
  \node [rectnode={1.1,1.1},right=10mm of t1,label={[shift={(0,-1.1)}]$C^{SR}$}] (c) {};
  \node [rectnode={1.1,1.1},right=7.5mm of c,label={[shift={(0,-1.1)}]$R^{SR}$}] (r) {};
  \node [rectnode={1.1,1.1},right=10mm of r,label={[shift={(0,-1.1)}]$T_2^{SR}$}] (t2) {};

  \node [rectnode={4,2},above=1.5cm of t1,xshift=8mm] (sch1) {$Sch_1\{\small{\alpha_1,\alpha_3,\alpha_5,\alpha_7}\}$};
  \node [rectnode={4,2},above=1.5cm of t2,xshift=-8mm] (sch2) {$Sch_2\{\small{\alpha_2,\alpha_4,\alpha_6,\alpha_8}\}$};
  \node [rectnode={8.5,2},above=5mm of sch1,xshift=22.5mm,label={[shift={(0,-10mm)}]$Conflict$ $Resolution$}] (crp) {};
  
  
  \node [triangle,label=-90:\small{$o_1$}] (ot1) at ($(t1.north west)!0.25!(t1.north east)$) {};
    \node [dots,label=-90:\small{$n_1$}] (nt1) at ($(t1.north west)!0.75!(t1.north east)$) {};
  
    \node [triangle,label=-90:\small{$o_2$}] (oc) at ($(c.north west)!0.25!(c.north east)$) {};
    \node [dots,label=-90:\small{$n_2$}] (nc) at ($(c.north west)!0.75!(c.north east)$) {};
  
    \node [triangle,label=-90:\small{$o_3$}] (or) at ($(r.north west)!0.25!(r.north east)$) {};
    \node [dots,label=-90:\small{$n_3$}] (nr) at ($(r.north west)!0.75!(r.north east)$) {};
  
    \node [triangle,label=-90:\small{$o_4$}] (ot2) at ($(t2.north west)!0.25!(t2.north east)$) {};
    \node [dots,label=-90:\small{$n_4$}] (nt2) at ($(t2.north west)!0.75!(t2.north east)$) {};

  %sch1
    %\node [dots,label=90:\small{$o_1$}] (s1ot1) at ($(sch1.south west)!0.15!(sch1.south east)$) {};
    %\node [triangle,label=-90:\small{$n_1$},rotate=180] (s1nt1) at ($(sch1.south west)!0.25!(sch1.south east)$) {};
    \node [dots,label=90:\small{$o_1$},above=1.25cm of ot1] (s1ot1)  {};
    \node [triangle,label=-90:\small{$n_1$},rotate=180,above=1.5cm of nt1] (s1nt1) {};
  
    \node [dots,label=90:\small{$o_2$}] (s1oc) at ($(sch1.south west)!0.5!(sch1.south east)$) {};
    \node [triangle,label=-90:\small{$n_2$},rotate=180] (s1nc) at ($(sch1.south west)!0.6!(sch1.south east)$) {};
  
    \node [dots,label=90:\small{$o_3$}] (s1or) at ($(sch1.south west)!0.75!(sch1.south east)$) {};
    \node [triangle,label=-90:\small{$n_3$},rotate=180] (s1nr) at ($(sch1.south west)!0.85!(sch1.south east)$) {};
  
  \node [dots,label=-90:\small{$ok_1$}] (ok1) at ($(sch1.north west)!0.25!(sch1.north east)$) {};
  \node [dots,label=-90:\small{$fail_1$}] (fail1) at ($(sch1.north west)!0.5!(sch1.north east)$) {};
  \node [triangle,label=-90:\small{$req_1$}] (res1) at ($(sch1.north west)!0.75!(sch1.north east)$) {};
  %crp1
  \node [triangle,label=-90:\small{$ok_1$},rotate=180,above=5mm of ok1] (ok1')  {};
  \node [triangle,label=-90:\small{$fail_1$},rotate=180,above=5mm of fail1] (fail1') {};
  \node [dots,label=90:\small{$req_1$},above=2.5mm of res1] (res1') {};
  %sch2
  \node [dots,label=90:\small{$o_2$}] (s2oc) at ($(sch2.south west)!0.15!(sch2.south east)$) {};
    \node [triangle,label=-90:\small{$n_2$},rotate=180] (s2nc) at ($(sch2.south west)!0.25!(sch2.south east)$) {};
  
    \node [dots,label=90:\small{$o_3$}] (s2or) at ($(sch2.south west)!0.4!(sch2.south east)$) {};
    \node [triangle,label=-90:\small{$n_3$},rotate=180] (s2nr) at ($(sch2.south west)!0.5!(sch2.south east)$) {};
  
    \node [dots,label=90:\small{$o_4$},above=1.25cm of ot2] (s2ot2)  {};
    \node [triangle,label=-90:\small{$n_4$},rotate=180,above=1.5cm of nt2] (s2nt2) {};
    %\node [dots,label=90:\small{$o_4$}] (s2ot2) at ($(sch2.south west)!0.75!(sch2.south east)$) {};
    %\node [triangle,label=-90:\small{$n_4$},rotate=180] (s2nt2) at ($(sch2.south west)!0.85!(sch2.south east)$) {};

  \node [dots,label=-90:\small{$ok_2$}] (ok2) at ($(sch2.north west)!0.25!(sch2.north east)$) {};
  \node [dots,label=-90:\small{$fail_2$}] (fail2) at ($(sch2.north west)!0.5!(sch2.north east)$) {};
  \node [triangle,label=-90:\small{$res_2$}] (res2) at ($(sch2.north west)!0.75!(sch2.north east)$) {};
  %crp2
  \node [triangle,label=-90:\small{$ok_2$},rotate=180,above=5mm of ok2] (ok2')  {};
  \node [triangle,label=-90:\small{$fail_2$},rotate=180,above=5mm of fail2] (fail2') {};
  \node [dots,label=90:\small{$res_2$},above=2.5mm of res2] (res2') {};
  
  
  %connection
  \draw[thick,-,red] (res1) -- (res1');
  \draw[thick,-,dgreen] (ok1) -- (ok1');
  \draw[thick,-,dgreen] (fail1) -- (fail1');
  
  \draw[thick,-,red] (res2) -- (res2');
  \draw[thick,-,dgreen] (ok2) -- (ok2');
  \draw[thick,-,dgreen] (fail2) -- (fail2');
  
  
  \draw[thick,-,red] (ot1) -- (s1ot1);
  \draw[thick,-,dgreen] (s1nt1) -- (nt1);
  \draw[thick,-,red] (ot2) -- (s2ot2);
  \draw[thick,-,dgreen] (s2nt2) -- (nt2);

  
  \path (oc) +(0,3.5mm) coordinate (c0);
  \draw[-,red,thick,name path=line1] (s1oc) |- (c0) -- (oc) -- (c0)-|(s2oc);

  \path (nc) +(0,5.5mm) coordinate (nc0);
  \path[-,name path=line2] (s1nc) |- (nc0) -- (nc) -- (nc0)-|(s2nc);
 
  
  \path[name intersections={of=line1 and line2, by={a,b,c}}];% here
  \coordinate (aux) at (nc0-|s2nc);
  \draw[thick,-,dgreen] (s1nc) |- (nc0);
  \draw[thick,-,dgreen,connect=(nc0) to (nc) over (a) by 2pt];
  \draw[thick,-,dgreen,connect=(nc0) to (aux) over (c) by 2pt];
  \draw[thick,-,dgreen] (aux)--(s2nc);
   
  
  \path (or) +(0,7.5mm) coordinate (or0) +(0,4.5mm) coordinate(or1) +(5mm,7.5mm) coordinate(or2);
  \path[-,name path=line3](s1or)|-(or0)--(or)--(or0)-|(s2or);
  \path[name intersections={of=line3 and line2, by={a,b,c}}];% here
  \path[name intersections={of=line3 and line1, by={d,e,f}}];% here
  \coordinate (aux) at (or0-|s2or);
  \draw[thick,-,red] (s1or) |- (or0);
  \draw[thick,-,red,connect=(or0) to (or1) over (a) by 2pt];
  \draw[thick,-,red,connect=(or1) to (or) over (d) by 2pt];
  \draw[thick,-,red,connect=(or0) to (or2) over (f) by 2pt];
  \draw[thick,-,red,connect=(or2) to (aux) over (c) by 2pt];
  \draw[thick,red,-] (aux)--(s2or); 

  \path (nr) +(0,9.5mm) coordinate (nr0) +(0,6.5mm) coordinate(nr1) +(5mm,9.5mm) coordinate(nr2)
 +(-5mm,9.5mm) coordinate(nr3);
  \path[-,name path=line4](s1nr)|-(nr0)--(nr)--(nr0)-|(s2nr);
  \path[name intersections={of=line4 and line1, by={a}}];% here
  \path[name intersections={of=line4 and line2, by={b,c,d}}];% here
  \path[name intersections={of=line4 and line3, by={e,f,g}}];% here
  \coordinate (aux) at (nr0-|s2nr);
  \draw[thick,-,dgreen] (s1nr) |- (nr3);
  \draw[thick,-,dgreen,connect=(nr3) to (nr0) over (a) by 2pt];
  \draw[thick,-,dgreen,connect=(nr0) to (nr1) over (e) by 2pt];
  \draw[thick,-,dgreen,connect=(nr1) to (nr) over (b) by 2pt];
  \draw[thick,-,dgreen,connect=(nr0) to (nr2) over (d) by 2pt];
  \draw[thick,-,dgreen,connect=(nr2) to (aux) over (g) by 2pt];
  \draw[thick,-,dgreen] (s2nr) -- (aux);




\end{tikzpicture} 
\caption{High Level Representation of a Decentralized Send-Receive Model of the Task Manager Example}
\label{fig:sr}
\end{figure*}

\begin{example}
  Figure~\ref{fig:sr} describes a Send/Receive model of the Task Manager system with a 
  decentralized scheduling. 
  The set of interactions is partitioned into two classes 
  $\gamma_1=\{\alpha_1,\alpha_3,\alpha_5,\alpha_7\}$ and
  $\gamma_2=\{\alpha_2,\alpha_4,\alpha_6,\alpha_8\}$, each one handled by a scheduler ($Sch_1$
  and $Sch_2$ respectively). Since $\gamma_1$ and $\gamma_2$ are conflicting (interactions of 
  $\gamma_1$ are conflicting with interactions of $\gamma_2$), for instance
  $\alpha_1$ and $\alpha_2$ are potentially conflicting, schedulers rely on the conflict
  resolution layer to resolve the conflicts. In this case, they emit a request ($req_k$ with
  $k\in\{1,2\}$) and wait for a notification granting (respectively denying) them the execution
  of an interaction ($ok_k$ respectively $fail_k$).
  Notice that components $C^{SR}$ and $R^{SR}$ send offers to both schedulers since
  they are participating in interactions handled in both schedulers.
\end{example}
\subsection{Send/Receive Components}
\begin{figure}[!h]
 \centering
  \begin{tikzpicture}[every node/.style={scale=0.8},scale=0.8]

  \node [place,label=left:\textcolor{red}{$x\le4$}](l0) {$\loc$};
  \node [above=7.5mm of l0](l01) {};
  \node [above=7.5mm of l0,xshift=7.5mm](l02) {};
  \node [above=7.5mm of l0,xshift=-7.5mm](l03) {};
  \node [place,below=1.3cm of l0,xshift=1cm] (l1) {\phantom{$\loc$}};
  \node [place,below=1.3cm of l0,xshift=-1cm] (l2) {\phantom{$\loc$}};

  \path (l01) edge (l0)
        (l02) edge (l0)
        (l03) edge (l0);
  \path (l0) edge [bend left] node[right,align=center]{$a$}(l1)
             edge [bend right] node[left,align=center]{$b$} (l2);
  
  \node [place,right=5cm of l0,yshift=1cm](l0) {\scriptsize $\locp$};
  \draw [-] ($(l0) + (180:4mm)$) arc (180:360:4mm);
  \node [above=7.5mm of l0](l01) {};
  \node [above=7.5mm of l0,xshift=7.5mm](l02) {};
  \node [above=7.5mm of l0,xshift=-7.5mm](l03) {};
  \node [place,label=left:\textcolor{red}{$x\le4$},below=1cm of l0](l1) {$\loc$};
  \node [place,below=1cm of l1,xshift=1cm] (l2) {\phantom{$\loc$}};
  \node [place,below=1cm of l1,xshift=-1cm] (l3) {\phantom{$\loc$}};


  \path (l01) edge (l0)
        (l02) edge (l0)
        (l03) edge (l0)
        (l0) edge node[right,align=center]{$o$}(l1)
        (l1) edge [bend left] node[right,align=center]{$a$} (l2)
        (l1) edge [bend right] node[left,align=center]{$b$} (l3);
\end{tikzpicture}
  \caption{Offer Construction}
 \label{fig:offer}
\end{figure}  



The transformation of a timed component $B$ into a Send/Receive component $B^{SR}$ relies 
on decomposing each transition of $B$ into two transitions: \emph{(1)} an offer (send)
transition and \emph{(2)} a notification (receive) transition. This is done by splitting
each location $\loc$ into two locations, $\loc$ itself and $\locp$ as shown in 
Figure~\ref{fig:offer}.

When at $\locp$ location, the Send/Receive component is not in a \say{stable} state, that is,
it is able only to send an offer to its respective scheduler(s). 
We require that offers are sent as soon as 
possible meaning that there is no delay when at location $\locp$. We call such location
\emph{urgent} location, graphically represented by a $\smile$ inside the location. 
From a semantics point of view, an urgent location is equivalent to adding an extra clock
$x$ that is reset on all incoming edges, and having an invariant $x\le0$ on the location 
as illustrated in Figure~\ref{fig:urgloc}. Thus, time is not allowed to pass when the system 
is in such locations.  
An offer contains the exact variables encoding the current state of a component. It includes
the following variables:
\begin{itemize}
  \item An invariant variable of the current location invariant 
  \item A guard variable for each action (port) that is set to the guard (over clocks and data)
    of each transition outgoing from the current location and labeled by this port if it exists, 
    otherwise to $\false$
  \item A \emph{Boolean} variable for each clock indicating whether the corresponding clock 
    was reset by the last transition or not 
  \item A \emph{participation number} variable used for conflict resolution
\end{itemize}
\begin{figure}[!h]
 \centering
  \begin{tikzpicture}[scale=0.8,every node/.style={scale=0.8}]


  \node [place] (l1) {$\loc_1$};
  \node [place,below=1.5cm of l1,label=left:\textcolor{red}{$x\le0$}](l2) {$\loc_2$};

  \path (l1) edge node[right,align=center]{$a$\\$x:=0$}(l2);
  
  \node [place,right=4cm of l1] (l1) {$\loc_1$};
  \node [place,below=1.5cm of l1](l2) {$\loc_2$};
  \draw [-] ($(l2) + (180:4mm)$) arc (180:360:4mm);
 % \node [place,label=left:\textcolor{red}{$x\le4$},below=1cm of l0](l1) {$\loc$};
%  \node [place,below=1cm of l1,xshift=1cm] (l2) {\phantom{$\loc$}};
%  \node [place,below=1cm of l1,xshift=-1cm] (l3) {\phantom{$\loc$}};
  \path (l1) edge node[right,align=center]{$a$}(l2);

\end{tikzpicture}
  \caption{Representation of an Urgent Locatoion}
 \label{fig:urgloc}
\end{figure}  



\begin{definition}[Send/Receive Component]\label{def:tc_SR}
  Let $B=(\Loc,\loc_{0},\X,\D,\A,\E,\{f_e\}_{e\in\E},\I)$ be a timed component. The 
  corresponding Send/Receive component is defined by the timed component
  $B^{SR}=(\Loc^{SR},\loc_0^{SR},\emptyset,\D^{SR},\mc{P}^{SR},\E^{SR},\{f_e\}_{e\in\E^{SR}},
  \emptyset)$, such that:
  \begin{itemize}
    \item $\Loc^{SR}=\Loc\cup\Loc^{\perp}$, where $\Loc^{\perp}=\{\locp|\loc\in\Loc\}$.
      Locations of $\Loc^{\perp}$ are urgent locations.
    \item $\loc_0^{SR}=\loc_{0_{\perp}}\in\Loc^{\perp}$ is the initial location.
    \item $\mc{P}^{SR}= P\cup\{o\}$, where $P=\{p_a|a\in\mc{A}\}$ is the set of ports for each
      action of $B$ and $o$ is the offer port. 
    \item $\mc{D}^{SR}=\mc{D}\cup\{g_{p_a}\}_{a\in\mc{A}}\cup\{inv_B\}\cup\{r_x\}_{x\in\mc{X}}
      \cup\{n_B\}$,
      where $g_{p_a}$ are guard variables, $inv_{B}$ is an invariant variable, $r_x$ are 
      Boolean reset variables and $n_B$ is a participation number variable.
      Variables $\mc{D}_o^{SR}\subset\mc{D}^{SR}=\{g_{p_a}\}_{a\in\mc{A}}\cup\{i\}_{B}\cup
      \{n_B\}\cup\{r_x\}_{x\in\mc{X}}$  are exported by the offer port.
    \item For each place $\loc\in\Loc$, we include an offer transition $e_{\loc}=(\locp,o,\true,
      \emptyset,\loc)$ in $\mc{E}^{SR}$
    \item For each transition $e=(\loc,a,g,r,\loc')\in\mc{E}$, we include a notification 
      transition $e_{p_a}=(\loc,p_a,\true,\emptyset,\locp')$. The transfer function $f_{e_{p_a}}$
      applies the original transfer function $f_e$ of e, then update guard variables, 
      invariant variable, reset variables and the participation number as follows:
      \begin{itemize}
        \item $\forall a'\in\mc{A}, g_{p_{a'}}:=\begin{cases}
            g_{a'} & \text{if } e'=(\loc',a',g_{a'},r',\loc'')\in\mc{E}\\
          \false & otherwise
        \end{cases}$
        \item $inv_B:=\Inv{\loc'}$
        \item $\forall x\in\mc{X}, r_x:=\begin{cases}
            \true & \text{if } x\in r\\
          \false & otherwise
        \end{cases}$
        \item $n_B:=n_B+1$
      \end{itemize}
  \end{itemize}
\end{definition}

This definition of a Send/Receive component relates the execution of a transition 
$e=(\loc,a,g,r,\loc')\in\mc{E}$ from the initial component $B$ to the following two 
execution steps in $B^{SR}$. First, an offer transition $e_{\loc}=(\locp,o,\true,\emptyset,\loc)$
sends for each port $p\in\mc{P}$ the guard over clocks and data corresponding to the enabledness
of $p$ at $\loc$, the location invariant $inv_B$, the participation number $n_B$ for
component $B$, as well as the reset variable $r_x$ for each clock $x\in\mc{X}$, such that,
$r_x=\true$, if $x$ has been reset by the previous transition execution. Reset variables
$r_x$ are used to update values of clocks in the Scheduling layer before computing guard of 
interactions.
In the second place, a notification transition $e_{p_a}=(\loc,p_a,\true,\emptyset,\locp')$  
is executed upon the execution of an interaction involving $p_a$ in the scheduling layer. 
In the same manner to $e$ in $B$, $e_{p_a}$ updates values of variables $\mc{D}$ according to
to the transfer function $f_e$, as well as variables needed for the next offer. 
Figure~\ref{fig:tcSR} depicts the Send/Receive transformation of the component $C$ of Example
~\ref{fig:tm}.
\begin{figure}[!h]
 \centering
  \shorthandoff{:!}
\begin{tikzpicture}[scale=0.8,every node/.style={scale=0.8}]

  \node [accepting, place] (l0b)  {$\locpb{0}{1}$};
  \draw [-] ($(l0b) + (180:4mm)$) arc (180:360:4mm);
  \node [place,right=1.5cm of l0b] (l0)  {$\loc_0^1$};
  \node [place,below=2cm of l0] (l1b)  {$\locpb{1}{1}$};
  \draw [-] ($(l1b) + (180:4mm)$) arc (180:360:4mm);
  \node [place,left=1.5cm of l1b] (l1)  {$\loc_1^1$};
  
  \path (l0b) edge node[above,align=center]{$o$}(l0)
        (l0) edge node[right,align=center,xshift=1.4cm,yshift=1cm](init){$init_0$}(l1b)
        (l1b) edge node[below,align=center]{$o$}(l1)
        (l1) edge node[left,align=center,xshift=-1.4cm,yshift=1cm](start){$start_0$}(l0b);
  \node [align=center,below=0cm of start,rectangle,fill=gray!30](u1) {$g_{init_0}:=[z>25]$\\
                        $g_{start_0}:=\false$\\
                        $r_z:=\true$\\
                        $\mc{I}_{C^{SR}}=\true$};
  \node [align=center,below=0cm of init,rectangle,fill=gray!30](u2) {$g_{init_0}:=\false$\\
                        $g_{start_0}:=\true$\\
                        $r_z:=\false$\\
                        $\mc{I}_{C^{SR}}=\true$};

  \node [rounded corners,inner xsep=11mm,inner ysep=11mm,draw, fit=(l0)(l1)(l0b)(l1b)(u1)(u2)] (rec1) {};
  \node [triangle,label=-90:$o$] (o) at ($(rec1.north west)!0.25!(rec1.north east)$) {};
  \node [dots,label=-90:$init_0$] (n1) at ($(rec1.north west)!0.75!(rec1.north east)$) {};
  \node [dots,label=-90:$start_0$] (n2) at ($(rec1.north west)!0.85!(rec1.north east)$) {};
\end{tikzpicture}
  \caption{Send/Receive Transformation of the Controller Component From Figure~\ref{fig:tm}}
 \label{fig:tcSR}
\end{figure}  





\subsection{Scheduling Layer}
As explained earlier, the scheduling layer works with a partial view of the global state of 
the system. Initially, every scheduler is waiting for offers form the corresponding components
(with respect to the interaction partition). Each received offer specifies to the schedulers
the state of the sender component. In what follows, we consider schedulers preserving the 
execution sequences of the initial model under
the standard semantics. For the sake of distributed implementation, it is
worth taking a decision as soon as possible. Thus, once a scheduler gathers enough information 
for scheduling interactions, it arbitrarily choose one and executes the corresponding 
transitions that will trigger the notification responses to the components involved in 
that interaction.
In what follows, we use Petri nets~\cite{petri} to describe the scheduling layer. 
Petri nets are a well suited formalism for encoding parallel and concurrent executions. 
Particularly, we focus on a class of Petri nets (\emph{1-Safe}) to encode the structure of 
the schedulers introduced by our method since they provide clear and compact representations. 
Note that any 1-Safe Petri net can be 
transformed in an equivalent automaton~\cite{petri2}. 

\subsubsection{Petri Nets}
\begin{definition}[Petri Net]\label{def:pn}
  A Petri net is a 3-tuple $\mc{P}=(\Loc,\mc{A},\mc{T})$ where $\Loc$ is a set of finite
  \emph{places}, $\mc{A}$ is a finite set of actions, and $\mc{T}\subseteq 2^{\Loc}\times
  \mc{A}\times 2^{\Loc}$ is a set of transitions. A transition $\tau$ is a triple 
  $(^\bullet\tau,a, \tau^{\bullet})$, where $^\bullet\tau$ is the set of input places of 
  $\tau$ and $\tau^{\bullet}$ is 
  the set of output places of $\tau$.
\end{definition}

A Petri net can be represented as a directed bipartite graph $\mc{G}=(\mc{V},\mc{E})$ where 
$\mc{V}$ denotes the set of vertices and $\mc{E}$ denotes the set of directed edges. 
The set of vertices is structured into two classes, namely places and transitions, 
that is, $\mc{V}=\Loc\cup\mc{T}$. 
Places are represented by circular vertices and transitions are represented by rectangular 
vertices as shown in Figure~\ref{fig:pn}. 
The set of directed edges $\mc{E}$ is the union of the sets $\{(\loc,\tau)\in\Loc\times\mc{T}|
\loc\in{^\bullet\tau}\}$ and $\{(\tau,\loc)\in\mc{T}\times\Loc|\loc\in\tau^\bullet\}$.
A \emph{marking} of a Petri net is a mapping $m:\Loc\to\naturals$ that describes the current
\emph{state} of a Petri net by assigning a non-negative integer to each of its places.
We use tokens~\cite{petri} to represent the marking (number of tokens). We say that a place is
\emph{marked} if it contains at least one token.
For a transition $\tau$, we say that $\tau$ is \emph{enabled} at a given state if all of 
its input places $^\bullet\tau$ are marked, that is, $\forall\loc\in{^\bullet\tau},\ m(\loc)>0$.
A \emph{firing} (execution) of an enabled transition removes one token from each input place
and adds one token to each output place. Formally, the firing of a transition from a marking 
$m$ results in a marking $m'$ such that:
\begin{displaymath}
  \forall\loc\in\Loc, \ m'(\loc)=m(\loc)-\tau^{-}(\loc)+\tau^{+}(\loc) 
\end{displaymath}
where
\begin{displaymath}
  \tau^{-}(\loc)=\begin{cases}
    1 & \text{if }\loc\in{^\bullet\tau}\\
    0 & \text{otherwise} 
  \end{cases}
 \quad \text{and}\quad
\tau^{+}(\loc)=\begin{cases}
    1 & \text{if }\loc\in{\tau^\bullet }\\
    0 & \text{otherwise} 
\end{cases}\end{displaymath}

We put $m\transitb{a}{\mc{P}}m'$ to denote that a transition $\tau=({^\bullet\tau},a,
  \tau^\bullet)$ can be executed at marking $m$ and reaches marking $m'$. 
  We also denote by $\transitb{}{\mc{P}}$
the set of triples $(m,a,m')$ such that $m\transitb{a}{\mc{P}}m'$.
  \begin{figure}[!h]
 \centering
  \begin{tikzpicture}[scale=0.8,every node/.style={scale=0.8}]

  \node [markplace](l0) {$\loc_0$};
  \node [place,right=1.5cm of l0,yshift=1.5cm] (l1) {$\loc_1$};
  \node [place,right=1.5cm of l0,yshift=-1.5cm] (l2) {$\loc_2$};
  \node [transition,right=7.5mm of l0,yshift=-4mm,rotate=-90,label=below:$\tau_2$] (t1) {};
  \node [transition,right=7.5mm of l0,yshift=4mm,rotate=90,label=above:$\tau_1$] (t2) {};
  \node [transition,above=5mm of l0,label=left:$\tau_3$] (t3) {};
  \node [transition,below=5mm of l0,label=left:$\tau_4$] (t4) {};

  \path (l0) edge (t1)
             edge (t2)
        (t2) edge (l1)
        (t1) edge (l2)
        (t3) edge (l0)
        (t4) edge (l0);
    \draw[-stealth,rc] (l1) -- ([h2n]t3.center) --(t3);
    \draw[-stealth,rc] (l2) -- ([h2s]t4.center) --(t4);
   % (l1) [bend right]edge (t3)
   %     (l2) [bend left]edge (t4);

  \node [place,right=5cm of l0](l0) {$\loc_0$};
  \node [markplace,right=1.5cm of l0,yshift=1.5cm] (l1) {$\loc_1$};
  \node [place,right=1.5cm of l0,yshift=-1.5cm] (l2) {$\loc_2$};
  \node [transition,right=7.5mm of l0,yshift=-4mm,rotate=-90,label=below:$\tau_2$] (t1) {};
  \node [transition,right=7.5mm of l0,yshift=4mm,rotate=90,label=above:$\tau_1$] (t2) {};
  \node [transition,above=5mm of l0,label=left:$\tau_3$] (t3) {};
  \node [transition,below=5mm of l0,label=left:$\tau_4$] (t4) {};
  \path (l0) edge (t1)
             edge (t2)
        (t2) edge (l1)
        (t1) edge (l2)
        (t3) edge (l0)
        (t4) edge (l0);
    \draw[-stealth,rc] (l1) -- ([h2n]t3.center) --(t3);
    \draw[-stealth,rc] (l2) -- ([h2s]t4.center) --(t4);

\end{tikzpicture}
 \caption{A Simple Petri Net with Two Succesive Markins}
 \label{fig:pn}
\end{figure}  




\begin{example}
  Figure~\ref{fig:pn} depicts an example of a Petri net with two successive markings. It includes
  three places $\{\loc_1,\cdots,\loc_3\}$ and four transitions $\{t_1,\cdots,t_4\}$. For clarity,
  places with tokens are represented with filled gray circles (blue = zero tokens, 
  gray = one token). The left side marking 
  shows the initial marking of the Petri net whereas the right side marking results from the
  execution of transition $t_1$.
\end{example}

Given a Petri net $\mc{P}=(\Loc,\mc{A},\mc{T})$ and an initial marking $m_0$, the marking $m$
is \emph{reachable} if there exists a sequence of transitions $m_0\transitb{a_1}{\mc{P}}m_1
\transitb{a_1}{\mc{P}}\cdots\transitb{a_1}{\mc{P}}m$. We say that $\mc{P}$ is \emph{1-Safe}
if there is at most one token per place in each reachable marking. This implies at most
$2^{|\Loc|}$ markings. In this thesis, we consider only this class of Petri nets.
The behavior of a 1-Safe Petri net  $\mc{P}=(\Loc,\mc{A},\mc{T})$ is defined by the finite
labeled transition system $(2^{\Loc},\mc{A},\to_{\mc{P}})$, where $2^{\Loc}$ is the set of
states, $\mc{A}$ is the set of actions, and $\to_{\mc{P}}\subseteq 2^{\Loc}\times\mc{A}\times
2^{\Loc}$ is the set of transitions defined as follows. We have $(m,a,m')\in\to_{\mc{P}}$, 
denoted by $m\transitb{a}{\mc{P}}m'$, if there exists $\tau=({^\bullet\tau},a,{\tau^\bullet })
  \in\mc{T}$ such that ${^\bullet\tau}\subseteq m$ and $m'=(m\textbackslash{^\bullet\tau})
  \cup{\tau^\bullet }$.
In this case, we say that $a$ is enabled at $m$.

\subsubsection{Building Schedulers}
  Given a timed system $\gamma(B_1,\cdots,B_n)$ and a partition of interactions 
  $\{\gamma_j\}^m_{j=1}$, each class of the interaction partition is handled by a single
  scheduler component, namely $Sch_j$.
  The behavior of each scheduler is described as a 1-Safe Petri net in which there is a token
  for each component flowing between three or four different types of places as shown
  in Figure~\ref{fig:schSR}:
  \begin{figure}[!h]
 \centering
  \captionsetup{justification=centering}
\subfloat[Scheduling Mechanism for Internally or not Conflicting Interactions]{\label{fig:sch1}
  \begin{tikzpicture}[scale=0.7,every node/.style={scale=0.7}]
  
  \node [markplace] (w1) {$w_i$};
  \node [transition,below=1cm of w1](o1) {};
  \node [place,below=1cm of o1] (r1){$r_i$};
  \node [right=2cm of r1] (r2){};
  \node [transition,below=1cm of r1,xshift=1.5cm](a) {};
  \node [place,below=2cm of r1](sp){$s_{i_p}$};
  \node [right=2cm of sp](spp){};
  \node [transition,left=5mm of r1](s1) {};

  \path (w1) edge node[right,yshift=-3mm]{receive offer}  (o1)
        (o1) edge node[right,yshift=3mm]{from component $B_i$}(r1)
        (r1) edge node[right,xshift=1.2cm,yshift=-4mm]{execute interaction}(a)
        (a) edge node[right,xshift=1.7cm,yshift=4mm]{involving $B_i$}(sp);

  \draw[->,dashed] (r2) -- (a);
  \draw[->,dashed] (a) -- (spp);
  \draw[->,rounded corners=4mm] (sp)--([sh2sw2]sp.center) -- (s1);
  \draw[->,rounded corners=4mm] (s1) --([sh2w2]w1.center)--(w1);
  \node[left=0cm of s1,align=center]{send notification\\to $B_i$ on port $p$};
\end{tikzpicture}}
\subfloat[Scheduling Mechanism for Externally Conflicting Interactions]{\label{fig:sch2}
  \begin{tikzpicture}[scale=0.7,every node/.style={scale=0.7}]
  
  \node [markplace] (w1) {$w_i$};
  \node [transition,below=5mm of w1](o1) {};
  \node [place,below=5mm of o1] (r1){$r_i$};
  \node [right=2cm of r1] (r2){};
  \node [transition,below=5mm of r1,xshift=1.5cm](a) {};
  \node [transition,right=30mm of a](f) {};
  \node [place,below=5mm of a](t){$t_a$};  
  \node [transition,below=5mm of t](aa) {};
  \node [place,below=5mm of aa,xshift=-1.5cm](sp){$s_{p_i}$};
  \node [right=2cm of sp](spp){};
  \node [left=2mm of f,yshift=11mm](r3){};
  \node [transition,left=18mm of a](s1) {};

  \path (w1) edge (o1)
        (o1) edge (r1)
        (r1) edge node[right,align=center,xshift=1cm,yshift=-5mm]{sending request\\ to the CRP}(a)
        (r1) edge [loop,in=50,out=0,looseness=5] node[right]{new offer} (r1)
        (a) edge (t)
        (t) edge node[right,align=center,xshift=5mm,yshift=-10mm]
{CRP grants the execution\\ of the interaction \\involving $B_i$}(aa)
        (t) edge [loop,looseness=5, out=-100, in=-140 ](t)
        (t) edge [loop left,dashed] node[left,yshift=-3mm,xshift=3mm]{new offers} (t)
        (t) edge[bend right=50] (f)
        (f) edge[bend right=80] node[above,align=center,xshift=1cm]
{CRP denies the execution\\ of the interaction \\involving $B_i$} (r1)
        (f) edge[dashed,-,bend right=40] (r3)
        (aa) edge (sp);

  \draw[->,dashed] (r2) -- (a);
  \draw[->,dashed] (aa) -- (spp);
  \draw[->,rounded corners=4mm] (sp)--([sh2sw2]sp.center) -- (s1);
  \draw[->,rounded corners=4mm] (s1) --([sh2w2]w1.center)--(w1);
\end{tikzpicture}}
 \caption{Scheduling Mechanism}
 \label{fig:schSR}
\end{figure}  


  \begin{itemize}
    \item \emph{Waiting place}: For each component participating in an interaction handled by
      a scheduler, the corresponding scheduler include a waiting place signifying that 
      it is waiting for the component offer. Waiting places are labeled by $w$. 
    \item \emph{Receive place}: When receiving an offer from a component, the corresponding
      token is moved from the corresponding waiting place to the receive place (one received
      place per component) and stays there until an interaction including this component is 
      scheduled or requested for scheduling (through the conflict resolution layer). 
      Receive places are labeled by $r$.
    \item \emph{Try place}: Try places (labeled by $t$) 
      concern only components that are participating in an interaction that is externally
      conflicting with another interaction (of another scheduler). 
      As explained in Subsection~\ref{sub:conf}, schedulers
      rely on the conflict resolution layer to resolve conflicts. For each externally
      conflicting interaction, a try place is inserted. When scheduling such interactions,
      tokens are moved from receive places of components to try place of that interaction, 
      meaning that a request has been sent to the CRP. Following this request, the CRP 
      either grants the execution
      of the interaction and the tokens are moved to the sending places, or denies the execution
      which results in moving the tokens back to the receive places.
      Moreover, loops for offer transitions are added on try places and receive places
      in order to accept new offers sent by the components if executed as part of an externally
      conflicting interaction.
    \item \emph{Sending place}: Once an interaction has been scheduled for execution,
      the corresponding components tokens are moved from receive places (or try place) to
      send places corresponding to ports of components participating in that interaction. 
      There is one send place for each port (excluding offer port) 
      for every Send/Receive component.
      Sending places are labeled by $s$.
  \end{itemize}
As illustrated in Figure~\ref{fig:schSR}, tokens are initially in waiting places. Once an offer
is received by a scheduler, the corresponding token moves to the receive place. The scheduler
copies then the values of the offer variables to its local variables. Once offers of all 
components involved in an interaction have been gathered, schedulers computes its guard. If 
the guard evaluates to $\true$, with respect to data variables and the global scheduler clock,
the scheduler can either execute the interaction if it is not externally conflicting with
another interaction (Figure~\ref{fig:sch1}). Tokens are then moved to send places of components
ports participating in that interaction. Otherwise (Figure~\ref{fig:sch2}), 
a request is sent to the CRP and the token is moved to the
corresponding $try$ place. Thereafter, either the CRP grants the execution of the interaction
and tokens are moved to send places, or the execution is denied and tokens are moved back
to receive places. Possibly, the scheduler may receive new offers when being in $try$
places. This corresponds to the execution of a conflicting interaction in another scheduler.


\begin{definition}[Scheduler]\label{def:sch_sr}
  Let $\gamma(B_1,\cdots,B_n)$ be a timed system and $\gamma_j\subset\gamma$ be a subset of
  interactions. The corresponding scheduler $Sch_j$ responsible for executing interactions
  of $\gamma_j$ is defined by the tuple  
  $Sch_j=(\Loc_j, \mc{P}_j,\mc{T}_j,\mc{X}_j, \mc{D}_j,\{g_{\tau}\}_{\tau\in\mc{T}_j},
  \{r_{\tau}\}_{\tau\in\mc{T}_j},\{f_{\tau}\}_{\tau\in\mc{T}_j},\{\mc{I}_{\loc}\}_{\loc\in\Loc_j})
  $, where $(\Loc_j,\mc{P}_j,\mc{T}_j)$ is a 1-Safe Petri net defining 
      the structure of the scheduler such that:
  \begin{itemize}
    \item $\mc{X}_j=\{t_j\}\cup\{z_j\}$ is the set of clocks of $Sch_j$, where $t_j$ is
      the global clock used for scheduling interactions of $\gamma_j$ (it is never reset)
      and $z_j$ is a clock used for enforcing urgent behavior. 
    \item $\mc{D}_j$ is the set of variables containing:
      \begin{itemize}
        \item Variables updated whenever an offer from a component $B_i$ participating in 
          interactions of $\gamma_j$ is received. These variables consist of: an invariant
          variable $inv_{B_i}$ and a participation number $n_{B_i}$ for each $B_i$, 
          a guard variable $g_{p_a}$ for each action of $B_i$ involved in an interaction of 
          $\gamma_j$, and a Boolean reset variable for each clock of $B_i$.
        \item Reset time variables that stores the absolute time of the last reset of 
          each component clocks. For each clock $x$ of $B_i$ we include a reset time 
          variable $\rho_x$.
      \end{itemize}
    \item For each transition $\tau\in\mc{T}_j$, $g_{\tau}$ is a guard over 
      $\mc{X}_j$ and $\mc{D}_j$.
    \item For each transition $\tau\in\mc{T}_j$, $r_{\tau}$ is a reset function over $\X_j$.
    \item For each transition $\tau\in\mc{T}_j$, $f_{\tau}$ is a transfer function over $\D_j$.
    \item For each place $\loc\in\Loc_j$, $\mc{I}_{\loc}$ is an invariant over $\mc{X}_j$. 
    \item The 1-Safe Petri net $(\Loc_j,\mc{P}_j,\mc{T}_j)$ is structured as follows:
      \begin{itemize}
        \item $\Loc_j$ is the set of places. It includes four types of places:
        \begin{itemize}
          \item For each component $B_i$ involved in interactions of $\gamma_j$, we include
            a waiting place $w_i^j$, a receive place $r_i^j$, where $\mc{I}_{w_i^j}=\true$
            and $\mc{I}_{r_i^j}$ is the invariant $inv_{B_i}$ expressed on $t_j$.
          \item For each action $a$ involved in interactions of $\gamma_j$, we include a sending
            place $s_{p_a}$, where $\mc{I}_{s_{p_a}}=z_j\le0$.
          \item For each interaction $\alpha\in\gamma_j$ that is externally conflicting 
            with another interaction, we include a try place $t_{\alpha}$ with 
            $\mc{I}_{t_{\alpha}}$ is the invariant $inv_{B_i}$ expressed on $t_j$.
        \end{itemize}
        \item $\mc{P}_j$ is the set of ports. It includes the following ports:
          \begin{itemize}
            \item For each component $B_i$ involved in interactions of $\gamma_j$, we include
              a receive port $o_i^j$. Each port $o_i^j$ is associated with the variables
              $g_{p_a}$ and $r_x$ for each action, respectively clock, of $B_i$, as well
              as the variables $inv_{B_i}$ and $n_i$.
            \item For each action $a$ involved in interactions of $\gamma_j$, we include 
              a send port $p_a$.
            \item For each interaction $\alpha\in\gamma_j$ that is externally conflicting with
              another interaction, we include a send port $rsv_{\alpha}$ (reservation port), 
              and receive ports $ok_{\alpha}$ (granted execution) and $fail_{\alpha}$ (denied 
              execution). The port $rsv_{\alpha}$ exports the variables $\{n_i\}_{B_i
              \in\p{\alpha}}$.
            \item For each interaction $\alpha\in\gamma_j$ that is internally or not conflicting
              with other interactions of $\gamma_j$, we include the unary port $\alpha$.
          \end{itemize}
        \item $\mc{T}_j$ is the set of transitions. It consists of the following:
          \begin{itemize}
            \item For each component $B_i$, $\mc{T}_j$ includes the offer transitions 
              $(w_i^j,o_i,r_i^j)$, $(r_i^j,o_i,r_i^j)$ and $\{(t_{\alpha},o_i,t_{\alpha})|
              B_i\in\p{\alpha}\}$ where $\alpha$ is an externally conflicting interaction.
              The execution of offer transitions copy the variables attached to components offer
              to its local variables. Offer transitions have no guards and no reset functions. 
              Their transfer functions update reset time 
              variables $\rho_x$ whenever $r_x=\true$, that is, $\rho_x:=t_j$. 
            \item For each action $a$ involved in interactions of $\gamma_j$, $\mc{T}_j$ includes
              a transition $(s_{p_a},p_a,w_i^j)$ where $i$ is the index of the component
              containing $a$. This transition notifies the corresponding Send/Receive component 
              to execute the transition labeled by $p_a$. It has no guard, no reset function,
              and no transfer function.
            \item For each interaction $\alpha=\{a_i\}_{i\in{I}}\in\gamma_j$, that is 
              internally conflicting or not conflicting with interactions of $\gamma_j$,
              $\mc{T}_j$ includes the transition $\tau_{\alpha}=(\{r_i^j\}_{B_i\in\p{\alpha}},
              \alpha,$\\$\{s_{p_{a_i}}\}_{a_i\in\alpha})$. The guard of this transition is 
              $g_{\tau_{\alpha}}=
              \bigwedge_{a_i\in\alpha}g_{p_{a_i}}$. Notice that guards over data are the same,
              whereas guard of clocks are expressed using the global clock $t_j$ and the reset
              time variables $\rho_x$. This transition has no transfer function 
              and its reset function resets the clock $z_j$.
            \item For each interaction $\alpha=\{a_i\}_{i\in{I}}\in\gamma_j$, that is 
              externally conflicting,
              $\mc{T}_j$ includes the following transitions:
              \begin{itemize}
                \item $\tau_{rsv_{\alpha}}=(\{r_i^j\}_{B_i\in\p{\alpha}},rsv_{\alpha},t_{\alpha}
                  )$. The guard of this transition is $g_{\tau_{rsv_{\alpha}}}=
                  \bigwedge_{a_i\in\alpha} g_{p_{a_i}}$. 
                \item $\tau_{ok_{\alpha}}=(t_{\alpha},ok_{\alpha},\{s_{p_{a_i}}\}_{a_i\in\alpha})
                  $. This transition has no guard, no transfer function, and its reset function is
                   $r_{\tau_{ok_{\alpha}}}=\{z_j:=0\}$.
                \item $\tau_{fail_{\alpha}}=(t_{\alpha},fail_{\alpha},\{s_{p_{a_i}}\}_{a_i
                  \in\alpha})$. This transition has no guard, no reset function, and no 
                  transfer function.
              \end{itemize}
          \end{itemize}
      \end{itemize}
  \end{itemize}
\end{definition}
\begin{figure}[!h]
 \centering
  \captionsetup{justification=centering}
  \begin{tikzpicture}[scale=0.7,every node/.style={scale=0.7}]
  
  \node [markplace] (w1) {$w_1^1$};
  \node [markplace,right=4cm of w1] (w2) {$w_2^1$};
  \node [markplace,right=4cm of w2] (w3) {$w_4^1$};
  \node [transition,below=7.5mm of w1](o11) {} ;
  \node [transition,below=7.5mm of w2](o21) {};
  \node [transition,below=7.5mm of w3](o31) {};
  \node [left=1mm of o11] {$o_1$} ;
  \node [left=1mm of o21] {$o_2$} ;
  \node [left=1mm of o31] {$o_4$} ;
  \node[place,below=15mm of w1] (r1){$r_1^1$}; 
  \node[place,below=15mm of w2] (r2){$r_2^1$}; 
  \node[place,below=15mm of w3] (r3){$r_4^1$}; 
  \node [transition,below=7.5mm of r1,xshift=1.75cm](a1) {};
  \node [transition,below=7.5mm of r2,xshift=-1.75cm](a3) {};
  \node [transition,below=7.5mm of r2,xshift=1.75cm](a5) {};
  \node [transition,below=7.5mm of r3,xshift=-1.75cm](a7) {};
  \node [left=1mm of a1] {$rsv_{\alpha_1}$} ;
  \node [left=1mm of a3] {$rsv_{\alpha_3}$} ;
  \node [right=1mm of a5] {$rsv_{\alpha_5}$} ;
  \node [right=1mm of a7] {$rsv_{\alpha_7}$} ;
  \node[place,below=7.5mm of a1] (t1){$t_{\alpha_1}$}; 
  \node[place,below=7.5mm of a3] (t3){$t_{\alpha_3}$}; 
  \node[place,below=7.5mm of a5] (t5){$t_{\alpha_5}$}; 
  \node[place,below=7.5mm of a7] (t7){$t_{\alpha_7}$}; 
  \node [transition,below=7.5mm of t1](a11) {};
  \node [transition,below=7.5mm of t3](a33) {};
  \node [transition,below=7.5mm of t5](a55) {};
  \node [transition,below=7.5mm of t7](a77) {};
  \node [left=1mm of a11] {$ok_{\alpha_1}$} ;
  \node [left=1mm of a33] {$ok_{\alpha_3}$} ;
  \node [right=1mm of a55] {$ok_{\alpha_5}$} ;
  \node [right=1mm of a77] {$ok_{\alpha_7}$} ;
  \node[place,below=7.5mm of a11,xshift=-3.8cm] (s1){$s_{init_0}$}; 
  \node[place,right=1cm of s1] (s2){$s_{\substack{sta-\\rt_0}}$}; 
  \node[place,right=1cm of s2] (s3){$s_{init_1}$}; 
  \node[place,right=1cm of s3] (s4){$s_{\substack{sta-\\rt_1}}$}; 
  \node[place,right=1cm of s4,align=center] (s5){$s_{\substack{pro-\\cess_1}}$}; 
  \node[place,right=1cm of s5] (s6){$s_{end_1}$}; 
  \node[place,right=1cm of s6] (s7){$s_{take}$}; 
  \node[place,right=1cm of s7] (s8){$s_{free}$}; 
  \node [transition,below=7.5mm of s1](s11) {};
  \node [transition,below=7.5mm of s2](s22) {};
  \node [transition,below=7.5mm of s3](s33) {};
  \node [transition,below=7.5mm of s4](s44) {};
  \node [transition,below=7.5mm of s5](s55) {};
  \node [transition,below=7.5mm of s6](s66) {};
  \node [transition,below=7.5mm of s7](s77) {};
  \node [transition,below=7.5mm of s8](s88) {};
  \node [left=1mm of s11] {$init_0$} ;
  \node [left=1mm of s22] {$start_0$} ;
  \node [left=1mm of s33] {$init_1$} ;
  \node [left=1mm of s44] {$start_1$} ;
  \node [right=1mm of s55] {$process_1$} ;
  \node [right=1mm of s66] {$end_1$} ;
  \node [right=1mm of s77] {$take$} ;
  \node [right=1mm of s88] {$free$} ;
  \node [place,below=8.5cm of w1,dashed] (w11) {$w_1^1$};
  \node [place,right=4cm of w11,dashed] (w22) {$w_2^1$};
  \node [place,right=4cm of w22,dashed] (w33) {$w_4^1$};
  \node[transition, left=2cm of t1] (f1){};
  \node[transition, right=2mm of t3] (f3){};
  \node[transition, left=2mm of t5] (f5){};
  \node[transition, right=2cm of t7] (f7){};
  \node[left=1mm of f1]{$fail_{\alpha_1}$};
  \node[below=5mm of f3]{$fail_{\alpha_3}$};
  \node[below=5mm of f5]{$fail_{\alpha_5}$};
  \node[right=1mm of f7]{$fail_{\alpha_7}$};
  \node[above=8mm of w1](ff1){};
  \node[above=8mm of w3](ff7){};
  
  \path (w1) edge (o11)
        (w2) edge (o21)
        (w3) edge (o31)
        (o11) edge (r1)
        (o21) edge (r2)
        (o31) edge (r3)
        (r1) edge (a1)
        (r1) edge (a3)
        (r2) edge (a1)
        (r2) edge (a3)
        (r2) edge (a5)
        (r2) edge (a7)
        (r3) edge (a5)
        (r3) edge (a7)
        (a1) edge (t1)
        (a3) edge (t3)
        (a5) edge (t5)
        (a7) edge (t7)
        (t1) edge (a11)
        (t3) edge (a33)
        (t5) edge (a55)
        (t7) edge (a77)
        (a11) edge (s1)
        (a11) edge (s3)
        (a33) edge (s2)
        (a33) edge (s4)
        (a55) edge (s5)
        (a55) edge (s7)
        (a77) edge (s6)
        (a77) edge (s8)
        (s1) edge (s11)
        (s2) edge (s22)
        (s3) edge (s33)
        (s4) edge (s44)
        (s5) edge (s55)
        (s6) edge (s66)
        (s7) edge (s77)
        (s8) edge (s88)
        (s11) edge (w11)
        (s22) edge (w11)
        (s33) edge (w22)
        (s44) edge (w22)
        (s55) edge (w22)
        (s66) edge (w22)
        (s77) edge (w33)
        (s88) edge (w33)
        (r1) edge [loop,in=180,out=120,looseness=4] node[left]{$o_1$} (r1)
        (r2) edge [loop,in=180,out=120,looseness=4] node[left]{$o_2$} (r2)
        (r3) edge [loop,in=70,out=10,looseness=4] node[right]{$o_4$} (r3)
        (t1) edge [loop,in=180,out=120,looseness=4] node[left,align=center]{$o_1$\\$o_2$} (t1)
        (t3) edge [loop,in=180,out=120,looseness=4] node[left,align=center]{$o_1$\\$o_2$} (t3)
        (t5) edge [loop,in=70,out=10,looseness=4] node[right,align=center]{$o_2$\\$o_4$} (t5)
        (t7) edge [loop,in=70,out=10,looseness=4] node[right,align=center]{$o_2$\\$o_4$} (t7)
        (t1) edge [in=-90,out=-120] (f1)
        (t3) edge [in=-90,out=-60] (f3)
        (t5) edge [in=-90,out=-120] (f5)
        (t7) edge [in=-90,out=-60] (f7)
        (f1) edge [in=-150,out=90] (r1)
        (f1) edge [-,in=-180,out=90] (ff1.center)
        (ff1.center) edge [in=100,out=0] (r2)
        (f7) edge [in=10,out=90] (r3)
        (f7) edge [-,in=0,out=90] (ff7.center)
        (ff7.center) edge [in=80,out=180] (r2)
        (f3) edge [in=-100,out=90] (r2)
        (f5) edge [in=-80,out=90] (r2)
        (f3) edge [in=0,out=90,looseness=1.5] (r1)
        (f5) edge [in=180,out=90,looseness=1.5] (r3);
 
  
  \node [rounded corners,inner xsep=60mm,inner ysep=40mm,draw,fit=(w1)(w11)(ff7)(ff1)] (rec1) {};
  \node [dots,label=90:$o_1$] (i1) at ($(rec1.south west)!0.15!(rec1.south east)$) {};
  \node [triangle,rotate=180,label=-90:$init_0$] (i1) at ($(rec1.south west)!0.20!(rec1.south east)$) {};
  \node [triangle,rotate=180,label=-90:$start_0$] (i1) at ($(rec1.south west)!0.25!(rec1.south east)$) {};
  \node [dots,label=90:$o_2$] (i1) at ($(rec1.south west)!0.40!(rec1.south east)$) {};
  \node [triangle,rotate=180,label=-90:$init_1$] (i1) at ($(rec1.south west)!0.45!(rec1.south east)$) {};
  \node [triangle,rotate=180,label=-90:$start_1$] (i1) at ($(rec1.south west)!0.50!(rec1.south east)$) {};
  \node [triangle,rotate=180,label=-90:$process_1$] (i1) at ($(rec1.south west)!0.575!(rec1.south east)$) {};
  \node [triangle,rotate=180,label=-90:$end_1$] (i1) at ($(rec1.south west)!0.65!(rec1.south east)$) {};
  \node [dots,label=90:$o_4$] (i1) at ($(rec1.south west)!0.75!(rec1.south east)$) {};
  \node [triangle,rotate=180,label=-90:$take$] (i1) at ($(rec1.south west)!0.80!(rec1.south east)$) {};
  \node [triangle,rotate=180,label=-90:$free$] (i1) at ($(rec1.south west)!0.85!(rec1.south east)$) {};
      
  \node [triangle,label=-90:$rsv_{\alpha_1}$] (i1) at ($(rec1.north west)!0.15!(rec1.north east)$) {};
  \node [dots,label=-90:$ok_{\alpha_1}$] (i1) at ($(rec1.north west)!0.20!(rec1.north east)$) {};
  \node [dots,label=-90:$fail_{\alpha_1}$] (i1) at ($(rec1.north west)!0.25!(rec1.north east)$) {};
  \node [triangle,label=-90:$rsv_{\alpha_3}$] (i1) at ($(rec1.north west)!0.35!(rec1.north east)$) {};
  \node [dots,label=-90:$ok_{\alpha_3}$] (i1) at ($(rec1.north west)!0.4!(rec1.north east)$) {};
  \node [dots,label=-90:$fail_{\alpha_3}$] (i1) at ($(rec1.north west)!0.45!(rec1.north east)$) {};
  \node [triangle,label=-90:$rsv_{\alpha_5}$] (i1) at ($(rec1.north west)!0.55!(rec1.north east)$) {};
  \node [dots,label=-90:$ok_{\alpha_5}$] (i1) at ($(rec1.north west)!0.6!(rec1.north east)$) {};
  \node [dots,label=-90:$fail_{\alpha_5}$] (i1) at ($(rec1.north west)!0.65!(rec1.north east)$) {};
  \node [triangle,label=-90:$rsv_{\alpha_7}$] (i1) at ($(rec1.north west)!0.75!(rec1.north east)$) {};
  \node [dots,label=-90:$ok_{\alpha_7}$] (i1) at ($(rec1.north west)!0.8!(rec1.north east)$) {};
  \node [dots,label=-90:$fail_{\alpha_7}$] (i1) at ($(rec1.north west)!0.85!(rec1.north east)$) {};
  \end{tikzpicture}
 \caption{Internal Representation of Scheduler $Sch_1$ from Figure~\ref{fig:sr}}
 \label{fig:schSR}
\end{figure}  



\begin{example}
  Figure~\ref{fig:schSR} depicts the internal representation of scheduler $Sch_1$ from Figure
  ~\ref{fig:sr}. The scheduler $Sch_1$ is responsible of the interaction class $\gamma_1=\{
    \alpha_1,\alpha_3,\alpha_5,\alpha_7\}$, that is, he is responsible of notifying components
  $C$, $T_1$ and $R$ whose indexes are respectively $1$, $2$ and $4$. Notice that since 
  every interaction of $\gamma_1$ is potentially conflicting with an interaction of $\gamma_2$
  handled by scheduler $Sch_2$, scheduling interactions of $\gamma_1$ requires the intervention
  of the conflict resolution layer.
\end{example}
Notice that Definition~\ref{def:sch_sr} presents only the syntax of a scheduler. It uses 
the Petri net formalism only for compactness purposes and is not to be confused with 
any other Petri Net formalism such as Time Petri Nets or Timed Petri nets.
\begin{property}[Scheduler Semantics]\label{pr:sch_sr_sem}
  Let $Sch=(\Loc, \mc{P},\T,\X, \D,\{g_{\tau}\}_{\tau\in\mc{T}},\{r_{\tau}\}_{\tau\in\mc{T}},
    \{f_{\tau}\}_{\tau\in\mc{E}},$\\$\{\mc{I}_{\loc}\}_{\loc\in\Loc})$ be a tuple defining 
  a scheduler. Let $(2^{\Loc},\mc{P},\to_{\mc{P}})$ be the finite labeled transition
  system of its underlying 1-Safe Petri net. The semantics of the scheduler $Sch$ 
  is equivalent to the semantics of the timed component 
  $(2^{\Loc},\loc_0,\mc{X},\mc{D},\mc{P},\mc{E},\{f_{e}\}_{e\in\mc{T}},\I)$ such that:
  \begin{itemize}
    \item $\loc_0=\otimes_{\loc\in\Loc |\loc\in m_0}\loc$, where $m_0$ is the initial marking 
      of the Petri net $(\Loc, \mc{P},\mc{T})$.
    \item For each $(m_1,p,m_2)\in\to_{\mc{P}}$ we include a transition $e\in\mc{E}=
      (\loc_1,p,g_p,r,\loc_2)$ such that:
      \begin{itemize}
        \item $\loc_1=\otimes_{\loc\in \Loc|\loc\in m_1}\loc$ and
              $\loc_2=\otimes_{\loc\in \Loc|\loc\in m_2}\loc$. The invariant of $\loc_1$ and 
              $\loc_2$ are respectively 
              $\Inv{\loc_1}=\bigwedge_{\loc\in m_1|m_1(\loc)=1}I_{\loc}$ and
              $\Inv{\loc_1}=\bigwedge_{\loc\in m_2|m_2(\loc)=1}I_{\loc}$.
            \item $g_p=g_{\tau}$, $f_e=f_{\tau}$, and $r=r_{\tau}$ where 
              $\tau=({^\bullet\tau},p,{\tau^\bullet })\in\mc{T}$ such that
            ${^\bullet\tau}\subseteq m$ and $m'=(m\textbackslash{^\bullet\tau})
  \cup{\tau^\bullet }$.
      \end{itemize}
  \end{itemize}
\end{property}



\subsection{Conflict Reservation Protocol}
In this subsection, we present the third layer of our Send/Receive architecture, namely
the conflict resolution layer. The purpose of this layer is to resolve the conflict
that occur between interactions of separate schedulers at run time. 
Since interactions may compete on resources (here sharing components), the conflict resolution
layer implements a protocol, inspired from~\cite{msg_cnt} and based on messages counting 
technique,
that allows to check the freshness of requests received for the execution of an interaction
from schedulers. It ensures that two externally conflicting interactions
cannot execute with the same offers by checking that the participation numbers of the involved 
components have not been yet consumed. Particularly, the protocol keeps the last participation
number of each component and compares it with the participation number from the reservation 
request of a scheduler and thereafter, decides whether to grant a scheduler or not the execution
of an interaction.

There exists several implementations of the conflict resolution protocol~\cite{crp}. 
We present here only one variant since our interest is not in studying the 
conflict resolution layer. It is centralized variant based on Bagrodia's protocol.

\begin{definition}[Conflict Resolution Protocol]
  Let $\gamma(B_1,\cdots,B_n)$ be a timed system and $\{\gamma_j\}^m_{j=1}$ be an interaction
  partition. The corresponding centralized conflict resolution protocol component is defined
  by the timed component $CP=(\Loc^{CP},\loc_0^{CP},\emptyset,\mc{D}^{CP},\mc{P}^{CP},\mc{E}^{CP},
  \{f_e\}_{e\in\E^{CP}},\emptyset)$ such that:
  \begin{itemize}
    \item $\Loc^{CP}$ contains for each externally conflicting interaction $\alpha$ a waiting
      location $w_{\alpha}$ and a receive location $r_{\alpha}$. Receive locations are 
      urgent locations.
    \item $\mc{D}^{CP}$ includes for each component $B_i$ participating in conflicting 
      interactions $\alpha$ its current participation number $n_i^{\alpha}$ as well as the last 
      participation number $N_i$.
    \item $\mc{P}^{CP}$ includes for each externally conflicting interaction a reservation port
      $rsv_{\alpha}$, $ok_{\alpha}$ and $fail_{\alpha}$. The port $rsv_{\alpha}$ imports
      the variables $\{n_i^{\alpha}|B_i\in\p{\alpha}\}$.
    \item $\mc{E}^{CP}$ includes for each externally conflicting interaction $\alpha$ the 
      following transitions:
      \begin{itemize}
        \item A reservation transition 
          $e_{rsv_{\alpha}}=(w_{\alpha},rsv_{\alpha},\true,\emptyset,r_{\alpha})$ 
        \item A transition granting the execution of $\alpha$, 
          $e_{ok_{\alpha}}=(r_{\alpha},ok_{\alpha},g_{ok_{\alpha}},\emptyset,w_{\alpha})$ 
          such that, $g_{ok_{\alpha}}=\bigwedge_{B_i\in\p{\alpha}} n_i^{\alpha}>N_i$ and
          $f_{e_{ok_{\alpha}}}=\{\forall B_i\in\p{\alpha}, N_i:=n_i^{\alpha}\}$.
        \item A transition denying the execution 
          $e_{fail_{\alpha}}=(r_{\alpha},fail_{\alpha},\true,\emptyset,w_{\alpha})$
      \end{itemize}
  \end{itemize}
\end{definition}
\begin{figure}[!h]
 \centering
  \captionsetup{justification=centering}
  \begin{tikzpicture}[scale=0.7,every node/.style={scale=0.7}]
  
  \node[accepting,place] (w1){$w_{\alpha_1}$}; 
  \node[place,below=2.5cm of w1] (r1){$r_{\alpha_1}$}; 
  \draw [-] ($(r1) + (180:4mm)$) arc (180:360:4mm);
  \path (w1) edge node[left]{$rsv_{\alpha_1}$}(r1)
        (r1) edge [out=180,in=180] node[left,align=center]{$n_1^{\alpha_1}>N_1$\\
                                                      $n_2^{\alpha_1}>N_2$\\
                                                      $ok_{\alpha_1}$\\
                                                      $N_1:=n_1^{\alpha_1}$\\
                                                      $N_2:=n_2^{\alpha_1}$} (w1)
        (r1) edge [out=0,in=0] node[right,align=center]{$n_1^{\alpha_1}\le N_1$\\
                                                      $n_2^{\alpha_1}\le N_2$\\
                                                      $fail_{\alpha_1}$} (w1);
  

  \node[accepting,place,right=5cm of w1] (w3){$w_{\alpha_3}$}; 
  \node[place,below=2.5cm of w3] (r3){$r_{\alpha_3}$}; 
  \draw [-] ($(r3) + (180:4mm)$) arc (180:360:4mm);
  \path (w3) edge node[left]{$rsv_{\alpha_3}$}(r3)
        (r3) edge [out=180,in=180] node[left,align=center]{$n_1^{\alpha_3}>N_1$\\
                                                      $n_2^{\alpha_3}>N_2$\\
                                                      $ok_{\alpha_3}$\\
                                                      $N_1:=n_1^{\alpha_3}$\\
                                                      $N_2:=n_2^{\alpha_3}$} (w3)
        (r3) edge [out=0,in=0] node[right,align=center]{$n_1^{\alpha_3}\le N_1$\\
                                                      $n_2^{\alpha_3}\le N_2$\\
                                                      $fail_{\alpha_3}$} (w3);
  \node[accepting,place,below=5cm of w3] (w7){$w_{\alpha_7}$}; 
  \node[place,below=2.5cm of w7] (r7){$r_{\alpha_7}$}; 
  \draw [-] ($(r7) + (180:4mm)$) arc (180:360:4mm);
  \path (w7) edge node[left]{$rsv_{\alpha_7}$}(r7)
        (r7) edge [out=180,in=180] node[left,align=center]{$n_1^{\alpha_7}>N_1$\\
                                                      $n_2^{\alpha_7}>N_2$\\
                                                      $ok_{\alpha_7}$\\
                                                      $N_1:=n_1^{\alpha_7}$\\
                                                      $N_2:=n_2^{\alpha_7}$} (w7)
        (r7) edge [out=0,in=0] node[right,align=center]{$n_1^{\alpha_7}\le N_1$\\
                                                      $n_2^{\alpha_7}\le N_2$\\
                                                      $fail_{\alpha_7}$} (w7);
  \node[accepting,place,left=5cm of w7] (w5){$w_{\alpha_5}$}; 
  \node[place,below=2.5cm of w5] (r5){$r_{\alpha_5}$}; 
  \draw [-] ($(r5) + (180:4mm)$) arc (180:360:4mm);
  \path (w5) edge node[left]{$rsv_{\alpha_5}$}(r5)
        (r5) edge [out=180,in=180] node[left,align=center]{$n_1^{\alpha_5}>N_1$\\
                                                      $n_2^{\alpha_5}>N_2$\\
                                                      $ok_{\alpha_5}$\\
                                                      $N_1:=n_1^{\alpha_5}$\\
                                                      $N_2:=n_2^{\alpha_5}$} (w5)
        (r5) edge [out=0,in=0] node[right,align=center]{$n_1^{\alpha_5}\le N_1$\\
                                                      $n_2^{\alpha_5}\le N_2$\\
                                                      $fail_{\alpha_5}$} (w5);
  \node [rounded corners,inner xsep=50mm,inner ysep=30mm,draw,fit=(w1)(r7)] (rec1) {};
  \node [dots,label=90:$rsv_{\alpha_5}$] (i1) at ($(rec1.south west)!0.175!(rec1.south east)$) {};
  \node [triangle,rotate=180,label=-90:$ok_{\alpha_5}$] (i1) at ($(rec1.south west)!0.25!(rec1.south east)$) {};
  \node [triangle,rotate=180,label=-90:$fail_{\alpha_5}$] (i1) at ($(rec1.south west)!0.325!(rec1.south east)$) {};
  \node [dots,label=90:$rsv_{\alpha_7}$] (i1) at ($(rec1.south west)!0.675!(rec1.south east)$) {};
  \node [triangle,rotate=180,label=-90:$ok_{\alpha_7}$] (i1) at ($(rec1.south west)!0.75!(rec1.south east)$) {};
  \node [triangle,rotate=180,label=-90:$fail_{\alpha_7}$] (i1) at ($(rec1.south west)!0.825!(rec1.south east)$) {};
  
  \node [dots,label=-90:$rsv_{\alpha_1}$] (i1) at ($(rec1.north west)!0.175!(rec1.north east)$) {};
  \node [triangle,label=-90:$ok_{\alpha_1}$] (i1) at ($(rec1.north west)!0.25!(rec1.north east)$) {};
  \node [triangle,label=-90:$fail_{\alpha_1}$] (i1) at ($(rec1.north west)!0.325!(rec1.north east)$) {};
  \node [dots,label=-90:$rsv_{\alpha_3}$] (i1) at ($(rec1.north west)!0.675!(rec1.north east)$) {};
  \node [triangle,label=-90:$ok_{\alpha_3}$] (i1) at ($(rec1.north west)!0.75!(rec1.north east)$) {};
  \node [triangle,label=-90:$fail_{\alpha_3}$] (i1) at ($(rec1.north west)!0.825!(rec1.north east)$) {};

 \end{tikzpicture}
 \caption{Sub-Part of the Timed Component for the Centralized CRP of Figure~\ref{fig:sr} 
 Handeling Interactions of $\gamma_1$}
 \label{fig:crp}
\end{figure}  


In what follows, we present the Send/Receive interactions that link the three layer of the 
presented Send/Receive model.

\begin{definition}[Send/Receive Interactions]
  Let $\gamma(B_1,\cdots,B_n)$ be a timed system and $\{\gamma_j\}^m_{j=1}$ be an interaction
  partition. The Send/Receive interactions $\gamma^{SR}$ connecting the three layers of the
  Send/Receive models are:
  \begin{itemize}
    \item For each component $B_i^{SR}$, we include an offer interaction involving $B_i^{SR}$
      and its respective schedulers $\{B_i^{SR}.o,Sch_{j_1}.o_i,\cdots,Sch_{j_k}.o_i\}$.
    \item For each port $p$ of a component $B_i^{SR}$ and for each scheduler $Sch_j$ handling
      an interaction involving $p$, we include a notification interaction 
      $\{B_i^{SR}.p,Sch_{j}.p\}$.
    \item For each internally conflicting or not interaction $\alpha\in\gamma$ handled by 
      a scheduler $Sch_j$, we include the unary interaction $\{Sch_j.a\}$.
    \item For each externally conflicting interaction $\alpha\in\gamma$, we include the following
      interactions:
      \begin{itemize}
        \item $\{Sch_j.rsv_{\alpha},CP.rsv_{\alpha}\}$
        \item $\{Sch_j.ok_{\alpha},CP.ok_{\alpha}\}$
        \item $\{Sch_j.fail_{\alpha},CP.fail_{\alpha}\}$
      \end{itemize}
  \end{itemize}
\end{definition}

The correctness of the Send/Receive transformation is proved using observational equivalence, 
that is, weak bisimulation. 
\begin{theorem}[Correctness~\cite{ahlem_these}]
  $T\dot{\sim}\mc{T}^{SR}$.
\end{theorem}
The correctness of the presented approach is necessary to attest that both the initial
and resulting systems have the same behavior. Nonetheless, the proof of correctness has
already been established and its details are not relevant to the content of this thesis.
The interested reader can find all the steps of the proof in Chapter 5 of~\cite{ahlem_these}.
\section{Modeling Distributed Real-Time Constraints}

Distributed real-time systems are prone to different kind of problems. The immediate concern 
is the communication delays inherent to distributed platforms. The latter increase considerably
the effort of coordinating the parallel activities of running components. Thus, scheduling
such systems must cope with the induced delays by proposing execution strategies ensuring 
global consistency while satisfying the imposed timing constraints.
Another phenomenon intrinsic to distributed platforms is clock drift. A clock is a device
that consists of a counter that is incremented periodically according to the frequency of 
an oscillator. This implies that clocks are not perfect since the oscillator frequencies may
vary during their lifetime due to several factors such as aging, temperature, humidity, etc.
Consequently, clocks trend to \emph{drift} or gradually desynchronize from a given reference 
time. Moreover, when having multiple clocks running in the same system, which is usually the
case in distributed real-time systems, the relative clock drift between these clocks
may result in an unexpected (even undesirable) behavior. The common practice is to 
resynchronize the clocks~\cite{ptp} (internally or externally) in order to bring the difference 
to a certain threshold to minimize the impact of this phenomenon. 

\subsection{Communication Delays}

The Send/Receive model as presented in Section~\ref{sec:3.2} assumes implicitly that 
communication between the different layers is timeless. This restricts the applicability
of such approach to applications where communication delays imposed by the target platform 
can be neglected with respect to timing constraints of components.
To cope with those delays, a variant of the Send/Receive approach was presented 
in~\cite{ahlem_these}. 
It is based on an early decision making mechanism where schedulers plan interactions executions 
ahead and notify components of their execution dates in advance. 
The main issue of this method is that when planning components to execute at a given time,
all the interaction including the planned components will be ineligible for execution (and even
for planning). This may block additional components besides the ones planned for execution, 
which may introduce deadlocks and/or violations of timing constraints. To overcome this problem,
this adaptation of the Send/Receive models suggests that each scheduler, additionally to the
components it is handling, observe a subset of components that may be blocked when 
scheduling interactions. However, because of the nature of the location invariants (local
constraints that propagate on the global level), this technique results in observing
all the components of the system, instead of only a subset as presented in~\cite{ahlem_these}.

In order to model the behavior of a system under some communication delays, 
we introduce the \emph{local planning 
semantics}. This semantics aims to distinguish between the decision dates for executing 
interactions and their actual execution dates by adding a notion of \emph{planning} on 
the semantics level. The delay between the planning of an interaction and its execution has to
be greater than the parameter $\hmin$ which represents an upper bound on the communication delays
(i.e worst case delay). Although this approach 
is based on the same idea of anticipating the executions of interactions, it differs from 
the approach of~\cite{ahlem_these} in the following points:
\begin{itemize}
  \item The class of system handled by the method of~\cite{ahlem_these} is restricted to 
    timed components with closed guards, that is, where clock constraints are of the form:
    \begin{displaymath}
    c:=\true \ | \ x\le k \ |\ x\ge k \
    \end{displaymath}
    where $x$ is a clock and $k\in\integerpoz$.
    Moreover, this method is restricted to timed components with non-decreasing deadlines. In
    other words, if time can progress by $d$ from a given state $(\loc,\val)$ of a component, 
    it can also progress by $d$ from any state $(\loc',\val')$ reached by executing an action
    $a$ from state $(\loc,\val)$. Our approach on the other hand, imposes only that 
    the system is free of modelling errors such as deadlocks or timelocks. 
  \item Our approach works on the semantics level whereas the method of~\cite{ahlem_these} 
    is based on 
    transformations and model constructions. The main advantage of working on the semantics
    level is that it allows to stay at certain level of abstraction, close enough to the 
    original model, which reduce considerably the chance of errors during the formalization.
    Furthermore, one can still imagine a Send/Receive like transformation that implements
    this semantics.
  \item Unlike the standard semantics of timed system, 
    the local planning semantics is based on a local view of the system which is more suited
    for the distributed context.  
\end{itemize}
Chapter~\ref{chap:5} presents a detailed description of the local planning semantics, its 
properties and relations with the standard semantics of timed systems. It also provides
sufficient conditions that guarantee the correctness (in terms of behaviour) of a given
application under some bounded communication delays.

\subsection{Clock Drift}

Reachability analysis has been used to test the behavior
of timed automata models against some desired properties. However, such analysis
techniques whether region-based or zone-based can be incorrect and misleading,
since they rely on several assumptions such as zero response times or infinitesimally  
precise clocks, which is generally not the case in reality. In practice, clocks are 
implemented using an oscillator and a counting register, and their precision based on 
the quality of the oscillator together with the operating environment.
The common practice when studying the effect of clock imperfections is to define a perturbation
model that approximates the behavior of a given model under clock drifts in order to study its 
robustness. 

Robust reachability has been introduced to check whether a given
system still satisfies the specification when
subject to different perturbations such as clocks drift.
In~\cite{drift:puri}, Puri introduced a model of clock drift for closed timed automata
by introducing a parameter $\epsilon>0$ that bounds the clocks drift rates.
This work showed that the standard reachability analysis approach is not correct
when clocks drift, even by infinitesimally small amount, and subsequently suggested a region 
based method for calculating $Reach^*(S)$, the set of reachable states for \emph{every} drift 
(the limit as $\epsilon\to 0$),
that is, $Reach^*(S)=\cap_{\epsilon>0} Reach(S_{\epsilon})$. Other 
works~\cite{drift:conrad,drift:puriR} proposed a zone based algorithm for computing this 
reach-set more efficiently, and generalized the approach for open timed automata 
model~\cite{drift:puriR}.

In~\cite{drift:wulf,drift:puri}, another perturbation model was considered. Here, 
the system model is syntactically modified by~\emph{relaxing} the guards through 
a parametric enlargement of $\delta$.
Dewulf~\cite{drift:wulf} showed that the notion of robustness defined in~\cite{drift:puri}, and
studied in other works~\cite{drift:conrad, drift:puriR}, is closely related to the notion of 
implementability introduced in~\cite{drift:wulf}, that is, whether for some $\delta>0$,
the enlarged system model still satisfies the requirements expressed by the considered 
properties.
This allowed to prove that the considered notion of implementability is decidable 
for timed automata.

Finally,~\cite{drift:surp} consider a more realistic model of drifting clocks by considering 
clock resynchronization, available now in most distributed real-time systems. It was proven
that standard zone-based reachability analysis  is exact when testing robust safety, provided
a uniform strictly positive robustness margin of 1.

In Chapter~\ref{chap:6}, we present a timed automata based model for distributed real-time
systems where the relative drift between clocks is assumed to be bounded (clocks are assumed
to be resynchronized with a certain threshold). The resulting timed transition system 
includes straightforwardly more states than the initial model. We then give interesting 
properties of the drifted model and provide a strategy that allows, for any resulting 
execution trace, to stay close enough to a \say{similar} trace of the initial model. 
