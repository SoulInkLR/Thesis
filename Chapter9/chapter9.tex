\chapter{Conclusion and Perspectives}
\label{chap:9}
In this chapter, we conclude this thesis by summarizing the achievements and pointing out
interesting future working directions.  

\section{Achievements}
\subsection*{Knowledge Based Optimization Approach}
In this thesis, we presented an intermediate distributed representation (Send/Receive model) for 
distributed ream-time systems. This representation proposes an implementation method for
systems with multiparty interactions through simpler primitives such as
messages passing. It also provides a scheduling mechanism, more appropriate to distributed
real-time systems, since it is based on partial information of a given system.

Our first contribution was to propose an optimization methods for such intermediate 
representation. We presented a methods that relies on invariants in order to refine the
set of conflicting interactions, required for building the Send/Receive model.
Based on the original model, an approximation of the set of reachable states of a timed system
is derived on the form of invariants. The latter are used later on for checking the 
simultaneous enabledness of two potentially conflicting interactions. 
Based on the obtained results, either the potential conflict between two interactions is cleared 
out meaning that both interactions are never enabled at the same time, or the real conflict 
cannot be attested which in this case a counter-example is proposed.
This optimization aims particularly at reducing the effort of scheduling interactions
by removing any unnecessary exchange of messages or evaluation computations. 

\subsection*{Modeling Communication Delays}
The Send/Receive model presented in Chapter~\ref{chap:3} assumes that the communication 
mechanisms are fast enough to not impact the behavior of the overall system. Effectively, this
is due to the zero delay between the decision of executing interactions in schedulers and
the concrete execution of the corresponding actions in the participating components.
The resulting model was proven to be observationally equivalent to the initial model.
To reduce the impact of communication delays on the system behavior,~\cite{ahlem_these} proposed
an approach based on the idea of \emph{early decision making}. In this solution, schedulers
plan ahead the execution of interactions and notify the corresponding components in advance.
Moreover, it was suggested that schedulers are required to observe an additional subset of
components called \emph{observed components}, not participating in the planned interactions, 
in order to achieve global deadlines.
However, besides the fact that this method is restricted to systems where components have non 
decreasing deadlines, the characterization of the set of observed components is incomplete. In 
fact this set is greater than the presented characterization, and in many cases is equal to 
all the component of the system. This is mainly due to the nature of the location invariants:
Local constraints that propagate on the global level.

Our second contribution was to propose a method, based on the same idea of early decision making,
but that differs from the approach of~\cite{ahlem_these} in the following points:
\begin{itemize}
  \item No restriction on components constraints was made
  \item No restriction on the form of timing constraints (In ~\cite{ahlem_these} timing 
    constraints are restricted to constraints of the form $x\le k$ or $x\ge k$)
  \item Our method works on the semantics level conversely to the aforementioned method that
    relies directly on transformations and model construction. Effectively, we introduced 
    the local planning semantics that plan the execution of interactions based only on 
    the set of its participating components. The planning operations are constrained by 
    the worst estimation of communication delays as well as the planned interactions. This
    semantics is suited for distributed real-time systems since it is based on local (partial)
    information. We also provided a proof for weak simulation, and a strategy based on 
    sufficient conditions guaranteeing the preservation of the system behavior from deadlocks. 
    Finally, we presented an alternative method based on real-time controller synthesis 
    paradigm and discussed its convenience with respect to our method. 
\end{itemize}

\subsection*{Robustness to Clock Imperfections}
We studied in Chapter~\ref{chap:6} the effect of clock imperfections (clock drift) on the 
behavior of a timed system. We considered a perturbation model where clock rates are not 
perfect (not equal and may change during the execution) but under the assumption that clocks 
are resynchronized which induces that their difference of clock values will stay within a certain
threshold. 
Our robustness analysis approach was based on the characterization of potential bad states
that invalidates the $\epsilon$-simulation between the initial model and the perturbated model.
We proposed a solution for verifying the robustness of a given timed system based on 
shrinkability analysis and suggested the \texttt{Shrinktech} tool for achieving such analysis
which is basically based on parameter synthesis.


\section{Future Works}
\subsection*{Partitioning and Mapping}

In this thesis, we presented the partitioning of interactions as being an important parameter
of the Send/Receive transformation. Interaction partitioning is of importance since the 
set of conflicting interactions is calculated based on the latter. The immediate question is
whether a given partitioning is better than an other. In other terms, how can we evaluate 
if a given partition of interaction is good or bad. Instinctively, the choice of partitioning
interactions is based on two main factors, namely the degree of parallelism and the effort
for solving conflicts (in terms of messages and computational evaluation).
This choice is generally guided by the choice of a specific target platform: how expensive
is the communication? This introduce necessarily the problem of mapping an application
on a given target platform~\cite{map}. This question is of great importance since its answer
will define how each application process will be mapped to the platform interconnected 
processing nodes. Consequently, it decides on workload of each processing nodes and 
the communication volume induced by such mapping.
Thus an optimal partition would be a partition that takes into account the target platform
and proposes a partitioning that optimize all the above points. This can be done either
by computing directly the optimal solution or by tuning a proposed partition by merging or 
splitting some processes of the application.

\subsection*{Invariants for Distributed Real-Time Systems}

In Chapter~\ref{chap:5} and Chapter~\ref{chap:6}, we presented different semantics reflecting
the behavior of a real-time system in a distributed setting (under communication delays or 
with clock drift). 

In Chapter~\ref{chap:5}, we relied on the invariants of the initial model
to verify the sufficient conditions for deadlock freedom of the local planning semantics.
This usage is due to the fact that invariants of the system under the standard semantics 
are proven to be invariants of the system under the local planning semantics.
An interesting direction is to try to derive invariants characterizing specifically the reachable
states of the planning semantics. This will help directly in the verification process in term of 
results precision.

In Chapter~\ref{chap:6}, we proposed the \texttt{Shrinktech} tool for the robustness analysis.
However, such tool relies on the computation of the region graph of a timed automaton, which
in term of complexity suffers from the state explosion problem when the increasing the number of
clocks. A solution to reduce this cost is to compute a parametric invariant of the system
with drifting clocks. The latter represents a cheap over-approximations may in some cases 
allow to avoid overly eager and expensive computations of the precise parametric images of 
the set of reachable states. 

\subsection*{Modeling vs Semantics}

Model-based development is one promising approach in building real-time systems today.
This paradigm relies on models with well defined semantics when building systems all the way
form the design phase until reaching a concrete implementation.
This approach is really important when working with large scale heterogeneous systems since
it reduces both the engineering efforts and the development costs and time.
Nevertheless, a model represents only an abstraction of the reality and is often based
on idealized assumptions, which will not hold at a given stage of the model-based workflow, 
precisely when reaching the implementation point.
In this thesis, we tried to bridge this gap by introducing new semantics to tackle the different
issues inherent to the distributed context.
An interesting working direction would be to answer the following question: 
Can similar results be obtained without introducing new semantics, but using modeling
instead? 
This question is of a big interest since it can indicate which direction efforts need to be made.
For instance, the introduction of a new semantics can be in some case complicated and 
errors-prone whereas modeling is known to be more general, much easier and trivial to understand.
Also, modeling has also the benefit of being modular in the sense that if we change 
the target execution platform on will have just to replace the corresponding model.

\subsection*{Scheduling Distributed Real-Time Systems}
In this thesis, we often mentioned the words \say{schedulers or scheduling} but without 
referring to any scheduling scheme or policy. An interesting working direction is to refine 
the behavior of given system by imposing some scheduling constraints. In fact, since we presented
methods based on over-approximations and sufficient conditions, the results of our analysis may
in some cases yield counter-examples that are either false-positives or real counter-examples.
The idea is to identify scheduling schemes that could identify the needs, based on a given set
of properties, in order to achieve a correct behavior.



