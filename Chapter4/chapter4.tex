\chapter{Knowledge Based Optimization of Distributed Real-Time Systems}\label{chap:4}
\minitoc

\section{Conflicting Interaction Calculation}

As explained in Subsection~\ref{sub:conf}, two interactions $\alpha_1$ and $\alpha_2$ 
sharing a subset of components cannot execute concurrently.
In fact, if these interactions are enabled from the same state they are conflicting, meaning
that they are competing on the same resources (shared components) and only one interaction will
be granted the execution and not the other.
Particularly, given a timed system $\gamma(B_1,\cdots,B_n)$ and an interaction partition
$\{\gamma_j\}_{j=1}^m$ (Definition~\ref{def:inter_part} of Chapter~\ref{chap:3}), 
if such interactions are part of two different classes of the interaction
partition (externally conflicting interactions), then the resulting Send/Receive model requires 
the intervention of the conflict resolution layer to resolve the conflict situation. 
The computation of the conflicting interactions set is based on syntactic pre-checks
(Definition~\ref{def:pconf}), that is, it is an over-approximation that in some cases  
induces an unnecessary conflict resolution.  
The calculation of the conflicting interactions set highly impacts the structure and the 
performance of the underlying Send/Receive model. For every interaction that is in fact
not conflicting, the corresponding scheduler includes an additional place and three transitions
involved in the Send/Receive interactions involving the conflict resolution layer.
In this case, executing an interaction adds not only an evaluation overhead, but also latency 
resulting from the communication delays between the two layers.
In order to refine the conflicting interactions set, we rely on the following definition
of conflicts.
\begin{definition}[Conflicting Interactions]
\label{def:conf}
Let $S = \gamma(B_1,\dotsc,B_n)$ be a timed system. Two interactions 
$\alpha_1$ and $\alpha_2$ of $\gamma$ are \emph{conflicting}, and we write $\alpha_1\#\alpha_2$, 
  if $\p{\alpha_1} \cap\p{\alpha_2} \neq \emptyset$ and there exists a reachable state from 
which both $\alpha_1$ and $\alpha_2$ are enabled, i.e., a state satisfying:
\begin{equation}\label{eq:conf1}
  Conflict(S,\alpha,\beta)=Reach(S) \wedge \enabled{\alpha_1} \wedge \enabled{\alpha_2}
\end{equation}
  where $Reach(S)$ is the set of reachable states of the S.
\end{definition}

The above definition of conflicts characterizes the exact set of conflicting interactions.
Especially, it considers that two interactions are not conflicting if the whole system
cannot reach a state from which both can potentially execute.
Clearly, such interactions do not require conflict resolution as they cannot be scheduled
based on common offers.
In what follows, we propose an approach that aims to reduce the set of potential conflicts.
In fact instead of calculating the exact set of reachable states of a given system,
we use static analysis techniques to extract a \emph{knowledge} that represents an
over approximation of the reachable states of the system on the form of invariants.

\section{Knowledge Based Reduction of Potentially Conflicting Interaction}

Knowledge as referred to it here can be interpreted as any information that gives 
a characterization of a given system. We distinguish two types of knowledge:
local knowledge that captures partial information of a system at components level, and
a global knowledge that builds upon local knowledge and takes into account components 
synchronizations.
Our approach includes two main steps: the first step \emph{(i)} consists of constructing the set
of potentially conflicting interactions based on Definition~\ref{def:pconf}. 
This step aims mainly to distinguish non conflicting interactions from those that can
potentially conflict in order to avoid unnecessary checks during the second step.
Then, the second step \emph{(ii)} computes then combines local and global knowledge 
of the system on the form of invariants. The latter will represents an over-approximation 
of the reachable state of the system. 
After that, by replacing $Reach(S)$ by its over-approximation $\reacha{S}$ in
Expression~\ref{eq:conf1}, potentially conflicting interactions are reduced by
checking the following precondition:

\begin{equation}\label{eq:conf2}
  \stacktxt{Conflict(S,\alpha_1,\alpha_2)}=\ \reacha{S} \wedge \enabled{\alpha_1} \wedge 
  \enabled{\alpha_2}
\end{equation}

Notice that since $Reach(S)\Rightarrow\reacha{S}$, we obtain that 
$Conflict(S,\alpha_1,\alpha_2) \Rightarrow$\\$\stacktxt{Conflict(S,\alpha_1,\alpha_2)}$.
Thus, if two interactions are established to be conflicting  
according to~\ref{eq:conf2}, then they are conflicting according to~\ref{eq:conf1}.
Hereinafter, \emph{false} conflicts refers to potential conflicts as defined in
Definition~\ref{def:pconf} but that are not conflicts with respect to Definition~\ref{def:conf}.

A potential conflict between two interactions is a \emph{false} conflict either: \emph{(i)} 
because the system cannot reach a global location configuration enabling both interactions, 
or \emph{(ii)} because both interactions are not enabled at the same time due to timing 
constraints.
In the following, we show how to compute invariants for removing false conflicts of
types \emph{(i)} and \emph{(ii)}. These invariants combined with
individual reachable states of components will represent our over-approximation of $Reach(S)$.
\subsection{Linear Invariants}

Linear invariants consist of linear constraints that allow to reason on complex properties.
For instance, they can be used to \emph{count} how many processes are at a given state
of a concurrent system. Such invariants have been widely used in different domain.
Particularly, we are interested in the so called linear state-invariants~\cite{petri,inv1,inv2} 
from the 
Petri net community. Linear state-invariants or (S-invariants) are determined to be 
appropriate for proving non-coverage of subsets of individual locations, 
which corresponds exactly to what is needed to prove that two interactions 
cannot be enabled from the same location configuration.
Precisely, linear invariants consist of linear equations of predicates $\al{\loc}$ interpreted
as integer values using the convention 1 for $\true$ and 0 for $\false$.
For instance $\al{\loc_0}+\al{\loc_1}+\al{\loc_2}=1$ is a linear constraint
for the Petri net of Figure~\ref{fig:pn} expressing the presence of on token circulating
between places $\loc_0$, $\loc_1$ and $\loc_2$.

Locations configurations reachable in a composition 
$S = \gamma(B_1,\dotsc,B_n)$ are necessary combinations 
of reachable locations of individual components $B_i$.
However, in general not all combinations are reachable in $S$ since components are not 
fully independent as they synchronize through interactions.
A typical example of that is a shared resource used in mutual exclusion by a set of 
components: any of them can potentially use it, but they should coordinate so that states in 
which two (or more) components use the resource are not reachable.
Another illustration of this can be found in example of Figure~\ref{fig:tm}: components 
$T_1$ (resp. $T_2$) may reach location $\loc_1^2$ (resp. $\loc_1^3$) by executing 
action $init_1$ (resp. $init_2$), but in the composition $T_1$ and $T_2$ cannot be 
simultaneously at locations $\loc_1^2$ and $\loc_1^3$.
This is due to interactions $\alpha_1 = \{ init_0, init_1 \}$ and 
$\alpha_2 = \{ init_0, init_2 \}$ with component $C$: executing $\alpha_1$ disables 
$\alpha_2$, and vice versa.
That is, the potential conflict between interactions $\alpha_3$ and $\alpha_4$ can be 
excluded if we consider reachable locations of the composed system.

\begin{definition}[Linear Invariant]
\label{def:inv}
Let $S = \gamma(B_1,\dotsc,B_n)$ be a timed system and $\Loc = \bigcup_{1 \leq i \leq n} \Loc_i$
being all components locations, with $\Loc_i$ the set of locations of $B_i$.
A \emph{linear invariant} of $S$ is a linear equality constraint 
which holds in all reachable global state of S. It is of the form:
$$ \sum_{\loc \in\Loc}u_{\loc}\cdot \al{\loc} = u_0, $$
where $u_{\loc}$, and $u_0$ are integers, in which predicates $\al{\loc}$,
$\loc \in \Loc$, are interpreted as $0$ for false and $1$ for true.
\end{definition}

To compute linear invariants for a timed system $S = \gamma(B_1,\dotsc,B_n)$, we consider its 
untimed version $\tilde{S}$ abstracting all data and timing aspects of $S$
(i.e. obtained from $S$ by relaxing guards and location invariants of components).
Note that linear invariants for $\tilde{S}$ are also linear invariants for $S$, since 
reachable locations of $S$ are necessary included in reachable locations of $\tilde{S}$.
Methods for calculating linear invariants are based on linear algebra, and more precisely
on the \emph{characteristic system}~\cite{inv1} (also known by the place-transition matrix in 
the Petri nets community). It consists of a system of linear equations representing 
the interactions of a given system.
\begin{definition}[Characteristic System]\label{def:chars}
  For a timed system $S=\gamma(B_1,\cdots,B_n)$ and $\Loc=\Loc_1\times\cdots\times\Loc_n$,
  the set of all global location configuration. The characteristic system is defined as 
  follows:
  $$\mc{M}(S)\equiv\bigwedge_{\alpha\in\gamma}\bigwedge_{\loc\in\Loc_{\alpha}}
\Big(\sum_{\loc_i\in\alpha^{\bullet}}x_{\loc_i}-\sum_{\loc_j\in{^\bullet}\alpha}x_{\loc_j}
  \Big)=0$$
  where $\Loc_{\alpha}$ is the subset of locations configurations from which $\alpha$ is 
  possible, and $\alpha^{\bullet}$, ${^\bullet}\alpha$ denotes respectively the destinations and 
  sources locations of components actions involved in $\alpha$.
\end{definition}

\begin{example}
  The characteristic system for Example~\ref{exp:run} following the enumeration of all 
  interactions of $\gamma$ is:

  \[\mc{M}(S)=\begin{cases}
    x_{\loc_1^1}-x_{\loc_0^1}+x_{\loc_1^2}-x_{\loc_0^2}=0\\ 
    x_{\loc_1^1}-x_{\loc_0^1}+x_{\loc_1^3}-x_{\loc_0^3}=0\\ 
    x_{\loc_0^1}-x_{\loc_1^1}+x_{\loc_2^2}-x_{\loc_1^2}=0\\ 
    x_{\loc_0^1}-x_{\loc_1^1}+x_{\loc_2^3}-x_{\loc_1^3}=0\\ 
    x_{\loc_1^4}-x_{\loc_0^4}+x_{\loc_3^2}-x_{\loc_2^2}=0\\ 
    x_{\loc_1^4}-x_{\loc_0^4}+x_{\loc_3^3}-x_{\loc_2^3}=0\\ 
    x_{\loc_0^4}-x_{\loc_1^4}+x_{\loc_0^2}-x_{\loc_3^2}=0\\ 
    x_{\loc_0^4}-x_{\loc_1^4}+x_{\loc_0^3}-x_{\loc_3^3}=0\\ 
  
  
  \end{cases}\]

\end{example}

The common techniques for solving homogeneous systems $A.x=0$ are the Guass-Jordan elimination,
Cholesky-, QR- or LU-factorization. These algorithms have low polynomial complexity and
can be directly applied to solve the characteristic system $\mc{M}(S)$.
In order to obtain the linear invariants of a given system, we use the algorithm 
proposed in~\cite{inv-lin}. It is a variant of Gauss-Jordan elimination that exploits 
the locality of unknowns as well as the particular form of the characteristic system equations. 
Equations are processed iteratively, one by one, while producing an equivalent left-bound system.

Let $LI(S)$ be the linear invariants characterizing a given system $S$.
A potential conflict between interactions $\alpha_1$ and $\alpha_2$ is a false conflict if the 
following formula is not satisfiable:

\begin{equation}\label{eq:linconf}
  \bigwedge_{1\le i\le n}Reach(B_i)\wedge LI(S)  \wedge \enabled{\alpha_1}\wedge\enabled{\alpha_2}
\end{equation}
where $Reach(B_i)$ denots the reachable states of component $B_i$.
\begin{example}
Let us reconsider the example of Figure~\ref{fig:tm}.
Among the resulting linear invariants, we focus on the following:
\begin{numcases}{}
1\cdot\al{\loc_1^2} + 1\cdot\al{\loc_1^3} - 1\cdot\al{\loc_1^1} = 0 \label{eq:inv1} \\
1\cdot\al{\loc_3^2} + 1\cdot\al{\loc_3^3} - 1\cdot\al{\loc_1^4} = 0 \label{eq:inv2}.
\end{numcases}
We deduce from the invariant~(\ref{eq:inv1}) that $\al{\loc_1^2}$ and $\al{\loc_1^3}$
cannot be true simultaneously, that is, components $T_1$ and $T_2$ cannot be simultaneously 
at the corresponding locations. Consequently, we can directly infer that interactions 
$\alpha_3$ and $\alpha_4$ are not conflicting, even though they are potentially conflicting.
Likewise, with~(\ref{eq:inv2}) we exclude the conflict between $\alpha_7$ and $\alpha_8$.
\end{example}

\subsection{History Clocks Inequalities}
As they completely abstract time, linear invariants presented above are only partially 
capturing system dynamics.
For example, a global location may not be reachable because components locations have 
disjoint clock constraints, or an interaction may not be enabled from a state because 
of its empty timing constraint. Such properties require extra relationships relating clocks of 
different components that are not available in $Reach(B_i)$ as it is is restricted to 
clocks of a single component: a zone of one of its symbolic states is a conjunction of its 
related clocks constraints, as explained in Section~\ref{sec:2.4}.

We follow the approach of~\cite{hs} for reinforcing our approach with global invariants on 
clocks.
They are induced by simultaneity of transitions when executing an 
interaction and the synchrony of time progress. To compute such invariants, additional 
\emph{history} clocks are first introduced in components. History clocks are associated to 
actions of components and to interactions, and reset upon their executions.
They do not modify the behavior since they are not involved in timing constraints.
They only reveal local timing of components, relevant to the interaction layer, which 
allows to infer further properties referred to as \emph{history clocks inequalities} 
in~\cite{hs}, expressing the fact that history clock of an interaction is necessary 
equal to history clocks of its actions after its execution and until the execution of another 
interaction involving these actions.
\begin{figure}[!h]
 \centering
  \begin{tikzpicture}[every node/.style={scale=0.8},scale=0.8]

  \node [place](l0) {$\loc_1$};
  \node [place,below=1.3cm of l0] (l1) {$\loc_2$};

  \path (l0) edge node[right,align=center]{$a$}(l1);
  
  \node [place,right=5cm of l0](l0) {$\loc_1$};
  \node [place,below=1.3cm of l0] (l1) {$\loc_2$};

  \path (l0) edge node[right,align=center]{$a$\\$h_a:=0$}(l1);
\end{tikzpicture}
  \caption{Example of a History Clock for Action a}
 \label{fig:hca}
\end{figure}  



\begin{definition}[History Clocks Inequalities for Actions]
Given a timed system $S=\gamma(B_1,\cdots,B_n)$, the history clocks inequalities for actions 
  are defined as follows:
  $$\mc{HA}(S)=\bigvee_{\alpha\in\gamma}\big[\big(\bigwedge_{\substack{a_i,a_j\in\alpha\\
  a_k\in Act(\gamma\ominus\alpha)}}h_{a_i}=h_{a_j}\le h_{a_k}\big)\wedge(\mc{HA}(\gamma\ominus
  \alpha)\big]$$
\end{definition}
\noindent where $Act(\gamma\ominus\alpha)$ is the set of actions involved in the interactions
$\gamma\ominus\alpha=\{\beta\setminus\alpha \ | \ \beta\in\gamma\wedge\beta\nsubseteq\alpha\}$
and $\mc{HA}(\emptyset)=\true$.
The predicate $\mc{HA}(S)$ can be interpreted as follows. Assuming that $\alpha\in\gamma$
is the last interaction executed in the system. Then, all history clocks of its involved
actions are reset at the same time. Moreover, they are smaller than all the other history
clocks, contained in $\gamma\ominus\alpha$.
The history clocks for actions are additionally strengthened by \emph{separation constraints}
for conflicting interactions. In fact, an action involved in two conflicting interactions
is exclusively executed by one of these interactions at a given time. Particularly,
in some cases a minimum time lapse is required between two executive occurrences of
the same action. Similarly to history clocks for actions, we introduce a history clock for
each interaction that will be reset on the execution of the latter.
Separation constraints are then formalized as follows:

\begin{definition}[Separation Constraints]
Given a timed system $S=\gamma(B_1,\cdots,B_n)$, the separation constraints are defined
as follows:
  \begin{displaymath}
  HI(S)=\bigwedge_{\substack{\alpha\neq\beta\in\gamma\\a\in\alpha\cap\beta}}
  \bigwedge_{a\in\alpha} |h_{\alpha}-h_{\beta}|\ge k_a
  \end{displaymath}
    where $|h|$ denotes the absolute value of $h$ and $k_a$ is a constant computed locally
on the component containing action $a$. It represents the minimum time lapse between
two occurrences of $a$.
\end{definition}

Our method combines history clocks inequalities $\mc{H}(S)=\mc{HA}(S)\wedge\mc{HI}(S)$ 
and symbolic states of components to identify false conflicts where the following is not
satisfiable:
\begin{equation}\label{eq:histconf}
  \bigwedge_{1 \leq i \leq n} Reach(B_i) \ \wedge \mc{H}(S) \  \wedge \ \enabled{\alpha} 
  \wedge \ \enabled{\beta}
\end{equation}

\begin{example}
We illustrate the application of (\ref{eq:histconf}) for checking conflicts by 
considering again the example of Figure~\ref{fig:tm}.
It can be shown that the potential conflict between $\alpha_5$ and $\alpha_6$ 
cannot be removed using (only) linear invariants.
In the following, we prove that these interactions are actually not conflicting using 
history clocks inequalities.
Since action $start_0$ of $C$ is synchronized with either $start_1$ of $T_1$ or $start_2$ of 
$T_2$, and since history clocks $h_a$ of an action $a$ is reset whenever $a$ is executed, 
by~\cite{hs} the history clock inequalities for $start_0$ are:
\begin{equation}\label{eq:histconst}
\begin{split}
( h_{start_0} = h_{start_1} &\le h_{start_2} - 25 ) \vee 
  ( h_{start_0} = h_{start_2} \le h_{start_1} - 25 )
\end{split}
\end{equation}
Equality~(\ref{eq:histconst}) states that $h_{start_0}$ is equal to the history clock 
corresponding to the last synchronization, i.e., either $h_{start_1}$ or $h_{start_2}$, and is 
lower than history clocks of previous synchronizations. Value $25$ in~(\ref{eq:histconst}) is 
obtained considering \emph{separation constraints} computed from symbolic states of components 
and interactions: two occurrences of $start_0$ are separated by at least $25$ 
time units because of timing constraints of $C$, and so too occurrences of $start_1$ or 
$start_2$ which can only execute jointly with $start_0$.
To relate history clocks with components clocks, we simply include history clocks when 
computing symbolic states of components (i.e. $Reach$ for components), which is used to 
establish here that $x = h_{start_1}$ and $y = h_{start_2}$ when components $T_1$ and $T_2$ 
are respectively at locations $\loc_2^2$ and $\loc_2^3$.
That is, with~(\ref{eq:histconst}) we obtain $x \le y - 25$ or $y \le x - 25$.
  By definition of $Enabled$ we have $\enabled{\alpha_5} = \al{\loc_2^2} \wedge 
(10 \leq x \leq 30) $. 
  Similarly, $\enabled{\alpha_6} = \al{\loc_2^3} \wedge (10\le y \leq 30)$.
  We obtain then: \Big($\enabled{\alpha_5}\wedge\al{\loc_2^3}\Big)\Rightarrow\Big(y\le5 \
  \wedge\enabled{\alpha_6}\wedge\al{\loc_2^2}\Big)\Rightarrow x\le5$.
This proves that $\alpha_5$ and $\alpha_6$ are not conflicting. 
\end{example}

\section{Impact of Conflict Reduction on Send/Receive Models}
As explained earlier, refining the conflicting interactions set enables to minimize the
exchange of messages between the scheduling layer and the conflict resolution layer
of the corresponding Send/Receive model. More precisely, every false conflicting interaction
is scheduled using 2 exchange of messages between components participating in that interaction
and the corresponding schedulers (offers and notifications), 
instead of 4 by considering the unnecessary exchange of messages between schedulers and 
the conflict resolution protocol.
As a result, the Scheduler $Sch_1$ and the conflict resolution components 
from Figure~\ref{fig:schSR} and Figure~\ref{fig:crp} respectively become:
\begin{figure}[!h]
 \centering
  \captionsetup{justification=centering}
  \begin{tikzpicture}[scale=0.7,every node/.style={scale=0.7}]
  
  \node[accepting,place] (w1){$w_{\alpha_1}$}; 
  \node[place,below=2.5cm of w1] (r1){$r_{\alpha_1}$}; 
  \draw [-] ($(r1) + (180:4mm)$) arc (180:360:4mm);
  \path (w1) edge node[left]{$rsv_{\alpha_1}$}(r1)
        (r1) edge [out=180,in=180] node[left,align=center]{$n_1^{\alpha_1}>N_1$\\
                                                      $n_2^{\alpha_1}>N_2$\\
                                                      $ok_{\alpha_1}$\\
                                                      $N_1:=n_1^{\alpha_1}$\\
                                                      $N_2:=n_2^{\alpha_1}$} (w1)
        (r1) edge [out=0,in=0] node[right,align=center]{$n_1^{\alpha_1}\le N_1$\\
                                                      $n_2^{\alpha_1}\le N_2$\\
                                                      $fail_{\alpha_1}$} (w1);
  

  \node[accepting,place,right=5cm of w1] (w2){$w_{\alpha_2}$}; 
  \node[place,below=2.5cm of w2] (r2){$r_{\alpha_2}$}; 
  \draw [-] ($(r2) + (180:4mm)$) arc (180:360:4mm);
  \path (w2) edge node[left]{$rsv_{\alpha_2}$}(r2)
        (r2) edge [out=180,in=180] node[left,align=center]{$n_1^{\alpha_2}>N_1$\\
                                                      $n_3^{\alpha_2}>N_3$\\
                                                      $ok_{\alpha_2}$\\
                                                      $N_1:=n_1^{\alpha_2}$\\
                                                      $N_3:=n_3^{\alpha_2}$} (w2)
        (r2) edge [out=0,in=0] node[right,align=center]{$n_1^{\alpha_2}\le N_1$\\
                                                      $n_3^{\alpha_2}\le N_3$\\
                                                      $fail_{\alpha_2}$} (w2);
  \node [rounded corners,inner xsep=60mm,inner ysep=15mm,draw,fit=(w1)(r2)] (rec1) {};
  \node [dots,label=90:$rsv_{\alpha_1}$] (i1) at ($(rec1.south west)!0.175!(rec1.south east)$) {};
  \node [triangle,rotate=180,label=-90:$ok_{\alpha_1}$] (i1) at ($(rec1.south west)!0.25!(rec1.south east)$) {};
  \node [triangle,rotate=180,label=-90:$fail_{\alpha_1}$] (i1) at ($(rec1.south west)!0.325!(rec1.south east)$) {};
  \node [dots,label=90:$rsv_{\alpha_2}$] (i1) at ($(rec1.south west)!0.675!(rec1.south east)$) {};
  \node [triangle,rotate=180,label=-90:$ok_{\alpha_2}$] (i1) at ($(rec1.south west)!0.75!(rec1.south east)$) {};
  \node [triangle,rotate=180,label=-90:$fail_{\alpha_2}$] (i1) at ($(rec1.south west)!0.825!(rec1.south east)$) {};
  
 \end{tikzpicture}
 \caption{Refined Version of the Conflict Resolution Layer from Figure~\ref{fig:crp}}
 \label{fig:crp}
\end{figure}  


\begin{figure}[!h]
 \centering
  \captionsetup{justification=centering}
  \begin{tikzpicture}[scale=0.7,every node/.style={scale=0.7}]
  
  \node [markplace] (w1) {$w_1^1$};
  \node [markplace,right=4cm of w1] (w2) {$w_2^1$};
  \node [markplace,right=4cm of w2] (w3) {$w_4^1$};
  \node [transition,below=7.5mm of w1](o11) {} ;
  \node [transition,below=7.5mm of w2](o21) {};
  \node [transition,below=7.5mm of w3](o31) {};
  \node [left=1mm of o11] {$o_1$} ;
  \node [left=1mm of o21] {$o_2$} ;
  \node [left=1mm of o31] {$o_4$} ;
  \node[place,below=15mm of w1] (r1){$r_1^1$}; 
  \node[place,below=15mm of w2] (r2){$r_2^1$}; 
  \node[place,below=15mm of w3] (r3){$r_4^1$}; 
  \node [transition,below=10mm of r1,xshift=1.75cm](a1) {};
  \node [transition,below=20mm of r2,xshift=-1.75cm](a3) {};
  \node [transition,below=20mm of r2,xshift=1.75cm](a5) {};
  \node [transition,below=20mm of r3,xshift=-1.75cm](a7) {};
  \node [left=1mm of a1] {$rsv_{\alpha_1}$} ;
  \node [left=1mm of a3] {$\alpha_3$} ;
  \node [right=1mm of a5] {$\alpha_5$} ;
  \node [right=1mm of a7] {$\alpha_7$} ;
  \node[place,below=7.5mm of a1] (t1){$t_{\alpha_1}$}; 
  \node [transition,below=7.5mm of t1](a11) {};
  \node [left=1mm of a11] {$ok_{\alpha_1}$} ;
  \node[place,below=7.5mm of a11,xshift=-3.8cm] (s1){$s_{init_0}$}; 
  \node[place,right=1cm of s1] (s2){$s_{\substack{sta-\\rt_0}}$}; 
  \node[place,right=1cm of s2] (s3){$s_{init_1}$}; 
  \node[place,right=1cm of s3] (s4){$s_{\substack{sta-\\rt_1}}$}; 
  \node[place,right=1cm of s4,align=center] (s5){$s_{\substack{pro-\\cess_1}}$}; 
  \node[place,right=1cm of s5] (s6){$s_{end_1}$}; 
  \node[place,right=1cm of s6] (s7){$s_{take}$}; 
  \node[place,right=1cm of s7] (s8){$s_{free}$}; 
  \node [transition,below=7.5mm of s1](s11) {};
  \node [transition,below=7.5mm of s2](s22) {};
  \node [transition,below=7.5mm of s3](s33) {};
  \node [transition,below=7.5mm of s4](s44) {};
  \node [transition,below=7.5mm of s5](s55) {};
  \node [transition,below=7.5mm of s6](s66) {};
  \node [transition,below=7.5mm of s7](s77) {};
  \node [transition,below=7.5mm of s8](s88) {};
  \node [left=1mm of s11] {$init_0$} ;
  \node [left=1mm of s22] {$start_0$} ;
  \node [left=1mm of s33] {$init_1$} ;
  \node [left=1mm of s44] {$start_1$} ;
  \node [right=1mm of s55] {$process_1$} ;
  \node [right=1mm of s66] {$end_1$} ;
  \node [right=1mm of s77] {$take$} ;
  \node [right=1mm of s88](l) {$free$} ;
  \node [place,below=8.5cm of w1,dashed] (w11) {$w_1^1$};
  \node [place,right=4cm of w11,dashed] (w22) {$w_2^1$};
  \node [place,right=4cm of w22,dashed] (w33) {$w_4^1$};
  \node[transition, left=2cm of t1] (f1){};
  \node[left=1mm of f1]{$fail_{\alpha_1}$};
  \node[above=8mm of w1](ff1){};
  
  \path (w1) edge (o11)
        (w2) edge (o21)
        (w3) edge (o31)
        (o11) edge (r1)
        (o21) edge (r2)
        (o31) edge (r3)
        (r1) edge (a1)
        (r1) edge (a3)
        (r2) edge (a1)
        (r2) edge (a3)
        (r2) edge (a5)
        (r2) edge (a7)
        (r3) edge (a5)
        (r3) edge (a7)
        (a1) edge (t1)
        (t1) edge (a11)
        (a11) edge (s1)
        (a11) edge (s3)
        (a3) edge (s2)
        (a3) edge (s4)
        (a5) edge (s5)
        (a5) edge (s7)
        (a7) edge (s6)
        (a7) edge (s8)
        (s1) edge (s11)
        (s2) edge (s22)
        (s3) edge (s33)
        (s4) edge (s44)
        (s5) edge (s55)
        (s6) edge (s66)
        (s7) edge (s77)
        (s8) edge (s88)
        (s11) edge (w11)
        (s22) edge (w11)
        (s33) edge (w22)
        (s44) edge (w22)
        (s55) edge (w22)
        (s66) edge (w22)
        (s77) edge (w33)
        (s88) edge (w33)
        (r1) edge [loop,in=180,out=120,looseness=4] node[left]{$o_1$} (r1)
        (r2) edge [loop,in=180,out=120,looseness=4] node[left]{$o_2$} (r2)
      %  (r3) edge [loop,in=70,out=10,looseness=4] node[right]{$o_4$} (r3);
        (t1) edge [loop,in=180,out=120,looseness=4] node[left,align=center]{$o_1$\\$o_2$} (t1)
      %  (t3) edge [loop,in=180,out=120,looseness=4] node[left,align=center]{$o_1$\\$o_2$} (t3)
      %  (t5) edge [loop,in=70,out=10,looseness=4] node[right,align=center]{$o_2$\\$o_4$} (t5)
      %  (t7) edge [loop,in=70,out=10,looseness=4] node[right,align=center]{$o_2$\\$o_4$} (t7)
        (t1) edge [in=-90,out=-120] (f1)
      %  (t3) edge [in=-90,out=-60] (f3)
      %  (t5) edge [in=-90,out=-120] (f5)
      %  (t7) edge [in=-90,out=-60] (f7)
        (f1) edge [in=-150,out=90] (r1)
        (f1) edge [-,in=-180,out=90] (ff1.center)
        (ff1.center) edge [in=100,out=0] (r2);
      %  (f7) edge [in=10,out=90] (r3)
      %  (f7) edge [-,in=0,out=90] (ff7.center)
      %  (ff7.center) edge [in=80,out=180] (r2)
      %  (f3) edge [in=-100,out=90] (r2)
      %  (f5) edge [in=-80,out=90] (r2)
      %  (f3) edge [in=0,out=90,looseness=1.5] (r1)
      %  (f5) edge [in=180,out=90,looseness=1.5] (r3);
 
  
  \node [rounded corners,inner xsep=60mm,inner ysep=40mm,draw,fit=(w1)(w11)(s88)(ff1)(l),xshift=-1.5cm] (rec1) {};
  \node [dots,label=90:$o_1$] (i1) at ($(rec1.south west)!0.15!(rec1.south east)$) {};
  \node [triangle,rotate=180,label=-90:$init_0$] (i1) at ($(rec1.south west)!0.20!(rec1.south east)$) {};
  \node [triangle,rotate=180,label=-90:$start_0$] (i1) at ($(rec1.south west)!0.25!(rec1.south east)$) {};
  \node [dots,label=90:$o_2$] (i1) at ($(rec1.south west)!0.40!(rec1.south east)$) {};
  \node [triangle,rotate=180,label=-90:$init_1$] (i1) at ($(rec1.south west)!0.45!(rec1.south east)$) {};
  \node [triangle,rotate=180,label=-90:$start_1$] (i1) at ($(rec1.south west)!0.50!(rec1.south east)$) {};
  \node [triangle,rotate=180,label=-90:$process_1$] (i1) at ($(rec1.south west)!0.575!(rec1.south east)$) {};
  \node [triangle,rotate=180,label=-90:$end_1$] (i1) at ($(rec1.south west)!0.65!(rec1.south east)$) {};
  \node [dots,label=90:$o_4$] (i1) at ($(rec1.south west)!0.75!(rec1.south east)$) {};
  \node [triangle,rotate=180,label=-90:$take$] (i1) at ($(rec1.south west)!0.80!(rec1.south east)$) {};
  \node [triangle,rotate=180,label=-90:$free$] (i1) at ($(rec1.south west)!0.85!(rec1.south east)$) {};
      
  \node [triangle,label=-90:$rsv_{\alpha_1}$] (i1) at ($(rec1.north west)!0.15!(rec1.north east)$) {};
  \node [dots,label=-90:$ok_{\alpha_1}$] (i1) at ($(rec1.north west)!0.20!(rec1.north east)$) {};
  \node [dots,label=-90:$fail_{\alpha_1}$] (i1) at ($(rec1.north west)!0.25!(rec1.north east)$) {};
  \end{tikzpicture}
 \caption{Refined Version of Scheduler $Sch_1$ from Figure~\ref{fig:schSR}}
 \label{fig:schSR}
\end{figure}  






