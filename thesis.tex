% ******************************* PhD Thesis Template **************************
% Please have a look at the README.md file for info on how to use the template

\documentclass[a4paper,Times,twoside,index,12pt]{Classes/PhDThesisPSnPDF}

% ******************************************************************************
% ******************************* Class Options ********************************
% *********************** See README for more details **************************
% ******************************************************************************

% `a4paper'(The University of Cambridge PhD thesis guidelines recommends a page
% size a4 - default option) or `a5paper': A5 Paper size is also allowed as per
% the Cambridge University Engineering Deparment guidelines for PhD thesis
%
% `11pt' or `12pt'(default): Font Size 10pt is NOT recommended by the University
% guidelines
%
% `oneside' or `twoside'(default): Printing double side (twoside) or single
% side.
%
% `print': Use `print' for print version with appropriate margins and page
% layout. Leaving the options field blank will activate Online version.
%
% `index': For index at the end of the thesis
%
% `draft': For draft mode without loading any images (same as draft in book)
%
% `draftmode': Special draft mode with line numbers, images, and water mark with
% timestamp and custom text. Position of the text can also be modified.
%
% `abstract': To generate only the title page and abstract page with
% dissertation title and name, to submit to the Student Registry
%
% `chapter`: This option enables only the specified chapter and it's references
%  Useful for review and corrections.
%
% ************************* Custom Page Margins ********************************
%
% `custommargin`: Use `custommargin' in options to activate custom page margins,
% which can be defined in the preamble.tex. Custom margin will override
% print/online margin setup.
%
% *********************** Choosing the Fonts in Class Options ******************
%
% `times' : Times font with math support. (The Cambridge University guidelines
% recommend using times)
%
% `fourier': Utopia Font with Fourier Math font (Font has to be installed)
%            It's a free font.
%
% `customfont': Use `customfont' option in the document class and load the
% package in the preamble.tex
%
% default or leave empty: `Latin Modern' font will be loaded.
%
% ********************** Choosing the Bibliography style ***********************
%
% `authoryear': For author-year citation eg., Krishna (2013)
%
% `numbered': (Default Option) For numbered and sorted citation e.g., [1,5,2]
%
% `custombib': Define your own bibliography style in the `preamble.tex' file.
%              `\RequirePackage[square, sort, numbers, authoryear]{natbib}'.
%              This can be also used to load biblatex instead of natbib
%              (See Preamble)
%
% **************************** Choosing the Page Style *************************
%
% `default (leave empty)': For Page Numbers in Header (Left Even, Right Odd) and
% Chapter Name in Header (Right Even) and Section Name (Left Odd). Blank Footer.
%
% `PageStyleI': Chapter Name next & Page Number on Even Side (Left Even).
% Section Name & Page Number in Header on Odd Side (Right Odd). Footer is empty.
%
% `PageStyleII': Chapter Name on Even Side (Left Even) in Header. Section Number
% and Section Name in Header on Odd Side (Right Odd). Page numbering in footer


% ********************************** Preamble **********************************
% Preamble: Contains packages and user-defined commands and settings
\usepackage{lipsum}   % Dummytext
\usepackage{xargs}    % Use more than one optional parameter in a new commands
\usepackage[pdftex,dvipsnames]{xcolor}
\usepackage[colorinlistoftodos,prependcaption,textsize=tiny]{todonotes}

\usepackage{hyperref}
\usepackage{color}


% For french language
\usepackage[french,english]{babel}
\usepackage[utf8]{inputenc}
\usepackage[T1]{fontenc}
\usepackage{lettrine}
\usepackage{aurical}
\usepackage{csquotes}
\usepackage{dirtytalk}


% Package for the content of each chapter
\usepackage{minitoc}


% Package for adding uga cover page
\usepackage{epstopdf}
\usepackage{pdfpages}

%Maths
\usepackage{amsthm}
\usepackage{amssymb}
\usepackage{environ}
\usepackage{mathtools}
\usepackage{amsfonts}
\usepackage{stackrel}
\usepackage{stackengine}
\usepackage{scalerel}
\usepackage{ifthen}
\usepackage{cases}

%spacing
\usepackage{setspace}

%adding description to part
\usepackage{etoolbox}

%Figures

\usepackage{pgfplots}
\usepackage{tikz}
\usepackage{caption}
\usepackage{subfig}
\usetikzlibrary{patterns,intersections,plotmarks,calc,arrows,shapes,snakes,automata,backgrounds,petri,positioning,fit,arrows.meta}
\usepackage{siunitx}


%Font
%\newcommand{\ackfont}{\fontfamily{\Fontskrivan}\selectfont}
\newcommand{\ackfont}{\Fontskrivan}
%Colors
\definecolor{dkgreen}{RGB}{0, 153, 0}
\definecolor{dgreen}{rgb}{.2,.7,.2}

\makeatletter
\let\LaTeXStandardPart\part%
\newcommand{\unstarredpart@@noopt}[1]{%
    \unstarredpart@@opt[#1]{#1}%
}%
    
\newcommand{\unstarredpart@@opt}[2][]{%
  \cleardoublepage% (For clearing content before!!!)
  \begingroup%
  \let\newpage\relax%
  \LaTeXStandardPart[#1]{#2}%
  \endgroup%
}%
                
\newcommand{\starredpart}[1]{%
  \LaTeXStandardPart*{#1}%
}%

\newcommand{\unstarredpart}{%
  \@ifnextchar[{\unstarredpart@@opt}{\unstarredpart@@noopt}%
}%
\renewcommand{\part}{%
  \@ifstar{\starredpart}{\unstarredpart}%
}%of part

%cliquable links
\hypersetup{
colorlinks,
citecolor=black,
filecolor=black,
linkcolor=black,
urlcolor=black
}


%Todo notes: use \listoftodos[Notes] to display all the notes
\newif\ifreview
\reviewtrue
\ifreview
  \newcommandx{\unsure}[2][1=]{\todo[linecolor=red,backgroundcolor=red!25,bordercolor=red,#1]{#2}}
  \newcommandx{\change}[2][1=]{\todo[inline,linecolor=blue,backgroundcolor=blue!25,bordercolor=blue,#1]{#2}}
  \newcommandx{\info}[2][1=]{\todo[linecolor=OliveGreen,backgroundcolor=OliveGreen!25,bordercolor=OliveGreen,#1]{#2}}
  \newcommandx{\improvement}[2][1=]{\todo[linecolor=Plum,backgroundcolor=Plum!25,bordercolor=Plum,#1]{#2}}
\else
  \newcommandx{\unsure}[2][1=]{\todo[disable,linecolor=red,backgroundcolor=red!25,bordercolor=red,#1]{#2}}
  \newcommandx{\change}[2][1=]{\todo[disable,linecolor=blue,backgroundcolor=blue!25,bordercolor=blue,#1]{#2}}
  \newcommandx{\info}[2][1=]{\todo[disable,linecolor=OliveGreen,backgroundcolor=OliveGreen!25,bordercolor=OliveGreen,#1]{#2}}
  \newcommandx{\improvement}[2][1=]{\todo[disable,linecolor=Plum,backgroundcolor=Plum!25,bordercolor=Plum,#1]{#2}}
\fi

%Theorems and definition environment 
\theoremstyle{plain}
\newtheorem{corollary}{Corollary}
\numberwithin{corollary}{section} % important bit

\newtheorem{proposition}{Proposition}
\numberwithin{proposition}{section} % important bit

\newtheorem{theorem}{Theorem}
\numberwithin{theorem}{section} % important bit

\newtheorem{lemma}{Lemma}
\numberwithin{lemma}{section} % important bit

\newtheorem{property}{Property}
\numberwithin{property}{section} % important bit

\theoremstyle{definition}
\newtheorem{definition}{Definition}
\numberwithin{definition}{section} % important bit

\newtheorem{example}{Example}
\numberwithin{example}{section} % important bit

\theoremstyle{remark}
\newtheorem{remark}{Remark}
\numberwithin{remark}{section} % important bit



%TA semantics and predicates
\stackMath
\newcommand{\mf}[1]{\mathbf{#1}}
\newcommand{\mc}[1]{\mathcal{#1}}
\newcommand{\mb}[1]{\mathbb{#1}}
\newcommand{\q}{q} 
\newcommand{\X}{\mc{X}} 
\newcommand{\D}{\mc{D}} 
\newcommand{\E}{\mc{E}} 
\newcommand{\A}{\mc{A}}
\newcommand{\T}{\mc{T}}
\newcommand{\V}{\mc{V}}
\newcommand{\I}{\mc{I}}
\newcommand{\Q}{\mc{Q}} 
\newcommand{\tts}{\texttt{T}}
\newcommand{\TTS}{\tts=(\Q,q_0,\Sigma\cup\mb{K},\to)}
\newcommand{\TTSg}{\tts_g=(\Q_g,q_{0_g},\gamma\cup\realpos,\to_{\gamma})}
\newcommand{\TTSb}[1]{\tts_{#1}=(Q_{#1},q_{0_{#1}},\Sigma\cup\mb{K},\to_{#1})}
\newcommand{\TTSs}{\tts=(\Q,q_0,\A\cup\realpos,\to)}

\newcommand{\TTSbw}[1]{\tts_{#1}=(\Q_{#1},q_{0_{#1}},\Sigma
\cup\{\tau\}\cup\mb{K},\to_{#1})}

\newcommand{\transit}[1]{\xrightarrow{#1}}
\newcommand{\transitb}[2]{\xrightarrow{#1}_{#2}}
\newcommand{\exec}{\varrho}
\newcommand{\texec}{\mf{time(\exec)}}
\newcommand{\integer}{\mb{Z}}
\newcommand{\naturals}{\mb{N}}
\newcommand{\integerpoz}{\mb{Z}_{\ge 0}}
\newcommand{\integerpos}{\mb{Z}_{> 0}}
\newcommand{\realpoz}{\mb{R}_{\ge 0}}
\newcommand{\realpos}{\mb{R}_{> 0}}
\newcommand{\simu}[1]{\sqsubseteq_{#1}}
\newcommand{\simuw}[1]{\dot{\sqsubseteq}_{#1}}
\newcommand{\eqv}[1]{\sim_{#1}}
\newcommand{\eqvw}[1]{\dot{\sim}_{#1}}
\newcommand{\val}{\upsilon}
\newcommand{\Val}{\realpoz^{\X}}
\newcommand{\false}{\textnormal{\textit{false}}}
\newcommand{\true}{\textnormal{\textit{true}}}
\newcommand{\backward}{\raisebox{2pt}{$\swarrow$}}
\newcommand{\backwardp}[2]{\backward^{#2}_{\hspace{-2mm}#1}}
\newcommand{\forward}{\raisebox{2pt}{$\nearrow$}}
\newcommand{\enabledforward}[1]{Enabled\hspace{-0.8pt}\forward\hspace{-1mm}(#1)}
\newcommand{\enabledbackward}[1]{Enabled\hspace{-0.8pt}\backward\hspace{-1mm}(#1)}
\newcommand{\enabledbackwardb}[3]{Enabled\hspace{-0.8pt}\backwardp{#2}{#3}\hspace{-1mm}(#1)}
\newcommand{\Loc}{\mc{L}}
\newcommand{\loc}{\ell}
\newcommand{\locp}{\loc_{\perp}}
\newcommand{\locpb}[2]{\loc_{\perp_{#1}^{#2}}}
\newcommand{\Invb}[2]{\I_{#1}(#2)}
\newcommand{\Inv}[1]{\I(#1)}
\newcommand{\tc}{B=(\Loc,\loc_0,\X,\A,\E,\I)}
\newcommand{\tci}[1]{B_{#1}=(\Loc_{#1},\loc_{0_{#1}},\X_{#1},\A_{#1},\E_{#1},\I_{#1})}
\newcommand{\enabled}[1]{Enabled(#1)}
\newcommand{\p}[1]{part(#1)}
\newcommand{\urgent}{Urgent}
\newcommand{\al}[1]{\textsf{at}(#1)}
\newcommand{\action}{\textsf{action}}
\newcommand{\source}{\textsf{source}}
\newcommand{\target}{\textsf{target}}
\newcommand{\guard}{\textsf{guard}}
\newcommand{\reset}{\textsf{reset}}
%\newcommand{\hmn}{h_{\min}}
\newcommand{\reacha}[1]{\overline{Reach(#1)}}
%\newcommand{\stacktxt}[1]{\mathit{\stackon[0pt]{#1}{\hstretch{7.0}{\sim}}}}
\newcommand{\lpsabrb}{LPS }
\newcommand{\lpsabr}{LPS}
\newcommand{\lps}{local planning semantics }
\newcommand{\lpsb}{local planning semantics}
\NewEnviron{myequation}{%
\begin{displaymath}
\scalebox{1.5}{$\BODY$}
\end{displaymath}
}


%%%%%%%%%%%%%%%%%%PLANNING%%%%%%%%%%%%%%%%%%%%%%%%%%%%%%%%
\newcommand{\squig}{{\scriptstyle\sim\mkern-3.9mu}}
\newcommand{\lsquigend}{{\lhd\mkern-3mu}}
\newcommand{\rsquigend}{{\scriptstyle\rule{.1ex}{0ex}>}}
\newcounter{sqindex}
\newcommand\squigs[1]{
    \setcounter{sqindex}{0}
      \whiledo {\value{sqindex}< #1}{\addtocounter{sqindex}{1}\squig}
    }
\newcommand{\tranbp}[2]{
  \mathbin{\stackon[2pt]{\squigs{#2}\rsquigend}{\scriptstyle{#1\,}}}_{\gamma}
}

 
\makeatletter
\def\namedlabel#1#2{\begingroup
  \mbox{\hspace{1cm}}#2%
    \def\@currentlabel{#2}%
    \phantomsection\label{#1}\endgroup
}
\makeatother

\makeatletter
\newcommand{\mytag}[2]{%
 \text{#1}%
  \@bsphack
    \protected@write\@auxout{}%
   {\string\newlabel{#2}{{#1}{\thepage}}}%
  \@esphack
}          
\makeatother


\newcommand{\confl}{\mathit{conf}}
\newcommand{\pmin}{\min(\pi)}
\newcommand{\hmin}{h_{\min}}
\newcommand{\hmax}{h_{\max}}
\newcommand{\hmaxt}{h_{\max}^{\infty}}
\newcommand{\hmaxm}{h_{\max}^{\hmin}}
\newcommand{\hmxb}[1]{h_{\max}(#1)}
\newcommand{\hmx}{h_{\max}(\alpha)}
\newcommand{\plntxt}[1]{\mathit{Plannable(#1)}}
\newcommand{\plntxtb}[1]{\mathit{\stackon[0pt]{Plannable(#1)}{\hstretch{7.0}{\sim}}}}


\newcommand{\tcal}[1]{\mathcal{#1}}
\newcommand{\lto}{\longrightarrow}
\newcommand{\urgb}{\tsf{urg}}
\newcommand{\urg}{\textnormal{\textit{urg}}}
\newcommand{\tpcp}[1]{\textsf{tpc}(#1)}
\newcommand{\tpc}{\textsf{tpc}}

\newcommand{\pln}[1]{\mathit{Plannable(#1)}=\bigvee_{\loc\in\Loc_{#1}}
\al{\loc}\wedge\backhtxt(\bigwedge_{a_i\in#1} \guard{a_i}{\loc_i})}

\newcommand{\plnIntxt}[2]{\mathit{Plannable(#1,#2)}}
\newcommand{\plnIn}[2]{\mathit{Plannable(#1,#2)}=\bigvee_{\loc\in\Loc_{#1}}
\al{\loc}\wedge\bigwedge_{a_i\in#1} \Big(\guard{a_i}{\loc_i}+#2\Big)}
%%%%%%%%%%%%%%%%%%%%%%%%%%%%%%%%%%%%%%%%%%%%%%%%%%%%%%%%%%%%%%%%%%%%%%%%%%



\newcommand{\clock}[1]{\textsf{clock}(#1)}

\newcommand{\valdt}{v^{dt}}                              % valuation of the clocks
\newcommand{\valdtb}{v^{dt'}}                              % valuation of the clocks
\newcommand{\valg}{\bar{v}}                              % valuation of the clocks
\newcommand{\valgp}{\bar{v'}}  
\newcommand{\qdt}{q^{dt}}
\newcommand{\qdtb}{q^{dt'}}
\newcommand{\RS}{R}
\newcommand{\enabledb}{\mathit{Enabled}^{{\scriptscriptstyle
\swarrow^{\epsilon}}}}
\newcommand{\enabledf}{\mathit{Enabled}^{{\scriptscriptstyle
\nearrow^{\epsilon}}}}
\newcommand{\reach}{\textit{Reach}}
\newcommand{\simdt}{\sqsubseteq^{\epsilon*}_{\RS}}

%%%%%%%%%%%%%%%%%%%%%%%%%%%%%%%%%%%%%%%%%%%%DRIFT%%%%%%%%%%%%%%%%%%%%%%%%%%%
\makeatletter
\pgfdeclareshape{document}{
\inheritsavedanchors[from=rectangle] % this is nearly a rectangle
\inheritanchorborder[from=rectangle]
\inheritanchor[from=rectangle]{center}
\inheritanchor[from=rectangle]{north}
\inheritanchor[from=rectangle]{south}
\inheritanchor[from=rectangle]{west}
\inheritanchor[from=rectangle]{east}
\inheritanchor[from=rectangle]{north east}
\inheritanchor[from=rectangle]{north west}
\inheritanchor[from=rectangle]{south east}
\inheritanchor[from=rectangle]{south west}
% ... and possibly more
\backgroundpath{% this is new
% store lower right in xa/ya and upper right in xb/yb
\southwest \pgf@xa=\pgf@x \pgf@ya=\pgf@y
\northeast \pgf@xb=\pgf@x \pgf@yb=\pgf@y
% compute corner of ‘‘flipped page’’
\pgf@xc=\pgf@xb \advance\pgf@xc by-7pt % this should be a parameter
\pgf@yc=\pgf@yb \advance\pgf@yc by-7pt
% construct main path
\pgfpathmoveto{\pgfpoint{\pgf@xa}{\pgf@ya}}
\pgfpathlineto{\pgfpoint{\pgf@xa}{\pgf@yb}}
\pgfpathlineto{\pgfpoint{\pgf@xc}{\pgf@yb}}
\pgfpathlineto{\pgfpoint{\pgf@xb}{\pgf@yc}}
\pgfpathlineto{\pgfpoint{\pgf@xb}{\pgf@ya}}
\pgfpathclose
% add little corner
\pgfpathmoveto{\pgfpoint{\pgf@xc}{\pgf@yb}}
\pgfpathlineto{\pgfpoint{\pgf@xc}{\pgf@yc}}
\pgfpathlineto{\pgfpoint{\pgf@xb}{\pgf@yc}}
\pgfpathlineto{\pgfpoint{\pgf@xc}{\pgf@yc}}
}
}
\makeatother


\tikzstyle{doc}=[%
  draw,
  thick,
  align=center,
  color=black,
  shape=document,
  minimum width=11mm,
  minimum height=6mm,
  shape=document,
  inner sep=1ex,
]


\tikzset{                                 
  connect/.style args={(#1) to (#2) over (#3) by #4}{
    insert path={                         
    let                             
    \p1=($(#1)-(#3)$),            
    \n1={veclen(\x1,\y1)},        
    \n2={atan2(\y1,\x1)},         
    \n3={abs(#4)},                
    \n4={#4>0 ?180:-180}          
      in                            
      (#1) -- ($(#1)!\n1-\n3!(#3)$) 
      arc (\n2:\n2+\n4:\n3) -- (#2) 
    }                                     
  },                                       
  rc/.style={rounded corners=2mm,line width=1pt},
  ->,
  node distance=1.3cm,
  >=stealth',
  bend angle=20,
  auto,
  place/.style={text width=6mm,align=center,circle,thick,draw=blue!75,fill=blue!20,minimum size=8mm},
  redplace/.style={place,draw=red!75,fill=red!20},
  markplace/.style={place,draw=black!75,fill=black!20},
  dots/.style={fill=black,circle,inner sep=2pt},
  initial text={},
  sh2w/.style={shift={(-0.7,0)}},
  sh2w2/.style={shift={(-1.5,0)}},
  sh2nw/.style={shift={(-0.7,0.5)}},
  sh2ne/.style={shift={(0.7,0.5)}},
  sh2sw/.style={shift={(-0.7,-0.5)}},
  sh2sw2/.style={shift={(-1.5,-0.5)}},
  sh2se/.style={shift={(0.7,-0.5)}},
  h2n/.style={shift={(0,.5)}},
  h2s/.style={shift={(0,-.5)}},
  h2se/.style={shift={(0.5,-1.1)}},
  h2e/.style={shift={(0.5,0)}},
  rc/.style={rounded corners=2mm},
  triangle/.style={fill=black,regular polygon,regular polygon sides=3,
   minimum size=10pt,inner sep=0pt},
  transition/.style={
    rectangle,
    thick,
    fill=black,
    minimum width=7mm,
    inner ysep=0.5mm
  },
  rectnodes/.style args={#1,#2}{
  draw=black,        
  rounded corners,
  thick,             
  minimum width=#1 cm,
  minimum height=#2 cm,
  },                   
  rectnode/.style args={#1,#2}{
  rounded corners,
  draw=red!50,       
  fill=yellow!30,    
  thick,             
  minimum width=#1 cm,
  minimum height=#2 cm,
  }        
}       

\definecolor{mymagenta}{RGB}{226,0,116}
\definecolor{mygray}{RGB}{208,208,208}
\newcommand*{\mytextstyle}{\color{black}}
\newcommand{\arcarrow}[4]{%
   % inner radius, middle radius, outer radius, start angle,
   % end angle, tip protusion angle, options, text
   \pgfmathsetmacro{\rin}{3}
   \pgfmathsetmacro{\rmid}{3.5}
   \pgfmathsetmacro{\rout}{4}
   \pgfmathsetmacro{\astart}{#1}
   \pgfmathsetmacro{\aend}{#2}
   \pgfmathsetmacro{\atip}{1}
   \fill[#4, very thick] (\astart+\atip:\rin)
                         arc (\astart+\atip:\aend:\rin)
      -- (\aend-\atip:\rmid)
      -- (\aend:\rout)   arc (\aend:\astart+\atip:\rout)
      -- (\astart:\rmid) -- cycle;
   \path[
      decoration = {
         text along path,
         text = {|\mytextstyle|#3},
         text align = {align = center},
         raise = -1.0ex
      },
      decorate
   ](\astart+\atip:\rmid) arc (\astart+\atip:\aend+\atip:\rmid);
}

\tcbset{
frame code={}
center title,
left=0pt,
right=0pt,
top=0pt,
bottom=0pt,
colback=gray!10,
colframe=white,
width=\dimexpr\textwidth\relax,
enlarge left by=0mm,
boxsep=5pt,
arc=0pt,outer arc=0pt,
}



\newcommand{\nt}[1]{|\textbf{#1}|}

%\addbibresource{References/references.bib}
% ************************ Thesis Information & Meta-data **********************
% Thesis title and author information, refernce file for biblatex
% ************************ Thesis Information & Meta-data **********************
%% The title of the thesis
\crest{\includegraphics[scale=0.15]{uga}}
\title{Formal Methods for Distributed Real-Time SYstems}
%\title{Writing your PhD thesis in \texorpdfstring{\\ \LaTeX2e}{LaTeX2e}}
%\texorpdfstring is used for PDF metadata. Usage:
%\texorpdfstring{LaTeX_Version}{PDF Version (non-latex)} eg.,
%\texorpdfstring{$sigma$}{sigma}

%% Subtitle (Optional)
%\subtitle{Using the CUED template}

%% The full name of the author
\author{Dellabani Mahieddine}

%% Department (eg. Department of Engineering, Maths, Physics)
\dept{Laboratoire Verimag}

%% University and Crest
\university{Université Grenoble Alpes}

%% You can redefine the submission text:
% Default as per the University guidelines:
% ``This dissertation is submitted for the degree of''
%\renewcommand{\submissiontext}{change the default text here if needed}

%% Full title of the Degree
\degree{Doctor of Philosophy}

%% College affiliation (optional)
\college{Ecole Doctorale Mathématique, Sciences et Technologies de
l’Information, Informatique (MSTII)}
%% Submission date
% Default is set as {\monthname[\the\month]\space\the\year}
%\degreedate{September 2014} 

%% Meta information
%\subject{LaTeX} \keywords{{LaTeX} {PhD Thesis} {Engineering} {University of
%Cambridge}}

\listfiles

% ***************************** Abstract Separate ******************************
% To printout only the titlepage and the abstract with the PhD title and the
% author name for submission to the Student Registry, use the `abstract' option in
% the document class.

\ifdefineAbstract
 \pagestyle{empty}
 \includeonly{Declaration/declaration, Abstract/abstract}
\fi

% ***************************** Chapter Mode ***********************************
% The chapter mode allows user to only print particular chapters with references
% Title, Contents, Frontmatter are disabled by default
% Useful option to review a particular chapter or to send it to supervisior.
% To use choose `chapter' option in the document class

\ifdefineChapter
 \includeonly{Chapter3/chapter3}
\fi


% ***************************** Minitoc Settings ********************************

\setcounter{minitocdepth}{2} 
\setlength{\mtcindent}{24pt} 
\renewcommand{\mtcfont}{\small\rm} 
\renewcommand{\mtcSfont}{\small\bf} 

\nobibliography*
% ******************************** Front Matter ********************************
\begin{document}

\frontmatter

\includepdf[pages=-]{cover.pdf}


\begin{titlepage}
  \maketitle
\end{titlepage}

\mainmatter

%\include{Dedication/dedication}
%\include{Declaration/declaration}
\begin{acknowledgements}
\begin{otherlanguage}{french}
\ackfont
\large
\lettrine[lines=3,loversize=0.2]{\color{BrickRed}F}{} irstly, I would like to express my 
sincere gratitude to my advisor Prof. Saddek Bensalem for giving me the opportunity to carry 
out my PhD at the Verimag laboratory, for providing ideal research
conditions and for all his trust and support during this period. 
I would like to deepely thank my co-advisor Dr. Jacques Combaz for his continuous support, 
patience, motivation, and immense knowledge. His guidance as well as the tremendous 
amount of time spent with me all along my PhD are pricesless.
I want to thank Dr. Marius Bozga for his precious suggestions, advices and for 
sharing with me all his knowledge in formal methods.

I would like to thank Prof. Stavros Tripakis and Prof. Oleg Sokolsky for accepting to reviw this
thesis. Many thanks to Prof. Florence Maraninchi, Prof. Patricia Bouyer-Decitre for accepting 
to be part of the jury.
  
This journey would not have been possible without my family. To my parents, who always favored
their children without asking or expecting anything in return. Thank you for all your 
encouragement, assistance and for the emotional and financial supports. Special thanks to my 
sisters for all the time and efforts spent while trying to cheer me up during difficult moments.
Special thoughts to my maternal grandmother who passed away during my first year of PhD.
  
My sincere thanks to Lotfi Mediouni for all the time spent together during coffee breaks, for
his help and especially for his optimism (I think I have never seen a person that optimistic
in my life). Special thanks to all my colleagues Hosein Nazarpour, Ayoub Nouri, Rany Kahil, 
Rim El-Belouli and all my Verimag collegues.

Last but not least, thanks you Anne for her continuous support, and to her familly
for all their kindness and the endless invitations to family dinners, christmass and every other
occasion. They made me feel I am home.
     
\end{otherlanguage}
\end{acknowledgements}


\begin{abstract}


Nowadays, real-time systems are ubiquitous in several application domains. 
Such an emergence led to an increasing need of performance (resources, 
availability, concurrency, etc) and initiated a shift from the
use of single processor based hardware platforms, to large sets 
of interconnected and distributed computing nodes. This trend introduced the birth 
of a new family of systems known as \emph{Networked Embedded Systems}, 
that are intrinsically distributed.
Such an evolution stems from the growing complexity of real-time software 
embedded on such platforms (e.g. electronic control in avionics 
and automotive domains), and the need to integrate formerly isolated systems so that 
they can cooperate, as well as, share resources improving thus functionality 
and reducing costs.
Undoubtedly, the design, implementation and verification of such systems are 
acknowledged to be very hard tasks since they are
are prone to different kinds of factors 
that increases considerably the complexity of coordinating parallel activities.
  %recurrent in suchlike applications. 

In this thesis, we propose a rigorous design flow that addresses part of the aspects that
needs to be considered when building distributed real-time applications.
We investigate formal timed automata based models, with well defined semantics, in order 
to study the behavior of a given system with some imposed timing constraints when deployed 
in a distributed environment. Particularly, we study \emph{(i)} the impact of the communication 
delays by introducing a minimum latency between actions execution and their effective 
scheduling date, and \emph{(ii)} the effect of hardware imperfections, more precisely clock 
drifts, on systems execution by breaking the synchrony hypothesis, often adopted during 
the modeling phase. Nevertheless, timed automata formalism is intended to describe a high
level abstraction of the behavior of a given application, free from the physical constraints of
the its deployment environment. Therefore, we use an intermediate representation of 
the initial application, that besides having \say{equivalent} behavior, explicitly expresses
implementation mechanism, and thus reduce the gap between the modeling and the concrete
implementation. Additionally,  we contribute in building such models by \emph{(iii)} 
proposing a knowledge based optimization that aims to eliminate unnecessary computation time 
or exchange of messages during execution. 
  
We compare the behavior of each proposed model to the initial high level model and study the
relationships between both. Then, we identify and formally characterize the potential problems 
resulting from these additional constraints and propose execution strategies that allows
to preserve some desired properties and achieve a \say{similar} execution scenario, faithfull
to the original specifications.   
  
\end{abstract}


\begin{abstractfr}
\end{abstractfr}


% *********************** Adding TOC and List of Figures ***********************

\dominitoc
\dominilof
\dominilot
\renewcommand{\baselinestretch}{0.9}\normalsize
\tableofcontents
\renewcommand{\baselinestretch}{1.0}\normalsize
%\tableofcontents

%\listoffigures\mtcaddchapter 

%\listoftables\mtcaddchapter 
%\adjustmtc
% \printnomencl[space] space can be set as 2em between symbol and description
%\printnomencl[3em]

%\printnomencl

% ******************************** Main Matter *********************************
\chapter{Introduction}
\section{Distributed Real-Time Systems}
\section{From Modelling to Implementation of Distributed Real-Time Systems}
\subsection{Modeling}
\subsection{Implementation}
\subsection{Validation}
\section{Contribution and Organization}

\part{Preliminaries}
{In this part, we give formal definitions and present results that will be used in subsequent
chapters. Chapter~\ref{chap:2} provides formal definitions of timed transition systems,
timed automata, their semantics and properties as well as a variant of the latter.
It also discusses the verification technique used in this thesis. Chapter~\ref{chap:3}
explains how an intermediate representation, based on the timed automata formalism,
can be used to represent a realistic view of a distributed real-time systems. It also tackles
two important constraints that an application may incur when being deployed in a distributed
environment under real-time restrictions.} 
\chapter{Timed Systems and Semantics}\label{chap:2} 
\minitoc
\section{Timed Transition Systems}
Transition systems provide a general and convenient method for modeling 
systems and have been used frequently to model the behavior of 
software and hardware systems. They define graphs where nodes
represent the possible \emph{states} of the system, and edges model 
\emph{transitions}, that is, state changes. A state encodes all the relevant 
information at a certain instant, whereas a transition describes how the system
evolves between two states. 

Nowadays, several variants of transition system formalisms have been 
proposed. For instance, a labeled transition system is a transition system
where the set of transitions is labeled by \emph{actions}.
In this thesis, we use \emph{Timed Transition Systems} (TTS)
to explicitly model the effect of time passage (besides the actions)
on the states of the system. Formally, it is defined as follows:  

\begin{definition}[Timed Transition System]\label{def:tts}
  A timed transition system is a tuple $\TTS$ such that $\mc{Q}$ is 
  a set of states, $q_0\in\mc{Q}$ is the initial state, $\Sigma$ is a set of
  actions, $\mb{K}$ is a time domain and $\to\subseteq\mc{Q}\times
  (\Sigma\cup\mb{K})\times\mc{Q}$ is the transition relation.
\end{definition}
Consequently, we distinguish two types of transition:
\begin{itemize}
  \item \emph{action step} $\transit{\sigma}$ for $\sigma\in\Sigma$ and we write 
    $q\transit{\sigma}q'$. 
  \item \emph{time step} $\transit{d}$ for $d\in\mb{K}$ and we write $q\transit{d}q'$. 
\end{itemize}
We write $q\transit{d,\sigma}q''$ if there exists $q'\in\mc{Q}$
such that $q\transit{d}q'\transit{\sigma}q''$. We say that $q'$ is
a time successor of $q$.

Given a TTS $\TTS$, a \emph{run} $\exec$ of $\tts$ (also called 
\emph{execution sequence}), is a path that alternates action steps and 
time steps, that is:
\begin{displaymath}
  \exec=s_0\sigma_0 s_1\sigma_1 s_2\cdots \text{ such that } 
  s_i\in\mc{Q}, \  s_i\transit{\sigma_{i}}s_{i+1},\text{ and } 
  i\in\integerpoz,\sigma_i\in\Sigma\cup\mb{K}
\end{displaymath}

We denote by $\texeci$ the total elapsed time until point $i$, that is,
$\sum_{j<i}\sigma_j$ such as $\sigma_j\in\mb{K}$. In the same way, $\texec$ represents
the total elapsed time during $\varrho$, and is defined to be the limit of $\texeci$
if the sequence converges and $\infty$ otherwise.
A run $\exec$ is said to be an \emph{initial run} if $s_0=q_0$. 

A state $q\in\mc{Q}$ is called reachable if there exists an initial run
that leads to state $q$. We put $Reach(\tts)$ to denote the set of all 
reachable states of $\tts$.
\subsection{Comparing Timed Transition Systems}
In this thesis, we use the concept of (bi)simulation~\cite{bsim} 
in order to attest the similarity of timed transition systems.
\begin{definition}[Simulation]\label{def:sim}
  Given two TTS $\TTSb{1}$ and $\TTSb{2}$, a simulation relation from 
  $\ttsb{1}$ to $\ttsb{2}$ is a binary relation $\mc{R}\subseteq\mc{Q}_1
  \times\mc{Q}_2$ such that:
  \begin{itemize}
    \item $\forall(q_1,q_2)\in\mc{R},\forall\sigma\in\Sigma,\ 
      q_1\transitb{\sigma}{1}q_1'\Rightarrow\exists q'_2\in\mc{Q}_2 
      \text{ such that } q_2\transitb{\sigma}{2}q_2'\wedge
      (q_1',q_2')\in\mc{R}$
    \item $\forall(q_1,q_2)\in\mc{R},\forall d\in\mb{K},\ 
      q_1\transitb{d}{1}q_1'\Rightarrow\exists q'_2\in\mc{Q}_2 
      \text{ such that } q_2\transitb{d}{2}q_2'\wedge
      (q_1',q_2')\in\mc{R}$
  \end{itemize}
\end{definition}
  $\ttsb{2}$ simulates $\ttsb{1}$, denoted by $\ttsb{1}\simu{\mc{R}}\ttsb{2}$
  means that $\ttsb{2}$ can do everything $\ttsb{1}$ does. Notice that if
  $\ttsb{1}\simu{\mc{R}}\ttsb{2}$ and $\ttsb{2}\simu{\mc{R}}\ttsb{1}$,
  we say that $\ttsb{1}$ and $\ttsb{2}$ are bisimilar, denoted by
  $\ttsb{1}\eqv{\mc{R}}\ttsb{2}$.  

In some cases, this notion of simulation is refined in order to consider only
a part of a system behavior. This is usually the case when a system
performs \emph{internal} (or \emph{silent}) actions not visible by external
observers. This variant of simulation is called \emph{weak simulation}.

\begin{definition}[Weak Simulation]\label{def:wsim}
  Given two TTS $\TTSbw{1}$ and $\TTSbw{2}$, where $\tau$ actions 
  represent silent (unobservable) actions, a weak simulation relation from 
  $\ttsb{1}$ to $\ttsb{2}$, denoted $\ttsb{1}\simuw{\mc{R}}\ttsb{2}$, 
  is a binary relation $\mc{R}\subseteq\mc{Q}_1
  \times\mc{Q}_2$ such that:
  \begin{itemize}
    \item $\forall(q_1,q_2)\in\mc{R},\ 
      q_1\transitb{\tau}{1}q_1'\Rightarrow\exists q'_2\in\mc{Q}_2 
      \text{ such that } q_2\transitb{\tau^*}{2}q_2'\wedge
      (q_1',q_2')\in\mc{R}$
    \item $\forall(q_1,q_2)\in\mc{R},\forall\sigma\in\Sigma,\ 
      q_1\transitb{\sigma}{1}q_1'\Rightarrow\exists q'_2\in\mc{Q}_2 
      \text{ such that } q_2\transitb{\tau^*\sigma\tau^*}{2}q_2'\wedge
      (q_1',q_2')\in\mc{R}$
    \item $\forall(q_1,q_2)\in\mc{R},\forall d\in\mb{K},\ 
      q_1\transitb{d}{1}q_1'\Rightarrow\exists q'_2\in\mc{Q}_2 
      \text{ such that } q_2\transitb{\tau^*d\tau^*}{2}q_2'\wedge
      (q_1',q_2')\in\mc{R}$
  \end{itemize}
\end{definition}
  We say that $\ttsb{1}$ and $\ttsb{2}$ are observationally equivalent,
  denoted $\ttsb{1}\eqvw{\mc{R}}\ttsb{2}$ , if it exists a weak simulation 
  from $\ttsb{1}$ to $\ttsb{2}$ and vice versa.
  
  In chapter~\ref{chap:6}, we will introduce an even weaker notion of simulation 
  that characterizes the degree of closeness (in term of delays)
  between two timed systems. 

\subsection{Reactive Timed Systems}
Reactive systems are supposed to execute forever, that is,
they are supposed to act infinitely often. We refer to this characteristic as 
the \emph{requirement of progress}~\cite{progress}. 
Particularly, a timed system evolves either through an action step or 
by letting the time pass (time step).
This evolution imposes thus two requirements of progress:
discrete progress (resp. time progress) meaning that a timed system
should be able to perform action steps (resp. time steps) infinitely
often. In the physical world however, no matter how fast a system can
evolve, it cannot be infinitely fast. This induces the following constraints
on the progress of time:
\begin{enumerate}
  \item Only a finite number of actions can happen in a certain amount of time
  \item Only a bounded number of actions can happen in zero time
\end{enumerate}

\subsubsection{Time Progress (Zeno runs \& Timelocks)} 
We distinguish two types of anomalies that infringe the time progress in
a timed system namely \emph{zeno} runs and \emph{timelocks}.
A run $\exec$ is called \emph{zeno} if it is 
an infinite run and if $\texec\neq\infty$. Such a run transgresses the 
first point of the time progress presented above.
Timelocks are states from which all infinite runs starting from these states
are zeno.
\subsubsection{Discrete Progress (Deadlocks)}
States violating the discrete progress are called \emph{deadlock} states.
Formally, a state is said to be deadlock if no action can be executed from 
that state nor from any of its time successors. 

Any model of a reactive timed system should properly capture the behavior 
of the latter. Particularly, the corresponding model must react infinitely often,
that is, it must not block time or execute an unbounded number of actions in zero 
time. In that sense, we can say that deadlocks and timelocks are modeling errors
that needs to be cleared, either during the modeling process (which is tedious 
for large scale systems) or by providing verification methods that guarantee
their absence. 


\section{Timed Systems Syntax and Semantics}\label{sec:2.2}
\subsection*{Clocks}
In order to represent and measure the dense time domain, we rely on positive 
real valued variables, the \emph{clocks}. Clocks are 
positive real variables increasing synchronously (with the same rate) in a 
given system. They are used to express the progress of time and impose 
a certain dynamic on the execution of a timed system.

Given a finite set of clocks $\mc{X}$, we define the valuation function
$\val:\mc{X}\to\realpoz$ assigning to each clock $x$ a positive real value 
$\val(x)$. We put $\Val$ to denote the set of all valuations.
For a valuation $\val\in\Val$ and $d\in\realpoz$, $\val+d$ is the valuation
satisfying $(\val+d)(x)=\val(x)+d$, while for a subset of clocks 
$r\subseteq\mc{X}$, $\val[r]$ is the valuation obtained from $\val$ by 
resetting clocks of $r$, that is, $\val[r](x)=0$ for $x\in r$ and
$\val[r](x)=\val(x)$ otherwise. We write {\bf 0} for the valuation 
that assigns 0 to every clock.

An \emph{atomic clock constraint} is an expression of the form:
\begin{displaymath}
  c:=\true \ | \ x\#k \ |\ x-y\#k \
\end{displaymath}
where $x$ and $y$ are clocks in $\mc{X}$, $\#\in\{<,\le,=,\ge,>\}$,
and $k\in\integer$. A clock constraint is a conjunction of atomic clock 
constraint, that is:
\begin{equation}\label{eq:cc}
  c:=\true \ | \ x\#k \ |\ x-y\#k \ | \ c_1\wedge c_2 
\end{equation}
with $c_1$ and $c_2$ being atomic clock constraints. We write $\mc{C(X)}$
for the set of clock constraints over $\mc{X}$.
Given a clock constraint $c$ and a valuation $\val$, we say that $\val$
satisfies $c$, denoted $\val\models c$, if all constraints are satisfied
when each $x\in\mc{X}$ is replaced by $\val(x)$. 
We also consider for a clock constraint $c$ the classical \emph{backward} 
and \emph{forward} operators on
clock constraints:
\begin{align*}
  &\text{ \bf Backward } &:& \val\models\backward c\Leftrightarrow 
  \exists d\in\realpoz.\ \val+d\models c\\ 
  &\text{ \bf Forward } &:& \val\models\forward c\Leftrightarrow 
  \exists d\in\realpoz.\ \val-d\models c\\ 
\end{align*}

Additionally, we also use another variant of the backward operator considering
lower bounds $l\in\integerpoz$ and upper bounds $u\in\integerpoz\cup
\{+\infty\}$:

\begin{displaymath}
  \val\models\backwardp{l}{u} c\Leftrightarrow 
  \exists d\in\realpoz, l\le d\le u. \ \val+d\models c\\ 
\end{displaymath}

\subsection{Timed Components Syntax and Semantics}
In this thesis, components are timed automata and systems are compositions of timed automata
with respect to multiparty interactions. The timed automata we use are essentially the ones from
~\cite{AlurD94}, however, slightly adapted to embrace a uniform notation the dissertation.
\begin{definition}[Timed Component]\label{def:tc}
  A component is a tuple $B=(\Loc,\loc_0,\X,\A,\E,\I)$, where:
  \begin{itemize}
    \item $\Loc$ is a finite set of locations, with $\loc_0\in\Loc$ is the 
      initial location,
    \item $\mc{X}$ is a finite set of clocks,
    \item $\mc{A}$ is a finite set of actions
    \item $\mc{E}\subseteq\Loc\times(\mc{A}\times\mc{C(X)}\times2^{\mc{X}})
      \times\Loc$ is a finite set of transitions labeled with an action, 
      a clock constraint (guard), and a set of clocks to be reset,
    \item $\mc{I}:\Loc\to\mc{C(X)}$ is the function assigning an invariant
      to each location. Notice that invariants are restricted to conjunction
      of atomic clock constraints of the form $x\le k$. 
  \end{itemize}
\end{definition}

Throughout this thesis, we consider 
\emph{deterministic} timed components, that is, at a given location
$\loc$ and for a given action $a$, there is up to one outgoing transition
from $\loc$ labeled by a. 
A transition $e=(\loc,(a,g,r),\loc')\in\mc{E}$ is also denoted by
$\loc\transit{a,g,r}\loc'$. We write $\source(e)$, $\target(e)$, $\action(e)$, $\guard(e)$
and $\reset(e)$ for $\loc$, $\loc'$, $a$, $g$ and $r$, respectively.
We also denote by $\Phi(a,\loc)$ the guard 
of the transition labeled by $a$ and outgoing from $\loc$
if it exists, and $\false$ otherwise. It is formalized as follows:
\[\Phi(a,\loc)=\begin{cases}
  g,& \text{ if } \exists e=(\loc,a,g,r,\loc')\in\mc{E}  \\
  \false,& otherwise
\end{cases}\]

\begin{example}
  Figure~\ref{fig:tc} depicts a timed component $B$ where locations
  are represented by circles and transitions are the directed arrows
  from a location to another. The initial location ($\loc_0$ here) is
  represented by a double circle. The component 
  $B=(\Loc,\loc_0,\mc{X},\mc{A},\mc{E},\mc{I})$ is defined such that:
  \begin{itemize}
    \item $\Loc=\{\loc_0,\loc_1\}$,
    \item $\mc{X}=\{x\}$,
    \item $\mc{A}=\{a,b\}$,
    \item $\mc{E}=\{e_1,e_2\}$ where
      \begin{itemize}
        \item $e_1=(\loc_0,a,2\le x\le4,\emptyset,\loc_1)$
        \item $e_2=(\loc_1,b,\true,\{x\},\loc_0)$
      \end{itemize}
    \item $\mc{I}$ assigns the following invariants for locations: 
      $\Inv{\loc_0}=x\le4$ and $\Inv{\loc_1}=\true$
  \end{itemize}
  Notice that when a clock constraint (respectively an invariant is not
  shown on a transition (respectively location) it is interpreted as $\true$.
  This is the case for transition $e_2$ and location $\loc_2$.
\end{example}
\begin{figure}[!h]
 \centering
  \begin{tikzpicture}[scale=0.8,every node/.style={scale=0.8}]

  \node [accepting, place,label=below:\textcolor{red}{$x\le4$},label={[shift={(-.9,.3)}]$B$}](l0) {$\loc_0$};
  \node [place,right=2cm of l0] (l1) {$\loc_1$};

  \path (l0) edge [bend left] node[above,align=center]{$a$\\$2\le x\le4$}(l1)
        (l1) edge [bend left] node[below,align=center]{$b$\\$x:=0$} (l0);

  \node [rounded corners,inner xsep=11mm,inner ysep=11mm,draw, fit=(l0)(l1)] (rec1) {};
\end{tikzpicture}
  \caption{Example of a Timed Component}
 \label{fig:tc}
\end{figure}  





\begin{definition}[Standard Semantics]\label{def:std_sem}
The semantics of a timed component $B=(\Loc,\loc_0,\X,\A,\E,\I)$ is given by the timed transition
  system $\TTSs$ where:
  \begin{itemize}
    \item $\mc{Q}=\Loc\times\Val$ denotes the states
      of $B$ with $q_0=(\loc_0,0)$ being the initial state, 
    \item $\to\subseteq\mc{Q}\times(\mc{A}\cup\realpos)\times\mc{Q}$
    denotes the set of transitions between states according to the rules:
  \begin{itemize}
    \item $(\loc,\val)\transit{a}(\loc',\val[r])$ if $\loc\transit{a,g,r}
      \loc'$, $\val\models g$, and $\val[r]\models\Inv{\loc'}$
      (action step) 
    \item $(\loc,\val)\transit{d}(\loc,\val+d)$ if $\forall d'\in[0,d]$,
      $\val+d'\models \Inv{\loc}$ (time step) 
  \end{itemize}
  \end{itemize}
\end{definition}
Notice that since the invariants are restricted to conjunction of upper bound
atomic constraints, the time step can be simplified to:
\begin{displaymath}
  (\loc,\val)\transit{d}(\loc,\val+d)\text{ if }\val+d\models \Inv{\loc} 
\end{displaymath}

In this thesis, we always assume components with $\emph{well formed}$
guards, that is, for a transition $\loc\transit{a,g,r}\loc'$,
$\big(\val\models g\big)\Rightarrow\big(\val\models\Inv{\loc}\wedge\val[r]\models
\Inv{\loc'}\big)$ for any $\val\in\Val$. This ensures that the reachable states
always satisfy the location invariants. The rule on action step becomes then:
\begin{displaymath}
  (\loc,\val)\transit{a}(\loc',\val[r])\text{ if } \loc\transit{a,g,r}
      \loc'\text{ and } \val\models g 
\end{displaymath}
\begin{remark}
  When used in predicate definition, clock constraints are straightforwardly 
  applied to clock valuations of states and thus interpreted as $\true$
  or $\false$.
\end{remark}
We define the predicate $\enabled{a}$ characterizing states $(\loc,\val)$
from which an action $a$ is enabled, that is, such that $(\loc,\val)
\transit{a}(\loc',\val')$. It is formalized as follows:
\begin{displaymath}
  \enabled{a}=\bigvee_{\loc\in\Loc}\al{\loc}\wedge\Phi(a,\loc)
\end{displaymath}
where $\al{\loc}$ is $\true$ on states whose location is $\loc$.
In the same way, we define the predicates $\enabledbackward{a}$,
$\enabledbackwardb{a}{l}{u}$ and $\enabledforward{a}$ describing, respectively,
states from which an action $a$ can be executed after some time step,
some bounded time step or states that are time successors of states satisfying $\enabled{\alpha}$
(see Figure~\ref{fig:zones}). These predicates can be formally written 
as follows:
\begin{align*}
  &\enabledbackward{a}\hspace{1.8mm}=\bigvee_{\loc\in\Loc}\al{\loc}\wedge
  \backward\Phi(a,\loc)  \\
  &\enabledbackwardb{a}{l}{u}=\bigvee_{\loc\in\Loc}\al{\loc}\wedge
  \backwardp{l}{u}\Phi(a,\loc)\\
  &\enabledforward{a}\hspace{1.9mm}=\bigvee_{\loc\in\Loc}\al{\loc}\wedge
  \forward\Phi(a,\loc)
\end{align*}
\begin{figure}[H]
\center
\begin{tikzpicture}


\draw[->]  (-2,0) -- (8,0) node[below left] {time};

\draw[very thick,-]   (2,0) |- (4,0.75) -- (4,0);


\draw[{Bar[width=4mm].Straight Barb[]}-{Straight Barb[].Bar[width=4mm]}]
    (2,-0.2) -- node[fill=white,below] {$g_{a}$}  (4,-0.2);


\end{tikzpicture}
\begin{tikzpicture}


\draw[->]  (-2,0) -- (8,0) node[below left] {time};


\draw[draw=white,fill=red!30,very thick,semitransparent,-]  % <--- changed  
                (-2,0) |- (4,0.75) -- (4,0); 
\draw[draw=red,very thick,semitransparent,-]  % <--- changed  
                (-2,0.75) -| (4,0); 



\draw[{Bar[width=4mm].Straight Barb[]}-{Straight Barb[].Bar[width=4mm]}]
    (-2,-0.2) -- node[fill=white,below] {$\backward g_a$}  (4,-0.2);



\end{tikzpicture}
\begin{tikzpicture}


\draw[->]  (-2,0) -- (8,0) node[below left,yshift=-2mm] {time};


\draw[draw=white,fill=blue!30,very thick,semitransparent,-]  % <--- changed  
                (2,0) |- (8,0.75) -- (8,0); 
\draw[draw=blue,very thick,semitransparent,-]  % <--- changed  
                (2,0) |- (8,0.75); 



\draw[{Bar[width=4mm].Straight Barb[]}-{Straight Barb[].Bar[width=4mm]}]
    (2,-0.2) -- node[fill=white,below] {$\forward g_a$}  (8,-0.2);



\end{tikzpicture}
\begin{tikzpicture}


\draw[->]  (-2,0) -- (8,0) node[below left] {time};

\draw[draw=dgreen,fill=dgreen!30,very thick,-]   (0,0) |- (3,0.75) -- (3,0);


\draw[{Bar[width=4mm].Straight Barb[]}-{Straight Barb[].Bar[width=4mm]}]
    (0,-0.2) -- node[fill=white,below] {$\backwardp{l}{u} g_{a}$}  (3,-0.2);

\draw[{Bar[width=4mm].Straight Barb[]}-{Straight Barb[].Bar[width=4mm]}]
    (3,-0.2) -- node[fill=white,below] {$l$}  (4,-0.2);
\draw[{Bar[width=4mm].Straight Barb[]}-{Straight Barb[].Bar[width=4mm]}]
    (0,1) -- node[fill=white,above] {$u$}  (2,1);




\end{tikzpicture}
\caption{Backward and Forward Operators on a Guard}\label{fig:zones}
\end{figure}

A state $(\loc,\val)$ is said \emph{urgent} if time cannot progress from
this state, that is, $\nexists d\in\realpos$ such that $(\loc,\val)\transit{d}
(\loc,\val+d)$. Urgent states of a component B are characterized by the predicate:
\begin{equation}\label{eq:urg}
  \urgent(B)=\bigvee_{\loc\in\Loc}\al{\loc}\wedge urg(\loc)
\end{equation}
where $urg(\loc)$ is a clock constraint characterizing the valuations from
which time cannot progress with respect to the location invariant 
$\Inv{\loc}$, that is, assuming that 
$\Inv{\loc}=\bigwedge_{i=1}^m x_i\le k_i$ then 
$urg(\loc)=\bigwedge_{i=1}^m x_i\ge k_i$. Notice that due to well formed
guards, an urgent reachable state satisfies also Expression~\ref{eq:urg} if inequalities
$x_i\ge k_i$ on clocks are replaced by equalities $x_i=k_i$ in the expression
of $urg(\loc)$.

\begin{definition}[Strongly Non-Zeno Timed Component]\label{def:snz}
  Given a timed component $B$, a \emph{structural loop} of $B$ is a sequence 
  of distinct transition $e_1\cdots e_m$ such that $\forall i\in\{1,\cdots,m
  \},\target(e_i)= \source(e_{i+1})$. $B$ is called \emph{strongly non-zeno} 
  if for every structural loop there exists a clock $x$ and some $0\le i,j
  \le m$ such that:
  \begin{itemize}
    \item $x$ is reset in step $i$, that is, $x\in\reset(e_i)$
    \item $x$ is bounded from below in step j, that is, $(x<1)\wedge\guard(e_j)
      =\false$
  \end{itemize}
\end{definition}
Intuitively, this definition implies that at least 1 time unit elapses 
at each loop of $B$.

\begin{lemma}[\cite{progress}]\label{lm:snz}
  If $B$ is strongly non-zeno then every infinite run of $B$ is non-zeno
\end{lemma}

The following corollary is an immediate consequence of lemma~\ref{lm:snz}. 
\begin{corollary}\label{cr:snz}
Given a timed component $B$, if B is strongly non-zeno then it is 
  timelock free.
\end{corollary}

Corollary~\ref{cr:snz} highlights an interesting fact of strong non-zenoness.
It discharges from the trouble of checking time progress. In particular, checking
the progress of a timed system is reduced to checking its deadlock freedom.

\begin{definition}[Deadlocks]\label{def:dl}
  Given a timed component $B$. We say that a state $(\loc,\val)$ of $B$ is 
  a \emph{deadlock} if and only if no action can be executed from this state
  or any of its time successors, that is:
  \begin{displaymath}
    \neg\Big(\exists a\in\mc{A}. \ (\loc,\val)\transit{a}(\loc',\val')\vee
    \exists d>0. \ (\loc,\val)\transit{d}(\loc,\val+d)\transit{a}(\loc',\val')\Big)
  \end{displaymath}
  Deadlock states are characterized by the following predicate:
  \begin{displaymath}
   \bigvee_{\loc\in\Loc}\al{\loc}\wedge
 \neg\Big(\bigvee_{a\in\mc{A}}
    \backward\big(\enabled{a}\wedge\Inv{\loc}\big)\Big)
  \end{displaymath}
  Because of well formed guards, the above can be simplified into:
  \begin{equation}\label{eq:dl}
    \bigwedge_{a\in\mc{A}}\neg\enabledbackward{a}
  \end{equation}
\end{definition}
A deadlock $(\loc,\val)$ is called an \emph{action-time-lock} when no action can
execute nor time can progress from $(\loc,\val)$, that is:
  \begin{displaymath}
    \neg\Big(\exists a\in\mc{A}. \ (\loc,\val)\transit{a}(\loc',\val')\vee
    \exists d>0. \ (\loc,\val)\transit{d}(\loc,\val+d)\Big)
  \end{displaymath}
  Action-time-locks verifies the following predicate:
  \begin{equation}\label{eq:atl}
    \bigwedge_{a\in\mc{A}}\neg\enabled{a}\wedge\bigvee_{\loc\in\Loc}
    \big(\al{\loc}\wedge urg(\loc)\big)
  \end{equation}
\subsection{Timed Systems}
The common practice in component-based timed systems is to have several components
executing in parallel while their clocks increase synchronously. Moreover,
it is often mandatory to restrict the components behaviors so as to achieve 
a given global property. This is usually achieved by coordinating the execution of 
actions in components using synchronization mechanisms.
In what follows, components communicate by means
of \emph{multiparty interactions}. A multiparty interaction is a rendez-vous 
synchronization between actions of a fixed subset of components. It takes place 
only if all the participants agree to execute the corresponding actions. Given 
$n$ components $B_i$, with disjoint sets of actions $\mc{A}_i$, an interaction
is a subset of actions $\alpha\subseteq\cup_{1\le i\le n}A_i$ containing at most
one action per component, that is, $\alpha\cap\mc{A}_i$ is either empty or a 
singleton $\{a_i\}$. Thus, an interaction $\alpha$ can be put in the form
$\alpha=\{a_i\}_i{\in{I}}$ with $I\subseteq\{1,\cdots,n\}$ and $a_i\in\mc{A}_i$
for all $i\in{I}$. We denote by $\p{\alpha}$, the set of components \emph{participating}
in $\alpha$, that is, $\p{\alpha}=\{B_i\}_{i\in{I}}$.

\begin{definition}[Timed System]\label{def:comp}
  Given $n$ components $\tci{i}$ with $\Loc_i\cap\Loc_j=\emptyset, \ \A_i\cap
  \A_j=\emptyset,\text{ and }\X_i\cap\X_j=\emptyset$ for any $i\neq j$,
  the composition with respect to the interaction set $\gamma$, denoted by 
  $S=\gamma(B_1,\cdots,B_n)$, is defined by the timed component 
  $(\Loc,\loc_0,\X,\gamma,\E_{\gamma},\I)$ where:
  \begin{itemize}
    \item $\Loc=\Loc_1\times\cdots\times\Loc_n$
    \item $\loc_0=(\loc_{0_1},\cdots,\loc_{0_n})$,
    \item $\mc{X}=\mc{X}_1\cup\cdots\cup\mc{X}_n$,
    \item $\Inv{\loc}=\Invi{\loc_1}{1}\wedge\cdots\wedge\Invi{\loc_n}{n}$, for $\loc=
      (\loc_1,\cdots,\loc_n)$,
    \item $\mc{E}_{\gamma}$ is defined by:
  \end{itemize}
      \begin{myequation}
        \mc{E}_{\gamma}=\left\{
          \substack{\loc\transit{\alpha,g,r}\loc'\\
            \text{ for } \alpha=\{a_i\}_{i\in{I}}\in\gamma}\Big\lvert
        \substack{\loc=(\loc_1,\cdots,\loc_n)\in\Loc, \  
        \loc'=(\loc_1',\cdots,\loc_n')\in\Loc \\
        \text{ if } i\not\in{I},\ \loc_i'=\loc_i,\text{ and for }i\in{I},\
        \loc_i\transit{a_i,g_i,r_i}\loc_i'\text{ and } \\ 
        g=\bigwedge_{i\in{I}}g_i,\  r=\cup_{i\in{I}}r_i} \right\} 
      \end{myequation}
      
\end{definition}

In a composition $S$ of $n$ components $B_i$, an action $a_i$ can execute only as part
of an interaction $\alpha$ such that $a_i\in\alpha$, that is, along with the 
execution of all other actions $a_j\in\alpha$. This corresponds to the usual 
notion of multiparty interactions.
In practice, we do not explicitly build compositions of timed components as 
presented in Definition~\ref{def:comp}. We rather interpret their semantics by 
evaluating enabled interactions based on current states of components. 
\begin{property}[Semantics of a Composition]\label{pr:std_sem}
  Given a set of components $\{B_1,\cdots,B_n\}$ and an interaction set $\gamma$,
  the semantics of the composite component $S=\gamma(B_1,\cdots,B_n)$
  with respect to the set of interaction $\gamma$, is defined by 
  the timed transition system $\TTSg$ where:
  \begin{itemize}
    \item $\mc{Q}_g=\Loc\times\Val$ is the set of global states, where
      $\Loc=\Loc_1\times\cdots\times\Loc_n$ and $\mc{X}=\bigcup_{i=1}^n\mc{X_i}$.
      We write a state $q=(\loc,\val)$ where $\loc=(\loc_1,\cdots,\loc_n)\in\Loc$
      is a global location and $\val=(\val_1,\cdots,\val_n)\in\Val$ is a global 
      clock valuation. The initial state is $q_0=(\loc_0,0)$,
    \item $\gamma$ is the set of interactions,
    \item $\to_{\gamma}$ is the set of transitions defined by the rules:
    \begin{itemize}
      \item Interaction step: \\
      \end{itemize}
  \end{itemize}
  \vspace*{-1cm}
  \begin{align*}
    & \alpha=\{a_i\}_{i\in I}\in\gamma,\quad\forall i\in{I}.(\loc_i,\val_i)
    \transitb{a_i}{\gamma}(\loc'_i,\val'_i),\quad\forall i\notin{I}.
    (\loc_i,\val_i)=(\loc'_i,\val'_i)\\
    \cline{1-2}
   &\hspace{4.5cm}(\loc,\val)\transitb{\alpha}{\gamma}(\loc',\val')
 \end{align*}
  \vspace*{-1.5cm}
  \begin{itemize}
    \item[] 
      \begin{itemize}
      \item Time step:
    \end{itemize}
  \end{itemize}
  \vspace*{-5mm}
  \begin{align*}
    & d\in\realpos,\quad\forall i\in\{1,\cdots,n\}
    .\val_i+d\models\Invi{\loc_i}{i}\\
    \cline{1-2}
   &\hspace{2cm}(\loc,\val)\transitb{d}{\gamma}(\loc,\val+d)
 \end{align*}
\end{property}

To simplify notations, predicates defined on individual components $B_i$ are
straightforwardly interpreted on global states $(\loc,\val)$ of the composition
by considering the projection $(\loc_i,\val_i)$ of $(\loc,\val)$ on $B_i$.
For instance, $\al{\loc_i}$ evaluates to true on $(\loc,\val)$ iff
$\loc\in\Loc_1,\times\cdots\times\Loc_{i-1}\times\{\loc_i\}\times\Loc_{i+1}\times
\cdots\times\Loc_n$. Similarly, clock constraints of component $B_i$ are applied to clock 
valuation functions of the composition
by restricting $\val$ to clocks in $\mc{X}_i$ of $B_i$. This allows to write
the predicate $\enabled{\alpha}$, characterizing states $(\loc,\val)$ from which 
an interaction $\alpha=\{a_i\}_{i\in{I}}\in\gamma$ can be executed, as:
\begin{align}\label{eq:en}
  \enabled{\alpha}&=\bigwedge_{i\in{I}}\enabled{a_i},\\
                  &=\bigwedge_{i\in{I}}\bigvee_{\loc_i\in\Loc_i}\al{\loc_i}\wedge
                  \Phi(a_i,\loc_i),\label{eq:en_1}\\
                  &=\bigwedge_{\substack{\loc\in\Loc\\ 
                  \loc=(\loc_1,\cdots,\loc_n)}}\al{\loc}\wedge
                  \bigwedge_{\substack{i\in I\\ a_i\in\alpha }}\Phi(a_i,\loc_i)\label{eq:en_2}
\end{align}
Expression~\ref{eq:en_1} expresses the predicate $\enabled{\alpha}$ using location of individual
components whereas Expression~\ref{eq:en_2} formalizes it on global location configurations.
Notice that the above formulation of $\enabled{\alpha}$ corresponds to locations 
enumeration of all components participating in interaction $\alpha$. In practice,
we rather consider only a subset of locations $\Loc_{\alpha}\subseteq\Loc$, from
which the execution of $\alpha$ is possible. This corresponds to $\Pi_{i\in{I}}
|\Loc_{a_i}|$ possible configuration, where $\Loc_{a_i}\subseteq\Loc_i$ is a subset
of locations from which there exists a transition labeled by action $a_i\in\alpha$,
and $|\Loc_{a_i}|$ denotes the cardinality of $\Loc_{a_i}$, which is reasonably
small in practical examples but can be (at the worst case) equal to $\Pi_{i\in{I}}
|\Loc|$. The predicates $\enabledbackward{\alpha}$, 
$\enabledbackwardb{\alpha}{l}{u}$ and $\enabledforward{\alpha}$ becomes:

\begin{align*}
  &\enabledbackward{\alpha}\hspace{1.8mm}=\bigvee_{\substack{\loc\in\Loc\\
  \loc=(\loc_1,\cdots,\loc_n)}}\al{\loc}\wedge \backward\big(\bigwedge_{\substack{
    i\in{I}\\a_i\in\alpha}}\Phi(a_i,\loc_i)\big) \\
  &\enabledbackwardb{\alpha}{l}{u}=\bigvee_{\substack{\loc\in\Loc\\
  \loc=(\loc_1,\cdots,\loc_n)}}\al{\loc}\wedge \backwardp{l}{u}\big(\bigwedge_{
    \substack{i\in{I}\\a_i\in\alpha}}\Phi(a_i,\loc_i)\big) \\
  &\enabledforward{\alpha}\hspace{1.9mm}=\bigvee_{\substack{\loc\in\Loc\\
  \loc=(\loc_1,\cdots,\loc_n)}}\al{\loc}\wedge \forward\big(\bigwedge_{\substack{
    i\in{I}\\a_i\in\alpha}}\Phi(a_i,\loc_i)\big) \\
\end{align*}

Notice that for a clock constraint $c=c_1\wedge c_2$, we have:
\begin{displaymath}
  \diamond c = \diamond (c_1\wedge c_2) \neq \diamond c_1\wedge \diamond c_2
\end{displaymath}
with $\diamond\in\{\backward,\forward\}$.

The definitions of deadlocks and action-time-locks are also trivially extended
to composition of timed components. Deadlocks can be characterized as follows:
\begin{displaymath}
  \bigvee_{\loc=(\loc_1,\cdots,\loc_i)\in\Loc}\al{\loc}\wedge
  \big[\bigwedge_{\alpha\in\gamma}\neg\backward\big(\enabled{\alpha}\wedge
  \bigwedge_{1\le i\le n}\Invb{i}{\loc_i}\big)\big]
\end{displaymath}
and action-time-locks by:
\begin{displaymath}
  \big(\bigwedge_{\alpha\in\gamma}\neg\enabled{\alpha}\big)\wedge\big(
  \bigvee_{1\le i\le n}\bigvee_{\loc_i\in\Loc_i}\al{\loc_i}\wedge urg(\loc_i)\big)
\end{displaymath}

\subsection{Example}\label{exp:run}
\begin{figure}[!h]
 \centering
  \shorthandoff{:!}
  \begin{tikzpicture}[every node/.style={scale=0.6},font=\small]

  \node [accepting, place] (l0)  {$\loc_0^1$};
  \node [place,below=1.5cm of l0,label={[shift={(-1.5,4)}]$C$}] (l1) {$\loc_1^1$};

  \path (l0) edge [bend left] node[right,align=center]{$init_0$\\$z>25$} (l1)
    (l1) edge [bend left] node[left,align=center]{$start_0$\\$z:=0$} (l0);

  \node [accepting, place] (p1-0) [right=2cm of l0] {$\loc_0^2$};
  \node [place] (p1-1) [right=1.5cm of p1-0]{$\loc_1^2$};
  \node [place] (p1-2) [below=1.5cm of p1-1,label=right:\textcolor{red}{$x\le 30$},label={[shift={(1.4,4)}]$T_1$}]            {$\loc_2^2$};
  \node [place] (p1-3) [left=1.5cm of p1-2,label=left:\textcolor{red}{$x\le 4$}] {$\loc_3^2$};

  \path (p1-0) edge node[align=center, pos=0.5]{$init_1$} (p1-1)
    (p1-1) edge node[align=center, pos=0.5]{$start_1$\\$x:=0$ } (p1-2)
    (p1-2) edge node[align=center, pos=0.5]{$process_1$ \\$10\le x\le30,$ $x:=0$ } (p1-3)
    (p1-3) edge node[align=center, pos=0.5]{$end_1$\\$x\le4$} (p1-0);


  \node [place] (p2-1) [left=2cm of l0]{$\loc_1^3$};
  \node [accepting, place] (p2-0) [left=1.5cm of p2-1]{$\loc_0^3$};
  \node [place] (p2-2) [below=1.5cm of p2-1,label=right:\textcolor{red}{$y\le 30$},label={[shift={(-5,4)}]$T_2$}]            {$\loc_2^3$};
  \node [place] (p2-3) [left=1.5cm of p2-2,label=left:\textcolor{red}{$y\le 4$}] {$\loc_3^3$};

  \path (p2-0) edge node[align=center, pos=0.5]{$init_2$} (p2-1)
    (p2-1) edge node[align=center, pos=0.5]{$start_2$\\ $y:=0$ } (p2-2)
    (p2-2) edge node[align=center, pos=0.5]{$process_2$\\$10\le y\le30,$ $y:=0$ } (p2-3)
    (p2-3) edge node[align=center, pos=0.5]{$end_2$\\$y\le4$} (p2-0);

  \node [accepting, place] (r0) [above=2cm of l0,xshift=-2.25cm,label={[shift={(-.7,.6)}]$R$}] {$\loc_0^4$};
  \node [place,right=2cm of r0] (r1) {$\loc_1^4$};

  \path (r0) edge [bend left] node[above]{take} (r1)
    (r1) edge [bend left] node[below]{free} (r0);

  \node [rounded corners,inner xsep=15mm,inner ysep=20mm,draw, fit=(l0)(l1)] (rec1) {};
  \node [rounded corners,inner xsep=15mm,inner ysep=15mm,draw, fit=(r0)(r1)] (rec2) {};
  \node [rounded corners,xshift=1mm,inner xsep=22mm,inner ysep=20mm,draw, fit=(p1-0)(p1-1)(p1-2)(p1-3)] (rec3) {};
  \node [rounded corners,xshift=1mm,inner xsep=22mm,inner ysep=20mm,draw, fit=(p2-0)(p2-1)(p2-2)(p2-3)] (rec4) {};
%  \node [inner xsep=4cm,inner ysep=2.5cm,draw, fit=(rec1)(rec2)(rec3)(rec4)] (rec5) {};
 % \node [inner xsep=1.5cm,inner ysep=5mm,draw,above=5mm of rec1] (rec5) {

  %  $\begin{aligned}
  %    \gamma &=\{
  %  init_1=\{init,init_1\}, start_1=\{start,start_1\}, process_1=\{enter,proces_1\}, 
  %  end_1=\{end,exit,end_1\}, \\
   %  & init_2=\{init, init_2\}, start_2=\{start,start_2\}, 
   % process_2=\{enter,process_2\},end_2=\{end,exit,end_2\}\} 
  %  \end{aligned}$
 % };

  \node [dots,label=90:$init_2$] (i2) at ($(rec4.south west)!0.6!(rec4.south east)$) {};
  \node [dots,label=90:$start_2$] (s2) at ($(rec4.south west)!0.8!(rec4.south east)$) {};
  \node [dots,label=-90:$end_2$] (e2) at ($(rec4.north east)!0.2!(rec4.north west)$) {};
  \node [dots,label=-90:$process_2$] (p2) at ($(rec4.north east)!0.5!(rec4.north west)$) {};

  \node [dots,swap,label=90:$init_1$] (i1) at ($(rec3.south east)!0.6!(rec3.south west)$) {};
  \node [dots,swap,label=90:$start_1$] (s1) at ($(rec3.south east)!0.8!(rec3.south west)$) {};
  \node [dots,swap,label=-90:$end_1$] (e1) at ($(rec3.north west)!0.2!(rec3.north east)$) {};
  \node [dots,swap,label=-90:$process_1$] (p1) at ($(rec3.north west)!0.5!(rec3.north east)$) {};

  \node [dots,label=-90:take] (tr) at ($(rec2.north west)!0.5!(rec2.north east)$) {};
  \node [dots,label=90:free] (fr) at ($(rec2.south west)!0.5!(rec2.south east)$) {};

  \node [dots,label=90:$init_0$] (ic) at ($(rec1.south west)!0.35!(rec1.south east)$) {};
  \node [dots,label=90:$start_0$] (rc) at ($(rec1.south west)!0.65!(rec1.south east)$) {};
  
  \path (tr) ++(0,0.25cm) +(-1cm,0) coordinate(xp2) +(1cm,0) coordinate(xp1);
  \draw  [-] (p1) |-node[above,xshift=-2.5cm]{$\alpha_5$} (xp1) -- (tr) -- (xp2)node[above,xshift=-2.5cm]{$\alpha_6$} -| (p2);

  \path (ic) ++(0,-0.5cm) +(1cm,0) coordinate(xi1) +(-1cm,0) coordinate(xi2);
  \draw[-,name path=line1] (i1) |-node[above,xshift=-2.5cm]{$\alpha_1$} (xi1) -- (ic) -- (xi2)node[above,xshift=-2.5cm]{$\alpha_2$} -| (i2); %here


  \path (rc) ++(0,-1cm) +(1cm,0) coordinate(sx1) +(-1cm,0) coordinate(sx2);

  \path[-,name path=line2] (s1) |-node[above,xshift=-1.5cm]{$\alpha_3$} (sx1) -- (rc) -- (sx2)node[above,xshift=-2cm]{$\alpha_4$} -| (s2); %here

 \path[name intersections={of=line1 and line2, by={a,b,c,d}}];% here

% draw semicircles at crossing points on the path
\coordinate (aux1) at (s2|-sx2);
\coordinate (aux2) at (s1|-sx1);
\draw[-,connect=(s2) to (aux1) over (d) by 3pt];
\draw[-] (aux1) -- (sx2);
\draw[-,connect=(sx2) to (rc) over (c) by 3pt];
\draw[-,connect=(rc) to (sx1) over (b) by 3pt];
\draw[-] (sx1) -- (aux2);
\draw[-,connect=(aux2) to (s1) over (a) by 3pt];

\path (fr) ++(0,-2.5mm) +(-1cm,0) coordinate(xe2) +(1cm,0) coordinate(xe1);
  \draw [-] (e1) |- node[above,xshift=-2cm]{$\alpha_7$}(xe1);
  \draw [-] (xe1) -- (fr);
  \draw [-] (e2) |- node[above,xshift=2cm]{$\alpha_8$}(xe2);
  \draw [-] (xe2) -- (fr);
\end{tikzpicture}
  \caption{Task Manager}
 \label{fig:tm}
\end{figure}  




Figure~\ref{fig:tm} depicts a timed system composed of four components $C$, $T_1$, $T_2$, 
and $R$.
Component $C$ represents a  controller that initializes then releases tasks $T_1$ and $T_2$.
Tasks use the shared resource $R$ during their executions.
To implement such behavior, we consider the following interactions between $C$, $R$, and 
$T_1$: $\alpha_1=\{init_0, init_1\}$,
$\alpha_3=\{start_0, start_1\}$, $\alpha_5=\{ take, process_1\}$, $\alpha_7 = 
\{free, end_1 \}$, 
and similar interactions $\alpha_2$, $\alpha_4$, $\alpha_6$, $\alpha_8$ for task $T_2$, 
as shown by connections on Figure~\ref{fig:tm}.
The controller is responsible for firing
the execution of each task. First, it non-deterministically initializes one
of the two tasks, i.e., executes $\alpha_1$ or $\alpha_2$, and then
releases it through interaction $\alpha_3$ or $\alpha_4$.
Tasks perform their processing independently of the controller, after being granted an access 
to the shared resource ($\alpha_5$ or $\alpha_6$).
When finished, a task releases the resource (interactions $\alpha_7$ or $\alpha_8$) and goes 
back to its initial location.
An example of execution sequence of this system is given below. Valuation $v$ of clocks $x$, 
$y$, and $z$ are represented as tuples 
$(\val(x),\val(y),\val(z))$:
\begin{displaymath}
\small{
\begin{split}
&((\loc_0^1,\loc_0^2,\loc_0^3,\loc_0^4),(0,0,0))\transit{26}_{\gamma}
((\loc_0^1,\loc_0^2,\loc_0^3,\loc_0^4),(26,26,26))\transit{\alpha_1}_{\gamma}
((\loc_1^1,\loc_1^2,\loc_0^3,\loc_0^4),(26,26,26))\transit{\alpha_3}_{\gamma}\\
&((\loc_0^1,\loc_2^2,\loc_0^3,\loc_0^4),(0,26,0))\transit{10}_{\gamma}
((\loc_0^1,\loc_2^2,\loc_0^3,\loc_0^4),(10,36,10))\transit{\alpha_5}_{\gamma}
((\loc_0^1,\loc_3^2,\loc_0^3,\loc_1^4),(0,36,10))\transit{2}_{\gamma}\\&
((\loc_0^1,\loc_3^2,\loc_0^3,\loc_1^4),(2,38,12))\transit{\alpha_2}_{\gamma}
((\loc_1^1,\loc_3^2,\loc_1^3,\loc_1^4),(2,38,12))
\end{split}
}
\end{displaymath}

\section{Timed Systems with Data}
Timed models introduced in the previous section focuses on the timing behavior of a given system.
In order to achieve a higher degree of expressiveness, we extend these models with data variables.
Data allows additional representations of complex behavior. Similarly to clock variables, they may
appear in the guards of transitions as additional conditions and may be updated when transitions
fire.

\begin{definition}[Guards on clocks and Data]\label{def:guard}
  Let $\mc{X}$ be a set of clock variables and $\mc{D}$ be a set of
  data variables. We denote by $\mc{G(X,D)}$ the set of guards induced
  by the following grammar:
  \begin{displaymath}
    g:=g_x \ | \ g_d \ | \ g_1\wedge g_2
  \end{displaymath}
  where $g_x\in\mc{C(X)}$, $g_d$ is a predicate on a subset of data variables
  of $\mc{D}$, and $g_1$, $g_2$ are guards over clocks and/or data variables.
\end{definition}

We extend the notion of valuation to data variables in the following manner:
\begin{itemize}
  \item Valuations assign values to data variable (in addition to clocks),
  \item The satisfaction of a valuation to a constraint is straightforwardly 
    extended to data variables,
  \item Data variables are insensitive to the progress of time, that is, for $k\in\mc{D}$ and
    $d\in\realpos$, $(v+d)(k)=(v)(k)$,
  \item Update operations are also defined for data variables using transfer functions

\end{itemize}

We use transfer functions to express update operations on data variables of $\D$. A transfer 
function $f:\V(\D)\to\V(\D)$ assigns to each variable $d\in\D$ a new value $f(v)$ based on the
current values of variables of $\D$. Additionally, given a set $\D'\supseteq\D$, applying $f$
on $\D'$ does not change the values of variables in $\D'\setminus\D$.

\begin{definition}[Timed Component with Data]\label{def:tce}
  A timed component with data is a tuple 
  $B_d=(\Loc,\loc_0,\X,\D,\A,\E,\{f_e\}_{e\in\E},\I)$ such that:
  \begin{itemize}
    \item $\Loc$, $\loc_0$, $\mc{X}$, $\mc{A}$ and $\mc{I}$ are defined as
      in Definition~\ref{def:tc},
    \item $\mc{D}$ is a finite set of data variables,
    \item $\mc{E}\subseteq\Loc\times(\mc{A}\times\mc{G(X,D)}\times2^{\X})\times\Loc$
    is a finite set of labeled transitions with an action, an extended guard, 
      and a set of clocks to be reset. For each transition $e\in\E$, we include the transfer
      function $f_e$ that updates elements of $\mc{D}$. 
  \end{itemize}
\end{definition}

\begin{figure}[!h]
 \centering
  \begin{tikzpicture}[scale=0.8,every node/.style={scale=0.8}]

  \node [accepting, place,label=below:\textcolor{red}{$x\le4$},label={[shift={(-.9,.3)}]$B$}](l0) {$\loc_0$};
  \node [place,right=2cm of l0] (l1) {$\loc_1$};

  \path (l0) edge [bend left] node[above,align=center]{$a$\\$2\le x\le4\wedge k<=5$\\$k:=k+10$}(l1)
        (l1) edge [bend left] node[below,align=center]{$b$\\$x:=0$} (l0)
        (l0) edge [loop left] node[left,align=center]{$c$\\$k>5\wedge1<x\le4$\\$k:=k-3\wedge x:=0$}(l0);
  \node [rounded corners,inner xsep=26mm,inner ysep=15mm,draw, fit=(l0)(l1),xshift=-2.1cm] (rec1) {};
\end{tikzpicture}
  \caption{Example of an Extended Timed Component}
 \label{fig:tcd}
\end{figure}  




\begin{example}
  Component of Figure~\ref{fig:tcd} is an extended timed component with actions $a$, $b$ and $c$.
  The set of data is $\mc{D}=\{k\}$. It is used on the guards of the transitions labeled by
  $a$ and $c$ ($k\le 5$ and $k>5$ respectively). The update operations for both transitions
  are respectively $\{k:=k+10\}$ and $\{k:=k-3\wedge x:=0\}$.
\end{example}
\begin{remark}
  Notice that extending a timed component with data will result in a restriction
  of the behavior of the initial timed components since extending guards to data
  variables will only constrain the execution of transitions. Consequently,
  for a timed component with data $B_d=(\Loc,\loc_0,\X,\D,\A,\E,\{f_e\}_{e\in\E},\I)$, 
  the timed component $B=(\Loc,\loc_0,\X,\A,\E,\I)$ represents an abstraction of the latter. 
\end{remark}

\begin{definition}[Timed System with Data]\label{def:tce_sem}
  Given $n$ components $B_{d_i}=(\Loc_i,\loc_{0_i},\X_i,\D_i,\A_i,$\\$
  \E_i,\{f_e\}_{e\in\E_i},\I_i)$ and an interaction set $\gamma$,
  the composition $S_d=\gamma(B_{d_1},\cdots,B_{d_n})$ with respect to 
  the interaction set $\gamma$ is defined by the timed component 
  $(\Loc,\loc_0,\X,\D,\gamma,\E_{\gamma},\{f_e\}_{e\in\gamma},\I)$ where:
  \begin{itemize}
    \item $\Loc$, $\loc_0$, $\mc{X}$, $\gamma$ and $\mc{I}$ are defined as
      in Definition~\ref{def:comp}.
    \item $\mc{D}=\mc{D}_1\cup\cdots\cup\mc{D}_n$
    \item $\mc{E}_{\gamma}$ is straightforwardly extended by considering guards
      on data variables and the application of transfer functions on executions of interactions
  \end{itemize}
\end{definition}
The semantics of timed system with data extends the semantics of Property~\ref{pr:std_sem}
in the sense that an interaction takes place if the guards on data variables also evaluates
to $\true$. Also, interactions executions apply the underlying transfer functions of the
corresponding components transitions.

In what follows, when referring to timed component or timed system we imply by that 
extended timed component or extend timed systems, unless explicitly stated. 

\section{Verification of Timed Systems}\label{sec:2.4}
\subsection{Symbolic Reachability}

The semantics of timed systems as presented in Property~\ref{pr:std_sem} 
defines states as pairs of locations configurations and clock
valuations. By considering a continuous time domain ($\realpoz$ here), the 
resulting timed transition system yields an infinity of states.
Consequently, the usual practice is to rely on a symbolic representation
of states to make this state space finite.
A symbolic state is defined by a pair $(\loc,\xi)$ where $\loc$ is a location
and $\xi$ is a zone, a set of clock valuations defined by clock constraints
(as defined in~\ref{eq:cc}). Consequently, the set of reachable states of a 
timed component B (system) can be put on the form:
\begin{displaymath}
  Reach(B)=\bigvee_{j\in{J}}\al{\loc_j}\wedge\xi_j
\end{displaymath}

Given a timed component with data $B_d$ and its abstraction of data $B$, we have
$Reach(B_d)\subseteq Reach(B)$. This means that $Reach(B)$ can be used as an over-approximation
of the reachable state of the timed component with data $B_d$.
\subsection{Compositional Verification}

Standard verification techniques such as model checking are based on explicit 
exploration and exhaustive enumeration of all the reachable symbolic states 
of a given system. The main issue with this 
method is the combinatorial explosion when considering large scale systems.
Compositional verification have been introduced to cope with state explosion
problem, and thus achieves scalability when verifying large scale systems.
This approach is based on the concept of divide-and-conquer in order to 
break up the verification
problems into smaller subsequent problems. Compositional verification have been 
extensively studied under different manners e.g., assume-guarantee 
reasoning~\cite{Lamport77,Owl76}, 
contract-based verification~\cite{contract1,contract2},  
deductive verification
~\cite{deductive}, etc.  
In this thesis, we choose to use a deductive compositional verification 
method that exploits compositionality for analysis of timed systems 
using \emph{invariants}. Invariants are symbolic approximations of the set 
of reachable states of the system as opposed to the exact 
reachability analysis in model-checking. The key principle of this 
approach is the computation of a global invariant as the conjunction of
other invariants (components invariants, interaction invariants, etc.).



%:\emph{(i)} local invariants of 
%the underlying components, \emph{(ii)} an interaction 
%invariant deduced from the synchronizations 
%(interactions) between components, and \emph{(iii)} an additional invariant 
%capturing the 
%global timing aspects and relating clocks of different 
%components.

Let $S=\gamma(B_1,\cdots,B_n)$ be a system composed of $n$ timed components $B_i$ 
synchronizing through the interaction 
set $\gamma$, and let $\psi$ be a property of interest. Assuming that 
$GI(S)$ is the global invariant for this system, 
the verification rule of $\psi$ can be intuitively
written as follows:

\begin{align*}
  &\hspace{1cm}GI(S)\Rightarrow\psi\\
    \cline{1-2}
   &\gamma(B_1,\cdots,B_n)\models\square\psi
\end{align*}
where the notation \enquote{$\gamma(B_1,\cdots,B_n)\models\square\psi$} is to 
be read as \enquote{$\psi$ holds in every reachable state of the
composition $\gamma(B_1,\cdots,B_n)$}.

Usually when verifying timed systems, the common practice is to verify the system 
against some \emph{given property} (safety property, liveness, deadlock, etc.).
These properties allows to assert that a given model satisfies the specifications.
We use satisfiability checking to verify that the global 
invariant of a system 
implies properties of interest. Particularly, for a system S characterized 
by the global invariant $GI(S)$, and 
given a property $\psi$, we check the unsatisfiability of 
$GI(S)\wedge\neg\psi$. 




\chapter{Modeling Distributed Real-Time Systems}\label{chap:3} 
\minitoc
In the previous chapter, we presented a timed automata based model for representing 
timed systems with multiparty interactions. The semantics of such model is based on 
the notion of \emph{global states}, that is, interactions execution is based not only on the 
state of its participating components but on the states of all components of the system.
Moreover, this type of model does not provide any details on how an implementation of 
multiparty interactions can be derived.
Conversely, a distributed system can be seen as a collection of loosely coupled independent
components communicating by explicit message passing (components state may be known only
through communication). 
In order to reduce the gap between the high level abstraction of a system and its concrete
implementation, we propose an intermediate model more suited for the distributed real-time
context and that is obtained by applying transformation rules on the initial model. 
It aims to explicitly express the ongoing communication mechanism as well as 
allowing interactions execution based only on their participating components.
The key concept of this approach is to structure a given system in two main layers: \emph{(1)}
an application layer that consists of a set of distributed components and \emph{(2)} a 
scheduling layer that is responsible for scheduling the execution of the latter. Additionally, 
a third layer may be needed by the scheduling layer in order to achieve global consistency.
\section{Target Architecture}

In a distributed context, we consider that components communicate through asynchronous 
message-passing. Consequently, each component is able either to send a message, to wait for 
a notification or to execute an internal computation. Our approach proposes an architecture for 
executing multiparty interactions as a two way handshake protocol~\cite{} involving asynchronous 
exchange of messages between \emph{distributed} components and a second layer responsible for 
triggering interactions, the scheduling layer (see Figure~\ref{fig:ta}). 
In order to evaluate the enabled interactions at a given state, distributed components are 
required to send their current local information (e.g. enabled actions, invariants, 
clock constraints, etc.) to the scheduling layer using an \emph{offer} messages. 
As offers are sent asynchronously,  
the scheduling layer may not have a global knowledge of the system. It may decide to execute
an interaction based only on a partial knowledge, that is, once it accumulates enough offers.
We require that the exchange of messages is sender-triggered and not blocking, that is,
each time a sender is ready to transmit the corresponding receiver is ready to receive.
The class of models satisfying this restriction are called Send/Receive models.
\begin{figure}[h!]
\centering
  \begin{tikzpicture}

  \node [rectnode={1.1,1.1},label={[shift={(0,-1.1)}]$T_1^{SR}$}] (t1) {};
  \node [rectnode={1.1,1.1},right=7.5mm of t1,label={[shift={(0,-1.1)}]$C^{SR}$}] (c) {};
  \node [rectnode={1.1,1.1},right=7.5mm of c,label={[shift={(0,-1.1)}]$R^{SR}$}] (r) {};
  \node [rectnode={1.1,1.1},right=7.5mm of r,label={[shift={(0,-1.1)}]$T_2^{SR}$}] (t2) {};

  \node [rectnode={7,2},above=1cm of c,xshift=8.75mm] (sch) {$Scheduling$ $Layer$};
  
  
  \node [triangle,label=-90:$o_1$] (ot1) at ($(t1.north west)!0.25!(t1.north east)$) {};
  \node [dots,label=-90:$n_1$] (nt1) at ($(t1.north west)!0.75!(t1.north east)$) {};
  
  \node [triangle,label=-90:$o_2$] (oc) at ($(c.north west)!0.25!(c.north east)$) {};
  \node [dots,label=-90:$n_2$] (nc) at ($(c.north west)!0.75!(c.north east)$) {};
  
  \node [triangle,label=-90:$o_3$] (or) at ($(r.north west)!0.25!(r.north east)$) {};
  \node [dots,label=-90:$n_3$] (nr) at ($(r.north west)!0.75!(r.north east)$) {};
  
  \node [triangle,label=-90:$o_4$] (ot2) at ($(t2.north west)!0.25!(t2.north east)$) {};
  \node [dots,label=-90:$n_4$] (nt2) at ($(t2.north west)!0.75!(t2.north east)$) {};


  \node [dots,label=90:$o_1$,above=7.5mm of ot1] (sot1) {};
  \node [triangle,label=-90:$n_1$,above=10mm of nt1,rotate=180] (snt1) {};

  \node [dots,label=90:$o_3$,above=7.5mm of oc] (soc) {};
  \node [triangle,label=-90:$n_3$,above=10mm of nc,rotate=180] (snc) {};

  \node [dots,label=90:$o_3$,above=7.5mm of or] (sor) {};
  \node [triangle,label=-90:$n_3$,above=10mm of nr,rotate=180] (snr) {};

  \node [dots,label=90:$o_4$,above=7.5mm of ot2] (sot2) {};
  \node [triangle,label=-90:$n_4$,above=10mm of nt2,rotate=180] (snt2) {};

  \draw [thick,red,-] (ot1) -- (sot1);
  \draw [thick,dgreen,-] (nt1) -- (snt1);

  \draw [thick,red,-] (oc) -- (soc);
  \draw [thick,dgreen,-] (nc) -- (snc);
 
  \draw [thick,red,-] (or) -- (sor);
  \draw [thick,dgreen,-] (nr) -- (snr);
  
  \draw [thick,red,-] (ot2) -- (sot2);
  \draw [thick,dgreen,-] (nt2) -- (snt2);
\end{tikzpicture} 
\caption{High Level Representation of the Target Architecture}
\label{fig:ta}
\end{figure}

\begin{example}
  Figure~\ref{fig:ta} depicts a high level representation of the timed system~\ref{fig:tm} in a 
  distributed setting. Components $C$, $T_1$, $T_2$ and $R$ are transformed into the distributed
  components $C^{SR}$, $T_1^{SR}$, $T_2^{SR}$ and $R^{SR}$ respectively. Each component send 
  information about its current state to the scheduling layer through offers messages
  ($\{o_1,\cdots,o_4\}$), and is notified through notifications messages ($\{n_1,\cdots,n_4\}$).
  Triangles (respectively dots) indicates the sender (respectively the receiver). 
\end{example}


\subsection{Interface}
In order to express the message-passing mechanism, we introduce the notion of 
\emph{communication ports}. A communication port defines the interface of a distributed 
components, that is how it interacts with the rest of the system. For our purpose, we distinguish
three types of ports: send ports, receive ports, and unary ports.
A send port is used to export data outside of the sender when sending offer, whereas
a receive port imports data inside the receiver  
when being notified. Unary ports corresponds to independent execution of a distributed 
component, which is formally expressed using a unary interaction (singleton). 
Effectively, each action of a timed component as presented in Definition~\ref{def:tc} 
will correspond to a receive port in a distributed component, which is responsible for triggering
the execution of the underlying action.

\subsection{From Local Time to Global Time}
In the timed systems model of Chapter~\ref{chap:2}, every timed component can define a set of 
local clocks to be used for expressing clock constraints on transitions or the allowed time 
progress on locations. In our intermediate model, we choose to make use of \emph{global} clocks.
In fact, a global clock measures the absolute time elapsed since the system startup and are never
reset. This approach allows to have a common timescale between the distributed components
and the scheduling layer, which reduces considerably the effort when keeping track of the
actual time progress since one needs to maintain only the global clock(s). 
Notice that any component clock $x$ can be derived from the global clock simply by shifting its 
value by an amount of time that is constant between successive resets of $x$ is not reset. 
Consequently, achieving 
the global time mechanism is done by removing local clocks from individual components 
and adding global clock(s) to the scheduling layer. Moreover, for each local clock $x$
we include a variable $\rho_x$ that stores the absolute time of its last reset with respect
to a given global clock $g$. This variable is updated each time a transition resetting $x$ 
is executed. Then, the value of $x$ can be found by the equality $x=g-\rho_x$.  
As a result, any clock constraint $c$ involved in a component can be expressed using clock $g$
as follows:
\begin{equation}\label{eq:g_clk}
  c=\bigwedge_{x\in\mc{X}} l_x\le x\le u_x=\bigwedge_{x\in\mc{X}}l_x+\rho_x\le g\le u_x+\rho_x
\end{equation}

\subsection{Conflicting Interactions and Interaction Partitioning}
\label{sub:conf}
In a distributed context, interaction execution may occur in parallel. However,
when two interactions share at least a component it is impossible to execute both
interaction concurrently. Particularly, if these interactions are enabled from the same state 
then they are \emph{conflicting} since they will compete on the same resources (shared 
components) at the same time.
In general, conflicts can be very hard to characterize for real life case studies since 
they depend on the reachability of particular states. In~\cite{}, the computation of
the conflicting interactions set relies on over-approximations. It is based on a notion 
of \emph{potential conflicts} that can be detected by simple syntactic pre-checks, as depicted
in Figure~\ref{fig:pconf}, and are used to quickly exclude conflicts since two interactions
that are not potentially conflicting are also not conflicting. 

\begin{definition}[Potential Conflict]\label{def:pconf}
  Two interactions $\alpha_1$ and $\alpha_2$ are potentially conflicting if 
  $\p{\alpha_1}\cap\p{\alpha_2}\neq\emptyset$ and for each component $B_i\in\p{\alpha_1}\cap
  \p{\alpha_2}$ there exists two transitions of $B_i$ 
  $e_1,e_2\in\mc{E}_i$ such that $\source(e_1)=\source(e_2)$ and $\action(e_1)\in\alpha_1$,
  $\action(e_2)\in\alpha_2$.
\end{definition}
\begin{figure}[h]
 \centering
\begin{tikzpicture}[->,node distance=1cm,>=stealth',bend angle=20,auto,
  place/.style={circle,thick,draw=black,minimum size=5mm},
  dots/.style={fill=black,circle,inner sep=1.5pt},
  initial text={}]

  \node [place] (l0) {};
  \node [place,below=1cm of l0,xshift=-5mm] (l1) {};
  \node [place,below=1cm of l0,xshift=5mm] (l2) {};
  \path (l0) edge  node[left]{$a_1$} (l1)
             edge  node[right]{$a_2$} (l2);
  \node [rounded corners,inner xsep=5mm,fit=(l0)(l1)(l2),draw,thick] (rec1) {};
  \node [rectnodes={2.6,2.3}, right=2cm of rec1] (rec2) {};
  
  \node [dots,label=-90:$a_1$] (a) at ($(rec1.north west)!0.25!(rec1.north east)$) {};
  \node [dots,label=-90:$a_2$] (b) at ($(rec1.north west)!0.75!(rec1.north east)$) {};
  
  \node [dots,label=-90:$a$] (c) at ($(rec2.north west)!0.5!(rec2.north east)$) {};

  \path (a) +(0,5mm) coordinate (a2) +(-5mm,5mm) coordinate(a3);
  \path (b) +(0,5mm) coordinate (b2) +(5mm,5mm) coordinate(b3);

  \path (c) +(5mm,5mm) coordinate (c2) +(-5mm,5mm) coordinate(c3) +(10mm,5mm) coordinate(c4)
+(-10mm,5mm) coordinate(c5);

  \draw[thick,-] (a) -- (a2); 
  \draw[thick,-] (a2) -- node[above]{$\alpha_1$}(a3); 
  \draw[thick,-] (b) -- (b2); 
  \draw[thick,-] (b2) -- node[above]{$\alpha_2$}(b3); 
  \draw[thick,-] (c) -- (c2); 
  \draw[thick,-] (c) -- (c3); 
  \draw[thick,-] (c2) -- node [above]{$\alpha_2$}(c4); 
  \draw[thick,-] (c3) -- node [above]{$\alpha_1$} (c5); 

\end{tikzpicture}
\caption{Potential Conflict Between Interactions $\alpha_1$ and $\alpha_2$}
\label{fig:pconf}
\end{figure}


In order to avoid a centralized scheduling and to introduce concurrency between interactions
execution, we propose to decentralize the scheduling layer into several schedulers each one
responsible of scheduling a subset of interaction. The purpose behind this practice is: 
\emph{(i)} to spread the workload across concurrent schedulers and
as much as possible independently, and \emph{(ii)} to map schedulers as close as possible to 
the components that they concretely handle (with respect to the corresponding subset of 
interactions), which brings back the communication overhead between components to the same 
magnitude. Our work does not address interaction partitioning, nonetheless it is a crucial 
concern for load-balancing and for tuning the system to achieve a desired level of performance. 

\begin{definition}[Interaction Partition]\label{def:inter_part}
  Given an interaction set $\gamma$, a partition of $\gamma$ is a set of subset 
  $\{\gamma_k\}_{k=1}^m$ such that $\gamma=\gamma_1\cup\cdots\cup\gamma_m$ and 
  $\forall i,j\in\{1,\cdots,m\}$ such that $i\neq j$, $\gamma_i\cap\gamma_j=\emptyset$.
\end{definition}

Decentralizing the schedulers generates situational conflict between interactions, that is, 
if two interactions handled in separate schedulers (from two class of the interactions partition)
are potentially conflicting, they cannot execute in parallel. We call such interactions,
\emph{externally conflicting} interactions. 
A simple solution to resolve such conflicts is to enforce
a \emph{conflict-free} partitioning of interactions. In spite of that, this solution will 
restrict the choice for distributing interactions across schedulers. Thus, another method
~\cite{} is to incorporate a third layer that will arbiter the execution of potentially 
conflicting interactions. The latter can be represented using a tiers component realizing 
a \emph{conflict resolution protocol} (CRP). This
component implements an algorithm based on the idea of message-counting technique~\cite{}.
This technique is based on counting the number of times that a component participates in an 
interaction. Conflicts are then resolved by ensuring that each participation number is used 
only once, which is achieved by counting the number of the interaction offer for each 
component. Then, conflicts are simply resolved by comparing the offer numbers of participating
components with the numbers of their last execution.
On the other hand, conflicts raised from interactions of the same class, that is, handled by 
the same scheduler, are resolved locally by the latter.
\begin{example}\label{exp:partition}
  Let us consider example of Figure~\ref{fig:tm}. For the interaction set $\gamma=\{\alpha_1,
  \cdots,\alpha_8\}$, let $\gamma_1=\{\alpha_{2\times i-1}\}_{i=1}^4\cup\gamma_2=
  \{\alpha_{2\times i}\}_{i=1}^4$ be an interaction partition. Then from 
  Definition~\ref{def:pconf} the set of potentially conflicting interactions two-by-two is:
  $\{(\alpha_1,\alpha_2);(\alpha_3,\alpha_4);(\alpha_5,\alpha_6);$\\$(\alpha_7,\alpha_8)\}$.
  This mean that the set of conflicting interactions of the whole system is $\gamma$.
\end{example}

\section{3-Layer Send/Receive Model}\label{sec:3.2}
Let $S=\gamma(B_1,\cdots,B_n)$ be a timed system. Given a partition of interaction
$\{\gamma_k\}^m_{k=1}$, the Send/Receive model corresponding to $S$ is based on the three
following layers:
\begin{itemize}
  \item The \emph{Distributed Component Layer} consists of a transformation of timed components
    $B_i$ into Send/Receive components $B_i^{SR}$ that send, asynchronously, offer messages 
    enclosing their current state to the scheduling layer 
  \item The \emph{Scheduling Layer} is responsible of interaction executions. Based on offers 
    received form the Send/Receive components, it may decide or not to execute an interaction.
    In case of conflicts, the scheduling layer rely on the conflict resolution layer to grant
    or deny the execution of an interaction
  \item The \emph{Conflict Resolution Layer} resolves conflicts between interaction based on
    the idea of message-count technique
\end{itemize}

\begin{figure*}[h!]
\centering
  \captionsetup{justification=centering}
  \shorthandoff{:!}
\begin{tikzpicture}
  \node [rectnode={1.1,1.1},label={[shift={(0,-1.1)}]$T_1^{SR}$}] (t1) {};
  \node [rectnode={1.1,1.1},right=10mm of t1,label={[shift={(0,-1.1)}]$C^{SR}$}] (c) {};
  \node [rectnode={1.1,1.1},right=7.5mm of c,label={[shift={(0,-1.1)}]$R^{SR}$}] (r) {};
  \node [rectnode={1.1,1.1},right=10mm of r,label={[shift={(0,-1.1)}]$T_2^{SR}$}] (t2) {};

  \node [rectnode={4,2},above=1.5cm of t1,xshift=8mm] (sch1) {$Sch_1\{\small{\alpha_1,\alpha_3,\alpha_5,\alpha_7}\}$};
  \node [rectnode={4,2},above=1.5cm of t2,xshift=-8mm] (sch2) {$Sch_2\{\small{\alpha_2,\alpha_4,\alpha_6,\alpha_8}\}$};
  \node [rectnode={8.5,2},above=5mm of sch1,xshift=22.5mm,label={[shift={(0,-10mm)}]$Conflict$ $Resolution$}] (crp) {};
  
  
  \node [triangle,label=-90:\small{$o_1$}] (ot1) at ($(t1.north west)!0.25!(t1.north east)$) {};
    \node [dots,label=-90:\small{$n_1$}] (nt1) at ($(t1.north west)!0.75!(t1.north east)$) {};
  
    \node [triangle,label=-90:\small{$o_2$}] (oc) at ($(c.north west)!0.25!(c.north east)$) {};
    \node [dots,label=-90:\small{$n_2$}] (nc) at ($(c.north west)!0.75!(c.north east)$) {};
  
    \node [triangle,label=-90:\small{$o_3$}] (or) at ($(r.north west)!0.25!(r.north east)$) {};
    \node [dots,label=-90:\small{$n_3$}] (nr) at ($(r.north west)!0.75!(r.north east)$) {};
  
    \node [triangle,label=-90:\small{$o_4$}] (ot2) at ($(t2.north west)!0.25!(t2.north east)$) {};
    \node [dots,label=-90:\small{$n_4$}] (nt2) at ($(t2.north west)!0.75!(t2.north east)$) {};

  %sch1
    %\node [dots,label=90:\small{$o_1$}] (s1ot1) at ($(sch1.south west)!0.15!(sch1.south east)$) {};
    %\node [triangle,label=-90:\small{$n_1$},rotate=180] (s1nt1) at ($(sch1.south west)!0.25!(sch1.south east)$) {};
    \node [dots,label=90:\small{$o_1$},above=1.25cm of ot1] (s1ot1)  {};
    \node [triangle,label=-90:\small{$n_1$},rotate=180,above=1.5cm of nt1] (s1nt1) {};
  
    \node [dots,label=90:\small{$o_2$}] (s1oc) at ($(sch1.south west)!0.5!(sch1.south east)$) {};
    \node [triangle,label=-90:\small{$n_2$},rotate=180] (s1nc) at ($(sch1.south west)!0.6!(sch1.south east)$) {};
  
    \node [dots,label=90:\small{$o_3$}] (s1or) at ($(sch1.south west)!0.75!(sch1.south east)$) {};
    \node [triangle,label=-90:\small{$n_3$},rotate=180] (s1nr) at ($(sch1.south west)!0.85!(sch1.south east)$) {};
  
  \node [dots,label=-90:\small{$ok_1$}] (ok1) at ($(sch1.north west)!0.25!(sch1.north east)$) {};
  \node [dots,label=-90:\small{$fail_1$}] (fail1) at ($(sch1.north west)!0.5!(sch1.north east)$) {};
  \node [triangle,label=-90:\small{$req_1$}] (res1) at ($(sch1.north west)!0.75!(sch1.north east)$) {};
  %crp1
  \node [triangle,label=-90:\small{$ok_1$},rotate=180,above=5mm of ok1] (ok1')  {};
  \node [triangle,label=-90:\small{$fail_1$},rotate=180,above=5mm of fail1] (fail1') {};
  \node [dots,label=90:\small{$req_1$},above=2.5mm of res1] (res1') {};
  %sch2
  \node [dots,label=90:\small{$o_2$}] (s2oc) at ($(sch2.south west)!0.15!(sch2.south east)$) {};
    \node [triangle,label=-90:\small{$n_2$},rotate=180] (s2nc) at ($(sch2.south west)!0.25!(sch2.south east)$) {};
  
    \node [dots,label=90:\small{$o_3$}] (s2or) at ($(sch2.south west)!0.4!(sch2.south east)$) {};
    \node [triangle,label=-90:\small{$n_3$},rotate=180] (s2nr) at ($(sch2.south west)!0.5!(sch2.south east)$) {};
  
    \node [dots,label=90:\small{$o_4$},above=1.25cm of ot2] (s2ot2)  {};
    \node [triangle,label=-90:\small{$n_4$},rotate=180,above=1.5cm of nt2] (s2nt2) {};
    %\node [dots,label=90:\small{$o_4$}] (s2ot2) at ($(sch2.south west)!0.75!(sch2.south east)$) {};
    %\node [triangle,label=-90:\small{$n_4$},rotate=180] (s2nt2) at ($(sch2.south west)!0.85!(sch2.south east)$) {};

  \node [dots,label=-90:\small{$ok_2$}] (ok2) at ($(sch2.north west)!0.25!(sch2.north east)$) {};
  \node [dots,label=-90:\small{$fail_2$}] (fail2) at ($(sch2.north west)!0.5!(sch2.north east)$) {};
  \node [triangle,label=-90:\small{$res_2$}] (res2) at ($(sch2.north west)!0.75!(sch2.north east)$) {};
  %crp2
  \node [triangle,label=-90:\small{$ok_2$},rotate=180,above=5mm of ok2] (ok2')  {};
  \node [triangle,label=-90:\small{$fail_2$},rotate=180,above=5mm of fail2] (fail2') {};
  \node [dots,label=90:\small{$res_2$},above=2.5mm of res2] (res2') {};
  
  
  %connection
  \draw[thick,-,red] (res1) -- (res1');
  \draw[thick,-,dgreen] (ok1) -- (ok1');
  \draw[thick,-,dgreen] (fail1) -- (fail1');
  
  \draw[thick,-,red] (res2) -- (res2');
  \draw[thick,-,dgreen] (ok2) -- (ok2');
  \draw[thick,-,dgreen] (fail2) -- (fail2');
  
  
  \draw[thick,-,red] (ot1) -- (s1ot1);
  \draw[thick,-,dgreen] (s1nt1) -- (nt1);
  \draw[thick,-,red] (ot2) -- (s2ot2);
  \draw[thick,-,dgreen] (s2nt2) -- (nt2);

  
  \path (oc) +(0,3.5mm) coordinate (c0);
  \draw[-,red,thick,name path=line1] (s1oc) |- (c0) -- (oc) -- (c0)-|(s2oc);

  \path (nc) +(0,5.5mm) coordinate (nc0);
  \path[-,name path=line2] (s1nc) |- (nc0) -- (nc) -- (nc0)-|(s2nc);
 
  
  \path[name intersections={of=line1 and line2, by={a,b,c}}];% here
  \coordinate (aux) at (nc0-|s2nc);
  \draw[thick,-,dgreen] (s1nc) |- (nc0);
  \draw[thick,-,dgreen,connect=(nc0) to (nc) over (a) by 2pt];
  \draw[thick,-,dgreen,connect=(nc0) to (aux) over (c) by 2pt];
  \draw[thick,-,dgreen] (aux)--(s2nc);
   
  
  \path (or) +(0,7.5mm) coordinate (or0) +(0,4.5mm) coordinate(or1) +(5mm,7.5mm) coordinate(or2);
  \path[-,name path=line3](s1or)|-(or0)--(or)--(or0)-|(s2or);
  \path[name intersections={of=line3 and line2, by={a,b,c}}];% here
  \path[name intersections={of=line3 and line1, by={d,e,f}}];% here
  \coordinate (aux) at (or0-|s2or);
  \draw[thick,-,red] (s1or) |- (or0);
  \draw[thick,-,red,connect=(or0) to (or1) over (a) by 2pt];
  \draw[thick,-,red,connect=(or1) to (or) over (d) by 2pt];
  \draw[thick,-,red,connect=(or0) to (or2) over (f) by 2pt];
  \draw[thick,-,red,connect=(or2) to (aux) over (c) by 2pt];
  \draw[thick,red,-] (aux)--(s2or); 

  \path (nr) +(0,9.5mm) coordinate (nr0) +(0,6.5mm) coordinate(nr1) +(5mm,9.5mm) coordinate(nr2)
 +(-5mm,9.5mm) coordinate(nr3);
  \path[-,name path=line4](s1nr)|-(nr0)--(nr)--(nr0)-|(s2nr);
  \path[name intersections={of=line4 and line1, by={a}}];% here
  \path[name intersections={of=line4 and line2, by={b,c,d}}];% here
  \path[name intersections={of=line4 and line3, by={e,f,g}}];% here
  \coordinate (aux) at (nr0-|s2nr);
  \draw[thick,-,dgreen] (s1nr) |- (nr3);
  \draw[thick,-,dgreen,connect=(nr3) to (nr0) over (a) by 2pt];
  \draw[thick,-,dgreen,connect=(nr0) to (nr1) over (e) by 2pt];
  \draw[thick,-,dgreen,connect=(nr1) to (nr) over (b) by 2pt];
  \draw[thick,-,dgreen,connect=(nr0) to (nr2) over (d) by 2pt];
  \draw[thick,-,dgreen,connect=(nr2) to (aux) over (g) by 2pt];
  \draw[thick,-,dgreen] (s2nr) -- (aux);




\end{tikzpicture} 
\caption{High Level Representation of a Decentralized Send-Receive Model of the Task Manager Example}
\label{fig:sr}
\end{figure*}

\begin{example}
  Figure~\ref{fig:sr} describes a Send/Receive model with a decentralized scheduling. 
  The set of interactions is partitioned into two classes 
  $\gamma_1=\{\alpha_1,\alpha_3,\alpha_5,\alpha_7\}$ and
  $\gamma_2=\{\alpha_2,\alpha_4,\alpha_6,\alpha_8\}$, each one handled by a scheduler ($Sch_1$
  and $Sch_2$ respectively). Since $\gamma_1$ and $\gamma_2$ are conflicting, for instance
  $\alpha_1$ and $\alpha_2$ are potentially conflicting, schedulers rely on the conflict
  resolution layer to resolve the conflicts. In this case, they emit a request ($req_k$ with
  $k\in\{1,2\}$) and wait for a notification granting (respectively denying) them the execution
  of an interaction ($ok_k$ respectively $fail_k$).
  Notice that components $C^{SR}$ and $R^{SR}$ send offers to both schedulers since
  they are participating in interactions handled in both schedulers.
\end{example}
\subsection{Send/Receive Components}
\begin{figure}[!h]
 \centering
  \begin{tikzpicture}[every node/.style={scale=0.8},scale=0.8]

  \node [place,label=left:\textcolor{red}{$x\le4$}](l0) {$\loc$};
  \node [place,below=1.3cm of l0,xshift=1cm] (l1) {\phantom{$\loc$}};
  \node [place,below=1.3cm of l0,xshift=-1cm] (l2) {\phantom{$\loc$}};

  \path (l0) edge [bend left] node[right,align=center]{$a$}(l1)
             edge [bend right] node[left,align=center]{$b$} (l2);
  
  \node [place,right=5cm of l0,yshift=1cm](l0) {\scriptsize $\locp$};
  \draw [-] ($(l0) + (180:4mm)$) arc (180:360:4mm);
  \node [place,label=left:\textcolor{red}{$x\le4$},below=1cm of l0](l1) {$\loc$};
  \node [place,below=1cm of l1,xshift=1cm] (l2) {\phantom{$\loc$}};
  \node [place,below=1cm of l1,xshift=-1cm] (l3) {\phantom{$\loc$}};

  \path (l0) edge node[right,align=center]{$o$}(l1)
        (l1) edge [bend left] node[right,align=center]{$a$} (l2)
        (l1) edge [bend right] node[left,align=center]{$b$} (l3);
\end{tikzpicture}
  \caption{Offer Construction}
 \label{fig:offer}
\end{figure}  



The transformation of a timed component $B$ into a Send/Receive component $B^{SR}$ relies 
on decomposing each transition of $B$ into two transitions: \emph{(1)} an offer (send)
transition and \emph{(2)} a notification (receive) transition. This is done by splitting
each location $\loc$ into two locations, $\loc$ itself and $\locp$ as shown in 
Figure~\ref{fig:offer}.

When at $\locp$ location, the Send/Receive component is not in a stable state and is able only
to send an offer to its respective scheduler(s). We require that offers are sent as soon as 
possible meaning that there is no delay when at location $\locp$. We call such location
\emph{urgent} location, graphically represented by a $\smile$ inside the location. 
From a semantics point of view, an urgent location is equivalent to adding an extra clock
$x$ that is reset on all incoming edges, and having an invariant $x\le0$ on the location 
as illustrated in Figure~\ref{fig:urgloc}. Thus, time is not allowed to pass when the system 
is in such locations.  
An offer contains the exact variables encoding the current state of a component. It includes
the following variables:
\begin{itemize}
  \item An invariant variable of the current location invariant 
  \item A guard variable for each action (port), that is set to the guard (over clocks and data)
    of each port if it exists from the current location, otherwise to $\false$.
  \item A \emph{Boolean} variable indicating whether the next transition reset clocks or not 
  \item A \emph{participation number} variable used for conflict resolution
\end{itemize}
\begin{figure}[!h]
 \centering
  \begin{tikzpicture}[scale=0.8,every node/.style={scale=0.8}]


  \node [place] (l1) {$\loc_1$};
  \node [place,below=1.5cm of l1,label=left:\textcolor{red}{$x\le0$}](l2) {$\loc_2$};

  \path (l1) edge node[right,align=center]{$a$\\$x:=0$}(l2);
  
  \node [place,right=4cm of l1] (l1) {$\loc_1$};
  \node [place,below=1.5cm of l1](l2) {$\loc_2$};
  \draw [-] ($(l2) + (180:4mm)$) arc (180:360:4mm);
 % \node [place,label=left:\textcolor{red}{$x\le4$},below=1cm of l0](l1) {$\loc$};
%  \node [place,below=1cm of l1,xshift=1cm] (l2) {\phantom{$\loc$}};
%  \node [place,below=1cm of l1,xshift=-1cm] (l3) {\phantom{$\loc$}};
  \path (l1) edge node[right,align=center]{$a$}(l2);

\end{tikzpicture}
  \caption{Representation of an Urgent Locatoion}
 \label{fig:urgloc}
\end{figure}  



\begin{definition}[Send/Receive Component]\label{def:tc_SR}
  Let $B=(\Loc,\loc_{0},\X,\D,\A,\E,\{f_e\}_{e\in\E},\I)$ be a timed component. The 
  corresponding Send/Receive component is defined by the timed component
  $B^{SR}=(\Loc^{SR},\loc_0^{SR},\emptyset,\D^{SR},\mc{P}^{SR},\E^{SR},\{f_e\}_{e\in\E^{SR}},
  \emptyset)$, such that:
  \begin{itemize}
    \item $\Loc^{SR}=\Loc\cup\Loc^{\perp}$, where $\Loc^{\perp}=\{\locp|\loc\in\Loc\}$.
      Locations of $\Loc^{\perp}$ are urgent locations.
    \item $\loc_0^{SR}=\loc_{0_{\perp}}\in\Loc^{\perp}$ is the initial location.
    \item $\mc{P}^{SR}= P\cup\{o\}$, where $P=\{p_a|a\in\mc{A}\}$ is the set of ports for each
      action of $B$ and $o$ is the offer port. 
    \item $\mc{D}^{SR}=\mc{D}\cup\{g_{p_a}\}_{a\in\mc{A}}\cup\mc{I}_{B}\cup\{r_x\}_{x\in\mc{X}}
      \cup\{n_B\}$,
      where $g_{p_a}$ are guard variables, $\mc{I}_{B}$ is an invariant variable, $r_x$ are 
      Boolean reset variables and $n_B$ is a participation number variable.
      Variables $\mc{D}_o^{SR}i\subset\mc{D}^{SR}=\{g_{p_a}\}_{a\in\mc{A}}\cup\mc{I}_{B}\cup
      \{n_B\}\cup\{r_x\}_{x\in\mc{X}}$  are exported by the offer port.
    \item For each place $\loc\in\Loc$, we include an offer transition $e_{\loc}=(\locp,o,\true,
      \emptyset,\loc)$ in $\mc{E}^{SR}$
    \item For each transition $e=(\loc,a,g,r,\loc')\in\mc{E}$, we include a notification 
      transition $e_{p_a}=(\loc,p_a,\true,\emptyset,\locp')$. The transfer function $f_{e_{p_a}}$
      applies the original transfer function $f_e$ of e, then update guard variables, 
      invariant variable, reset variables and the participation number as follows:
      \begin{itemize}
        \item $\forall a'\in\mc{A}, g_{p_{a'}}:=\begin{cases}
            g_{a'} & \text{if } e'=(\loc',a',g_{a'},r,\loc'')\in\mc{E}\\
          \false & otherwise
        \end{cases}$
        \item $\mc{I}_B:=\Inv{\loc'}$
        \item $\forall x\in\mc{X}, r_x:=\begin{cases}
            \true & \text{if } x\in r\\
          \false & otherwise
        \end{cases}$
        \item $n_B:=n_B+1$
      \end{itemize}
  \end{itemize}
\end{definition}

This definition of Send/Receive component relates the execution of a transition 
$e=(\loc,a,g,r,\loc')\in\mc{E}$ from the initial component $B$ to the following two 
execution steps in $B^{SR}$. First, an offer transition $e_{\loc}=(\locp,o,\true,\emptyset,\loc)$
sends for each port $p\in\mc{P}$ the guard over clocks and data corresponding to the enabledness
of $p$ at $\loc$, the location invariant $\Inv{\loc}$, the participation number $n_B$ for
component $B$, as well as the reset variable $r_x$ for each clock $x\in\mc{X}$, such that,
$r_x=\true$, if $x$ has been reset by the previous transition execution. Reset variables
$r_x$ are used to reset clocks in the Scheduling layer before computing guard of interactions.
In the second place, a notification transition $e_{p_a}=(\loc,p_a,\true,\emptyset,\locp')$  
is executed upon the execution of an interaction involving $p_a$ in the scheduling layer. 
In the same manner to $e$ in $B$, $e_{p_a}$ updates values of variables $\mc{D}$ according to
to the transfer function $f_e$, as well as variables needed for the next offer. 
Figure~\ref{fig:tcSR} depicts the Send/Receive transformation of the component $C$ of Example
~\ref{fig:tm}.
\begin{figure}[!h]
 \centering
  \shorthandoff{:!}
\begin{tikzpicture}[scale=0.8,every node/.style={scale=0.8}]

  \node [accepting, place] (l0b)  {$\locpb{0}{1}$};
  \draw [-] ($(l0b) + (180:4mm)$) arc (180:360:4mm);
  \node [place,right=1.5cm of l0b] (l0)  {$\loc_0^1$};
  \node [place,below=2cm of l0] (l1b)  {$\locpb{1}{1}$};
  \draw [-] ($(l1b) + (180:4mm)$) arc (180:360:4mm);
  \node [place,left=1.5cm of l1b] (l1)  {$\loc_1^1$};
  
  \path (l0b) edge node[above,align=center]{$o$}(l0)
        (l0) edge node[right,align=center,xshift=1.4cm,yshift=1cm](init){$init_0$}(l1b)
        (l1b) edge node[below,align=center]{$o$}(l1)
        (l1) edge node[left,align=center,xshift=-1.4cm,yshift=1cm](start){$start_0$}(l0b);
  \node [align=center,below=0cm of start,rectangle,fill=gray!30](u1) {$g_{init_0}:=[z>25]$\\
                        $g_{start_0}:=\false$\\
                        $r_z:=\true$\\
                        $\mc{I}_{C^{SR}}=\true$};
  \node [align=center,below=0cm of init,rectangle,fill=gray!30](u2) {$g_{init_0}:=\false$\\
                        $g_{start_0}:=\true$\\
                        $r_z:=\false$\\
                        $\mc{I}_{C^{SR}}=\true$};

  \node [rounded corners,inner xsep=11mm,inner ysep=11mm,draw, fit=(l0)(l1)(l0b)(l1b)(u1)(u2)] (rec1) {};
  \node [triangle,label=-90:$o$] (o) at ($(rec1.north west)!0.25!(rec1.north east)$) {};
  \node [dots,label=-90:$init_0$] (n1) at ($(rec1.north west)!0.75!(rec1.north east)$) {};
  \node [dots,label=-90:$start_0$] (n2) at ($(rec1.north west)!0.85!(rec1.north east)$) {};
\end{tikzpicture}
  \caption{Send/Receive Transformation of the Controller Component From Figure~\ref{fig:tm}}
 \label{fig:tcSR}
\end{figure}  





\subsection{Scheduling Layer}
As explained earlier, the scheduling layer works with a partial view of the global state of 
the system. Initially, every scheduler is waiting for offers form the corresponding components
(with respect to the interactions partition). Each received offer specifies to the schedulers
the state of the sender component. In what follows, we consider work-conservative scheduler,
that is, schedulers preserving the execution sequences of the initial model under
the standard semantics. For the sake of distributed implementation, it is
worth taking a decision as soon as possible. Thus, once a scheduler gathers enough information 
for scheduling interactions, it arbitrarily choose one and executes the corresponding 
transition that will trigger the notification responses to the components involved in 
that interaction.
In what follows, we use Petri nets~\cite{} to describe the scheduling layer. 
Petri nets are a well suited formalism for encoding parallel and concurrent executions. 
Particularly, we focus on a class of Petri nets (\emph{1-Safe}) to encode the structure of 
the schedulers introduced by our method since it is proven that any 1-Safe Petri net can be 
transformed in an equivalent automaton~\cite{}. 
Consequently, they provide a clear and compact representation 
of the scheduling layer.

\subsubsection{Petri Nets}
\begin{definition}[Petri Net]\label{def:pn}
  A Petri net is a 3-tuple $\mc{P}=(\Loc,\mc{A},\mc{T})$ where $\Loc$ is a set of finite
  \emph{places}, $\mc{A}$ is a finite set of actions, and $\mc{T}\subseteq 2^{\Loc}\times
  \mc{A}\times 2^{\Loc}$ is a set of transitions. A transition $\tau$ is a triple 
  $(^\bullet\tau,a, \tau^{\bullet})$, where $^\bullet\tau$ is the set of input places of 
  $\tau$ and $\tau^{\bullet}$ is 
  the set of output places of $\tau$.
\end{definition}

A Petri net can be represented as directed bipartite graph $\mc{G}=(\mc{V},\mc{E})$ where 
$\mc{V}$ denotes the set of vertices and $\mc{E}$ denotes the set of directed edges. 
The set of vertices is structured into two classes, namely places and transitions, 
that is, $\mc{V}=\Loc\cup\mc{T}$. 
Places are represented by circular vertices and transitions are represented by rectangular 
vertices as shown in Figure~\ref{fig:pn}. 
The set of directed edges $\mc{E}$ is the union of the sets $\{(\loc,\tau)\in\Loc\times\mc{T}|
\loc\in{^\bullet\tau}\}$ and $\{(\tau,\loc)\in\mc{T}\times\Loc|\loc\in\tau^\bullet\}$.
A \emph{marking} of a Petri net is a mapping $m:\Loc\to\naturals$ that describes the current
\emph{state} of a Petri net by assigning a non-negative integer to each of its places.
We use tokens~\cite{} to represent the marking (number of tokens). We say that a place is
\emph{marked} if it contains at least one token.
For a transition $\tau$, we say that $\tau$ is \emph{enabled} at a given state if all of 
its input places $^\bullet\tau$ are marked, that is, $\forall\loc\in{^\bullet\tau},\ m(\loc)>0$.
A \emph{firing} (execution) of an enabled transition removes one token from each input place
and adds one token to each output place. Formally, the firing of a transition from a marking 
$m$ results in a marking $m'$ such that:
\begin{displaymath}
  \forall\loc\in\Loc, \ m'(\loc)=m(\loc)-\tau^{-}(\loc)+\tau^{+}(\loc) 
\end{displaymath}
where
\begin{displaymath}
  \tau^{-}(\loc)=\begin{cases}
    1 & \text{if }\loc\in{^\bullet\tau}\\
    0 & \text{otherwise} 
  \end{cases}
 \quad \text{and}\quad
\tau^{+}(\loc)=\begin{cases}
    1 & \text{if }\loc\in{\tau^\bullet }\\
    0 & \text{otherwise} 
\end{cases}\end{displaymath}

We put $m\transitb{a}{\mc{P}}m'$ to denote that a transition $\tau=({^\bullet\tau},a,
  \tau^\bullet)$ can be executed at marking $m$ and reaches marking $m'$. 
  We also denote by $\transitb{}{\mc{P}}$
the set of triples $(m,a,m')$ such that $m\transitb{a}{\mc{P}}m'$.
  \begin{figure}[!h]
 \centering
  \begin{tikzpicture}[scale=0.8,every node/.style={scale=0.8}]

  \node [markplace](l0) {$\loc_0$};
  \node [place,right=1.5cm of l0,yshift=1.5cm] (l1) {$\loc_1$};
  \node [place,right=1.5cm of l0,yshift=-1.5cm] (l2) {$\loc_2$};
  \node [transition,right=7.5mm of l0,yshift=-4mm,rotate=-90,label=below:$\tau_2$] (t1) {};
  \node [transition,right=7.5mm of l0,yshift=4mm,rotate=90,label=above:$\tau_1$] (t2) {};
  \node [transition,above=5mm of l0,label=left:$\tau_3$] (t3) {};
  \node [transition,below=5mm of l0,label=left:$\tau_4$] (t4) {};

  \path (l0) edge (t1)
             edge (t2)
        (t2) edge (l1)
        (t1) edge (l2)
        (t3) edge (l0)
        (t4) edge (l0);
    \draw[-stealth,rc] (l1) -- ([h2n]t3.center) --(t3);
    \draw[-stealth,rc] (l2) -- ([h2s]t4.center) --(t4);
   % (l1) [bend right]edge (t3)
   %     (l2) [bend left]edge (t4);

  \node [place,right=5cm of l0](l0) {$\loc_0$};
  \node [markplace,right=1.5cm of l0,yshift=1.5cm] (l1) {$\loc_1$};
  \node [place,right=1.5cm of l0,yshift=-1.5cm] (l2) {$\loc_2$};
  \node [transition,right=7.5mm of l0,yshift=-4mm,rotate=-90,label=below:$\tau_2$] (t1) {};
  \node [transition,right=7.5mm of l0,yshift=4mm,rotate=90,label=above:$\tau_1$] (t2) {};
  \node [transition,above=5mm of l0,label=left:$\tau_3$] (t3) {};
  \node [transition,below=5mm of l0,label=left:$\tau_4$] (t4) {};
  \path (l0) edge (t1)
             edge (t2)
        (t2) edge (l1)
        (t1) edge (l2)
        (t3) edge (l0)
        (t4) edge (l0);
    \draw[-stealth,rc] (l1) -- ([h2n]t3.center) --(t3);
    \draw[-stealth,rc] (l2) -- ([h2s]t4.center) --(t4);

\end{tikzpicture}
 \caption{A Simple Petri Net with Two Succesive Markins}
 \label{fig:pn}
\end{figure}  




\begin{example}
  Figure~\ref{fig:pn} depicts an example of a Petri net with two successive marking. It includes
  three places $\{\loc_1,\cdots,\loc_3\}$ and four transitions $\{t_1,\cdots,t_4\}$. For clarity,
  places with tokens are represented with filled gray circles. The left side marking 
  shows the initial marking of the Petri net whereas the right side marking results from the
  execution of transition $t_1$.
\end{example}

Given a Petri net $\mc{P}=(\Loc,\mc{A},\mc{T})$ and an initial marking $m_0$, the marking $m$
is \emph{reachable} if there exists a sequence of transitions $m_0\transitb{a_1}{\mc{P}}m_1
\transitb{a_1}{\mc{P}}\cdots\transitb{a_1}{\mc{P}}m$. We say that $\mc{P}$ is one \emph{1-Safe}
if there is at most one token per place in each reachable marking. This implies at most
$2^{|\Loc|}$ markings. In this thesis, we consider only this class of Petri nets.
The behavior of a 1-Safe Petri net  $\mc{P}=(\Loc,\mc{A},\mc{T})$ is defined by the finite
labeled transition system $(2^{\Loc},\mc{A},\to_{\mc{P}})$, where $2^{\Loc}$ is the set of
states, $\mc{A}$ is the set of actions, and $\to_{\mc{P}}\subseteq 2^{\Loc}\times\mc{A}\times
2^{\Loc}$ is the set of transitions defined as follows. We have $(m,a,m')\in\to_{\mc{P}}$, 
denoted by $m\transitb{a}{\mc{P}}m'$, if there exists $\tau=({^\bullet\tau},a,{\tau^\bullet })
  \in\mc{T}$ such that ${^\bullet\tau}\subseteq m$ and $m'=(m\textbackslash{^\bullet\tau})
  \cup{\tau^\bullet }$.
In this case, we say that $a$ is enabled at $m$.

\subsubsection{Building Schedulers}
  Given a timed system $\gamma(B_1,\cdots,B_n)$ and a partition of interactions 
  $\{\gamma_j\}^m_{j=1}$, each class of the interactions partition is handled by a single
  scheduler component, namely $Sch_j$.
  The behavior of each scheduler is described as a 1-Safe Petri net in which there is a token
  for each component flowing between three or four different types of places as shown
  in Figure~\ref{fig:schSR}:
  \begin{figure}[!h]
 \centering
  \captionsetup{justification=centering}
\subfloat[Scheduling Mechanism for Internally or not Conflicting Interactions]{\label{fig:sch1}
  \begin{tikzpicture}[scale=0.7,every node/.style={scale=0.7}]
  
  \node [markplace] (w1) {$w_i$};
  \node [transition,below=1cm of w1](o1) {};
  \node [place,below=1cm of o1] (r1){$r_i$};
  \node [right=2cm of r1] (r2){};
  \node [transition,below=1cm of r1,xshift=1.5cm](a) {};
  \node [place,below=2cm of r1](sp){$s_{i_p}$};
  \node [right=2cm of sp](spp){};
  \node [transition,left=5mm of r1](s1) {};

  \path (w1) edge node[right,yshift=-3mm]{receive offer}  (o1)
        (o1) edge node[right,yshift=3mm]{from component $B_i$}(r1)
        (r1) edge node[right,xshift=1.2cm,yshift=-4mm]{execute interaction}(a)
        (a) edge node[right,xshift=1.7cm,yshift=4mm]{involving $B_i$}(sp);

  \draw[->,dashed] (r2) -- (a);
  \draw[->,dashed] (a) -- (spp);
  \draw[->,rounded corners=4mm] (sp)--([sh2sw2]sp.center) -- (s1);
  \draw[->,rounded corners=4mm] (s1) --([sh2w2]w1.center)--(w1);
  \node[left=0cm of s1,align=center]{send notification\\to $B_i$ on port $p$};
\end{tikzpicture}}
\subfloat[Scheduling Mechanism for Externally Conflicting Interactions]{\label{fig:sch2}
  \begin{tikzpicture}[scale=0.7,every node/.style={scale=0.7}]
  
  \node [markplace] (w1) {$w_i$};
  \node [transition,below=5mm of w1](o1) {};
  \node [place,below=5mm of o1] (r1){$r_i$};
  \node [right=2cm of r1] (r2){};
  \node [transition,below=5mm of r1,xshift=1.5cm](a) {};
  \node [transition,right=30mm of a](f) {};
  \node [place,below=5mm of a](t){$t_a$};  
  \node [transition,below=5mm of t](aa) {};
  \node [place,below=5mm of aa,xshift=-1.5cm](sp){$s_{p_i}$};
  \node [right=2cm of sp](spp){};
  \node [left=2mm of f,yshift=11mm](r3){};
  \node [transition,left=18mm of a](s1) {};

  \path (w1) edge (o1)
        (o1) edge (r1)
        (r1) edge node[right,align=center,xshift=1cm,yshift=-5mm]{sending request\\ to the CRP}(a)
        (r1) edge [loop,in=50,out=0,looseness=5] node[right]{new offer} (r1)
        (a) edge (t)
        (t) edge node[right,align=center,xshift=5mm,yshift=-10mm]
{CRP grants the execution\\ of the interaction \\involving $B_i$}(aa)
        (t) edge [loop,looseness=5, out=-100, in=-140 ](t)
        (t) edge [loop left,dashed] node[left,yshift=-3mm,xshift=3mm]{new offers} (t)
        (t) edge[bend right=50] (f)
        (f) edge[bend right=80] node[above,align=center,xshift=1cm]
{CRP denies the execution\\ of the interaction \\involving $B_i$} (r1)
        (f) edge[dashed,-,bend right=40] (r3)
        (aa) edge (sp);

  \draw[->,dashed] (r2) -- (a);
  \draw[->,dashed] (aa) -- (spp);
  \draw[->,rounded corners=4mm] (sp)--([sh2sw2]sp.center) -- (s1);
  \draw[->,rounded corners=4mm] (s1) --([sh2w2]w1.center)--(w1);
\end{tikzpicture}}
 \caption{Scheduling Mechanism}
 \label{fig:schSR}
\end{figure}  


  \begin{itemize}
    \item \emph{Waiting place}: For each component participating in an interaction handled by
      a scheduler, the corresponding scheduler include a waiting place signifying that 
      it is waiting for the component offer. Waiting places are labeled by $w$. 
    \item \emph{Receive place}: When receiving an offer from a component, the corresponding
      token is moved from the corresponding waiting place to the receive place (one received
      place per component) and stays there until an interaction including this component is 
      scheduled or requested for scheduling (through the conflict resolution layer). 
      Receive places are labeled by $r$.
    \item \emph{Try place}: Try places (labeled by $t$) 
      concern only components that are participating an interaction that is externally
      conflicting with another interaction (of another scheduler). 
      As explained in~\ref{sub:conf}, schedulers
      rely on the conflict resolution layer to resolve conflicts. For each externally
      conflicting interaction, a try place is inserted. When scheduling such interactions,
      tokens are moved from receive places of components to try place of that interaction, 
      meaning that a request has been sent to the CRP. Following this request, the CRP 
      either grants the execution
      of the interaction and the tokens are moved to the sending places, or denies the execution
      which results in moving the tokens back to the receive place.
      Moreover, loops for offer transitions are added on try places and receive places
      in order to take into account successive offers from components. 
    \item \emph{Sending place}: Once an interaction has been scheduled for execution,
      the corresponding components token are moved from receive places (or try place) to
      send places corresponding to ports of components participating in that interaction. 
      There is one send place for each port (excluding offer port) 
      for every Send/Receive component.
      Sending places are labeled by $s$.
  \end{itemize}
As illustrated in Figure~\ref{fig:schSR}, tokens are initially in waiting places. Once an offer
is received by a scheduler, the corresponding token moves to the receive place. The scheduler
copies then the values of the offer variables to its local variables. Once offer of all 
components involved in an interaction have been gathered, schedulers computes its guard. If 
the guard evaluates to $\true$, with respect to data variables and the global scheduler clock,
the scheduler can either execute the interaction if it is not externally conflicting with
another interaction (Figure~\ref{fig:sch1}). Tokens are then moved to send places of components
ports participating in that interaction. Otherwise (Figure~\ref{fig:sch2}), 
a request is sent to the CRP and the token is moved to the
corresponding $try$ place. Thereafter, either the CRP grants the execution of the interaction
and tokens are moved to send places, or the execution is denied and tokens are moved back
to receive places. Eventually, the scheduler may receive new offers when being in $try$
places. This corresponds to the execution of a conflicting interaction in another scheduler.


\begin{definition}[Scheduler]\label{def:sch_sr}
  Let $\gamma(B_1,\cdots,B_n)$ be a timed system and $\gamma_j\subset\gamma$ be a subset of
  interactions. The corresponding scheduler $Sch_j$ responsible for executing interactions
  of $\gamma_j$ is defined by the tuple  
  $Sch_j=(\Loc_j, \mc{P}_j,\mc{T}_j,\mc{X}_j, \mc{D}_j,\{g_{\tau}\}_{\tau\in\mc{T}},
  \{r_{\tau}\}_{\tau\in\mc{T}_j},\{f_{\tau}\}_{\tau\in\mc{T}_j},\{\mc{I}_{\loc}\}_{\loc\in\Loc_j})
  $, where:
  \begin{itemize}
    \item $\mc{X}_j=\{t_j\}\cup\{z_j\}$ is the set of clocks of $Sch_j$, where $t_j$ is
      the global clock used for scheduling interactions of $\gamma_j$ (it is never reset)
      and $z_j$ is a clock used for internal constraints. 
    \item $\mc{D}_j$ is the set of variables containing:
      \begin{itemize}
        \item Variables updated whenever an offer from a component $B_i$ participating in 
          interactions of $\gamma_j$ is received. These variables consist of: an invariant
          variable $\mc{I}_{B_i}$ and a participation number $n_{B_i}$ for each $B_i$, 
          a guard variable $g_{p_a}$ for each action of $B_i$ involved in an interaction of 
          $\gamma_j$, and a Boolean reset variable for each clock of $B_i$.
        \item Reset time variables that stores the absolute time of the last reset of 
          each component clock $B_i$. For each clock $x$ of $B_i$ we include a reset time 
          variable $\rho_x$.
      \end{itemize}
    \item For each transition $\tau\in\mc{T}_j$, $g_{\tau}$ is a guard over 
      $\mc{X}_j$ and $\mc{D}_j$.
    \item For each transition $\tau\in\mc{T}_j$, $r_{\tau}$ is a reset function over $\X_j$.
    \item For each transition $\tau\in\mc{T}_j$, $f_{\tau}$ is a transfer function over$\D_j$.
    \item For each place $\loc\in\Loc_j$, $\mc{I}_{\loc}$ is an invariant over $\mc{X}_j$. 
    \item $(\Loc_j,\mc{P}_j,\mc{T}_j)$ is a 1-Safe Petri net defining 
      the structure of the scheduler such that:
      \begin{itemize}
        \item $\Loc_j$ is the set of places. It includes four types of places:
        \begin{itemize}
          \item For each component $B_i$ involved in interactions of $\gamma_j$, we include
            a waiting place $w_i^j$, a receive place $r_i^j$, where $\mc{I}_{w_i^j}=\true$
            and $\mc{I}_{r_i^j}$ is the invariant $\mc{I}_{B_i}$ expressed on $t_j$.
          \item For each action $a$ involved in interactions of $\gamma_j$, we include a sending
            place $s_{p_a}$, where $\mc{I}_{s_{p_a}}=z\le0$.
          \item For each interaction $\alpha\in\gamma_j$ that is externally conflicting 
            with another interaction, we include a try place $t_{\alpha}$ with 
            $\mc{I}_{t_{\alpha}}$ is the invariant $\mc{I}_{B_i}$ expressed on $t_j$.
        \end{itemize}
        \item $\mc{P}_j$ is the set of ports. It includes the following ports:
          \begin{itemize}
            \item For each component $B_i$ involved in interactions of $\gamma_j$, we include
              a receive port $o_i^j$. Each port $o_i^j$ is associated with the variables
              $g_{p_a}$ and $r_x$ for each action, respectively clock, of $B_i$, as well
              as the variable $\mc{I}_{B_i}$ and $n_i$.
            \item For each action $a$ involved in interactions of $\gamma_j$, we include 
              a send port $p_a$.
            \item For each interaction $\alpha\in\gamma_j$ that is externally conflicting with
              another interaction, we include a send port $rsv_{\alpha}$ (reservation port), 
              and receive ports $ok_{\alpha}$ (granted execution) and $fail_{\alpha}$ (denied 
              execution). The port $rsv_{\alpha}$ exports the variables $\{n_i\}_{B_i
              \in\p{\alpha}}$.
            \item For each interaction $\alpha\in\gamma_j$ that is internally or not conflicting
              with other interactions of $\gamma_j$, we include the unary port $\alpha$.
          \end{itemize}
        \item $\mc{T}_j$ is the set of transitions. It consists of the following:
          \begin{itemize}
            \item For each component $B_i$, $\mc{T}_j$ includes the offer transitions 
              $(w_i^j,o_i,r_i^j)$, $(r_i^j,o_i,r_i^j)$ and $\{(t_{\alpha},o_i,t_{\alpha})|
              B_i\in\p{\alpha}\}$ where $\alpha$ is an externally conflicting interaction.
              These transitions have no guards and no reset functions. 
              Their transfer functions update reset time 
              variables $\rho_x$ whenever $r_x=\true$, that is, $\rho_x:=g$. 
            \item For each action $a$ involved in interactions of $\gamma_j$, $\mc{T}_j$ includes
              a transition $(s_{p_a},p_a,w_i^j)$ where $i$ is the index of the component
              containing $a$. This transition notifies the corresponding Send/Receive component 
              to execute the transition labeled by $p_a$. It has no guard, no reset function,
              and no transfer function.
            \item For each interaction $\alpha=\{a_i\}_{i\in{I}}\in\gamma_j$, that is 
              internally conflicting or not conflicting with interactions of $\gamma_j$,
              $\mc{T}_j$ includes the transition $\tau_{\alpha}=(\{r_i^j\}_{B_i\in\p{\alpha}},
              \alpha,$\\$\{s_{p_{a_i}}\}_{a_i\in\alpha})$. The guard of this transition is 
              $g_{\tau}=
              \bigwedge_{a_i\in\alpha}g_{p_{a_i}}$. Notice that guards over data are the same,
              whereas guard of clocks are expressed using the global clock $t_j$ and the reset
              time variables $\rho_x$. This transition has no transfer function 
              and its reset function resets clock $z$.
            \item For each interaction $\alpha=\{a_i\}_{i\in{I}}\in\gamma_j$, that is 
              externally conflicting another interaction,
              $\mc{T}_j$ includes the following transitions:
              \begin{itemize}
                \item $\tau_{rsv_{\alpha}}=(\{r_i^j\}_{B_i\in\p{\alpha}},rsv_{\alpha},t_{\alpha}
                  )$. The guard of this transition is $g_{\tau}=\bigwedge_{a_i\in\alpha}
                  g_{p_{a_i}}$.
                \item $\tau_{ok_{\alpha}}=(t_{\alpha},ok_{\alpha},\{s_{p_{a_i}}\}_{a_i\in\alpha})
                  $. This transition has no guard, no reset function, and its transfer function is
                   $r_{\tau_{ok_{\alpha}}}=\{z:=0\}$.
                \item $\tau_{fail_{\alpha}}=(t_{\alpha},fail_{\alpha},\{s_{p_{a_i}}\}_{a_i
                  \in\alpha})$. This transition has no guard, no reset function, and no 
                  transfer function.
              \end{itemize}
          \end{itemize}
      \end{itemize}
  \end{itemize}
\end{definition}
Notice that Definition~\ref{def:sch_sr} presents only the syntax of a scheduler. It uses 
the Petri net formalism only for compactness purposes and is not to be confused with 
any other Petri Net formalism such as Time Petri Nets or Timed Petri nets.
\input{Figures/sch.tex}
\begin{example}
  Figure~\ref{fig:schSR} depicts the internal representation of scheduler $Sch_1$ from Figure
  ~\ref{fig:sr}. The scheduler $Sch_1$ is responsible of the interaction class $\gamma_1=\{
    \alpha_1,\alpha_3,\alpha_5,\alpha_7\}$, that is, he is responsible of notifying components
  $C$, $T_1$ and $R$ whose indexes are respectively $1$, $2$ and $4$. Notice that since 
  every interaction of $\gamma_1$ is potentially conflicting with an interaction of $\gamma_2$
  handled by scheduler $Sch_2$, scheduling interactions of $\gamma_1$ requires the intervention
  of the conflict resolution layer.
\end{example}
\begin{property}[Scheduler Semantics]\label{pr:sch_sr_sem}
  Let $Sch=(\Loc, \mc{P},\T,\X, \D,\{g_{\tau}\}_{\tau\in\mc{T}},\{r_{\tau}\}_{\tau\in\mc{T}},
    \{f_{\tau}\}_{\tau\in\mc{T}},$\\$\{\mc{I}_{\loc}\}_{\loc\in\Loc})$ be a tuple defining 
  a scheduler. Let $(2^{\Loc},\mc{P},\to_{\mc{P}})$ be the finite labeled transition
  system of its underlying 1-Safe Petri net. The semantics of the scheduler $Sch$ 
  is equivalent to the semantics of the timed component 
  $(2^{\Loc},\loc_0,\mc{X},\mc{D},\mc{P},\mc{E},\{f_{e}\}_{e\in\mc{T}},\I)$ such that:
  \begin{itemize}
    \item $\loc_0=\otimes_{\loc\in m_0|m_0(\loc)=1}\loc$, where $m_0$ is the initial marking 
      of the Petri net $(\Loc, \mc{P},\mc{T})$.
    \item For each $(m_1,p,m_2)\in\to_{\mc{P}}$ we include a transition $e\in\mc{E}=
      (\loc_1,p,g_p,r,\loc_2)$ such that:
      \begin{itemize}
        \item $\loc_1=\otimes_{\loc\in m_1|m_1(\loc)=1}\loc$ and
              $\loc_2=\otimes_{\loc\in m_2|m_2(\loc)=1}\loc$. The invariant of $\loc_1$ and 
              $\loc_2$ are respectively 
              $\Inv{\loc_1}=\bigwedge_{\loc\in m_1|m_1(\loc)=1}I_{\loc}$ and
              $\Inv{\loc_1}=\bigwedge_{\loc\in m_2|m_2(\loc)=1}I_{\loc}$.
            \item $g_p=g_{\tau}$ and $r=r_{\tau}$ where 
              $\tau=({^\bullet\tau},p,{\tau^\bullet })\in\mc{T}$ such that
            ${^\bullet\tau}\subseteq m$ and $m'=(m\textbackslash{^\bullet\tau})
  \cup{\tau^\bullet }$.
      \end{itemize}
  \end{itemize}
\end{property}



\subsection{Conflict Reservation Protocol}
In this subsection, we present the third layer of our Send/Receive architecture, namely
the conflict resolution layer. The main purpose of this layer is to resolve the conflict
that occur between interaction of separate schedulers at run time. 
Since interactions may compete on resources (here sharing components), the conflict resolution
layer implements a protocol, inspired from~\cite{} and based on messages counting technique,
that allows to check the freshness of offers received for the execution of an interaction
from schedulers. In other words, it ensures that two externally conflicting interactions
cannot execute with the same offers by checking that the participation numbers of the involved 
components have not been yet consumed. Particularly, the protocol keeps the last participation
number of each component and compares it with the participation number from the reservation 
request of a scheduler and thereafter, decides whether to grant a scheduler or not the execution
of an interaction.

There exists several implementations of the conflict resolution protocol in the literature
~\cite{,,}. We present here only one variant since our interest is not in studying the 
conflict resolution layer. It is centralized variant based on Bagrodia's protocol.

\begin{definition}[Conflict Resolution Protocol]
  Let $\gamma(B_1,\cdots,B_n)$ be a timed system and $\{\gamma_j\}^m_{j=1}$ be an interaction
  partition. The corresponding centralized conflict resolution protocol component is defined
  by the timed component $CP=(\Loc^{CP},\loc_0^{CP},\emptyset,\mc{D}^{CP},\mc{P}^{CP},\mc{E}^{CP}
  ,\emptyset)$ such that:
  \begin{itemize}
    \item $\Loc^{CP}$ contains for each externally conflicting interaction $\alpha$ a waiting
      location $w_{\alpha}$ and a receive location $r_{\alpha}$. Receive locations are 
      urgent locations.
    \item $\mc{D}^{CP}$ includes for each component $B_i$ participating in conflicting 
      interactions $\alpha$ its current participation number $n_i^{\alpha}$ as well as the last 
      participation number $N_i$.
    \item $\mc{P}^{CP}$ includes for each externally conflicting interaction a reservation port
      $rsv_{\alpha}$, $ok_{\alpha}$ and $fail_{\alpha}$. The port $rsv_{\alpha}$ exports
      the variables $\{n_i^{\alpha}|B_i\in\p{\alpha}\}$.
    \item $\mc{E}^{CP}$ includes for each externally conflicting interaction $\alpha$ the 
      following transitions:
      \begin{itemize}
        \item A reservation transition 
          $e_{rsv_{\alpha}}=(w_{\alpha},rsv_{\alpha},\true,\emptyset,r_{\alpha})$ 
        \item A transition granting the execution of $\alpha$, 
          $e_{ok_{\alpha}}=(r_{\alpha},ok_{\alpha},g_{ok_{\alpha}},r_{ok_{alpha}},w_{\alpha})$ 
          such that, $g_{ok_{\alpha}}=\bigwedge_{B_i\in\p{\alpha}} n_i^{\alpha}>N_i$ and
          $r_{ok_{alpha}}=\{\forall B_i\in\p{\alpha}, N_i:=n_i^{\alpha}\}$.
        \item A transition denying the execution 
          $e_{fail_{\alpha}}=(r_{\alpha},fail_{\alpha},\true,\emptyset,w_{\alpha})$
      \end{itemize}
  \end{itemize}
\end{definition}
\begin{figure}[!h]
 \centering
  \captionsetup{justification=centering}
  \begin{tikzpicture}[scale=0.7,every node/.style={scale=0.7}]
  
  \node[accepting,place] (w1){$w_{\alpha_1}$}; 
  \node[place,below=2.5cm of w1] (r1){$r_{\alpha_1}$}; 
  \draw [-] ($(r1) + (180:4mm)$) arc (180:360:4mm);
  \path (w1) edge node[left]{$rsv_{\alpha_1}$}(r1)
        (r1) edge [out=180,in=180] node[left,align=center]{$n_1^{\alpha_1}>N_1$\\
                                                      $n_2^{\alpha_1}>N_2$\\
                                                      $ok_{\alpha_1}$\\
                                                      $N_1:=n_1^{\alpha_1}$\\
                                                      $N_2:=n_2^{\alpha_1}$} (w1)
        (r1) edge [out=0,in=0] node[right,align=center]{$n_1^{\alpha_1}\le N_1$\\
                                                      $n_2^{\alpha_1}\le N_2$\\
                                                      $fail_{\alpha_1}$} (w1);
  

  \node[accepting,place,right=5cm of w1] (w3){$w_{\alpha_3}$}; 
  \node[place,below=2.5cm of w3] (r3){$r_{\alpha_3}$}; 
  \draw [-] ($(r3) + (180:4mm)$) arc (180:360:4mm);
  \path (w3) edge node[left]{$rsv_{\alpha_3}$}(r3)
        (r3) edge [out=180,in=180] node[left,align=center]{$n_1^{\alpha_3}>N_1$\\
                                                      $n_2^{\alpha_3}>N_2$\\
                                                      $ok_{\alpha_3}$\\
                                                      $N_1:=n_1^{\alpha_3}$\\
                                                      $N_2:=n_2^{\alpha_3}$} (w3)
        (r3) edge [out=0,in=0] node[right,align=center]{$n_1^{\alpha_3}\le N_1$\\
                                                      $n_2^{\alpha_3}\le N_2$\\
                                                      $fail_{\alpha_3}$} (w3);
  \node[accepting,place,below=5cm of w3] (w7){$w_{\alpha_7}$}; 
  \node[place,below=2.5cm of w7] (r7){$r_{\alpha_7}$}; 
  \draw [-] ($(r7) + (180:4mm)$) arc (180:360:4mm);
  \path (w7) edge node[left]{$rsv_{\alpha_7}$}(r7)
        (r7) edge [out=180,in=180] node[left,align=center]{$n_1^{\alpha_7}>N_1$\\
                                                      $n_2^{\alpha_7}>N_2$\\
                                                      $ok_{\alpha_7}$\\
                                                      $N_1:=n_1^{\alpha_7}$\\
                                                      $N_2:=n_2^{\alpha_7}$} (w7)
        (r7) edge [out=0,in=0] node[right,align=center]{$n_1^{\alpha_7}\le N_1$\\
                                                      $n_2^{\alpha_7}\le N_2$\\
                                                      $fail_{\alpha_7}$} (w7);
  \node[accepting,place,left=5cm of w7] (w5){$w_{\alpha_5}$}; 
  \node[place,below=2.5cm of w5] (r5){$r_{\alpha_5}$}; 
  \draw [-] ($(r5) + (180:4mm)$) arc (180:360:4mm);
  \path (w5) edge node[left]{$rsv_{\alpha_5}$}(r5)
        (r5) edge [out=180,in=180] node[left,align=center]{$n_1^{\alpha_5}>N_1$\\
                                                      $n_2^{\alpha_5}>N_2$\\
                                                      $ok_{\alpha_5}$\\
                                                      $N_1:=n_1^{\alpha_5}$\\
                                                      $N_2:=n_2^{\alpha_5}$} (w5)
        (r5) edge [out=0,in=0] node[right,align=center]{$n_1^{\alpha_5}\le N_1$\\
                                                      $n_2^{\alpha_5}\le N_2$\\
                                                      $fail_{\alpha_5}$} (w5);
  \node [rounded corners,inner xsep=50mm,inner ysep=30mm,draw,fit=(w1)(r7)] (rec1) {};
  \node [dots,label=90:$rsv_{\alpha_5}$] (i1) at ($(rec1.south west)!0.175!(rec1.south east)$) {};
  \node [triangle,rotate=180,label=-90:$ok_{\alpha_5}$] (i1) at ($(rec1.south west)!0.25!(rec1.south east)$) {};
  \node [triangle,rotate=180,label=-90:$fail_{\alpha_5}$] (i1) at ($(rec1.south west)!0.325!(rec1.south east)$) {};
  \node [dots,label=90:$rsv_{\alpha_7}$] (i1) at ($(rec1.south west)!0.675!(rec1.south east)$) {};
  \node [triangle,rotate=180,label=-90:$ok_{\alpha_7}$] (i1) at ($(rec1.south west)!0.75!(rec1.south east)$) {};
  \node [triangle,rotate=180,label=-90:$fail_{\alpha_7}$] (i1) at ($(rec1.south west)!0.825!(rec1.south east)$) {};
  
  \node [dots,label=-90:$rsv_{\alpha_1}$] (i1) at ($(rec1.north west)!0.175!(rec1.north east)$) {};
  \node [triangle,label=-90:$ok_{\alpha_1}$] (i1) at ($(rec1.north west)!0.25!(rec1.north east)$) {};
  \node [triangle,label=-90:$fail_{\alpha_1}$] (i1) at ($(rec1.north west)!0.325!(rec1.north east)$) {};
  \node [dots,label=-90:$rsv_{\alpha_3}$] (i1) at ($(rec1.north west)!0.675!(rec1.north east)$) {};
  \node [triangle,label=-90:$ok_{\alpha_3}$] (i1) at ($(rec1.north west)!0.75!(rec1.north east)$) {};
  \node [triangle,label=-90:$fail_{\alpha_3}$] (i1) at ($(rec1.north west)!0.825!(rec1.north east)$) {};

 \end{tikzpicture}
 \caption{Sub-Part of the Timed Component for the Centralized CRP of Figure~\ref{fig:sr} 
 Handeling Interactions of $\gamma_1$}
 \label{fig:crp}
\end{figure}  


In what follows, we present the Send/Receive interactions that link the three layer of the 
presented Send/Receive model.

\begin{definition}[Send/Receive Interactions]
  Let $\gamma(B_1,\cdots,B_n)$ be a timed system and $\{\gamma_j\}^m_{j=1}$ be an interaction
  partition. The Send/Receive interactions $\gamma^{SR}$ connecting the three layers of the
  Send/Receive models are:
  \begin{itemize}
    \item For each component $B_i^{SR}$, we include an offer interaction involving $B_i^{SR}$
      and its respective schedulers $\{B_i^{SR}.o,Sch_{j_1}.o_i,\cdots,Sch_{j_k}.o_i\}$.
    \item For each port $p$ of a component $B_i^{SR}$ and for each scheduler $Sch_j$ handling
      an interaction involving $p$, we include a notification interaction 
      $\{B_i^{SR}.p,Sch_{j}.p\}$.
    \item For each internally conflicting or not interaction $\alpha\in\gamma$ handled by 
      a scheduler $Sch_j$, we include the unary interaction $\{Ssch_j.a\}$.
    \item For each externally conflicting interaction $\alpha\in\gamma$, we include the following
      interactions:
      \begin{itemize}
        \item $\{S_j.rsv_{\alpha},CP.rsv_{\alpha}\}$
        \item $\{S_j.ok_{\alpha},CP.ok_{\alpha}\}$
        \item $\{S_j.fail_{\alpha},CP.fail_{\alpha}\}$
      \end{itemize}
  \end{itemize}
\end{definition}

The correctness of the Send/Receive transformation is proved using observational equivalence, 
that is, weak bisimulation. 
\begin{theorem}[Correctness~\cite{}]
  $T\dot{\sim}\mc{T}^{SR}$.
\end{theorem}
The correctness of the presented approach is necessary to attest that both the initial
and resulting system have the same behavior. Nonetheless, the proof of correctness has
already been established and its details are not relevant to the content of this thesis.
The interested reader can find all the steps of the proof in Chapter ? of~\cite{}.
\section{Modeling Distributed Real-Time Constraints}

Distributed real-time systems are prone to different kind of problems. The immediate concern 
is the communication delays inherent to distributed platforms. The latter increases considerably
the effort of coordinating the parallel activities of running components. Thus, scheduling
such systems must cope with the induced delays by proposing execution strategies ensuring 
global consistency while satisfying the imposed timing constraints.
Another phenomenon intrinsic to distributed platforms is clock drift. A clock is a device
that consists of a counter that is incremented periodically according to the frequency of 
an oscillator. This implies that clocks are not perfect since the oscillator frequency may
vary during its lifetime due to several factors such as aging, temperature, humidity, etc.
Consequently, clocks trend to \emph{drift} or gradually desynchronize from a given reference 
time. Moreover, when having multiple clocks running in the same system, which is usually the
case in distributed real-time systems, the relative clock drift between these clocks
may result in an unexpected (even undesirable) behavior. The common practice is to 
resynchronize the clocks (internally or externally) in order to bring the difference to 
a certain threshold to minimize the impact of this phenomenon. 

\subsection{Communication Delays}

The Send/Receive model as presented in Section~\ref{sec:3.2} assumes implicitly that 
communication between the different layers is timeless. This restricts the applicability
of such approach to applications where the timing constraints are far bigger than the 
communication delays imposed by a given target platform.  
To cope with those delays, a variant of the Send/Receive approach was presented in~\cite{}. 
It is based on an early decision making mechanism where schedulers plan interactions execution 
ahead and notify components of their execution date in advance. 
The main issue of this method is that when planning components to execute at a given time,
all the interaction including the planned components will be ineligible for execution (and even
for planning). This may locally block components, especially if all the interactions 
involving these components are disallowed from execution. To overcome this problem,
this adaptation of the Send/Receive models suggests that each scheduler, additionally to the
components he is handling, observe a subset of components that may be blocked when 
scheduling interactions. However, because of the nature of the location invariants (local
constraints that propagate on the global level), this technique results in observing
all the components of the system, instead of only a subset as presented in~\cite{}.

In order to model the behavior of a system under some communication delays bounded by $\hmin$
($\hmin$ being the worst case communication delay), we introduce the \emph{local planning 
semantics}. This semantics aims to distinguish between the decision dates for executing 
interactions and their actual execution dates by adding a notion of \emph{planning} on 
the semantics level. The delay between the planning of interaction and its execution is
thus constrained by $\hmin$., which is a parameter of the semantics. Although this approach 
is based on the same idea of anticipating the execution of interactions, it differs from 
the approach of~\cite{} in the following point:
\begin{itemize}
  \item The class of system handled by the method of~\cite{} is restricted to timed components
    with closed guard, that is, with clock constraints are of the form:
    \begin{displaymath}
    c:=\true \ | \ x\le k \ |\ x\ge k \
    \end{displaymath}
    where $x$ is a clock and $k\in\integerpoz$.
    Moreover, this method is restricted to timed components with non-decreasing deadlines. In
    other words, if time can progress by $d$ from a given state $(\loc,\val)$ of a component, 
    it can also progress by $d$ from any state $(\loc',\val')$ reached by executing an action
    $a$ from state $(\loc,\val)$. Our approach on the other hand imposes only that 
    the system is free of modelling errors such as deadlock or timelock. 
  \item Our approach work on the semantics level whereas the method of~\cite{} is based on 
    transformation and model construction. The main advantage of working on the semantics
    level is that it allows to stay at certain level of abstraction, close enough to the 
    original model, which reduce considerably the chance of errors during the formalization.
    Furthermore, one can still imagine a Send/Receive like transformation that implements
    this semantics.
  \item Unlike the standard semantics of timed system as presented in Chapter~\ref{chap:2}, 
    the local planning semantics is based on a local view of the system which is more suited
    for the distributed context.  
\end{itemize}
Chapter~\ref{chap:5} presents a detailed description of the local planning semantics, its 
properties and relation with the standard semantics of timed semantics. It also provides
sufficient conditions that guarantee the correctness (in terms of behaviour) of a given
application under some bounded communication delays.

\subsection{Clock Drift}

Reachability analysis~\cite{} has been used to test the behavior
of timed automata model against some safety properties. However, such analysis
techniques whether region-based or zone-based can be incorrect and misleading,
since they rely on the several assumption such as zero response times or infinitesimally  
precise clocks which is generally not the case in reality. In practice, clocks are 
implemented using an oscillator and a counting registers, and their precision based on 
the quality of the oscillator together with the operating environment.
The common practice when studying the effect of clock imperfections is to define a perturbation
model that approximates the behavior of a given model under clock drifts in order to study its 
robustness. 
Robust reachability has been introduced to check whether a given
timed automata model (system) still satisfies the specification when
subject to different perturbations such as clocks drift.
In~\cite{drift:puri}, Puri introduced a model of clock drift for closed timed automata
by introducing a parameter $\epsilon>0$ that bounds the clocks drift rates.
This work showed that the standard reachability analysis approach is not correct
when clocks drift, even by infinitesimally small amount, and subsequently provide a region based
method for calculating $Reach^*(S,q_0)$, the set of reachable states for \emph{every} drift 
(the limit as $\epsilon\to 0$),
that is, $Reach^*(S,q_0)=\cap_{\epsilon>0} Reach(S_{\epsilon},q_0)$. Other 
works~\cite{drift:conrad,drift:puriR} proposed a zone based algorithm for computing this 
reach-set more efficiently and generalize the approach for open timed automata 
model~\cite{drift:puriR}.
In~\cite{drift:wulf,drift:puri}, another perturbation model was considered. Here, 
the system model is syntactically modified by~\emph{relaxing} the guards through 
a parametric enlargement of $\delta$.
Dewulf~\cite{drift:wulf} showed that the notion of robustness defined in~\cite{drift:puri} and
studied in other works~\cite{drift:conrad, drift:puriR} is closely related to the notion of 
implementability introduced in~\cite{drift:wulf}, that is, whether for some $\delta>0$,
the enlarged system model still satisfies the requirements expressed by the safety properties.
This allows to prove that the considered notion of implementability is decidable 
for timed automata.
Finally,~\cite{drift:surp} consider a more realistic model of drifting clocks by considering 
clock resynchronization available now in most distributed real-time systems. It was proven
that standard zone-based reachability analysis  is exact when testing robust safety, provided
a uniform strictly positive robustness margin of 1.
In Chapter~\ref{chap:6}, we present a timed automata based model for distributed real-time
systems where the relative drift between clocks is assumed to be bounded (clocks are assumed
to be resynchronized with a certain threshold). The resulting timed transition system 
includes straightforwardly more states than the initial model. We then give interesting 
properties of the drifted model and provide a strategy that allows to for any resulting 
execution trace to stay close enough to a similar trace of the initial model. 

\part{Contribution}
{This part includes our contributions to the field of modeling and validation of
distributed real-time systems. First, Chapter~\ref{chap:4} proposes a knowledge based
optimization of the Send/Receive transformation. It aims at reducing the interactions between
the scheduling layer and the conflict resolution layer through a reduction of the potentially
conflicting interactions set. Thereafter, Chapter~\ref{chap:5} and~\ref{chap:6} study the 
behavior of a given model when subject to constraints inherent to the distributed context.
Chapter~\ref{chap:5} tackles the problem of communication delays by proposing a strategy based
on anticipating the execution of components beforehand. It provides sufficient conditions
that allow to check whether a given system is robust or not (in the sense not guaranteed) 
to communication delays. We also propose an alternative method based on real-time controller
synthesis and explain how it differs from our approach. 
In the same way, Chapter~\ref{chap:6} investigates the clock drift problem and proposes a 
strategy that ensures that executions of the drifted system stay close enough from 
executions of the model with perfect clocks.
}
\chapter{Knowledge Based Optimization of Distributed Real-Time Systems}\label{chap:4}
\minitoc

\section{Conflicting Interaction Calculation}

As explained in Subsection~\ref{sub:conf}, two interactions $\alpha_1$ and $\alpha_2$ 
sharing a subset of components cannot execute concurrently.
Particularly, if these interactions are enabled from the same state they are conflicting, meaning
that they are competing on the same resources (shared components) and only one interaction will
be granted the execution and not the other.
Particularly, given a timed system $\gamma(B_1,\cdots,B_n)$ and an interaction partition
$\{\gamma_j\}_{j=1}^m$, if such interactions are part of two different class of the interaction
partition (externally conflicting interactions), then the resulting Send/Receive model requires 
the intervention of the conflict resolution layer to resolve the conflict situation. 
The computation of the conflicting interactions set is based on syntactic pre-checks
(Definition~\ref{def:pconf}), that is, it is an over-approximation that in some cases  
induces an unnecessary conflict resolution.  
The calculation of the conflicting interactions set highly impact the structure and the 
performance of the underlying Send/Receive model. For every interaction that is in fact
not conflicting, the corresponding scheduler include an additional place and three transitions
involved in the Send/Receive interactions involving the conflict resolution layer.
In this case, executing an interaction adds not only a evaluation overhead but also latency 
resulting from the communication delay between the two layers.
In order to refine the conflicting interactions set, we rely on the following definition
of conflicts.
\begin{definition}[Conflicting Interaction]
\label{def:conf}
Let $S = \gamma(B_1,\dotsc,B_n)$ be a timed system. Two interactions 
$\alpha_1$ and $\alpha_2$ of $\gamma$ are \emph{conflicting}, and we write $\alpha_1\#\alpha_2$, 
  if $\p{\alpha_1} \cap\p{\alpha_2} \neq \emptyset$ and there exists a reachable state from 
which both $\alpha_1$ and $\alpha_2$ are enabled, i.e. a state satisfying:
\begin{equation}\label{eq:conf1}
  Conflict(S,\alpha,\beta)=Reach(S) \wedge \enabled{\alpha_1} \wedge \enabled{\alpha_2}
\end{equation}
  where $Rreach(S)$ is the set of reachable states of the S.
\end{definition}

The above definition of conflict characterizes the exact set of conflicting interactions.
Especially, it considers that two interactions are not conflicting if the whole system
cannot reach a state from which both can potentially execute.
Clearly, such interactions do not require conflict resolution as they cannot be scheduled
based on common offers.
In what follows, we propose an approach that aims to reduce the set of potential conflicts.
In fact instead of calculating the exact set of reachable states of a given system,
we use static analysis techniques to extract a \emph{knowledge} that represents an
over approximation of the reachable states of the system on the form of invariants.

\section{Knowledge Based Reduction of Potentially Conflicting Interaction}

Knowledge as referred to it here can be interpreted as any information that gives 
a characterization of a given system. We distinguish two types of knowledge:
Local knowledge that captures partial information of a system on components level, and
a global knowledge that relates these local knowledge and link them together.
Our approach includes two main steps: the first step \emph{(i)} consists of constructing the set
of potentially conflicting interactions based on Definition~\ref{def:pconf}. 
This step aims mainly to distinguish non conflicting interactions from those that can
potentially conflict in order to ease and avoid unnecessary checks during the second step.
Then, the second step \emph{(ii)} calculates then combines local and global knowledge 
of the system on the form of invariants. The latter will represents an over-approximation 
of the reachable state of the system. 
After that, by replacing $Reach(S)$ by its over-approximation $\reacha{S}$ in
Equation~\ref{eq:conf1}, potentially conflicting interactions are reduced by
checking the following precondition:

\begin{equation}\label{eq:conf2}
  \overline{Conflict(S,\alpha_1,\alpha_2)}=\ \reacha{S} \wedge \enabled{\alpha_1} \wedge 
  \enabled{\alpha_2}
\end{equation}

Notice that since $Reach(S)\Rightarrow\reacha{S}$, we obtain that 
$Conflict(S,\alpha_1,\alpha_2) \Rightarrow$\\$\overline{Conflict(S,\alpha_1,\alpha_2)}$.
Thus, if two interactions are established to be conflicting  
according to Equation~\ref{eq:conf2}, then they are conflicting according to 
Equation~\ref{eq:conf1}.
Hereinafter, \emph{false} conflicts refers to potential conflicts as defined in
Definition~\ref{def:pconf} but that are not conflicts with respect to Definition~\ref{def:conf}.

A potential conflict between two interactions is a \emph{false} conflict either: \emph{(i)} 
because the system cannot reach a global location configuration enabling both interactions, 
or \emph{(ii)} because both interactions are not enabled at the same time due to timing 
constraints.
In the following, we show how to compute invariants for removing false conflicts of
types \emph{(i)} and \emph{(ii)}. This invariants combined with
individual reachable states of components will represent our over-approximation.
\subsection{Linear Invariants}

Linear invariants consists of linear constraints that allows to reason on complex properties.
For instance, they can be used to \emph{count} how many processes are at a given state
of a concurrent system. Such invariants have been widely used in different domain~\cite{,}.
Particularly, we are interested in the so called linear state-invariants~\cite{} from the 
Petri net community. Linear state-invariants or (S-invariants) are determined to be 
appropriate for proving non-coverage of subsets of individual locations, 
which corresponds exactly to what needed to prove that two interactions 
cannot be enabled from the same locations configuration.
They consists of a linear combination of $\al{\loc}$ predicates that is 
equal to a constant. For instance $\al{\loc_0}+\al{\loc_1}+\al{\loc_2}=1$ is a linear constraint
for the Petri net of Figure~\ref{fig:pn}.

Locations configurations reachable in a composition 
$S = \gamma(B_1,\dotsc,B_n)$ are necessary combinations 
of reachable locations of individual components $B_i$.
However, in general not all combinations are reachable in $S$ since components are not 
fully independent as they synchronize through interactions in the composition.
A typical example of that is a shared resource used in mutual exclusion by a set of 
components: any of them can potentially use it, but they should coordinate so that states in 
which two (or more) components use the resource are not reachable.
Another illustration of this can be found in example of Figure~\ref{fig:tm}: components 
$T_1$ (resp. $T_2$) may reach location $\loc_1^2$ (resp. $\loc_1^3$) by executing 
action $init_1$ (resp. $init_2$), but in the composition $T_1$ and $T_2$ cannot be 
simultaneously at locations $\loc_1^2$ and $\loc_1^3$.
This is due to interactions $\alpha_1 = \{ init_0, init_1 \}$ and 
$\alpha_2 = \{ init_0, init_2 \}$ with component $C$: executing $\alpha_1$ disables 
$\alpha_2$, and vice versa.
That is, the potential conflict between interactions $\alpha_3$ and $\alpha_4$ can be 
excluded if we consider reachable locations of the composed system.

\begin{definition}[Linear Invariant]
\label{def:inv}
Let $S = \gamma(B_1,\dotsc,B_n)$ be a timed system and $\Loc = \bigcup_{1 \leq i \leq n} \Loc_i$
all components locations, $\Loc_i$ being  the locations of $B_i$.
A \emph{linear invariant} of $S$ is a linear equality constraint 
which holds in all reachable global state of S. It is of the form:
$$ \sum_{\loc \in\Loc}u_{\loc}\cdot \al{\loc} = u_0, $$
where $u_{\loc}$, and $u_0$ are integers, in which predicates $\al{\loc}$,
$\loc \in \Loc$, are interpreted as $0$ for false and $1$ for true.
\end{definition}

To compute linear invariants for a timed system $S = \gamma(B_1,\dotsc,B_n)$, we consider its 
version $\tilde{S}$ abstracting all data and timing aspects of $S$
(i.e. obtained from $S$ by relaxing guards and location invariants of components).
Note that linear invariants for $\tilde{S}$ are also a linear invariants for $S$, since 
reachable locations of $S$ are necessary included in reachable locations of $\tilde{S}$.
Methods for calculating linear invariants are based on linear algebra, and more precisely
on the \emph{characteristic system}~\cite{} (also known by the place-transition matrix in 
the Petri nets community~\cite{}). It consists of a system of linear equations representing 
the interactions of a given system.
\begin{definition}\label{def:chars}[Characteristic System]
  For a timed system $S=\gamma(B_1,\cdots,B_n)$ and $\Loc=\Loc_1\times\cdots\times\Loc_n$,
  the set of all global location configuration. The characteristic system is defined as 
  follows:
  $$\mc{M}(S)=\bigwedge_{\alpha\in\gamma}\bigwedge_{\loc\in\Loc_{\alpha}}
\Big(\sum_{\loc_i\in\alpha^{\bullet}}x_{\loc_i}-\sum_{\loc_j\in{^\bullet}\alpha}x_{\loc_j}\Big)$$
  where $\Loc_{\alpha}$ is the subset of locations configurations from which $\alpha$ is 
  possible, and $\alpha^{\bullet}$, ${^\bullet}\alpha$ denotes respectively the destinations and 
  sources locations of components actions involved in $\alpha$.
\end{definition}

\begin{example}
  The characteristic system for Example~\ref{exp:run} following the enumeration of all 
  interactions of $\gamma$ is:

  \[\mc{M}(S)=\begin{cases}
    x_{\loc_1^1}-x_{\loc_0^1}+x_{\loc_1^2}-x_{\loc_0^2}=0\\ 
    x_{\loc_1^1}-x_{\loc_0^1}+x_{\loc_1^3}-x_{\loc_0^3}=0\\ 
    x_{\loc_0^1}-x_{\loc_1^1}+x_{\loc_2^2}-x_{\loc_1^2}=0\\ 
    x_{\loc_0^1}-x_{\loc_1^1}+x_{\loc_2^3}-x_{\loc_1^3}=0\\ 
    x_{\loc_1^4}-x_{\loc_0^4}+x_{\loc_3^2}-x_{\loc_2^2}=0\\ 
    x_{\loc_1^4}-x_{\loc_0^4}+x_{\loc_3^3}-x_{\loc_2^3}=0\\ 
    x_{\loc_0^4}-x_{\loc_1^4}+x_{\loc_0^2}-x_{\loc_3^2}=0\\ 
    x_{\loc_0^4}-x_{\loc_1^4}+x_{\loc_0^3}-x_{\loc_3^3}=0\\ 
  
  
  \end{cases}\]

\end{example}

The common techniques for solving homogeneous systems $Ax=0$ are the Guass-Jordan elimination,
Cholesky-, QR- or LU-factorization. These algorithm have low polynomial complexity and
can be directly applied to solve the characteristic system $\mc{M}(S)$.
In order to obtain the linear invariants of a given system, we use the algorithm 
proposed in~\cite{inv-lin}. It is a variant of Gauss-Jordan elimination that exploits 
the locality of unknowns as well as the particular form of the characteristic system equations. 
Equations are processed iteratively, one by one, while producing an equivalent left-bound system.

Let $LI(S)$ be the linear invariants characterizing a given system S.
A potential conflict between interactions $\alpha_1$ and $\alpha_2$ is a false conflict if the 
following formula is not satisfiable:

\begin{equation}\label{eq:linconf}
  \bigwedge_{1\le i\le n}Reach(B_i)\wedge LI(S)  \wedge \enabled{\alpha_1}\wedge\enabled{\alpha_2}
\end{equation}
where $Reach(B_i)$ denots the reachable states of component $B_i$.
\begin{example}
Let us reconsider the example of Figure~\ref{fig:tm}.
Among the resulting linear invariants, we focus on following:
\begin{numcases}{}
1\cdot\al{\loc_1^2} + 1\cdot\al{\loc_1^3} - 1\cdot\al{\loc_1^1} = 0 \label{eq:inv1} \\
1\cdot\al{\loc_3^2} + 1\cdot\al{\loc_3^3} - 1\cdot\al{\loc_1^4} = 0 \label{eq:inv2}.
\end{numcases}
We deduce from Equation~(\ref{eq:inv1}) that $\al{\loc_1^2}$ and $\al{\loc_1^3}$
cannot be true simultaneously, that is, components $T_1$ and $T_2$ cannot be simultaneously 
at the corresponding locations. Consequently, we can directly infer that interactions 
$\alpha_3$ and $\alpha_4$ are not conflicting, even though they are potentially conflicting.
Likewise, with~(\ref{eq:inv2}) we exclude the conflict between $\alpha_7$ and $\alpha_8$.
\end{example}

\subsection{History Clocks Inequalities}
As they completely abstract time, linear invariants presented above are only partially 
capturing system dynamics.
For example, a global location may be not reachable because components locations have 
disjoint clock constraints, or an interaction may not be enabled from a state because 
of an empty timing constraint. Such properties require extra relationships relating clocks of 
different components that are not available in $Reach(B_i)$ as it is is restricted to 
clocks of a single component: a zone of one of its symbolic states is a conjunction of its 
related clocks constraints, as explained in Section~\ref{sec:2.4}.

We follow the approach of~\cite{} for reinforcing our approach with global invariants on clocks.
They are induced by simultaneity of transitions when executing an 
interaction and the synchrony of time progress. To compute such invariants, additional 
\emph{history} clocks are first introduced in components. History clocks are associated to 
actions of components and to interactions, and reset upon their execution.
They do not modify the behavior since they are not involved in timing constraints.
They only reveal local timing of components, relevant to the interaction layer, which 
allows to infer further properties referred as \emph{history clocks inequalities} 
in~\cite{}, expressing the fact that history clock of an interaction are necessary 
equal to history clocks of its actions after its execution and until the execution of another 
interaction involving these actions.
\begin{figure}[!h]
 \centering
  \begin{tikzpicture}[every node/.style={scale=0.8},scale=0.8]

  \node [place](l0) {$\loc_1$};
  \node [place,below=1.3cm of l0] (l1) {$\loc_2$};

  \path (l0) edge node[right,align=center]{$a$}(l1);
  
  \node [place,right=5cm of l0](l0) {$\loc_1$};
  \node [place,below=1.3cm of l0] (l1) {$\loc_2$};

  \path (l0) edge node[right,align=center]{$a$\\$h_a:=0$}(l1);
\end{tikzpicture}
  \caption{Example of a History Clock for Action a}
 \label{fig:hca}
\end{figure}  



\begin{definition}[History Clocks for Actions]
Given a timed system $S=\gamma(B_1,\cdots,B_n)$, the history clocks for actions are defined
as follows:
  $$\mc{HA}(S)=\bigvee_{\alpha\in\gamma}\big[\big(\bigwedge_{\substack{a_i,a_j\in\alpha\\
  a_k\in Act(\gamma\ominus\alpha)}}h_{a_i}=h_{a_j}\le h_{a_k}\big)\wedge(\mc{HA}(\gamma\ominus
  \alpha)\big]$$
\end{definition}
where $Act(\gamma\ominus\alpha)$ is the set of actions involved in the interactions
$\gamma\ominus\alpha=\{\beta\setminus\alpha|\beta\in\gamma\wedge\beta\nsubseteq\alpha\}$
and $\mc{HA}(S)=\true$.
The predicate $\mc{HA}(S)$ can be interpreted as follows. Assuming that $\alpha\in\gamma$
is the last interaction executed in the system. Then, all history clocks of its involved
actions are reset at the same time. Moreover, they are smaller than all the other history
clocks, contained in $\gamma\ominus\alpha$.
The history clocks for actions are additionally strengthened by \emph{separation constraints}
for conflicting interactions. In fact, an action involved in two conflicting interactions
is exclusively executed by one of these interactions at a given time. Particularly,
in some cases a minimum time lapse is required between two executive occurrences of
the same action. Similarly to history clocks for actions, we introduce a history clock for
each interaction that will be reset on the execution of the latter.
Separation constraints are then formalized as follows:

\begin{definition}[Separation Constraints]
Given a timed system $S=\gamma(B_1,\cdots,B_n)$, the separation constraints are defined
as follows:
  $$HI(S)=\bigwedge{\substack{\alpha\neq\beta\in\gamma\\a\in\alpha\cap\beta}}
  \bigwedge_{a\in\alpha} |h_{\alpha}-h_{\beta}|\ge k_a$$
where $|h|$ denotes the absolute value of $h$ and $k_a$ is a constant computed locally
on the component containing action $a$. It represents the minimum time lapse between
two occurrences of $a$.
\end{definition}

Our method combines history clocks inequalities $\mc{H}(S)=\mc{HA}(S)\wedge\mc{HI}(S)$ 
and symbolic states of components to identify false conflicts where the following is not
satisfiable:
\begin{equation}\label{eq:histconf}
  \bigwedge_{1 \leq i \leq n} Reach(B_i) \ \wedge \mc{H}(S) \  \wedge \ \enabled{\alpha} 
  \wedge \ \enabled{\beta}
\end{equation}

\begin{example}
We illustrate the application of (\ref{eq:histconf}) for checking conflicts by 
considering again the example of Figure~\ref{fig:tm}.
It can be shown that the potential conflict between $\alpha_5$ and $\alpha_6$ 
cannot be removed using (only) linear invariants.
In the following, we prove that these interactions are actually not conflicting using 
history clocks inequalities.
Since action $start_0$ of $C$ is synchronized with either $start_1$ of $T_1$ or $start_2$ of 
$T_2$, and since history clocks $h_a$ of an action $a$ is reset whenever $a$ is executed, 
by~\cite{souha:hs} the history clock inequalities for $start_0$ are:
\begin{equation}\label{eq:histconst}
\begin{split}
( h_{start_0} = h_{start_1} &\le h_{start_2} - 25 ) \\ 
  &\vee \\ ( h_{start_0} = h_{start_2} &\le h_{start_1} - 25 ).
\end{split}
\end{equation}
Equation~(\ref{eq:histconst}) states that $h_{start_0}$ is equal to the history clock 
corresponding to the last synchronization, i.e. either $h_{start_1}$ or $h_{start_2}$, and is 
lower than history clocks of previous synchronizations. Value $25$ in~(\ref{eq:histconst}) is 
obtained considering \emph{separation constraints} computed from symbolic states of components 
and interactions~\cite{}: two occurrences of $start_0$ are separated by at least $25$ 
time units because of timing constraints of $C$, and so too occurrences of $start_1$ or 
$start_2$ which can only execute jointly with $start_0$.
To relate history clocks with components clocks, we simply include history clocks when 
computing symbolic states of components (i.e. $Reach$ for components), which is used to 
establish here that $x = h_{start_1}$ and $y = h_{start_2}$ when components $T_1$ and $T_2$ 
are respectively at locations $\loc_2^2$ and $\loc_2^3$.
That is, with~(\ref{eq:histconst}) we obtain $x \le y - 25$ or $y \le x - 25$.
  By definition of $Enabled$ we have $\enabled{\alpha_5} = \al{\loc_2^2} \wedge 
(10 \leq x \leq 30) $. 
  Similarly, $\enabled{\alpha_6} = \al{\loc_2^3} \wedge (10\le y \leq 30)$.
  We obtain then: $\enabled{\alpha_5}\wedge\al{\loc_2^3}\Rightarrow y\le5 \
  \wedge\enabled{\alpha_6}\wedge\al{\loc_2^2}\Rightarrow x\le5$.
This proves that $\alpha_5$ and $\alpha_6$ are not conflicting. 
\end{example}

\section{Impact of Conflict Reduction on Send/Receive Models}
As explained earlier, refining the conflicting interaction set enables to minimize the
exchange of messages between the scheduling layer and the conflict resolution layer
of the corresponding Send/Receive model. More precisely, every false conflicting interaction
is scheduled using 2 exchange of messages between components participating in that interaction
and the corresponding schedulers (offers and notifications), 
instead of 4 by considering the unnecessary exchange of messages between schedulers and 
the conflict resolution protocol.
As a result, the Scheduler $Sch_1$ and the conflict resolution components 
from Figure~\ref{fig:schSR} and Figure~\ref{fig:crp} respectively become:
\begin{figure}[!h]
 \centering
  \captionsetup{justification=centering}
  \begin{tikzpicture}[scale=0.7,every node/.style={scale=0.7}]
  
  \node[accepting,place] (w1){$w_{\alpha_1}$}; 
  \node[place,below=2.5cm of w1] (r1){$r_{\alpha_1}$}; 
  \draw [-] ($(r1) + (180:4mm)$) arc (180:360:4mm);
  \path (w1) edge node[left]{$rsv_{\alpha_1}$}(r1)
        (r1) edge [out=180,in=180] node[left,align=center]{$n_1^{\alpha_1}>N_1$\\
                                                      $n_2^{\alpha_1}>N_2$\\
                                                      $ok_{\alpha_1}$\\
                                                      $N_1:=n_1^{\alpha_1}$\\
                                                      $N_2:=n_2^{\alpha_1}$} (w1)
        (r1) edge [out=0,in=0] node[right,align=center]{$n_1^{\alpha_1}\le N_1$\\
                                                      $n_2^{\alpha_1}\le N_2$\\
                                                      $fail_{\alpha_1}$} (w1);
  

  \node[accepting,place,right=5cm of w1] (w2){$w_{\alpha_2}$}; 
  \node[place,below=2.5cm of w2] (r2){$r_{\alpha_2}$}; 
  \draw [-] ($(r2) + (180:4mm)$) arc (180:360:4mm);
  \path (w2) edge node[left]{$rsv_{\alpha_2}$}(r2)
        (r2) edge [out=180,in=180] node[left,align=center]{$n_1^{\alpha_2}>N_1$\\
                                                      $n_3^{\alpha_2}>N_3$\\
                                                      $ok_{\alpha_2}$\\
                                                      $N_1:=n_1^{\alpha_2}$\\
                                                      $N_3:=n_3^{\alpha_2}$} (w2)
        (r2) edge [out=0,in=0] node[right,align=center]{$n_1^{\alpha_2}\le N_1$\\
                                                      $n_3^{\alpha_2}\le N_3$\\
                                                      $fail_{\alpha_2}$} (w2);
  \node [rounded corners,inner xsep=60mm,inner ysep=15mm,draw,fit=(w1)(r2)] (rec1) {};
  \node [dots,label=90:$rsv_{\alpha_1}$] (i1) at ($(rec1.south west)!0.175!(rec1.south east)$) {};
  \node [triangle,rotate=180,label=-90:$ok_{\alpha_1}$] (i1) at ($(rec1.south west)!0.25!(rec1.south east)$) {};
  \node [triangle,rotate=180,label=-90:$fail_{\alpha_1}$] (i1) at ($(rec1.south west)!0.325!(rec1.south east)$) {};
  \node [dots,label=90:$rsv_{\alpha_2}$] (i1) at ($(rec1.south west)!0.675!(rec1.south east)$) {};
  \node [triangle,rotate=180,label=-90:$ok_{\alpha_2}$] (i1) at ($(rec1.south west)!0.75!(rec1.south east)$) {};
  \node [triangle,rotate=180,label=-90:$fail_{\alpha_2}$] (i1) at ($(rec1.south west)!0.825!(rec1.south east)$) {};
  
 \end{tikzpicture}
 \caption{Refined Version of the Conflict Resolution Layer from Figure~\ref{fig:crp}}
 \label{fig:crp}
\end{figure}  


\begin{figure}[!h]
 \centering
  \captionsetup{justification=centering}
  \begin{tikzpicture}[scale=0.7,every node/.style={scale=0.7}]
  
  \node [markplace] (w1) {$w_1^1$};
  \node [markplace,right=4cm of w1] (w2) {$w_2^1$};
  \node [markplace,right=4cm of w2] (w3) {$w_4^1$};
  \node [transition,below=7.5mm of w1](o11) {} ;
  \node [transition,below=7.5mm of w2](o21) {};
  \node [transition,below=7.5mm of w3](o31) {};
  \node [left=1mm of o11] {$o_1$} ;
  \node [left=1mm of o21] {$o_2$} ;
  \node [left=1mm of o31] {$o_4$} ;
  \node[place,below=15mm of w1] (r1){$r_1^1$}; 
  \node[place,below=15mm of w2] (r2){$r_2^1$}; 
  \node[place,below=15mm of w3] (r3){$r_4^1$}; 
  \node [transition,below=10mm of r1,xshift=1.75cm](a1) {};
  \node [transition,below=20mm of r2,xshift=-1.75cm](a3) {};
  \node [transition,below=20mm of r2,xshift=1.75cm](a5) {};
  \node [transition,below=20mm of r3,xshift=-1.75cm](a7) {};
  \node [left=1mm of a1] {$rsv_{\alpha_1}$} ;
  \node [left=1mm of a3] {$\alpha_3$} ;
  \node [right=1mm of a5] {$\alpha_5$} ;
  \node [right=1mm of a7] {$\alpha_7$} ;
  \node[place,below=7.5mm of a1] (t1){$t_{\alpha_1}$}; 
  \node [transition,below=7.5mm of t1](a11) {};
  \node [left=1mm of a11] {$ok_{\alpha_1}$} ;
  \node[place,below=7.5mm of a11,xshift=-3.8cm] (s1){$s_{init_0}$}; 
  \node[place,right=1cm of s1] (s2){$s_{\substack{sta-\\rt_0}}$}; 
  \node[place,right=1cm of s2] (s3){$s_{init_1}$}; 
  \node[place,right=1cm of s3] (s4){$s_{\substack{sta-\\rt_1}}$}; 
  \node[place,right=1cm of s4,align=center] (s5){$s_{\substack{pro-\\cess_1}}$}; 
  \node[place,right=1cm of s5] (s6){$s_{end_1}$}; 
  \node[place,right=1cm of s6] (s7){$s_{take}$}; 
  \node[place,right=1cm of s7] (s8){$s_{free}$}; 
  \node [transition,below=7.5mm of s1](s11) {};
  \node [transition,below=7.5mm of s2](s22) {};
  \node [transition,below=7.5mm of s3](s33) {};
  \node [transition,below=7.5mm of s4](s44) {};
  \node [transition,below=7.5mm of s5](s55) {};
  \node [transition,below=7.5mm of s6](s66) {};
  \node [transition,below=7.5mm of s7](s77) {};
  \node [transition,below=7.5mm of s8](s88) {};
  \node [left=1mm of s11] {$init_0$} ;
  \node [left=1mm of s22] {$start_0$} ;
  \node [left=1mm of s33] {$init_1$} ;
  \node [left=1mm of s44] {$start_1$} ;
  \node [right=1mm of s55] {$process_1$} ;
  \node [right=1mm of s66] {$end_1$} ;
  \node [right=1mm of s77] {$take$} ;
  \node [right=1mm of s88](l) {$free$} ;
  \node [place,below=8.5cm of w1,dashed] (w11) {$w_1^1$};
  \node [place,right=4cm of w11,dashed] (w22) {$w_2^1$};
  \node [place,right=4cm of w22,dashed] (w33) {$w_4^1$};
  \node[transition, left=2cm of t1] (f1){};
  \node[left=1mm of f1]{$fail_{\alpha_1}$};
  \node[above=8mm of w1](ff1){};
  
  \path (w1) edge (o11)
        (w2) edge (o21)
        (w3) edge (o31)
        (o11) edge (r1)
        (o21) edge (r2)
        (o31) edge (r3)
        (r1) edge (a1)
        (r1) edge (a3)
        (r2) edge (a1)
        (r2) edge (a3)
        (r2) edge (a5)
        (r2) edge (a7)
        (r3) edge (a5)
        (r3) edge (a7)
        (a1) edge (t1)
        (t1) edge (a11)
        (a11) edge (s1)
        (a11) edge (s3)
        (a3) edge (s2)
        (a3) edge (s4)
        (a5) edge (s5)
        (a5) edge (s7)
        (a7) edge (s6)
        (a7) edge (s8)
        (s1) edge (s11)
        (s2) edge (s22)
        (s3) edge (s33)
        (s4) edge (s44)
        (s5) edge (s55)
        (s6) edge (s66)
        (s7) edge (s77)
        (s8) edge (s88)
        (s11) edge (w11)
        (s22) edge (w11)
        (s33) edge (w22)
        (s44) edge (w22)
        (s55) edge (w22)
        (s66) edge (w22)
        (s77) edge (w33)
        (s88) edge (w33)
        (r1) edge [loop,in=180,out=120,looseness=4] node[left]{$o_1$} (r1)
        (r2) edge [loop,in=180,out=120,looseness=4] node[left]{$o_2$} (r2)
      %  (r3) edge [loop,in=70,out=10,looseness=4] node[right]{$o_4$} (r3);
        (t1) edge [loop,in=180,out=120,looseness=4] node[left,align=center]{$o_1$\\$o_2$} (t1)
      %  (t3) edge [loop,in=180,out=120,looseness=4] node[left,align=center]{$o_1$\\$o_2$} (t3)
      %  (t5) edge [loop,in=70,out=10,looseness=4] node[right,align=center]{$o_2$\\$o_4$} (t5)
      %  (t7) edge [loop,in=70,out=10,looseness=4] node[right,align=center]{$o_2$\\$o_4$} (t7)
        (t1) edge [in=-90,out=-120] (f1)
      %  (t3) edge [in=-90,out=-60] (f3)
      %  (t5) edge [in=-90,out=-120] (f5)
      %  (t7) edge [in=-90,out=-60] (f7)
        (f1) edge [in=-150,out=90] (r1)
        (f1) edge [-,in=-180,out=90] (ff1.center)
        (ff1.center) edge [in=100,out=0] (r2);
      %  (f7) edge [in=10,out=90] (r3)
      %  (f7) edge [-,in=0,out=90] (ff7.center)
      %  (ff7.center) edge [in=80,out=180] (r2)
      %  (f3) edge [in=-100,out=90] (r2)
      %  (f5) edge [in=-80,out=90] (r2)
      %  (f3) edge [in=0,out=90,looseness=1.5] (r1)
      %  (f5) edge [in=180,out=90,looseness=1.5] (r3);
 
  
  \node [rounded corners,inner xsep=60mm,inner ysep=40mm,draw,fit=(w1)(w11)(s88)(ff1)(l),xshift=-1.5cm] (rec1) {};
  \node [dots,label=90:$o_1$] (i1) at ($(rec1.south west)!0.15!(rec1.south east)$) {};
  \node [triangle,rotate=180,label=-90:$init_0$] (i1) at ($(rec1.south west)!0.20!(rec1.south east)$) {};
  \node [triangle,rotate=180,label=-90:$start_0$] (i1) at ($(rec1.south west)!0.25!(rec1.south east)$) {};
  \node [dots,label=90:$o_2$] (i1) at ($(rec1.south west)!0.40!(rec1.south east)$) {};
  \node [triangle,rotate=180,label=-90:$init_1$] (i1) at ($(rec1.south west)!0.45!(rec1.south east)$) {};
  \node [triangle,rotate=180,label=-90:$start_1$] (i1) at ($(rec1.south west)!0.50!(rec1.south east)$) {};
  \node [triangle,rotate=180,label=-90:$process_1$] (i1) at ($(rec1.south west)!0.575!(rec1.south east)$) {};
  \node [triangle,rotate=180,label=-90:$end_1$] (i1) at ($(rec1.south west)!0.65!(rec1.south east)$) {};
  \node [dots,label=90:$o_4$] (i1) at ($(rec1.south west)!0.75!(rec1.south east)$) {};
  \node [triangle,rotate=180,label=-90:$take$] (i1) at ($(rec1.south west)!0.80!(rec1.south east)$) {};
  \node [triangle,rotate=180,label=-90:$free$] (i1) at ($(rec1.south west)!0.85!(rec1.south east)$) {};
      
  \node [triangle,label=-90:$rsv_{\alpha_1}$] (i1) at ($(rec1.north west)!0.15!(rec1.north east)$) {};
  \node [dots,label=-90:$ok_{\alpha_1}$] (i1) at ($(rec1.north west)!0.20!(rec1.north east)$) {};
  \node [dots,label=-90:$fail_{\alpha_1}$] (i1) at ($(rec1.north west)!0.25!(rec1.north east)$) {};
  \end{tikzpicture}
 \caption{Refined Version of Scheduler $Sch_1$ from Figure~\ref{fig:schSR}}
 \label{fig:schSR}
\end{figure}  







\Chapter{The Local Planning Semantics}\label{chap:5}
\minitoc
\section{Local Planning of Interactions}
\label{sec3}
High-level coordination primitives, such as multiparty synchronizations (interactions) defined in the previous section, are rarely built-in primitives of distributed platforms.
Hence, their implementation on a distributed platform requires synchronization protocols responsible for realizing global synchronizations
using simpler primitives such as point-to-point messages passing as explained in~\cite{bip-par}. 
This is classically implemented using one or more additional coordination component(s) observing the system state and deciding on interactions execution.
However, due to communication delays, to meet timing constraints of components, scheduling decisions must be taken before actual executions.

This motivates the introduction of the \emph{\lpsb}, which distinguishes between the execution decision of an interaction (its \emph{planning}), and the execution itself.
The delay between the planning of an interaction and its execution is constrained by the (maximal) communication latency induced by the execution platform, which is a parameter of the semantics.
In the proposed semantics, interaction planning is based only on the states of its participating components, which allows to decide locally without monitoring the entire system.
It is correct in the sense that it refines (it is included in) the semantics of Section~\ref{sec2}.
However, being based on local states, planning decisions are too permissive and may introduce deadlocks when they are not compatible with the global state of the system.

\subsection{Definition of the LPS}
\label{subsec:wp} 
Let $S=\gamma(B_1,\cdots,B_n)$ be a composition of components $B_1$, \ldots, $B_n$ as defined in Section~\ref{sec2}.
We define the predicate $\plnIntxt{\alpha}{\delta}$ characterizing states $(\loc, \val)$
from which an interaction $\alpha = \{ a_i \}_{i \in I} \in \gamma$ is enabled in $\delta \in \realposz$ units of time (if time progresses by $\delta$ units of time),
that is, such that $\enabled(\alpha)$ evaluates to true on state $(\loc, \val + \delta)$.
It is characterized by:
  \begin{equation}\label{eq:enf1}
    \plnIntxt{\alpha}{\delta} = \bigvee_{\substack{\loc\in\Loc\\\loc=(\loc_1,\cdots,\loc_n)}}\at{\loc} \wedge \bigwedge_{\substack{i\in I\\a_i\in\alpha}}  \big(\guard{a_i}{\loc_i} + \delta\big)
  \end{equation}
Notice that for an interaction $\alpha$ the predicate $\plnIntxt{\alpha}{\delta}$ depends only on states of components of $\p{\alpha}$, which motivates the following property.

\begin{property}\label{pt:plnIn1}
  Let $(\loc,\val)$ be a state of the composition $S$. For any interactions $\alpha,\beta\in\gamma$ such that, $(\loc,\val)\transit{\beta}_{\gamma}(\loc',\val')$
  and $\p{\alpha}\cap\p{\beta}=\emptyset$, if $\plnIntxt{\alpha}{\delta}$ holds at state $(\loc,\val)$ then it still holds at state $(\loc',\val')$.
\end{property}
This property derives directly from the fact that executing an interaction $\beta$ does not change the states of components participating in an interaction $\alpha$, provided that 
$\alpha$ and $\beta$ have disjoint sets of participating components, and thus $\plnIntxt{\alpha}{\delta}$ is not affected by the execution of $\beta$ in this case.
In the following, we say that two interactions $\alpha$ and $\beta$ \emph{conflicts} when they have common participating components, that is, when $\p{\alpha}\cap\p{\beta}\neq\emptyset$, and we write $\alpha\#\beta$.
We denote by $conf(\alpha)$ the set of interactions conflicting with $\alpha$, that is, $conf(\alpha) = \{ \beta \in \gamma \ | \ \alpha\#\beta \}$.

\begin{property}\label{pt:plnIn2}
  Let $(\loc,\val)$ and $(\loc,\val+\delta')$, with $\delta'\in\realpos$ be two states of the composition $S$. 
  For an interaction $\alpha\in\gamma$, if $\plnIntxt{\alpha}{\delta}$ holds at state $(\loc,\val)$ then $\plnIntxt{\alpha}{\delta-\delta'}$ also holds at any state $(\loc,\val+\delta')$ such that $\delta'\le\delta$.
\end{property}
This property can be found directly by writing Equation~\ref{eq:enf1} on state $(\loc,\val+\delta')$.

As previously explained, due to communication latencies induced by the platform we assume that interactions cannot be planned in $\delta$ units of time if $\delta < \hmin$, where $\hmin \in \integerpoz$ is a parameter representing the minimal \emph{planning horizon}, which should represent the upper bound communication latencies.
Notice that for the sake of simplicity, we consider a global parameter $\hmin$ but we could also assume different parameters for each interaction.
For an interaction $\alpha$ we define the predicate $\plntxt{\alpha}$ characterizing states from which $\alpha$ can be planned in a delay respecting the planning horizon $\hmin$, that is:
\begin{displaymath}
\plntxt{\alpha} \Leftrightarrow \exists \delta \geq \hmin \ . \ \plnIntxt{\alpha}{\delta},
\end{displaymath}
%or equivalently:
%\begin{equation}\label{eq:enf4}
%\bigvee_{\delta \geq \hmin} \bigvee_{ \{\loc_i \in \Loc_{a_i} \}_{i \in I} } \bigwedge_{i \in I} \at{\loc_i} \wedge \big( \guard{a_i}{\loc_i} + \delta \big).
%\end{equation}
%Notice Equation~(\ref{eq:enf4}) can be rewritten as:
It can be written formally as follows:
\begin{equation}\label{eq:pln}
  \plntxt{\alpha} = \bigvee_{\substack{\loc\in\Loc\\\loc=(\loc_1,\cdots,\loc_n)}}\at{\loc} \ \wedge \ \backhmin \Big(\bigwedge_{\substack{i\in I\\a_i\in\alpha}}\guard{a_i}{\loc_i}\Big)
%\bigvee_{ \{\loc_i \in \Loc_{i} \}_{i \in I} } \Big( \bigwedge_{i \in I} \at{\loc_i} \Big) \wedge \Big( \backhmin \big[ \bigwedge_{i \in I} \guard{a_i}{\loc_i} \big] \Big).
\end{equation}

\begin{definition}[Plan]\label{def:plan}
A plan $\pi$ is a function $\pi:\gamma \to\realposz \cup \{ +\infty \}$ defining relative times for 
executing interactions, with the convention that an interaction $\alpha$ is planned to execute in $\pi(\alpha)$ time units only if $\pi(\alpha) < +\infty$.
Plans satisfy that for any two interactions $\alpha \neq \beta$ such that $\pi(\alpha) < +\infty$ and $\pi(\beta) < +\infty$, then the interactions $\alpha$ and $\beta$ are not conflicting (i.e. $\neg(\alpha\#\beta)$).
\end{definition}
We denote by $\confl(\pi)$ the set of interactions conflicting with the plan $\pi$, i.e. $\confl(\pi) = \{ \alpha \ | \ \exists \beta \# \alpha.$ \\$ \pi (\beta) < +\infty \}$, and $\p{\pi}$ the set of components participating in interactions planned by $\pi$, i.e. $\p{\pi} = \{ B_i \ | \ \exists \alpha \ . \ \pi(\alpha) < +\infty \wedge B_i \in \p{\alpha} \}$.
%We write $(\alpha,\delta)\in\pi$ the planning of interaction $\alpha$ in $\delta$ units of 
%Notice that $min \ \pi$ corresponds closest relative execution time of interactions in the plan $\pi$, i.e. $\min \ \pi = \textnormal{ min } \{ \pi(\alpha) \ | \ \alpha \in \gamma \}$.
Notice that since $\pi$ stores relative times, whenever time progresses by $\delta$, the value $\pi(\alpha)$ assigned by $\pi$ to an interaction $\alpha$ should be decreased by $\delta$ 
until it reaches $0$, meaning that $\alpha$ has to execute.
We write $\pi-\delta$ to describe the progress of time 
over the plan, that is, $(\pi-\delta)(\alpha) = \pi(\alpha) - \delta$ for interactions $\alpha$ such that $\pi(\alpha) < +\infty$.
Similarly, $\pi [ \alpha \mapsto \delta ]$ assigns relative time $\delta$ to $\alpha$, $\alpha \notin conf(\pi)$, into existing plan $\pi$, i.e. $(\pi [ \{ \alpha \mapsto \delta ])(\beta) = \delta$ for $\beta = \alpha$, $(\pi [ \alpha \mapsto \delta ])(\beta) = \pi(\beta)$ otherwise.

\begin{definition}[Local Planning Semantics]\label{def:pln_sem}
Given a set of components $\{B_1,\cdots,B_n\}$ and an interaction set $\gamma$,
we define the \lps (LPS) of the composition $\gamma(B_1,\cdots,B_n)$,
as the LTS $LTS_p=(\Q_p,\gamma\cup\realpos\cup(\gamma\times\realpoz),\tranbp{}{3})$ where:
\begin{itemize}
  \item $\Q_p=\Loc\times\mathcal{V}(\X)\times\Pi$, where $\Loc$ is the set of global locations,
    $\mathcal{V}(\X)$ is the set of global clock valuation, and $\Pi$ is the set of plans.
    Again, in the following to simplify notations predicates defined on states $(\loc,\val) \in \Q_g=\Loc\times\mathcal{V}(\X)$ of the standard semantics are straightforwardly interpreted on states $(\loc,\val,\pi) \in \Q_p$ considering the projection $(\loc,\val)$ of $(\loc,\val,\pi)$ on $\Q_g$.
  \item $(\gamma\times\realpoz)$ defines the action of planning interactions of $\gamma$ and their relative times.
  \item $\tranbp{}{3}$ is the set of labeled transitions defined by the rules:
    \begin{itemize}%[leftmargin=0em]
      \item $(\loc,\val,\pi)\tranbp{(\alpha,\delta)}{4}(\loc,\val,\pi [ \alpha\mapsto\delta ])$ for $\alpha \in \gamma$ and $\delta \geq \hmin$ if $\alpha \notin\confl(\pi)$ and $\plnIntxt{\alpha}{\delta}$ holds on $(\loc,\val,\pi)$.

    \item $(\loc,\val,\pi)\tranbp{\alpha}{3}(\loc',\val', \pi [ \alpha\mapsto+\infty ])$ for $\alpha \in \gamma$ if $\pi(\alpha) = 0$.

    \item $(\loc,\val,\pi)\tranbp{\delta}{3}(\loc,\val+\delta,\pi-\delta)$ for $\delta \le \min \ \pi$, $\loc = (\loc_1, \ldots, \loc_n)$, if $\tpc{\loc_i}(\val + \delta)$ for components $B_i \in \p{\pi}$ and $\tpc{\loc_i}(\val + \delta + \hmin)$ for components $B_i \notin \p{\pi}$.
  \end{itemize}
  \end{itemize}
\end{definition}

States of the \lpsabr do not include only locations and clocks valuation, but also the relative execution times of the planned interactions stored by $\pi$.
Initially, no interaction is planned, that is, initial states $(\loc_0,\val_0,\pi_0)$ satisfy $\pi_0 = +\infty$.
Planning an interaction $\alpha$ to be executed at a relative time $\delta \geq \hmin$ corresponds to the operation $\pi [ \alpha\mapsto\delta ]$ on the plan, which can only be done if $\alpha$ is not conflicting with the plan, and becomes enabled if time progresses by $\delta$ (i.e. if $\plnIntxt{\alpha}{\delta}$).
On the other hand, time progress not only updates clocks value but also the plan by decreasing the relative execution times of the planned interactions.
To force the execution of planned interactions when their relative execution times reach $0$, time cannot progress more than the relative execution times of the interactions (more than $\delta \le \min \ \pi$).
As for the standard semantics, time progress is limited by the time progress conditions of the components, but with the following significant difference:
Components $B_i \in \p{\pi}$ participating in planned interactions behave as in the standard semantics, that is, time can progress until their time progress conditions expire.
For components $B_i \notin \p{\pi}$ that are not participating in planned interactions, we take into account the minimal delay $\hmin$ needed for planning and then executing an interaction: in components $B_i \notin \p{\pi}$ time can progress only up to $\hmin$ time units before their time progress conditions expire.
By doing so we ensure that there always remains enough time to plan interactions involving $B_i \notin \p{\pi}$ , if they exist, and execute them before their time progress conditions expire.
 
\begin{example}
  \label{exp:dl}
  Let us consider the following execution sequence for example of Figure~\ref{fig:run} under the LPS with $\hmin = 2$. 
\begin{displaymath}
    \begin{split}
      &((\loc_0^1,\loc_0^2,\loc_0^3,\loc_0^4),(0,0,0),+\infty)\tranbp{(\alpha_1,26)}{6}((\loc_0^1,\loc_0^2,\loc_0^3,\loc_0^4),(0,0,0),\{\alpha_1\mapsto26\})\tranbp{26}{3}\\
      &((\loc_0^1,\loc_0^2,\loc_0^3,\loc_0^4),(26,26,26),\{\alpha_1\mapsto0\})\tranbp{\alpha_1}{3}((\loc_1^1,\loc_1^2,\loc_0^3,\loc_0^4),(26,26,26),+\infty)\\
      &\tranbp{(\alpha_3,2)}{6}((\loc_1^1,\loc_1^2,\loc_0^3,\loc_0^4),(26,26,26),\{\alpha_3\mapsto2\})\tranbp{2}{3}((\loc_1^1,\loc_1^2,\loc_0^3,\loc_0^4),(28,28,28),\\
      &\{\alpha_3\mapsto0\})\tranbp{\alpha_3}{3}((\loc_0^1,\loc_2^2,\loc_0^3,\loc_0^4),(0,28,0),+\infty)\tranbp{(\alpha_2,26)}{6}((\loc_0^1,\loc_2^2,\loc_0^3,\loc_0^4),\\
      &(0,28,0),\{\alpha_2\mapsto26\})\tranbp{26}{3}((\loc_0^1,\loc_2^2,\loc_0^3,\loc_0^4),(26,54,26),\{\alpha_2\mapsto0\})\tranbp{\alpha_2}{3}\\
      &((\loc_1^1,\loc_2^2,\loc_1^3,\loc_0^4),(26,54,26),+\infty)\tranbp{(\alpha_4,2)}{6}((\loc_1^1,\loc_2^2,\loc_1^3,\loc_0^4),(26,54,26),\{\alpha_4\mapsto2\})\\
      &\tranbp{2}{3}((\loc_1^1,\loc_2^2,\loc_1^3,\loc_0^4),(28,56,28),\{\alpha_4\mapsto0\})\tranbp{\alpha_4}{3}((\loc_0^1,\loc_2^2,\loc_2^3,\loc_0^4),(28,0,0),+\infty)\\
      &\tranbp{(\alpha_6,30)}{6}((\loc_0^1,\loc_2^2,\loc_2^3,\loc_0^4),(28,0,0),\{\alpha_6\mapsto30\}) 
    \end{split}
  \end{displaymath}
This execution sequence represents a path that alternates plan actions, time progress and execution of some interactions, and leads to the action-time-lock state
$((\loc_0^1,\loc_2^2,\loc_2^3,\loc_0^4),(0,0,28),\{\alpha_6\mapsto30\})$. In fact, the time progress condition $x\leq30$ in component $T_1$, imposes the 
planning of interaction $\alpha_7$ at the latest $h_{min}$ units of time before it becomes urgent. However, since interaction $\alpha_6$ was planned in
28 units of time, $\alpha_7$ cannot be planned since it is conflicting with $\alpha_6$.
This execution sequence shows that a given system action-time-locks under the \lps, even if it is deadlock-free in the standard semantics. 
\end{example}

\subsection{Properties of the LPS}\label{subsec:planprop}
We use weak simulation to compare the model under
the standard semantics and the local planning semantics
by considering the planning transitions unobservable.
As shown in Example~\ref{exp:dl}, the \lpsabr does not preserve the deadlock freedom property of our system.
Nevertheless, the following proves weak simulation relations between the two semantics.

\begin{lemma}\label{lem:pi_pln}
  Given a reachable state $(\loc,\val,\pi)$ of the \lpsabrb. If for $\alpha\in\gamma$, $\pi(\alpha) < +\infty \Rightarrow \plnIntxt{\alpha}{\pi(\alpha)}$.
\end{lemma}

%Let $LTS_g=(\Q_g,\gamma\cup\realpos,\transit{}_{\gamma})$ (resp. $LTS_{p}=(\Q_{p},\gamma\cup\realpos\cup\{\plana\},\tranbp{}{3})$)
%the labeled transition system characterizing the global (resp. planning) semantics.

\begin{proposition}\label{prop:r1}
An interaction can execute from a state $(\loc,\val,\pi)$ in the \lpsabr semantics only if it can execute from $(\loc,\val)$ in the standard semantics, that is:
\begin{displaymath}
      \forall\alpha\in\gamma.(\loc,\val,\pi)\tranbp{\alpha}{3}(\loc',\val',\pi')
      \Rightarrow (\loc,\val)\transit{\alpha}_{\gamma}(\loc',\val').
\end{displaymath}
\end{proposition}

Proposition~\ref{prop:r1} is a consequence of Lemma~\ref{lem:pi_pln}: an interaction $\alpha$ can execute in the \lps only if $\pi(\alpha) = 0$ (see Definition~\ref{def:plan}).
That is, a state $(\loc,\val,\pi)$ of the \lpsabr from which $\alpha$ can execute satisfies $\plnIntxt{\alpha}{0}$ or equivalently $\enabled(\alpha)$, which demonstrates that $\alpha$ can execute from $(\loc,\val)$ in the standard semantics.

\begin{proposition}\label{prop:r2}
Time can progress by $\delta$ at a state $(\loc,\val,\pi)$ in the \lps only if time can progress by $\delta$ at $(\loc,\val)$ in the standard semantics, that is:
\begin{displaymath}
      \forall\delta\in\realpos.(\loc,\val,\pi)\tranbp{\delta}{3}(\loc',\val',\pi')
      \Rightarrow (\loc,\val)\transit{\delta}_{\gamma}(\loc',\val').
\end{displaymath}
\end{proposition}

Proposition~\ref{prop:r2} is a direct consequence of the definition of time progress in the \lps which is a restriction of the one in the standard semantics.

\begin{corollary}\label{cr:reach}
If a state $(\loc,\val,\pi)$ is reachable in the \lpsb, then the state $(\loc,\val)$ is reachable in the standard semantics.
\end{corollary}

Corollary~\ref{cr:reach} is obtained from Propositions~\ref{prop:r1} and~\ref{prop:r2} and the fact that planning transitions (labeled by $(\alpha,\delta)$) affect only the plan $\pi$ in states $(\loc,\val,\pi)$ of the \lpsabr.

\begin{definition}[Weak Simulation]
  A weak simulation over $A=(\Q_A,\sum\cup\{\beta\},\to_A)$ and $B=(\Q_B,\sum\cup\{\beta\},
  \to_B)$ is a relation $R\subseteq \Q_A\times \Q_B$ such that we have: 
  $\forall(q,r)\in R, a\in \sum .q\transit{a}_A q' \implies\exists r':(q',r')\in R\wedge r\transit{
  \beta^*a\beta^*}_B r' \text{ and } \forall(q,r)\in R: q\transit{\beta}_Aq'\implies\exists r':
  (q',r')\in R\wedge r\transit{\beta^*}r'$.
  B simulates A, denoted by $A\simu{R}B$, means that B can do everything A does.
\end{definition}
The definition of weak simulation is based on the unobservability of $\beta-$transitions. In our case, $\beta-$transitions corresponds to planning transitions.
\begin{corollary}\label{cr:sim}
  $LTS_p\simu{R} LTS_g$ with $R=\{(\q,\pi);\q)\in\Q_p\times\Q_g\}$.
\end{corollary}

Corollary~\ref{cr:sim} corresponds to a notion of correctness of the \lpsb: any execution in the \lpsabr corresponds to an execution in the standard semantics.
In addition, if interactions are allowed to be planned with relative execution times of $0$ (i.e. $\hmin = 0$) then the planning semantics simulates the standards semantics~\cite{FM:plan}.
However, this is no longer true in general if  $\hmin > 0$ which means that not all execution sequences of the standard semantics are preserved by the \lps.
Notice that the \lps also preserves non zenoness of the standard semantics
(by Corollary~\ref{cr:sim} and the fact that it is not possible to have infinite sequences of planning transitions without interaction execution).

\begin{proposition}\label{prop:deadlock-timelock}
For a given composition $\gamma(B_1,\cdots,B_n)$, if it is deadlock-free under the standard semantics then it is deadlock-free under the \lps if and only if it is action-time-lock free.
\end{proposition}
\begin{proof}[Proof of Propostion~\ref{prop:deadlock-timelock}]
  We prove Proposition~\ref{prop:deadlock-timelock} by contradiction.
  Let us assume that the system under the standard (resp. local planning) semantics is
  deadlock free (resp. action-time-lock free).
  Let $(\loc,\val,\pi)$ be a reachable deadlock state of the \lpsabr. We have:
  \begin{displaymath}
    \nexists\sigma\in\gamma\cup(\gamma\times\realpoz),\exists\delta.\ (\loc,\val,\pi)\tranbp{\sigma}{3}(\loc',\val',\pi')
    \vee(\loc,\val,\pi)\tranbp{\delta}{3}(\loc,\val+\delta,\pi-\delta)\tranbp{\sigma}{3}
    (\loc',\val',\pi')
  \end{displaymath}

  We denote by $wait(\loc,\val,\pi)$ the set of allowed waiting times at state $(\loc,\val,\pi)$, that is:
\begin{displaymath}
  wait(\loc,\val,\pi)=\{0\}\cup\{\delta\in\realpos|(\loc,\val,\pi)\tranbp{\delta}{3}(\loc,\val+\delta,\pi-\delta)\}
\end{displaymath}
We also put $\max(wait(\loc,\val,\pi))$ to denote the maximal waiting time at state $(\loc,\val,\pi)$.
\begin{lemma}\label{lemma:wait}
  Let $(\loc,\val,\pi)$ be a reachable state of the \lpsb. For $k\in\realpoz$, such
  that $k=\max(wait(\loc,\val,\pi))$, we have the following properties:
  \begin{description}[labelwidth=1.5cm]
    \item[\namedlabel{p1}{P1}] If $k<+\infty$ then $(\loc,\val,\pi)\tranbp{k}{3}(\loc,\val+k,\pi-k)\wedge wait(\loc,\val+k,\pi-k)=\{0\}$
    \item[\namedlabel{p2}{P2}] If $\pi\neq+\infty$ then $k<+\infty\wedge k\le\min\pi$
  \end{description}
 
\end{lemma}

We distinguish 2 cases:
\paragraph*{Case 1: no interaction was planned (i.e. $\pi = +\infty$)}
By definition of the \lpsabrb, it is clear that for $\pi=+\infty$, there is no interaction to execute from
$(\loc,\val,\pi)$ or any of its successor $(\loc,\val+\delta,\pi-\delta)$.
\begin{enumerate}
  \item $wait(\loc,\val,\pi)=\{0\}$:\\
    This means that time progress is not allowed at state $(\loc,\val,\pi)$. We also have
    $\nexists\sigma\in(\gamma\times\realpoz).(\loc,\val,\pi)\tranbp{\sigma}{3}(\loc',\val',\pi')$ (deadlock assumption). We can conclude
    that $(\loc,\val,\pi)$ is a reachable action-time-lock state, which contradicts the assumption that the system under the \lps is action-time-lock free.
  \item $wait(\loc,\val,\pi)\neq\{0\}$:
    \begin{enumerate}
      \item $\max(wait(\loc,\val,\pi))=+\infty$:\\
        \begin{lemma}\label{lemma:deadlock}
        Let $(\loc,\val,\pi)$ be a reachable state of the \lps. 
        If $ \ \forall\delta\in\realpos. \ (\loc,\val,\pi)\tranbp{\delta}{3}(\loc,\val+\delta,\pi-\delta)\wedge\neg\plntxt{\alpha}$ at $(\loc,\val,\pi)$,
        then we have $\neg\enabled(\alpha)$ at $(\loc,\val+\delta,\pi-\delta)$ with $\delta\ge h_{\min}$.
      \end{lemma}
        %Since the time progress conditions of components are restricted to constraints of the form $x\le k$,
        %we can deduce that $\forall i\in\{1,\cdots,n\}.\tpc{\loc_i}(v)=\true$ ($wait(\loc,\val,\pi)=+\infty$). i
      By~\ref{p1} of Lemma~\ref{lemma:wait} we can deduce that $\exists\delta\ge h_{\min}$ such that $(\loc,\val,\pi)\tranbp{\delta}{3}(\loc,\val+\delta,\pi-\delta)$. 
      We also have from the deadlock assumption and Lemma~\ref{lemma:deadlock}:
        $\bigwedge_{\alpha\in\gamma}\neg\enabled(\alpha)$. Finally, since the state $(\loc,\val+\delta,\pi-\delta)$ is reachable in the standard
        semantics, and by evaluating the deadlock characterization~\ref{eq:standard_dl} on state $(\loc,\val+\delta,\pi-\delta)$, 
        we can conclude that the system under the standard semantics deadlocks, which contradicts the assumption of deadlock freedom of the system under the standard semantics.
      \item $\max(wait(\loc,\val,\pi))<+\infty$:\\
        Considering that $k=\max(wait(\loc,\val,\pi))$, then we have by~\ref{p1} of Lemma~\ref{lemma:wait}:
        $(\loc,\val,\pi)\tranbp{k}{3}(\loc,\val+k,\pi-k)\wedge wait(\loc,\val+k,\pi-k)=\{0\}$.
    Using the deadlock assumption we have: $\bigwedge_{\alpha\in\gamma}\neg\plntxt{\alpha}$ at state $(\loc,\val+k,\pi-k)$.
        Since the system cannot progress beyond this state ($wait(\loc,\val+k,\pi-k)=\{0\}$), we can conclude that $(\loc,\val+k,\pi-k)$ is 
    a reachable action-time-lock state, which contradicts the assumption that the system under the \lps is action-time-lock free.
      
    \end{enumerate}
\end{enumerate}

\paragraph*{Case 2: at least an interaction was planned (i.e. $\pi \neq +\infty$)}
Considering that $k=\max(wait(\loc,\val,\pi))$, since $\pi\neq+\infty$, we have by~\ref{p2} of Lemma~\ref{lemma:wait}:
$k<+\infty\wedge k\le\min\pi$. Using the deadlock assumption we can infer that $k<\min\pi$, since 
no execution is possible from $(\loc,\val,\pi)$ or any of its successors. 
This means that $(\loc,\val+k,\pi-k)$ is a reachable action-time-lock state, which contradicts the assumption that the system under the \lpsabr is action-time-lock free.
\end{proof}

\begin{proposition}\label{prop:timelocks}
A state $(\loc,\val,\pi)$ of the \lps is an action-time-lock if and only if:
\begin{displaymath}
  \pi>0 \ \wedge \bigwedge_{\alpha \notin conf(\pi)} \hspace*{-2ex} \neg\plntxt{\alpha} \ \wedge
  \hspace*{-1ex} \bigvee_{\substack{\loc_i \in \Loc_i \\ B_i \notin \p{\pi}}} \hspace*{-2ex} \at{\loc_i} \ \wedge (\urg(\loc_i) + \hmin).
\end{displaymath}
\end{proposition}
The above proposition derives directly from the definition of action-time-locks on a state of the \lps. 


As shown in Example~\ref{exp:dl}, the \lps may also introduce deadlocks.
The source of deadlocks is twofold: \emph{(i)} due to communication delays, consecutive execution in a component are separated by at least $\hmin$ units of time which may be incompatible with its timings constraints, and \emph{(ii)} conditions for planning interactions are too permissive as they only take into account timing constraints of participating components whereas they may block additional components, namely the ones participating in conflicting interactions.
In the rest of the paper, we study how to generate planning strategies for preserving deadlock-freedom by restricting the planning transitions of the \lpsabr so that deadlock states become unreachable.
Such a strategy may not exist when timing constraints cannot accommodate with the communication delays $\hmin$.
Notice that by Proposition~\ref{prop:deadlock-timelock}, it is sufficient to focus on the action-time-locks of the \lpsabr for systems that are initially deadlock-free.

\section{Enforcing Deadlock-Free Planning}
\label{sec4}
%Effectively, local planning of interactions consists in applying a local time step, followed
%by the execution of an interaction. However, even if local time steps are allowed in planned components,
%it may be disallowed in the rest of the system. Moreover, since planning horizons 
%insert a certain latency between interactions, models execution under the presented semantics 
%can result in missing interactions deadlines in some cases.
%Therefore, the planning semantics may exhibit deadlock situations even if the system under the global state semantics,
%is deadlock-free.
As explained in previous section, the \lps is based on local conditions for planning interactions and may exhibit deadlocks even when the system is deadlock-free with the standard semantics.
Such deadlocks are partly due to the fact that planning an interaction may block, in addition to the participating components, extra components whose timing constraints are not considered by these local conditions.
In this section, we investigate simple execution strategies that only restrict the horizon used for planning interactions with upper bounds.
By reducing the period of time during which components are blocked, they tend to remove deadlocks from the reachable states.
In addition, we provide sufficient conditions for checking deadlock-freedom of the \lpsabr subject to upper bounded horizons.
In what follows, we consider a composition of components $S = \gamma(B_1,\cdots,B_n)$ such that it is deadlock-free in the standard semantics.
\subsection{Planning with Upper Bounded Horizon}
We slightly modify the \lpsabr of Definition~\ref{def:pln_sem} of Section~\ref{subsec:wp} by considering
upper bounds $\hmax : \gamma \to \integerpoz \cup \{ +\infty \}$ for interactions such that for any $\alpha\in\gamma$ we have $\hmax(\alpha) \geq \hmin$.
In the modified semantics, interactions $\alpha$ can be planned only
using planning horizon $\delta$ satisfying $\hmin \leq \delta \leq \hmax(\alpha)$.
Notice that the \lps as presented in Section~\ref{sec2} corresponds to the choice $\hmax(\alpha) = +\infty$ for all interactions $\alpha \in \gamma$.
The predicate $\plntxt{\alpha}$ introduced in Section~\ref{subsec:wp} and characterizing states from which $\alpha$ can be planned in a delay respecting the constraints on the planning horizon becomes:
\begin{displaymath}
\plntxt{\alpha} \Leftrightarrow \exists \delta \in \realpoz \ . \ \hmin \leq \delta \leq \hmax(\alpha) \ \wedge \ \plnIntxt{\alpha}{\delta},
\end{displaymath}
more formally:
\begin{displaymath}
  \plntxt{\alpha} = \bigvee_{\substack{\loc\in\Loc\\\loc=(\loc_1,\cdots,\loc_n)}}\at{\loc} \ \wedge \ \backhminhmax{\alpha} \Big(\bigwedge_{\substack{i\in I\\a_i\in\alpha}}\guard{a_i}{\loc_i}\Big)
\end{displaymath}

It can easily be shown that all results of Section~\ref{subsec:planprop} also apply to this variant of the \lpsabrb.
The characterization of action-time-locks of Proposition~\ref{prop:timelocks} is still valid provided $\plntxt{\alpha}$ is computed using its refined version presented above.
In the rest of the section, we always consider this modified version of the \lpsb.

\subsection{Checking Deadlock-Freedom}
As explained in Section~\ref{subsec:planprop}, action-time-lock-freedom is a sufficient condition for deadlock-freedom of the \lpsabrb.
By Proposition~\ref{prop:timelocks}, a state $(\loc,\val,\pi)$ is an action-time-lock in the \lps if and only if 
no time progress is allowed nor planning or execution of interactions from $(\loc,\val,\pi)$, that is:
\begin{displaymath}
  \pi>0 \ \wedge \ \bigwedge_{\alpha \in \gamma \setminus conf(\pi)} \hspace*{-2ex} \neg\plntxt{\alpha} \ \wedge
  \hspace*{-1ex} \bigvee_{\substack{\loc_i \in \Loc_i \\ B_i \notin \p{\pi}}} \hspace*{-2ex} \at{\loc_i} \ \wedge(\urg(\loc_i) + \hmin).
\end{displaymath}
The above predicate characterizes the fact that no interaction can be executed or planned, nor time can progress in component $B_i \notin \p{\pi}$.
Consequently, we deduce that a necessary condition of action-time-lock is the existence of a component $B_i \notin \p{\pi}$ such that time cannot progress in $B_i$ and $B_i$ cannot be planned in an interaction, that is:
\begin{displaymath}
  \bigwedge_{\substack{\alpha \in \gamma(B_i) \setminus conf(\pi)}} \hspace*{-4ex} \Big(\neg\plntxt{\alpha} \ \wedge \ \bigvee_{\loc_i \in \Loc_i} \hspace*{-1ex} \at{\loc_i} \ \wedge(\urg(\loc_i) + \hmin)\Big).
\end{displaymath}
where $\gamma(B_i)$ denotes the subset of interactions in which $B_i$ participates, that is, $\gamma(B_i) = \{ \beta \in \gamma \ | \ B_i \in \p{\beta} \}$.
Notice that the above equation strongly depends on the plan $\pi$, which is difficult to characterize in practice.
The following theorem proposes sufficient plan-independent condition characterizing action-time-lock states of the \lpsabrb.
\begin{theorem}\label{thm:dla}
  Let $\phi$ be the following predicate:
\begin{displaymath}
  \bigvee_{1\le i\le n}\Big[\bigvee_{\loc_i \in \Loc_i}\at{\loc_i}\ \wedge(\urg(\loc_i)+\hmin) \ \wedge\bigwedge_{\alpha \in \gamma(B_i)}\hspace*{-1ex}\Big(  \neg\plntxt{\alpha} \
  \vee\hspace*{-2ex}\bigvee_{\substack{\beta\in\confl(\alpha)\\B_i\notin\p{\beta}}}\hspace*{-1ex}\overline{\plntxt{\beta}} \Big)\Big].
\end{displaymath}
We prove that a reachable action-time-lock state $(\loc,\val,\pi)$ satisfies $\phi$.
\end{theorem}
\begin{proof}[Proof of Theorem~\ref{thm:dla}]
A reachable action-time-lock state of the \lpsabr satisfies:
\begin{displaymath}
  \pi>0 \ \wedge \ \bigwedge_{\substack{\alpha \in \gamma(B_i) \setminus conf(\pi)}} \hspace*{-4ex} \Big(\neg\plntxt{\alpha} \ \wedge \ \bigvee_{\substack{\loc_i \in \Loc_i \\ B_i \notin \p{\pi}}} \hspace*{-1ex} \at{\loc_i} \ \wedge(\urg(\loc_i) + \hmin)\Big).
\end{displaymath}

In order to approximate the above equation, we distinguish two cases:
\paragraph*{Case 1: no interaction was planned (i.e. $\pi = +\infty$)\\}
From $\pi = +\infty$ we deduce directly that there exists an urgent component $B_i$ such that no interaction $\alpha$
involving $B_i$ can be planned, that is:
\begin{equation}
  \label{eq:case1}\tag{1}
  \bigvee_{\substack{1\le i\le n}}\Big[\bigvee_{\loc_i \in \Loc_i}\at{\loc_i}\ \wedge(\urg(\loc_i)+\hmin) \ \wedge\bigwedge_{\substack{\alpha \in \gamma(B_i)}} \hspace*{-1ex} \neg\plntxt{\alpha}\Big].
\end{equation}

\paragraph*{Case 2: at least an interaction was planned (i.e. $\pi \neq +\infty$)\\}
In this case, there exists an urgent component $B_i \notin \p{\pi}$ such that no interaction $\alpha$ involving $B_i$ can be planned, either because
it conflicts with a planned interaction $\beta$ ($0<\pi(\beta)<+\infty$) or because $\plntxt{\alpha}$ is not satisfied, that is $\exists\beta\in\pi,\exists B_i\notin\p{\beta}$ satisfying:
\begin{displaymath}
  (0<\pi(\beta)<+\infty) \ \wedge\bigwedge_{\substack{\alpha \in \gamma(B_i) \setminus conf(\beta)}} \hspace*{-4ex} \neg\plntxt{\alpha} \ \wedge \ \bigvee_{\substack{\loc_i \in \Loc_i \\ B_i \notin \p{\beta}}} \hspace*{-1ex} \at{\loc_i} \ \wedge(\urg(\loc_i) + \hmin).
\end{displaymath}
or equivalently $\exists\beta\in\pi,\exists B_i\notin\p{\beta}$ satisfying:
\begin{displaymath}
  \bigvee_{\substack{\loc_i \in \Loc_i \\ B_i \notin \p{\beta}}} \hspace*{-1ex} \at{\loc_i} \ \wedge(\urg(\loc_i) + \hmin) \ \wedge\bigwedge_{\alpha \in \gamma(B_i) }
 \Big(\neg\plntxt{\alpha} \vee \big(\beta\in\confl(\alpha)\wedge (0<\pi(\beta)<+\infty)\big)\Big) .
\end{displaymath}
By noticing that we have the following implication between quantifiers 
$\exists y,\forall x. Q(x,y)\implies\forall x,\exists y. Q(x,y)$, we can deduce that the 
above condition implies:
\begin{displaymath}
  \bigvee_{1\le i\le n}\Big[\bigvee_{\loc_i \in \Loc_i}\at{\loc_i}\ \wedge(\urg(\loc_i)+\hmin) \ \wedge\bigwedge_{\alpha \in \gamma(B_i)}\hspace*{-1ex}\Big(  \neg\plntxt{\alpha} \
  \vee\hspace*{-2ex}\bigvee_{\substack{\beta\in\confl(\alpha)\\B_i\notin\p{\beta}}}\hspace*{-1ex}
0<\pi(\beta)<+\infty \Big)\Big].
\end{displaymath}


As $\pi > 0$, and if we consider only reachable action-time-locks, we have $0 < \pi(\beta) \leq \hmax(\beta)$, and by Lemma~\ref{lem:pi_pln} we have $\plnIntxt{\beta}{\pi(\beta)}$.
That is, $\beta$ satisfies  $\plntxt{\beta}$ in which the lower bound $\hmin$ is replaced by the strict lower bound 0, i.e.:
\begin{displaymath}
\overline{\plntxt{\beta}} \Leftrightarrow \exists \delta > 0\ . \ \delta \leq \hmax(\beta) \ \wedge \ \plnIntxt{\beta}{\delta}.
\end{displaymath}
Then, the above equation becomes:
\begin{equation}
  \label{eq:case2}\tag{2}
  \bigvee_{1\le i\le n}\Big[\bigvee_{\loc_i \in \Loc_i}\at{\loc_i}\ \wedge(\urg(\loc_i)+\hmin) \ \wedge\bigwedge_{\alpha \in \gamma(B_i)}\hspace*{-1ex}\Big(  \neg\plntxt{\alpha} \
  \vee\hspace*{-2ex}\bigvee_{\substack{\beta\in\confl(\alpha)\\B_i\notin\p{\beta}}}\hspace*{-1ex}\overline{\plntxt{\beta}} \Big)\Big].
\end{equation}

By remarking that Equations~\ref{eq:case1} implies Equation~\ref{eq:case2}, we can conclude that an
action-time-lock of the \lps satisfies:
\begin{displaymath}
  \bigvee_{1\le i\le n}\Big[\bigvee_{\loc_i \in \Loc_i}\at{\loc_i}\ \wedge(\urg(\loc_i)+\hmin) \ \wedge\bigwedge_{\alpha \in \gamma(B_i)}\hspace*{-1ex}\Big(  \neg\plntxt{\alpha} \
  \vee\hspace*{-2ex}\bigvee_{\substack{\beta\in\confl(\alpha)\\B_i\notin\p{\beta}}}\hspace*{-1ex}\overline{\plntxt{\beta}} \Big)\Big].
\end{displaymath}
\end{proof}

Notice that due to the monotony of the condition $\phi$ on upper bound horizons~\cite{FM:plan}, we obtain the following lemma: 
\begin{lemma}
  Let $LTS_p=(\Q_p,\gamma\cup\realpos\cup(\gamma\cup\realpoz),\tranbp{}{3})$ be the LTS of the composition $\gamma(B_1,\cdots,B_n)$ under the 
  \lpsabrb. If $LTS_p$ is action-time-lock free for the upper bounds horizon function $\hmax$, then it is action-time-lock free
  for any upper bound horizon function $\hmax'\le\hmax$.
\end{lemma}

In order to attest that planning interactions does not introduce deadlocks,
we use an SMT solver to check the satisfiability of $\phi$.
As explained earlier, a given system is deadlock-free under
the restricted \lpsabr if $Reach(LTS_p)\wedge\phi$ is unsatisfiable. Since
$Reach(LTS_p)\subset Reach(LTS_g)$ (Corollary~\ref{cr:sim}), we can verify the above on $Reach(LTS_g)$.

\noindent Effectively, we do not compute $Reach(LTS_g)$ to avoid the combinatorial explosion problem,
inherent to composition of timed automata. In fact, we rather build an over-approximation,
$\reacha$ of the latter, and use it during our verification.
The computed over-approximation combines the reachable states of individual components.
However, in general not all combinations are reachable since components are not fully independent 
and may synchronize together through interactions. Moreover, individual invariants alone do not
express the fact that time progresses the same way in components.
In order to capture these additional information, we use different kind of invariants:
\emph{(1)} linear invariants~\cite{inv-lin} that capture the constraints on location 
configurations of the components induced by their synchronizations (interactions), and
\emph{(2)} interaction inequalities for history
clocks~\cite{inv-comp}, allowing to relate the local constraints obtained individually on
components, as well as separation constraints for interaction~\cite{inv-comp} that describe
dis-equalities (or separations) constraints between interactions.

\section{Planning Semantics as Real-Time Controller Synthesis}
\label{sec5}

In Section~\ref{sec5}, we presented a method that provides execution strategies
by restricting the upper bounds planning horizon for each interaction.
Since the given approach checks a sufficient condition for deadlock-freedom,
it may give false-positive results, that is, it will find a state verifying condition
$\phi$ of Theorem~\ref{thm:dla}
but which is not a deadlock state. In such cases, an alternative is to tackle the problem
as a real-time controller synthesis problem.
Real-time controller synthesis is a common method used to extract an execution strategy from
formal specifications satisfying certain properties. Usually, these properties express the
reachability (resp. non-reachability) of a set of winning states (resp. bad states).
In case of planning interactions with bounded horizons, the idea is to restrict the transition relation so that all the remaining 
behaviors do not lead to states where a component is urgent and no possible execution
including this component may occur. This can be formalized as a reachability game for a timed
game automaton~\cite{tiga:alg}, where the main idea consists in trying to find an execution 
strategy guaranteeing that a given set of namely \emph{bad states} of the system are never
reached.

In order to apply this approach, it is required to encode the planning of interactions and their effects 
on the system, that is, \emph{(i)} encode interactions planning as synchronizations between 
components, \emph{(ii)} reserve the components of the planned interactions until their chosen execution
date, i.e, keep track of the planned interactions and their execution dates,
and \emph{(iii)} characterize the set of bad states. 
Thereafter, tools such as UPPAALL-Tiga~\cite{tiga} can be used to find an execution strategy of 
the planning semantics avoiding the set of bad states, that is, deadlock states.
Expressing the planning problem as a real-time controller synthesis problem is not 
an easy task. Hereinafter, we discuss the different issues met during the formalization 
process and provide suggestions for solving them.
\subsection{Planning Zones}

From Equation~\ref{eq:pln}, we can see that the clocks values for planning an interaction $\alpha$
are calculated at a global level, that is, by applying the $\backhminhmax{\alpha}$ on the conjunction
of its participating actions timing constraints.
Notice that for a timing constraint $g=g_1\wedge g_2$, we have:
\begin{equation*}
  \backhminhmin g=\backhminhmin(g_1\wedge g_2) \not\equiv\backhminhmin g_1\wedge\backhminhmin g_2 
\end{equation*}

The above equation bears out the fact that planning states must be encoded on the composition
of the system model and not on individual components. Therefore, a simple solution to avoid 
computing the composition is to consider
models with interactions having timing constraints on up to one of their participating
actions. In fact, considering interactions including up to one action with timing constraints, will
allow to encode the planning on individual components that, additionally to the defined 
synchronizations (interactions), will also synchronize their planning actions. 

The idea is to split each transition of the initial model into two transitions:
\emph{(1)} a planning transition, followed by \emph{(2)} an execution transition
after the plan transition being performed. Additionally, each component is equipped with an additional clock $x_p$, 
that will be used to track the planning dates.
Finally, time progress conditions must also be translated to enforce planning
at the latest $\hmin$ units of time before their expiry.
Figure~\ref{fig:enc} depicts an overview of such transformation for $\delta=\hmin$ horizon: 
\begin{figure}[!h]
  \centering
  \subfloat[Part of a timed automaton]{
    \begin{tikzpicture}[->,node distance=1.3cm,>=stealth',
  place/.style={circle,thick,draw=black,minimum size=7mm},
  initial text={},
baseline]
    \node [place,initial, initial where=above,label={[shift={(-1,-0.6)},scale=0.8]$x\le k$}] (l1)  {$\loc_1$};
    \node [place,below=2.5cm of l1] (l2) {$\loc_2$};
      \node [left=1.5cm of l1]{};
      \node [right=1.5cm of l1]{};
      \path (l1) edge node[left,scale=0.8]{$a,g_a,r_a$} (l2);
    \end{tikzpicture}}\hspace{4cm}
    \captionsetup[subfigure]{oneside,margin={1cm,0cm}}
  \subfloat[Planning encoding]{
  \begin{tikzpicture}[->,node distance=1.3cm,>=stealth',
  place/.style={circle,thick,draw=black,minimum size=7mm},
  initial text={},
baseline]
    \node [place,initial, initial where=above,label={[shift={(-1.2,-0.5)},scale=0.8]$x\le k-h_{\min}$}] (t1)  {$\loc_1$};
    \node [place,below=10mm of t1,align=center,label={[shift={(-1.5,-0.6)},scale=0.8]$x_p\le h_{\min}\wedge x\le k$},scale=0.9] (t2) {$\loc_{1}^{a}$};
    \node [place,below=7mm of t2] (t2b) {$\loc_2$};
    \path (t1) edge node[xshift=2mm,right,align=center,xshift=-2mm,scale=0.8]{$plan_a$\\$\backwardp{\hmin}{\hmin} g_a$\\$x_p:=0$} (t2)
      (t2) edge node[xshift=1mm,right,align=center,scale=0.8]{$a$\\$x_p=h_{\min}$\\$r_a$} (t2b);
  \end{tikzpicture}}
  \caption{Planning as a Timed Automaton}\label{fig:enc}
\end{figure}





\subsection{Infinite Planning Transitions} 

Effectively, in order to encode the planning in timed automata, horizons values must
be integer. Moreover, for timing constraints not restricted with upper bounds, we end up with an infinity of plan
transitions. Consequently, the first thing to do is to discretize the planning horizons in order
to obtain finite values in $\integerpos$ (Figure~\ref{fig:disc}). In what follows, we denote
by $Disc:\gamma\lto\mathcal{D}$ the discretized horizon function defining for each interaction
its respective discretized planning horizons $\mathcal{D}\subset\integerpos$.

\begin{figure*}[!h]
  \centering
  \begin{tikzpicture}
  \begin{axis}[
   axis y line=none,
   y=0.3cm,
   x=1cm,
   xtick={15,15.5,16,...,18},
   restrict y to domain=1:5,
   axis lines=left,
   enlarge x limits=upper,
   cycle list name=exotic,
   scatter/classes={
   o={mark=*}
   },
   scatter,
   scatter src=explicit symbolic,
   every axis plot post/.style={mark=*,thick},
   xlabel=Time,
   x label style={at={(axis description cs:0.95,-0.1)},anchor=south},
   xticklabels={,,},
   legend style={
      draw=none,
      at={(-0.1,-1)},
      anchor=south east
  },
  legend image post style={mark=none}
  ]
  \addplot table [y expr=1,meta index=1, header=false] {
17.5 c
18 c
};\addlegendentry{$g_{\alpha}$}
\addplot table [y expr=1,meta index=1, header=false] {
15 c
16.5 c
};\addlegendentry{$\backwardp{\hmin}{\hmx} g_{\alpha}$}
\addplot table [y expr=2,meta index=1, header=false] {
16.5 c
16 c
};\addlegendentry{$\backwardp{\hmin}{\hmin}g_{\alpha}$}
\addplot table [y expr=3,meta index=1, header=false] {
16.4 c
15.9 c
};\addlegendentry{$\backwardp{\hmin+\varepsilon}{\hmin+\varepsilon}g_{\alpha}$}
\addplot table [y expr=4,meta index=1, header=false] {
16.3 c
15.8 c
};\addlegendentry{$\backwardp{\hmin+2\varepsilon}{\hmin+2\varepsilon}g_{\alpha}$}
  \end{axis}
  \node (titile)[align=center,yshift=-1.75cm] {\footnotesize An interaction guard $g_{\alpha}$ \\ \footnotesize and its planning intervals};
\end{tikzpicture}
%  \caption{}
  %\captionsetup{justification=centering}
  \begin{tikzpicture}
  \begin{axis}[
   axis y line=none,
   y=0.3cm,
   x=1cm,  
   xtick={15,15.5,16,...,18},
   restrict y to domain=1:5,
   axis lines=left,
   enlarge x limits=upper,
   scatter/classes={
   o={mark=*,fill=white}
   },
   scatter,
   cycle list name=exotic,
   scatter src=explicit symbolic,
   every axis plot post/.style={mark=*,thick},
   xlabel=Time,
   x label style={at={(axis description cs:0.95,-0.1)},anchor=south},
   xticklabels={,,},
   legend style={
      draw=none,
      at={(2.2,-1)},
      anchor=south east
  },
  legend image post style={mark=none}
  ]
\addplot table [y expr=1,meta index=1, header=false] {
17.5 c
18 c
};\addlegendentry{$g_{\alpha}$}
\addplot table [y expr=1,meta index=1, header=false] {
15 c
16.5 c
};\addlegendentry{$\backwardp{\hmin}{\hmx} g_{\alpha}$}
\addplot table [y expr=2,meta index=1, header=false] {
16.5 c
16 c
};\addlegendentry{$\backwardp{\hmin}{\hmin}g_{\alpha}$}
\addplot table [y expr=3,meta index=1, header=false] {
16 c
15.5 c
};\addlegendentry{$\backwardp{\hmin+d}{\hmin+d}g_{\alpha}$}
\addplot table [y expr=4,meta index=1, header=false] {
15.5 c
15 c
};\addlegendentry{$\backwardp{\hmin+2d}{\hmin+2d}g_{\alpha}$}
  \end{axis}

  \node (titile)[align=center,yshift=-2cm,xshift=3cm] {\footnotesize Discretized planning intervals for $g_{\alpha}$};
  
\end{tikzpicture}
  %\caption{}
\caption{Discretizing Planning Horizons for Interaction}
\label{fig:disc}
\end{figure*}


\begin{definition}[Planning Timed Automaton]
  \label{def:plan_aut}
  Given n timed components $\tcal{B}_i=(\Loc_i,\loc_0^i,\A_i,\T_i,\X_i,\I_i)$ synchronizing through the interaction set  $\gamma$ such that,
  for each interaction $\alpha\in\gamma$, the guard of $\alpha$ is equal to the guard of one of its included actions.   
  We define the corresponding planning model as the composition of the n timed automata $\tcal{B}_i^p=(\Loc_i^p,\loc_0,\A_i\cup\tcal{P}_i,\T_i^p,\X_i\cup\{x_i^p\},\I_i^p)$,
  w.r.t the interaction set $\gamma\cup\tcal{P}$, where:
  \begin{itemize}
    \item $\tcal{P}_i=\cup_{a\in\A_i} \ p_{a}$ is the set of \emph{Planning Actions}
    \item $\tcal{P}=\{p_{\alpha}=\{p_{a_i}\}_{i\in I}|\alpha\in\gamma\wedge\alpha=\{a_i\}_{i\in I}\}$ is the set of \emph{Planning Interactions}  
    \item $x_i^p$ is a \emph{Tracking Clock} for interactions execution in each component
    \item $\Loc_i^{p}=(\Loc_i\cup\Loc_{i_p})$ is the set of control locations, where $\Loc_{i_p}$ is the set of locations 
      following planning actions
    \item $\T_i^p$ is such that for each $(\loc_i,a_i,g_i,r_i,\loc'_i)\in\T_i$, $a_i\in\alpha$ and for each
      $\delta\in Disc(\alpha)$:
      \begin{itemize}
        \item if $g_{\alpha}\neq\true$ we have:\\
          Planning transitions: $\begin{cases}
            \loc_i\transit{p_{a_i},\true,\emptyset}\loc_{a_i}, \text{ if } g=\true\\
          \loc_i\transit{p_{a_i},\zonep{\delta} g_i,r(x_i^p)}\loc^{\delta}_{a_i}, \text{ otherwise}\end{cases}$\\ 
          Execution transitions: $\begin{cases}
            \loc_{a_i}\transit{a,\true,r_i}\loc'_i, \text{ if } g=\true\\
            \loc_{a_i}^{\delta}\transit{a,g_a\wedge x_i^p=\delta,r_i}\loc'_i, \text{ otherwise}
            \end{cases}$\\
            where $\loc_{a},\loc_{a_i}^{\delta}\in\Loc_{i_p}$.\\
        \item if $g_{\alpha}=\true$, we choose one action $b\in\alpha$:\\
           Planning transitions: $\begin{cases}
            \loc_i\transit{p_{a_i},\true,\emptyset}\loc_{a_i}, \text{ if } a\neq b\\
          \loc_i\transit{p_{a_i},\true,r(x_i^p)}\loc_{a_i}^{\delta}, \text{ otherwise}
           \end{cases}$\\ 
           Execution transitions: $\begin{cases}
            \loc_{a_i}\transit{a_i,\true,r_i}\loc'_i, \text{ if } a\neq b\\
            \loc_{a_i}^{\delta}\transit{a_i,g_i\wedge x_i^p=\delta,r_i}\loc'_i, \text{ otherwise}
            \end{cases}$\\
      \end{itemize}
    \item $\I_i^p$ is the set of \emph{Location Invariants} , such that 
      $\forall\loc_i^p\in\Loc_i^p$, we have:\\
      $\I_i^p(\loc_i^p)= \begin{cases}
        \tpc{}(\loc_i)-\hmin, \text{ if }\loc_i^p=\loc_i\in\Loc_i\\
        x_{i}^{p}\le\delta\wedge\tpc{}(\loc_i), \text{ if } \loc_i^p=\loc_{a_i}^{\delta}\in
        \Loc_{i_p}$ such that $\loc_i\in\Loc_i\wedge\loc_i\transit{p_{a_i}}\loc_{a_i}^{\delta},
      \end{cases}$
  \end{itemize}

\end{definition}

For a composition $\gamma(B_1,\cdots,B_n)$, let $LTS_{p'}=(\Q_{p'},\gamma'\cup\realpos,\lto_{\gamma'})$, where $\gamma'=\gamma\cup\tcal{P}$, be the corresponding labeled transition system of its planning model under the 
standard semantics.  
\begin{theorem}\label{correctness}
  $LTS_{p'}\sqsubseteq_{R'} LTS_g$ where $R'$ is the relation defined as follows:
 
  For $q^p=(\loc^p,\val^p)\in\Q_{p'}$ and $q^g=(\loc^g,\val^g)\in\Q_g$, such that $(q^p,q^g)\in R'$, we have: 

  \begin{itemize}
    \item $\loc^p=(\loc^p_1,\cdots,\loc^p_n),\ \loc_g=(\loc^g_1,\cdots,\loc^g_n)$:
      \[\forall i\in\{1,\cdots,n\},\ \loc^g_i=\begin{cases}
        \loc^p_i, \text{ if } \loc^p_i\in\Loc_i,\\
        \loc_i, \text{ if } \loc^p_i\in\Loc_{i_p}\text{ with }\loc_i\transit{a,g,r}\loc_i^p\in\T_i^p\wedge\loc_i\in\Loc_i,
    \end{cases} 
      \]
      Notice that for the case where $\loc_i^p\in\Loc_{i_p}$, $\loc_i$ is unique by construction of the planning model.
    \item $\val^g=equ(\val^p)$, where $equ(\val^p)$ is the projection of $\val^p$ on clocks of $\val^g$ 
  \end{itemize}
 
  \end{theorem}
  \begin{proof}[Proof of Theorem~\ref{correctness}]
    To prove that $LTS_{p'}\sqsubseteq_{R'} LTS_g$, we need to prove that:

  \begin{enumerate}
    \item $\forall(q^p,q^g)\in R',\sigma\in\gamma\cup\realpos\text{ such that }q^p\transit{\sigma}_{\gamma'}q'^{p}\Rightarrow\exists q'^g.(q'^{p},q'^g)\in R'\wedge 
      q^g\transit{\sigma}_{\gamma}q'^g$
    \item $\forall(q^p,q^g)\in R',p_{\alpha}\in\tcal{P}\text{ such that }q^p\transit{p_{\alpha}}_{\gamma'}q'^{p}\Rightarrow(q'^{p},q^g)\in R'$ 
  \end{enumerate}

  \begin{enumerate}
    \item
  \begin{enumerate}
    \item Suppose that $(q^p,q^g)\in R',\sigma=\alpha\in\gamma$ and $q^p\transit{\alpha}_{\gamma'}q'^{p}$ with $q'^p=(({\loc'}_1^p,\cdots,{\loc'}_n^p),{\val'}^p)$. We have:
      $q^p\transit{\alpha}_{\gamma'}q'^{p}\Rightarrow g_{\alpha}$ is $\true$, and for 
      $\alpha=\{a_i\}_{i\in\I}$, by construction of the planning automaton, we have:
      $\loc_i^g\transit{a_i,g_i,r_i}{\loc'}_i^g$ such that ${\loc'}_i^g={\loc'}_i^p$. Moreover, 
     since the same clocks are reset by the execution of $\alpha$ in both models, we deduce that $\val'^g=equ(\val'^p)$. 
      By remarking that the state of components not participating in $\alpha$ remains the same, we conclude that $\exists {q'}^g$ such that 
      $q^g\transit{\alpha}_{\gamma}{q'}^g\wedge({q'}^p,{q'}^g)\in R'$. 
    \item Suppose that $(q^p,q^g)\in R',\sigma\in\realpos$ and $q^p\transit{\sigma}_{\gamma'}
      q'^{p}$. For $q^p_i=(\loc_i^p,\val_i^p)$, we define $\I_g$ the set of indexes such that 
       $\loc_i^p\in\Loc_i$, and $\I_p$ the set of indexes such that $\loc_i^p\in\Loc_{p_i}$. 
       \begin{itemize}
         \item $\forall i\in\I_g$.$\loc_i^p=\loc_i^g\wedge\q_i^p\transit{\sigma}
           {q'}_i^p\Rightarrow q_i^g\transit{\sigma}{q'}_i^g$. This implication is a direct result of the planning model definition since: $\sigma\le\I(\loc_i^p)\le\tpc{}(\loc_i^g)-\hmin$.
           
         \item $\forall i\in\I_p.\loc_i^g=\loc_i$ such that $\loc^p_i\in\Loc_{i_p}\text{ with }\loc_i\transit{a,g,r}\loc_i^p\in\T_i^p\wedge\loc_i\in\Loc_i$.
           Thus $\q_i^p\transit{\sigma}{q'}_i^p\Rightarrow q_i^g\transit{\sigma}{q'}_i^g$, since $\I(\loc_i^p)\implies\tpc{}(\loc_i^g)$.

         \end{itemize}
       We conclude that $\exists q'^g$ such that $q^g\transit{\sigma}_{\gamma}{q'}^g\wedge({q'}^
         p,{q'}^g)\in R'$.
  \end{enumerate}

\item Suppose that $(q^p,q^g)\in R'$ and $q^p\transit{p_{\alpha}}_{\gamma'}q'^{p}$, with 
  $p_{\alpha}\in\tcal{P}$ and $q'^p=(({\loc'}_1^p,\cdots,{\loc'}_n^p),{\val'}^p)$. We have:
      $q^p\transit{p_{\alpha}}_{\gamma'}q'^{p}\Rightarrow$ for $\alpha=\{a_i\}_{i\in\I}$
      $\loc_i^g=\loc_i^p\wedge\loc_i^g\transit{p_{a_i}}{\loc'}_i^p$. Moreover, since planning
      actions reset only the clocks $x_i^p$ for tracking execution time, we can deduce that
      $({q'}^p,q^g)\in R'$.
  \end{enumerate}

\end{proof}

Once interactions planning encoded, one last thing to do is to add the set of bad states to each
planning automaton (if needed) and find a strategy to avoid those states. 
Figure~\ref{fig:task} depicts the corresponding planning automaton for a task
component of the running example w.r.t Definition~\ref{def:plan_aut}. Here, $x_p$ is the clock
used for tracking the execution date of an interaction within the task component.
Locations suffixed by \emph{p}, correspond to locations following 
planning actions, whereas locations ending with \emph{err} define the bad states, that is,
states with urgent time progress condition(s) and no possible execution removing the urgency. In this example, we put $\hmin = 2$. 

For a matter of simplicity, we restrict the planning horizon values
by putting for each interaction $\alpha\in\gamma/ \hmx=h_{\min}$.
Notice that in this case,
the method checks if a given system is deadlock-free when being planned with
a $h_{\min}$ horizon.
\begin{figure}[h]
  \centering
  \includegraphics[scale=0.4]{../images/task}
  \caption{Planning automaton for the task component}
  \label{fig:task}
\end{figure}

\subsection{Discussion}

In this section, we explained how the problem of planning interactions can be formalized into
a real-time controller synthesis approach. However, this approach has some drawbacks.
In order to encode planning of interactions in components as timed automata, this approach
restricts its scope to discretized horizon values which results in having less control
over the planning dates of interactions, and leads in case of a high number of discretized values,
to an explosion in the number of planning transitions. 
Additionally, it considers only a class of systems where interactions have timing constraints on 
up to one of their participating components action. Otherwise, the planning should be encoded
on the composition, which represents a tedious work. 

\section{Communication Delays}
\section{Local Planning Semantics}
\section{Correctness}








\chapter{Clock Drift}\label{chap:6}
\minitoc
\section{Perturbation Models}
\section{Robustness Analysis}
\section{Robustness of Send/Receive Models to imprecisions}



\chapter{Implementation}
\minitoc
\section{BIP Framework}
\section{The BIP Toolbox}
\section{Distributed Real-Time BIP}

%\chapter{Experimental Results}
\label{chap:8}
\minitoc
\section{Case Studies}


\part{Conclusion}
{}
\chapter{Conclusion and Perspectives}
\label{chap:9}
In this chapter, we conclude this thesis by summarizing the achievements and pointing out
interesting future working directions.  

\section{Achievements}
\subsection*{Knowledge Based Optimization Approach}
In this thesis, we presented an intermediate distributed representation (Send/Receive model) for 
distributed ream-time systems. This representation proposes an implementation method for
systems with multiparty interactions through simpler primitives such as
messages passing. It also provides a scheduling mechanism, more appropriate to distributed
real-time systems, since it is based on partial information of a given system.

Our first contribution was to propose an optimization methods for such intermediate 
representation. We presented a methods that relies on invariants in order to refine the
set of conflicting interactions, required for building the Send/Receive model.
Based on the original model, an approximation of the set of reachable states of a timed system
is derived on the form of invariants. The latter are used later on for checking the 
simultaneous enabledness of two potentially conflicting interactions. 
Based on the obtained results, either the potential conflict between two interactions is cleared 
out meaning that both interactions are never enabled at the same time, or the real conflict 
cannot be attested which in this case a counter-example is proposed.
This optimization aims particularly at reducing the effort of scheduling interactions
by removing any unnecessary exchange of messages or evaluation computations. 

\subsection*{Modeling Communication Delays}
The Send/Receive model presented in Chapter~\ref{chap:3} assumes that the communication 
mechanisms are fast enough to not impact the behavior of the overall system. Effectively, this
is due to the zero delay between the decision of executing interactions in schedulers and
the concrete execution of the corresponding actions in the participating components.
The resulting model was proven to be observationally equivalent to the initial model.
To reduce the impact of communication delays on the system behavior,~\cite{ahlem_these} proposed
an approach based on the idea of \emph{early decision making}. In this solution, schedulers
plan ahead the execution of interactions and notify the corresponding components in advance.
Moreover, it was suggested that schedulers are required to observe an additional subset of
components called \emph{observed components}, not participating in the planned interactions, 
in order to achieve global deadlines.
However, besides the fact that this method is restricted to systems where components have non 
decreasing deadlines, the characterization of the set of observed components is incomplete. In 
fact this set is greater than the presented characterization, and in many cases is equal to 
all the component of the system. This is mainly due to the nature of the location invariants:
Local constraints that propagate on the global level.

Our second contribution was to propose a method, based on the same idea of early decision making,
but that differs from the approach of~\cite{ahlem_these} in the following points:
\begin{itemize}
  \item No restriction on components constraints was made
  \item No restriction on the form of timing constraints (In ~\cite{ahlem_these} timing 
    constraints are restricted to constraints of the form $x\le k$ or $x\ge k$)
  \item Our method works on the semantics level conversely to the aforementioned method that
    relies directly on transformations and model construction. Effectively, we introduced 
    the local planning semantics that plan the execution of interactions based only on 
    the set of its participating components. The planning operations are constrained by 
    the worst estimation of communication delays as well as the planned interactions. This
    semantics is suited for distributed real-time systems since it is based on local (partial)
    information. We also provided a proof for weak simulation, and a strategy based on 
    sufficient conditions guaranteeing the preservation of the system behavior from deadlocks. 
    Finally, we presented an alternative method based on real-time controller synthesis 
    paradigm and discussed its convenience with respect to our method. 
\end{itemize}

\subsection*{Robustness to Clock Imperfections}
We studied in Chapter~\ref{chap:6} the effect of clock imperfections (clock drift) on the 
behavior of a timed system. We considered a perturbation model where clock rates are not 
perfect (not equal and may change during the execution) but under the assumption that clocks 
are resynchronized which induces that their difference of clock values will stay within a certain
threshold. 
Our robustness analysis approach was based on the characterization of potential bad states
that invalidates the $\epsilon$-simulation between the initial model and the perturbated model.
We proposed a solution for verifying the robustness of a given timed system based on 
shrinkability analysis and suggested the \texttt{Shrinktech} tool for achieving such analysis
which is basically based on parameter synthesis.


\section{Future Works}
\subsection*{Partitioning and Mapping}

In this thesis, we presented the partitioning of interactions as being an important parameter
of the Send/Receive transformation. Interaction partitioning is of importance since the 
set of conflicting interactions is calculated based on the latter. The immediate question is
whether a given partitioning is better than an other. In other terms, how can we evaluate 
if a given partition of interaction is good or bad. Instinctively, the choice of partitioning
interactions is based on two main factors, namely the degree of parallelism and the effort
for solving conflicts (in terms of messages and computational evaluation).
This choice is generally guided by the choice of a specific target platform: how expensive
is the communication? This introduce necessarily the problem of mapping an application
on a given target platform~\cite{map}. This question is of great importance since its answer
will define how each application process will be mapped to the platform interconnected 
processing nodes. Consequently, it decides on workload of each processing nodes and 
the communication volume induced by such mapping.
Thus an optimal partition would be a partition that takes into account the target platform
and proposes a partitioning that optimize all the above points. This can be done either
by computing directly the optimal solution or by tuning a proposed partition by merging or 
splitting some processes of the application.

\subsection*{Invariants for Distributed Real-Time Systems}

In Chapter~\ref{chap:5} and Chapter~\ref{chap:6}, we presented different semantics reflecting
the behavior of a real-time system in a distributed setting (under communication delays or 
with clock drift). 

In Chapter~\ref{chap:5}, we relied on the invariants of the initial model
to verify the sufficient conditions for deadlock freedom of the local planning semantics.
This usage is due to the fact that invariants of the system under the standard semantics 
are proven to be invariants of the system under the local planning semantics.
An interesting direction is to try to derive invariants characterizing specifically the reachable
states of the planning semantics. This will help directly in the verification process in term of 
results precision.

In Chapter~\ref{chap:6}, we proposed the \texttt{Shrinktech} tool for the robustness analysis.
However, such tool relies on the computation of the region graph of a timed automaton, which
in term of complexity suffers from the state explosion problem when the increasing the number of
clocks. A solution to reduce this cost is to compute a parametric invariant of the system
with drifting clocks. The latter represents a cheap over-approximations may in some cases 
allow to avoid overly eager and expensive computations of the precise parametric images of 
the set of reachable states. 

\subsection*{Modeling vs Semantics}

Model-based development is one promising approach in building real-time systems today.
This paradigm relies on models with well defined semantics when building systems all the way
form the design phase until reaching a concrete implementation.
This approach is really important when working with large scale heterogeneous systems since
it reduces both the engineering efforts and the development costs and time.
Nevertheless, a model represents only an abstraction of the reality and is often based
on idealized assumptions, which will not hold at a given stage of the model-based workflow, 
precisely when reaching the implementation point.
In this thesis, we tried to bridge this gap by introducing new semantics to tackle the different
issues inherent to the distributed context.
An interesting working direction would be to answer the following question: 
Can similar results be obtained without introducing new semantics, but using modeling
instead? 
This question is of a big interest since it can indicate which direction efforts need to be made.
For instance, the introduction of a new semantics can be in some case complicated and 
errors-prone whereas modeling is known to be more general, much easier and trivial to understand.
Also, modeling has also the benefit of being modular in the sense that if we change 
the target execution platform on will have just to replace the corresponding model.

\subsection*{Scheduling Distributed Real-Time Systems}
In this thesis, we often mentioned the words \say{schedulers or scheduling} but without 
referring to any scheduling scheme or policy. An interesting working direction is to refine 
the behavior of given system by imposing some scheduling constraints. In fact, since we presented
methods based on over-approximations and sufficient conditions, the results of our analysis may
in some cases yield counter-examples that are either false-positives or real counter-examples.
The idea is to identify scheduling schemes that could identify the needs, based on a given set
of properties, in order to achieve a correct behavior.




\listoffigures 

\listoftables 


% ********************************** Back Matter *******************************
% Backmatter should be commented out, if you are using appendices after References
%\backmatter

% ********************************** Bibliography ******************************
\begin{spacing}{0.9}

% To use the conventional natbib style referencing
% Bibliography style previews: http://nodonn.tipido.net/bibstyle.php
% Reference styles: http://sites.stat.psu.edu/~surajit/present/bib.htm

\bibliographystyle{alpha}
\cleardoublepage
\bibliography{References/references} % Path to your References.bib file


% If you would like to use BibLaTeX for your references, pass `custombib' as
% an option in the document class. The location of 'reference.bib' should be
% specified in the preamble.tex file in the custombib section.
% Comment out the lines related to natbib above and uncomment the following line.
%\printbibliography[heading=bibintoc, title={References}]


\end{spacing}

% ********************************** Appendices ********************************

\begin{appendices} % Using appendices environment for more functunality

\chapter{The Gear Controller System}
\label{ap1}


\begin{figure}[H]       
\centering            
\includegraphics[scale=0.25]{Figures/gbgb}
\caption{The Gearbox Component}
\label{fig:gbgb}         
\end{figure}  


\begin{figure}[H]       
\centering            
\includegraphics[scale=0.29]{Figures/gbc}
\caption{The Clutch Component}
\label{fig:gbc}         
\end{figure}  

\begin{figure}[H]       
\centering            
\includegraphics[scale=.55]{Figures/gbgc}
\caption{The Gearbox Controller Component}
\label{fig:gbgc}         
\end{figure}  

\begin{figure}[H]       
\centering            
\includegraphics[scale=0.3]{Figures/gbe}
\caption{The Engine Component}
\label{fig:gbe}         
\end{figure}  

\begin{figure}[H]       
\centering            
\includegraphics[scale=0.5]{Figures/gbi}
\caption{The Interface Component}
\label{fig:gbi}         
\end{figure}  


%\include{Appendix2/appendix2}

\end{appendices}

% *************************************** Index ********************************
\printthesisindex % If index is present

\end{document}
