\chapter{Timed Systems and Semantics}\label{chap:2} 
\minitoc
\section{Timed Transition Systems}
Transition systems provide a general and convenient method for modeling 
systems and have been used frequently to model the behavior of 
software and hardware systems. They define graphs where nodes
represent the possible \emph{states} of the system, and edges model 
\emph{transitions}, that is, state changes. A state encode all the relevant 
information at certain instant, whereas a transition describes how the system
evolve between two states. 

Nowadays, several variants of transition systems formalism have been 
proposed. For instance, a labeled transition system is a transition system
where the set of transition is labeled by \emph{actions}.
In this thesis, we use \emph{Timed Transition Systems} (TTS)
to explicitly model the effect of time passage (besides the actions)
on the states of the system. Formally, it is defined as follows:  

\begin{definition}[Timed Transition System]
  A timed transition system is a tuple $\TTS$ such that $\mc{Q}$ is 
  a set of states, $q_0\in\mc{Q}$ is the initial state, $\Sigma$ is a set of
  labeled actions, $\mb{K}$ is a time domain and $\to\subseteq\mc{Q}\times
  (\Sigma\cup\mb{K})\times\mc{Q}$ is the transition relation.
\end{definition}
Consequently, we distinguish two types of transition:
\begin{itemize}
  \item \emph{action step} $\transit{\sigma}$ for $\sigma\in\Sigma$ 
  \item \emph{time step} $\transit{d}$ for $d\in\mb{K}$
\end{itemize}
We write $q\transit{d,\sigma}q''$ if there exists $q'\in\mc{Q}$
such that $q\transit{d}q'\transit{\sigma}q''$. We say that $q'$ is
a time successor of $q$.

Given a TTS $\TTS$, a \emph{run} $\exec$ of $\mc{T}$ (also called 
\emph{execution sequence}), is a path that alternates action steps and 
time steps, that is:
\begin{displaymath}
  \exec=s_0\sigma_0 s_1\sigma_1 s_2\cdots \text{ such that } 
  s_i\in\mc{Q}, \  s_i\transit{\sigma_{i}}s_{i+1},\text{ and } 
  i\in\integerpoz,\sigma_i\in\Sigma\cup\mb{K}
\end{displaymath}
We denote by $\texec$ the sum of all the time steps along this run,
that is, $\texec=\Sigma_{\sigma_i\in\mb{K}}\sigma_i$. 
A run $\exec$ is said to be an \emph{initial run} if $s_0=q_0$. 

A state $q\in\mc{Q}$ is called reachable if there exists an initial run
that leads to state $q$. We put $Reach(\mc{T})$ to denote the set of all 
reachable states of $\mc{T}$.
\subsection{Comparing Timed Transition Systems}
In this thesis, we use the concept of (bi)simulation~\cite{} 
in order to attest the similarity of timed transition systems.
\begin{definition}[Simulation]
  Given two TTS $\TTSb{1}$ and $\TTSb{2}$, a simulation relation from 
  $\mc{T}_1$ to $\mc{T}_2$ is a binary relation $\mc{R}\subseteq\mc{Q}_1
  \times\mc{Q}_2$ such that:
  \begin{itemize}
    \item $\forall(q_1,q_2)\in\mc{R},\forall\sigma\in\Sigma,\ 
      q_1\transitb{\sigma}{1}q_1'\Rightarrow\exists q'_2\in\mc{Q}_2 
      \text{ such that } q_2\transitb{\sigma}{2}q_2'\wedge
      (q_1',q_2')\in\mc{R}$
    \item $\forall(q_1,q_2)\in\mc{R},\forall d\in\mb{K},\ 
      q_1\transitb{d}{1}q_1'\Rightarrow\exists q'_2\in\mc{Q}_2 
      \text{ such that } q_2\transitb{d}{2}q_2'\wedge
      (q_1',q_2')\in\mc{R}$
  \end{itemize}
\end{definition}
  $\mc{T}_2$ simulates $\mc{T}_1$, denoted by $\mc{T}_1\simu{\mc{R}}\mc{T}_2$
  means that $\mc{T}_2$ can do everything $\mc{T}_1$ does. Notice that if
  $\mc{T}_1\simu{\mc{R}}\mc{T}_2$ and $\mc{T}_2\simu{\mc{R}}\mc{T}_1$,
  we say that $\mc{T}_1$ and $\mc{T}_2$ are bisimilar, denoted by
  $\mc{T}_1\eqv{\mc{R}}\mc{T}_2$   

In some cases, this notion of simulation is refined in order consider only
a part of a system behavior. This is usually the case when a system
performs \emph{internal} (or \emph{silent}) actions not visible by external
observers. This variant of simulation is called \emph{weak simulation}
~\cite{}.

\begin{definition}[Weak Simulation]
  Given two TTS $\TTSbw{1}$ and $\TTSbw{2}$, where $\tau$ actions 
  represent silent (unobservable) actions, a weak simulation relation from 
  $\mc{T}_1$ to $\mc{T}_2$, denoted $\mc{T}_1\simuw{\mc{R}}\mc{T}_2$, 
  is a binary relation $\mc{R}\subseteq\mc{Q}_1
  \times\mc{Q}_2$ such that:
  \begin{itemize}
    \item $\forall(q_1,q_2)\in\mc{R},\ 
      q_1\transitb{\tau}{1}q_1'\Rightarrow\exists q'_2\in\mc{Q}_2 
      \text{ such that } q_2\transitb{\tau*}{2}q_2'\wedge
      (q_1',q_2')\in\mc{R}$
    \item $\forall(q_1,q_2)\in\mc{R},\forall\sigma\in\Sigma,\ 
      q_1\transitb{\sigma}{1}q_1'\Rightarrow\exists q'_2\in\mc{Q}_2 
      \text{ such that } q_2\transitb{\tau*\sigma\tau*}{2}q_2'\wedge
      (q_1',q_2')\in\mc{R}$
    \item $\forall(q_1,q_2)\in\mc{R},\forall d\in\mb{K},\ 
      q_1\transitb{d}{1}q_1'\Rightarrow\exists q'_2\in\mc{Q}_2 
      \text{ such that } q_2\transitb{\tau*d\tau*}{2}q_2'\wedge
      (q_1',q_2')\in\mc{R}$
  \end{itemize}
\end{definition}
  we say that $\mc{T}_1$ and $\mc{T}_2$ are observationally equivalent,
  denoted $\mc{T}_1\eqvw{\mc{R}}\mc{T}_2$ , if it exists a weak simulation 
  from $\mc{T}_1$ to $\mc{T}_2$ and vice versa.
  
  In chapter~\ref{chap:6}, we will introduce a weaker notion of simulation 
  that characterize the degree of closeness (in term of delays)
  between two timed systems. 

\subsection{Reactive Timed Systems}
Reactive systems are supposed to execute forever, that is,
they are supposed to act infinitely often. We refer to this characteristic as 
the \emph{requirement of progress}. 
Particularly, timed systems evolves either through an action step or 
by letting the time pass (time step).
This evolution imposes thus two requirements of progress
discrete progress (resp. time progress) meaning that a timed system
should be able to perform action steps (resp. time steps) infinitely
often. In the physical world however, no matter how fast a system can
evolve, it cannot be infinitely fast. This induces the following constraints
on the progress of time:
\begin{enumerate}
  \item Finite number of actions can happen in a certain amount of time
  \item Infinite number of actions cannot occur in zero time
\end{enumerate}

\subsubsection{Time Progress (Zeno runs \& Timelocks)} 
We distinguish two types of anomalies that infringe the time progress of in
a timed system namely \emph{zeno} runs and \emph{timelocks}.
A run $\exec$ is called \emph{zeno} if it is 
an infinite run and if $\texec\neq\infty$. Such a run transgresses the 
first point of the time progress presented above.
Timelocks are states from which all infinite runs starting from these states
are zeno.
\subsubsection{Discrete Progress (Deadlocks)}
States violating the discrete progress are called \emph{deadlock} states.
Formally, a state is said to be deadlock if no action can be executed from 
that state nor from any of it time successor. 

Any model of a reactive timed system should properly capture the behavior 
of the latter. Particularly, the corresponding model must react infinitely often,
that is, it must not block time or execute a bounded number of actions in zero 
time. In that sens, we can say that deadlocks and timelocks are modeling errors
that needs to be cleared, either during the modeling process (which is tedious 
for large scale systems) or by providing verification methods that guarantee
their absence. 


\section{Timed Automata Syntax and Semantics}
\subsection{Preliminary Definitions}
\subsubsection{Clocks}
In order to represent the dense time domain, we rely on positive 
real valued variables, the \emph{clocks}. Clocks are 
positive real variable increasing synchronously (with the same rate) in a 
given system. They are used to express the progress of time and impose 
a certain dynamic on the execution of a timed system.

Given a finite set of clocks $\mc{X}$, we define the valuation function
$\val:\mc{X}\to\realpoz$ assigning to each clock $x$ a positive real value 
$\val(x)$. We put $\Val$ to denote the set of all valuations.
For a valuation $\val\in\Val$ and $d\in\realpoz$, $\val+d$ is the valuation
satisfying $(\val+d)(x)=\val(x)+d$, while for a subset of clocks 
$r\subseteq\mc{X}$, $\val[r]$ is the valuation obtained from $\val$ by 
resetting clocks of $r$, that is, $\val[r](x)=0$ for $x\in r$ and
$\val[r](x)=\val(x)$ otherwise. We write {\bf 0} for the valuation 
that assigns 0 to every clock.

An \emph{atomic clock constraint} is an expression of the form:
\begin{displaymath}
  c:=\true \ | \ x\#k \ |\ x-y\#k \
\end{displaymath}
where $x$ and $y$ are clocks in $\mc{X}$, $\#\in\{<,\le,=,\ge,>\}$,
and $k\in\integer$. A clock constraint is a conjunction of atomic clock 
constraint, that is:
\begin{equation}\label{eq:cc}
  c:=\true \ | \ x\#k \ |\ x-y\#k \ | \ c_1\wedge c_2 
\end{equation}
with $c_1$ and $c_2$ being atomic clock constraints. We write $\mc{C(X)}$
for the set of clock constraints over $\mc{X}$.
Given a clock constraint $c$ and a valuation $\val$, we say that $\val$
satisfies $c$, denoted $\val\models c$, if all constraints are satisfied
when each $x\in\mc{X}$ is replaced by $\val(x)$. 
We also consider for a clock constraint $c$ the classical \emph{backward} 
and \emph{forward} operators~\cite{} on
clock constraints:
\begin{align*}
  &\text{ \bf Backward } &:& \backward c\models\val\Leftrightarrow 
  \exists d\in\realpoz.\ \val+d\models c\\ 
  &\text{ \bf Forward } &:& \forward c\models\val\Leftrightarrow 
  \exists d\in\realpoz.\ \val-d\models c\\ 
\end{align*}

Additionally, we also use two variants of the backward operator considering
lower bounds $l\in\integerpoz$ and upper bounds $u\in\integerpoz\cup
\{+\infty\}$:

\begin{displaymath}
  \backwardp{l}{u} c\models\val\Leftrightarrow 
  \exists d\in\realpoz, l\le d\le u. \ \val+d\models c\\ 
\end{displaymath}

\subsection{Standard Semantics for Timed Automata}

\begin{definition}[Timed Component]\label{def:tc}
  A component is a tuple $B=(\Loc,\loc_0,\mc{X},\mc{A},\mc{E},\mc{I})$, where
  \begin{itemize}
    \item $\Loc$ is a finite set of locations, with $\loc_0\in\Loc$ is the 
      initial location,
    \item $\mc{X}$ is a finite set of clocks,
    \item $\mc{A}$ is a finite set of actions
    \item $\mc{E}\subseteq\Loc\times(\mc{A}\times\mc{C(X)}\times2^{\mc{X}})
      \times\Loc$ is a finite set of transitions labeled with an action, 
      a clock constraint (guard), and a set of clocks to be reset,
    \item $\mc{I}:\Loc\to\mc{C(X)}$ is the function assigning an invariant
      to each location. Notice that invariants are restricted to conjunction
      of atomic clock constraints of the form $x\le k$. 
  \end{itemize}
\end{definition}

Through this thesis, we consider 
\emph{deterministic} timed components, that is, at a given location
$\loc$ and for a given action $a$, there is up to one outgoing transition
from $\loc$ labeled by a. 
A transition $e=(\loc,(a,g,r),\loc')\in\mc{E}$ is also denoted by
$\loc\transit{a,g,r}\loc'$. We write $\source{e}$, $\target(e)$, $\action(e)$, $\guard(e)$
and $\reset(e)$ for $\loc$, $\loc'$, $a$, $g$ and $r$, respectively.
We also denote by $\Phi(a,\loc)$ the guard 
of the transition labeled by $a$ and outgoing from $\loc$
if it exists, and $\false$ otherwise. It is formalized as follows:
\[\Phi(a,\loc)=\begin{cases}
  g,& \text{ if } \exists e=(\loc,a,g,r,\loc')\in\mc{E}  \\
  \false,& otherwise
\end{cases}\]

\begin{example}
  Figure~\ref{fig:tc} depicts a timed component $B$ where locations
  are represented by circles and transitions are the directed arrows
  from a location to another. The initial location ($\loc_0$ here) is
  represented by a double circle. The component 
  $B=(\Loc,\loc_0,\mc{X},\mc{A},\mc{E},\mc{I})$ is defined such that:
  \begin{itemize}
    \item $\Loc=\{\loc_0,\loc_1\}$,
    \item $\mc{X}=\{x\}$,
    \item $\mc{A}=\{a,b\}$,
    \item $\mc{E}=\{e_1,e_2\}$ where
      \begin{itemize}
        \item $e_1=(\loc_0,a,2\le x\le4,\emptyset,\loc_1)$
        \item $e_2=(\loc_1,b,\true,\{x\},\loc_0)$
      \end{itemize}
    \item $\mc{I}$ assigns the following invariants for locations: 
      $\Inv{\loc_0}=x\le4$ and $\Inv{\loc_1}=\true$
  \end{itemize}
  Notice that when a clock constraint (respectively an invariant is not
  shown on a transition (respectively location) it is interpreted as $\true$.
  This is the case for transition $e_2$ and location $\loc_2$.
\end{example}
\begin{figure}[!h]
 \centering
  \begin{tikzpicture}[scale=0.8,every node/.style={scale=0.8}]

  \node [accepting, place,label=below:\textcolor{red}{$x\le4$},label={[shift={(-.9,.3)}]$B$}](l0) {$\loc_0$};
  \node [place,right=2cm of l0] (l1) {$\loc_1$};

  \path (l0) edge [bend left] node[above,align=center]{$a$\\$2\le x\le4$}(l1)
        (l1) edge [bend left] node[below,align=center]{$b$\\$x:=0$} (l0);

  \node [rounded corners,inner xsep=11mm,inner ysep=11mm,draw, fit=(l0)(l1)] (rec1) {};
\end{tikzpicture}
  \caption{Example of a Timed Component}
 \label{fig:tc}
\end{figure}  





\begin{definition}[Standard Semantics]
The semantics of a timed component $\tc$ is given by the timed transition
  system $\TTSs$ where $\mc{Q}\subseteq\Loc\times\Val$ denotes the states
  of $B$ with $q_0=(\loc_0,0)$ being the initial state, 
  and $\to\subseteq\mc{Q}\times(\mc{A}\cup\realpos)\times\mc{Q}$
  denotes the set of transitions between states according to the rules:
  \begin{itemize}
    \item $(\loc,\val)\transit{a}(\loc',\val[r])$ if $\loc\transit{a,g,r}
      \loc'$ and $\val\models g\wedge\val[r]\models\Inv{\loc'}$
      (action step) 
    \item $(\loc,\val)\transit{d}(\loc,\val+d)$ if $\forall d'\in[0,d]$,
      $\val+d'\models \Inv{\loc}$ (time step) 
  \end{itemize}
\end{definition}
Notice that since the invariants are restricted to conjunction of upper bound
atomic constraints, the time step can be simplified to:
\begin{displaymath}
  (\loc,\val)\transit{d}(\loc,\val+d)\text{ if }\val+d\models \Inv{\loc} 
\end{displaymath}

In this thesis, we always assume components with $\emph{well formed}$
guard, that is, for a transition $\loc\transit{a,g,r}\loc'$,
$\val\models g\Rightarrow\val\models\Inv{\loc}\wedge\val[r]\models
\Inv{\loc'}$ for any $\val\in\Val$. This ensures that the reachable states
always satisfy the location invariant. The rule on action step becomes then:
\begin{displaymath}
  (\loc,\val)\transit{a}(\loc',\val[r])\text{ if } \loc\transit{a,g,r}
      \loc'\text{ and } \val\models g 
\end{displaymath}
\begin{remark}
  When used in predicate definition, clock constraints are straightforwardly 
  applied to clock valuation functions of states and thus interpreted as $\true$
  or $\false$.
\end{remark}
We define the predicate $\enabled{a}$ characterizing states $(\loc,\val)$
from which an action $a$ is enabled, that is, such that $(\loc,\val)
\transit{a}(\loc',\val')$. It is formalized as follows:
\begin{displaymath}
  \enabled{a}=\bigvee_{\loc\in\Loc}\al{\loc}\wedge\Phi(a,\loc)
\end{displaymath}
where $\al{\loc}$ is $\true$ on states whose location is $\loc$.
In the same way, we define the predicates $\enabledbackward{a}$,
$\enabledbackwardb{a}{l}{u}$ and $\enabledforward{a}$ describing, respectively,
states from which an action $a$ can be executed after some time step,
some bounded time step or states with clocks valuations greater or equal than those
of $\enabled{a}$ (~\ref{fig:zones}). These predicates can be formally written 
as follows:
\begin{align*}
  &\enabledbackward{a}\hspace{1.8mm}=\bigvee_{\loc\in\Loc}\al{\loc}\wedge
  \backward\Phi(a,\loc)  \\
  &\enabledbackwardb{a}{l}{u}=\bigvee_{\loc\in\Loc}\al{\loc}\wedge
  \backwardp{l}{u}\Phi(a,\loc)\\
  &\enabledforward{a}\hspace{1.9mm}=\bigvee_{\loc\in\Loc}\al{\loc}\wedge
  \forward\Phi(a,\loc)
\end{align*}
\begin{figure}[H]
\center
  \begin{tikzpicture}%[remember picture,overlay]


\draw[->]  (-2,6) -- (8,6) node[below left] {time};

\draw[very thick,-]   (2,6) |- (4,6.75) -- (4,6);


\draw[{Bar[width=4mm].Straight Barb[]}-{Straight Barb[].Bar[width=4mm]}]
    (2,5.8) -- node[fill=white,below] {$g_{a}$}  (4,5.8);


%%%%%%%%%%%%%%%%%%%%%%%%%%%%%%%%%%%%%%%%%%%%%%%%%%%%%%%%%%%%%%%%%%%%%%%%%%%%%%%

\draw[->]  (-2,4) -- (8,4) node[below left] {time};


\draw[draw=white,fill=red!30,very thick,semitransparent,-]  % <--- changed  
                (-2,4) |- (4,4.75) -- (4,4); 
\draw[draw=red,very thick,semitransparent,-]  % <--- changed  
                (-2,4.75) -| (4,4); 



\draw[{Bar[width=4mm].Straight Barb[]}-{Straight Barb[].Bar[width=4mm]}]
    (-2,3.8) -- node[fill=white,below] {$\backward g_a$}  (4,3.8);



%%%%%%%%%%%%%%%%%%%%%%%%%%%%%%%%%%%%%%%%%%%%%%%%%%%%%%%%%%%%%%%%%%%%%%%%%%%%%%%


\draw[->]  (-2,1.5) -- (8,1.5) node[below left] {time};

\draw[draw=dgreen,fill=dgreen!30,very thick,-]   (0,1.5) |- (3,2.25) -- (3,1.5);


\draw[{Bar[width=4mm].Straight Barb[]}-{Straight Barb[].Bar[width=4mm]}]
    (0,1.3) -- node[fill=white,below] {$\backwardp{l}{u} g_{a}$}  (3,1.3);

\draw[{Bar[width=4mm].Straight Barb[]}-{Straight Barb[].Bar[width=4mm]}]
    (3,1.3) -- node[fill=white,below] {$l$}  (4,1.3);
\draw[{Bar[width=4mm].Straight Barb[]}-{Straight Barb[].Bar[width=4mm]}]
    (0,2.5) -- node[fill=white,above] {$u$}  (2,2.5);



%%%%%%%%%%%%%%%%%%%%%%%%%%%%%%%%%%%%%%%%%%%%%%%%%%%%%%%%%%%%%%%%%%%%%%%%%%%%%%%



\draw[->]  (-2,-0.5) -- (8,-0.5) node[below left,yshift=-2mm] {time};


\draw[draw=white,fill=blue!30,very thick,semitransparent,-]  % <--- changed  
                (2,-.5) |- (8,0.25) -- (8,-.5); 
\draw[draw=blue,very thick,semitransparent,-]  % <--- changed  
                (2,-.50) |- (8,0.25); 



\draw[{Bar[width=4mm].Straight Barb[]}-{Straight Barb[].Bar[width=4mm]}]
    (2,-0.7) -- node[fill=white,below] {$\forward g_a$}  (8,-0.7);


%%%%%%%%%%%%%%%%%%%%%%%%%%%%%%%%%%%%%%%%%%%%%%%%%%%%%%%%%%%%%%%%%%%%%%%%%%%%%%%



\draw[->]  (-2,-3) -- (8,-3) node[below left] {time};

\draw[draw=orange,fill=orange!30,very thick,-]   (3,-3) |- (6,-2.25) -- (6,-3);


\draw[{Bar[width=4mm].Straight Barb[]}-{Straight Barb[].Bar[width=4mm]}]
    (3,-3.2) -- node[fill=white,below] {$\forwardp{l}{u} g_{a}$}  (6,-3.2);

\draw[{Bar[width=4mm].Straight Barb[]}-{Straight Barb[].Bar[width=4mm]}]
    (2,-3.2) -- node[fill=white,below] {$l$}  (3,-3.2);
\draw[{Bar[width=4mm].Straight Barb[]}-{Straight Barb[].Bar[width=4mm]}]
    (4,-2) -- node[fill=white,above] {$u$}  (6,-2);

\coordinate (a1) at (2,7);
\coordinate (a2) at (4,7);
\coordinate (a3) at (2,-3.5);
\coordinate (a4) at (4,-3.5);
\draw[dashed,-] (a1)--(a3);
\draw[dashed,-] (a2)--(a4);

\end{tikzpicture}
\caption{Backward and Forward Operators on a Guard}\label{fig:zones}
\end{figure}

A state $(\loc,\val)$ is said \emph{urgent} if time cannot progress from
this state, that is, $\nexists d\in\realpos$ such that $(\loc,\val)\transit{d}
(\loc,\val+d)$. Urgent states are characterized by the predicate:
\begin{equation}\label{eq:urg}
  \urgent=\bigvee_{\loc\in\Loc}\al{\loc}\wedge urg(\loc)
\end{equation}
where $urg(\loc)$ is a clock constraint characterizing the valuations from
which time cannot progress with respect to the location invariant 
$\Inv{\loc}$, that is, assuming that 
$\Inv{\loc}=\bigwedge_{i=1}^m x_i\le k_i$ then 
$urg(\loc)=\bigwedge_{i=1}^m x_i\ge k_i$. Notice that due to well formed
guards, an urgent reachable state satisfies also~\ref{eq:urg} if inequalities
$x_i\ge k_i$ on clocks are replaced by equalities $x+i=k_i$ in the expression
of $urg(\loc)$.

\begin{definition}[Strongly Non-Zeno Timed Component]
  Given a timed component $B$. A \emph{structural loop} of $B$ is a sequence 
  of distinct transition $e_1\cdots e_m$ such that $\forall i\in\{1,\cdots,m
  \},\target{e_i}= \source{e_{i+1}}$. $B$ is called \emph{strongly non-zeno} 
  if for every structural loop there exists a clock $x$ and some $0\le i,j
  \le m$ such that:
  \begin{itemize}
    \item $x$ is reset in step $i$, that is, $x\in\reset(e_i)$
    \item $x$ is bounded from below in step j, that is, $(x<1)\cap\guard(e_j)
      =\false$
  \end{itemize}
\end{definition}
Intuitively, this definition implies that at least 1 time unit is elapsed 
at each loop of $B$.

\begin{lemma}[\cite{}]\label{lm:snz}
  If $B$ is strongly non-zeno then every infinite run of $B$ is non-zeno
\end{lemma}

The following corollary is an immediate consequence of lemma~\ref{lm:snz}. 
\begin{corollary}\label{cr:snz}
Given a timed component $B$. If B is strongly non-zeno then it is 
  timelock-free.
\end{corollary}

Corollary~\ref{cr:snz} highlights an interesting fact of strong non-zenoness.
It discharges from the trouble of checking time progress. In particular, checking
the progress of a timed system is reduced to checking deadlock freedom.

\begin{definition}[Deadlocks]
  Given a timed component $B$. We say that a state $(\loc,\val)$ of $B$ is 
  \emph{deadlock} if and only if no action can be executed from this state
  or any of its time succesors, that is:
  \begin{displaymath}
    \nexists a\in\mc{A}, \ (\loc,\val)\transit{a}(\loc',\val')\wedge
    \forall d>0, \ (\loc,\val)\transit{d}(\loc,\val+d)\transit{a}(\loc',\val')
  \end{displaymath}
  Deadlock states are characterized by the following predicate:
  \begin{displaymath}
    \bigvee_{\loc\in\Loc}\al{\loc}\wedge\Big(\bigwedge_{a\in\mc{A}}
    \neg \backward\big(\enabled{a}\wedge\Inv{\loc}\big)\Big)
  \end{displaymath}
  Because of well formed guards, the above can be simplified into:
  \begin{displaymath}
    \bigwedge_{a\in\mc{A}}\neg\enabledbackward{a}
  \end{displaymath}
\end{definition}
A deadlock $(\loc,\val)$ is called an \emph{action-time-lock} when no action can
execute nor time can progress from $(\loc,\val)$, that is:
  \begin{displaymath}
    \nexists a\in\mc{A}, \ (\loc,\val)\transit{a}(\loc',\val')\wedge
    \nexists d>0, \ (\loc,\val)\transit{d}(\loc,\val+d)
  \end{displaymath}
  Action-time-locks verifies the following predicate:
  \begin{displaymath}
    \bigwedge_{a\in\mc{A}}\neg\enabled{a}\wedge\bigvee_{\loc\in\Loc}
    \big(\al{\loc}\wedge urg(\loc)\big)
  \end{displaymath}

The common practice in component-based timed systems is to have several components
executing in parallel while their clocks increase synchronously. Moreover,
it is often mandatory to restrict the components behaviors so as to achieve 
a given global property. In what follows, components communicate by means
of \emph{multiparty interactions}. A multiparty interaction is a rendez-vous 
synchronization between actions of a fixed subset of components. It takes place 
only if all the participants agree to execute the corresponding actions. Given 
$n$ components $B_i$, with disjoint sets of actions $\mc{A}_i$, an interaction
is a subset of actions $\alpha\subseteq\cup_{1\le i\le n}A_i$ containing at most
one action per component, that is, $\alpha\cap\mc{A}_i$ is either empty or a 
singleton $\{a_i\}$. Thus, an interaction $\alpha$ can be put in the form
$\alpha=\{a_i\}_i{\in{I}}$ with $I\subseteq\{1,\cdots,n\}$ and $a_i\in\mc{A}_i$
for all $i\in{I}$. We denote by $\p{\alpha}$, the set of components \emph{participating}
in $\alpha$, that is, $\p{\alpha}=\{B_i\}_{i\in{I}}$.

\begin{definition}[Timed System]\label{def:comp}
  Given $n$ components $\tci{i}$ with $\Loc_i\cap\Loc_j=\emptyset, \ \mc{A}_i\cap
  \mc{A}_j=\emptyset,\text{ and }\mc{X}_i\cap\mc{X}_j=\emptyset$ for any $i\neq j$,
  the composition $S=\gamma(B_1,\cdots,B_n)$ with respect to the interaction set 
  $\gamma$ is defined by the timed component $(\Loc,\loc_0,\mc{X},\gamma,
  \mc{E}_{\gamma},\mc{I})$ where:
  \begin{itemize}
    \item $\Loc=\Loc_1\times\cdots\times\Loc_n$
    \item $\loc_0=(\loc_{0_1},\cdots,\loc_{0_n})$,
    \item $\mc{X}=\mc{X}_1\cup\cdots\cup\mc{X}_n$,
    \item $\Inv{\loc}=\Inv{\loc_1}\wedge\cdots\wedge\Inv{\loc_n}$, for $\loc=
      (\loc_1,\cdots,\loc_n)$,
    \item $\mc{E}_{\gamma}$ is defined by:
  \end{itemize}
      \begin{myequation}
        \mc{E}_{\gamma}=\left\{
          \substack{\loc\transit{\alpha,g,r}\loc'\\
            \text{ for } \alpha=\{a_i\}_{i\in{I}}}\Big\lvert
        \substack{\loc=(\loc_1,\cdots,\loc_n)\in\Loc, 
        \loc'=(\loc_1',\cdots,\loc_n')\in\Loc \\
        \text{ if } i\not\in{I},\ \loc_i'=\loc_i,\text{ and for }i\in{I},\
        \loc_i\transit{a_i,g_i,r_i}\loc_i'\text{ and } \\ 
        g=\bigwedge_{i\in{I}}g_i,\  r=\cup_{i\in{I}}r_i} \right\} 
      \end{myequation}
      
\end{definition}

In practice, we do not explicitly build compositions of timed components as 
presented in Definition~\ref{def:comp}. We rather interpret their semantics by 
evaluating enabled interactions based on current states of components. In a 
composition $S$ of $n$ components $B_i$, an action $a_i$ can execute only as part
of an interaction $\alpha$ such that $a_i\in\alpha$,that is, along with the 
execution of all other actions $a_j\in\alpha$, which corresponds to the usual 
notion of multiparty interactions.

\begin{property}[Semantics of a Composition]\label{pr:sem_comp}
  Given a set of components $\{B_1,\cdots,B_n\}$ and an interaction set $\gamma$,
  the semantics of the composite component $S=\gamma(B_1,\cdots,B_n)$
  with respect to the set of interaction $\gamma$, is defined by 
  the timed transition system $\TTSg$ where:
  \begin{itemize}
    \item $\mc{Q}_g=\Loc\times\Val$ is the set of global states, where
      $\Loc=\Loc_1\times\cdots\times\Loc_n$ and $\mc{X}=\bigcup_{i=1}^n\mc{X}$.
      We write a state $q=(\loc,\val)$ where $\loc=(\loc_1,\cdots,\loc_n)\in\Loc$
      is a global location and $\val=(\val_1,\cdots,\val_n)\in\Val$ is a global 
      clocks valuation. The initial state is $q_0=(\loc_0,0)$,
    \item $\gamma$ is the set of interactions,
    \item $\to_{\gamma}$ is the set of transitions defined by the rules:
    \begin{itemize}
      \item Interaction step: \\
      \end{itemize}
  \end{itemize}
  \vspace*{-1cm}
  \begin{align*}
    & \alpha=\{a_i\}_{i\in I}\in\gamma,\quad\forall i\in{I}.(\loc_i,\val_i)
    \transitb{\alpha}{\gamma}(\loc'_i,\val'_i),\quad\forall i\notin{I}.
    (\loc_i,\val_i)=(\loc'_i,\val'_i)\\
    \cline{1-2}
   &\hspace{4.5cm}(\loc,\val)\transitb{\alpha}{\gamma}(\loc',\val')
 \end{align*}
  \vspace*{-1.5cm}
  \begin{itemize}
    \item[] 
      \begin{itemize}
      \item Time step:
    \end{itemize}
  \end{itemize}
  \vspace*{-5mm}
  \begin{align*}
    & d\in\realpos,\quad\forall i\in\{1,\cdots,n\}
    .\val_i+d\models\Inv{\loc_i}\\
    \cline{1-2}
   &\hspace{2cm}(\loc,\val)\transitb{d}{\gamma}(\loc,\val+d)
 \end{align*}
\end{property}

To simplify notations, predicates defined on individual components $B_i$ are
straightforwardly interpreted on global states $(\loc,\val)$ of the composition
by considering the projection $(\loc,\val_i)$ of $(\loc,\val)$ on $B_i$.
For instance, $\al{\loc_i}$ evaluates to true on $(\loc,\val)$ iff
$\loc\in\Loc_1,\times\cdots\times\Loc_{i-1}\times\{\loc_i\}\times\Loc{i+1}\times
\cdots\times\Loc_n$. Similarly, clock constraints of component $B_i$ are applied
to clocks $\mc{X}_i$ are applied to clock valuation functions of the composition
by restricting $\val$ to clock of $\mc{X}_i$ of $B_i$. This allows to write
the predicate $\enabled{\alpha}$, characterizing states $(\loc,\val)$ from which 
an interaction $\alpha=\{a_i\}_{i\in{I}}\in\gamma$ can be executed, as:

\begin{align*}
  \enabled{\alpha}&=\bigwedge_{i\in{I}}\enabled{a_i},\\
                  &=\bigwedge_{i\in{I}}\bigvee_{\loc_i\in\Loc_i}\al{\loc_i}\wedge
                  \Phi(a_i,\loc_i),\\
                  &=\bigwedge_{\substack{\loc\in\Loc\\ 
                  \loc=(\loc_1,\cdots,\loc_n)}}\al{\loc}\wedge
                  \bigwedge_{\substack{i\in I\\ a_i\in\alpha }}\Phi(a_i,\loc_i)
\end{align*}

Notice that the above formulation of $\enabled{\alpha}$ corresponds to locations 
enumeration of all components participating in interaction $\alpha$. In practice,
we rather consider only a subset of locations $\Loc_{\alpha}\subseteq\Loc$, from
which the execution of $\alpha$ is possible. This corresponds to $\Pi_{i\in{I}}
|\Loc_{a_i}|$ possible configuration, where $\Loc_{a_i}\subseteq\Loc_i$ is a subset
of locations from which there exists a transition labeled by action $a_i\in\alpha$,
and $|\Loc_{a_i}|$ denotes the cardinality of $\Loc_{a_i}$, which is reasonably
small in practical examples bu can be (at the worst case) equal to $\Pi_{i\in{I}}
|\Loc|$. The predicates $\enabledbackward{\alpha}$, 
$\enabledbackwardb{\alpha}{l}{u}$ and $\enabledforward{\alpha}$ becomes:

\begin{align*}
  &\enabledbackward{\alpha}\hspace{1.8mm}=\bigvee_{\substack{\loc\in\Loc\\
  \loc=(\loc_1,\cdots,\loc_n)}}\al{\loc}\wedge \backward\big(\bigwedge_{\substack{
    i\in{I}\\a_i\in\alpha}}\Phi(a_i,\loc_i)\big) \\
  &\enabledbackwardb{\alpha}{l}{u}=\bigvee_{\substack{\loc\in\Loc\\
  \loc=(\loc_1,\cdots,\loc_n)}}\al{\loc}\wedge \backwardp{l}{u}\big(\bigwedge_{
    \substack{i\in{I}\\a_i\in\alpha}}\Phi(a_i,\loc_i)\big) \\
  &\enabledforward{\alpha}\hspace{1.9mm}=\bigvee_{\substack{\loc\in\Loc\\
  \loc=(\loc_1,\cdots,\loc_n)}}\al{\loc}\wedge \forward\big(\bigwedge_{\substack{
    i\in{I}\\a_i\in\alpha}}\Phi(a_i,\loc_i)\big) \\
\end{align*}

Notice that for a clock constraint $c=c_1\wedge c_2$, we have:
\begin{displaymath}
  \diamond c = \diamond (c_1\wedge c_2) \neq \diamond c_1\wedge \diamond c_2
\end{displaymath}
with $\diamond\in\{\backward,\forward\}$.

The definitions of deadlocks an action-time-locks are also trivially extended
to composition of timed components. Deadlocks can be characterized as follows:
\begin{displaymath}
  \bigvee_{\loc=(\loc_1,\cdots,\loc_i)\in\Loc}\al{\loc}\wedge
  \big[\bigwedge_{\alpha\in\gamma}\neg\backward\big(\enabled{\alpha}\wedge
  \bigwedge_{1\le i\le n}\Invb{i}{\loc_i}\big)\big]
\end{displaymath}
and action-time-locks by:
\begin{displaymath}
  \big(\bigwedge_{\alpha\in\gamma}\neg\enabled{\alpha}\big)\wedge\big(
  \bigvee_{1\le i\le n}\bigvee_{\loc_i\in\Loc_i}\al{\loc_i}\wedge urg(\loc_i)\big)
\end{displaymath}

\subsection{Example}\label{exp:run}
\begin{figure}[!h]
 \centering
  \shorthandoff{:!}
  \begin{tikzpicture}[every node/.style={scale=0.6},font=\small]

  \node [accepting, place] (l0)  {$\loc_0^1$};
  \node [place,below=1.5cm of l0,label={[shift={(-1.5,4)}]$C$}] (l1) {$\loc_1^1$};

  \path (l0) edge [bend left] node[right,align=center]{$init_0$\\$z>25$} (l1)
    (l1) edge [bend left] node[left,align=center]{$start_0$\\$z:=0$} (l0);

  \node [accepting, place] (p1-0) [right=2cm of l0] {$\loc_0^2$};
  \node [place] (p1-1) [right=1.5cm of p1-0]{$\loc_1^2$};
  \node [place] (p1-2) [below=1.5cm of p1-1,label=right:\textcolor{red}{$x\le 30$},label={[shift={(1.4,4)}]$T_1$}]            {$\loc_2^2$};
  \node [place] (p1-3) [left=1.5cm of p1-2,label=left:\textcolor{red}{$x\le 4$}] {$\loc_3^2$};

  \path (p1-0) edge node[align=center, pos=0.5]{$init_1$} (p1-1)
    (p1-1) edge node[align=center, pos=0.5]{$start_1$\\$x:=0$ } (p1-2)
    (p1-2) edge node[align=center, pos=0.5]{$process_1$ \\$10\le x\le30,$ $x:=0$ } (p1-3)
    (p1-3) edge node[align=center, pos=0.5]{$end_1$\\$x\le4$} (p1-0);


  \node [place] (p2-1) [left=2cm of l0]{$\loc_1^3$};
  \node [accepting, place] (p2-0) [left=1.5cm of p2-1]{$\loc_0^3$};
  \node [place] (p2-2) [below=1.5cm of p2-1,label=right:\textcolor{red}{$y\le 30$},label={[shift={(-5,4)}]$T_2$}]            {$\loc_2^3$};
  \node [place] (p2-3) [left=1.5cm of p2-2,label=left:\textcolor{red}{$y\le 4$}] {$\loc_3^3$};

  \path (p2-0) edge node[align=center, pos=0.5]{$init_2$} (p2-1)
    (p2-1) edge node[align=center, pos=0.5]{$start_2$\\ $y:=0$ } (p2-2)
    (p2-2) edge node[align=center, pos=0.5]{$process_2$\\$10\le y\le30,$ $y:=0$ } (p2-3)
    (p2-3) edge node[align=center, pos=0.5]{$end_2$\\$y\le4$} (p2-0);

  \node [accepting, place] (r0) [above=2cm of l0,xshift=-2.25cm,label={[shift={(-.7,.6)}]$R$}] {$\loc_0^4$};
  \node [place,right=2cm of r0] (r1) {$\loc_1^4$};

  \path (r0) edge [bend left] node[above]{take} (r1)
    (r1) edge [bend left] node[below]{free} (r0);

  \node [rounded corners,inner xsep=15mm,inner ysep=20mm,draw, fit=(l0)(l1)] (rec1) {};
  \node [rounded corners,inner xsep=15mm,inner ysep=15mm,draw, fit=(r0)(r1)] (rec2) {};
  \node [rounded corners,xshift=1mm,inner xsep=22mm,inner ysep=20mm,draw, fit=(p1-0)(p1-1)(p1-2)(p1-3)] (rec3) {};
  \node [rounded corners,xshift=1mm,inner xsep=22mm,inner ysep=20mm,draw, fit=(p2-0)(p2-1)(p2-2)(p2-3)] (rec4) {};
%  \node [inner xsep=4cm,inner ysep=2.5cm,draw, fit=(rec1)(rec2)(rec3)(rec4)] (rec5) {};
 % \node [inner xsep=1.5cm,inner ysep=5mm,draw,above=5mm of rec1] (rec5) {

  %  $\begin{aligned}
  %    \gamma &=\{
  %  init_1=\{init,init_1\}, start_1=\{start,start_1\}, process_1=\{enter,proces_1\}, 
  %  end_1=\{end,exit,end_1\}, \\
   %  & init_2=\{init, init_2\}, start_2=\{start,start_2\}, 
   % process_2=\{enter,process_2\},end_2=\{end,exit,end_2\}\} 
  %  \end{aligned}$
 % };

  \node [dots,label=90:$init_2$] (i2) at ($(rec4.south west)!0.6!(rec4.south east)$) {};
  \node [dots,label=90:$start_2$] (s2) at ($(rec4.south west)!0.8!(rec4.south east)$) {};
  \node [dots,label=-90:$end_2$] (e2) at ($(rec4.north east)!0.2!(rec4.north west)$) {};
  \node [dots,label=-90:$process_2$] (p2) at ($(rec4.north east)!0.5!(rec4.north west)$) {};

  \node [dots,swap,label=90:$init_1$] (i1) at ($(rec3.south east)!0.6!(rec3.south west)$) {};
  \node [dots,swap,label=90:$start_1$] (s1) at ($(rec3.south east)!0.8!(rec3.south west)$) {};
  \node [dots,swap,label=-90:$end_1$] (e1) at ($(rec3.north west)!0.2!(rec3.north east)$) {};
  \node [dots,swap,label=-90:$process_1$] (p1) at ($(rec3.north west)!0.5!(rec3.north east)$) {};

  \node [dots,label=-90:take] (tr) at ($(rec2.north west)!0.5!(rec2.north east)$) {};
  \node [dots,label=90:free] (fr) at ($(rec2.south west)!0.5!(rec2.south east)$) {};

  \node [dots,label=90:$init_0$] (ic) at ($(rec1.south west)!0.35!(rec1.south east)$) {};
  \node [dots,label=90:$start_0$] (rc) at ($(rec1.south west)!0.65!(rec1.south east)$) {};
  
  \path (tr) ++(0,0.25cm) +(-1cm,0) coordinate(xp2) +(1cm,0) coordinate(xp1);
  \draw  [-] (p1) |-node[above,xshift=-2.5cm]{$\alpha_5$} (xp1) -- (tr) -- (xp2)node[above,xshift=-2.5cm]{$\alpha_6$} -| (p2);

  \path (ic) ++(0,-0.5cm) +(1cm,0) coordinate(xi1) +(-1cm,0) coordinate(xi2);
  \draw[-,name path=line1] (i1) |-node[above,xshift=-2.5cm]{$\alpha_1$} (xi1) -- (ic) -- (xi2)node[above,xshift=-2.5cm]{$\alpha_2$} -| (i2); %here


  \path (rc) ++(0,-1cm) +(1cm,0) coordinate(sx1) +(-1cm,0) coordinate(sx2);

  \path[-,name path=line2] (s1) |-node[above,xshift=-1.5cm]{$\alpha_3$} (sx1) -- (rc) -- (sx2)node[above,xshift=-2cm]{$\alpha_4$} -| (s2); %here

 \path[name intersections={of=line1 and line2, by={a,b,c,d}}];% here

% draw semicircles at crossing points on the path
\coordinate (aux1) at (s2|-sx2);
\coordinate (aux2) at (s1|-sx1);
\draw[-,connect=(s2) to (aux1) over (d) by 3pt];
\draw[-] (aux1) -- (sx2);
\draw[-,connect=(sx2) to (rc) over (c) by 3pt];
\draw[-,connect=(rc) to (sx1) over (b) by 3pt];
\draw[-] (sx1) -- (aux2);
\draw[-,connect=(aux2) to (s1) over (a) by 3pt];

\path (fr) ++(0,-2.5mm) +(-1cm,0) coordinate(xe2) +(1cm,0) coordinate(xe1);
  \draw [-] (e1) |- node[above,xshift=-2cm]{$\alpha_7$}(xe1);
  \draw [-] (xe1) -- (fr);
  \draw [-] (e2) |- node[above,xshift=2cm]{$\alpha_8$}(xe2);
  \draw [-] (xe2) -- (fr);
\end{tikzpicture}
  \caption{Task Manager}
 \label{fig:tm}
\end{figure}  




Figure~\ref{fig:tm} depicts a timed system composed of four components $C$, $T_1$, $T_2$, 
and $R$.
Component $C$ represents a  controller that initializes then releases tasks $T_1$ and $T_2$
Tasks use the shared resource $R$ during their execution.
To implement such behavior, we consider the following interactions between $C$, $R$, and 
$T_1$: $\alpha_1=\{init_0, init_1\}$,
$\alpha_3=\{start_0, start_1\}$, $\alpha_5=\{ take, process_1\}$, $\alpha_7 = 
\{free, end_1 \}$, 
and similar interactions $\alpha_2$, $\alpha_4$, $\alpha_6$, $\alpha_8$ for task $T_2$, 
as shown by connections on Figure~\ref{fig:tm}.
The controller is responsible for firing
the execution of each task. First, it non-deterministically initializes one
of the two tasks, i.e. executes $\alpha_1$ or $\alpha_2$, and then
releases it through interaction $\alpha_3$ or $\alpha_4$.
Tasks perform their processing independently of the controller, after being granted an access 
to the shared resource ($\alpha_5$ or $\alpha_6$).
When finished, a task releases the resource (interactions $\alpha_7$ or $\alpha_8$) and goes 
back to its initial location.
An example of execution sequence of this system is given below. Valuations $v$ of clocks $x$, 
$y$, and $z$ are represented as a tuples 
$(\val(x),\val(y),\val(z))$:
\begin{displaymath}
\small{
\begin{split}
&((\loc_0^1,\loc_0^2,\loc_0^3,\loc_0^4),(0,0,0))\transit{26}_{\gamma}
((\loc_0^1,\loc_0^2,\loc_0^3,\loc_0^4),(26,26,26))\transit{\alpha_1}_{\gamma}
((\loc_1^1,\loc_1^2,\loc_0^3,\loc_0^4),(26,26,26))\transit{\alpha_3}_{\gamma}\\
&((\loc_0^1,\loc_2^2,\loc_0^3,\loc_0^4),(0,26,0))\transit{10}_{\gamma}
((\loc_0^1,\loc_2^2,\loc_0^3,\loc_0^4),(10,36,10))\transit{\alpha_5}_{\gamma}
((\loc_0^1,\loc_3^2,\loc_0^3,\loc_1^4),(0,36,10))\transit{2}_{\gamma}\\&
((\loc_0^1,\loc_3^2,\loc_0^3,\loc_1^4),(2,38,12))\transit{\alpha_2}_{\gamma}
((\loc_1^1,\loc_3^2,\loc_1^3,\loc_1^4),(2,38,12))
\end{split}
}
\end{displaymath}

\section{Timed Systems with Data}
Timed models introduced in the previous section focuses on the timing behavior of a given system.
In order to achieve a higher degree of expressiveness, we extend these models wit data variables.
Data allows additional representation of complex behavior. Similarly to clock variables, they may
appear in the guard of transitions as additional conditions and may be updated when transitions
fire.

\begin{definition}[Guards on clocks and Data]
  Let $\mc{X}$ be a set of clock variables and $\mc{D}$ be a set of
  data variables. We denote by $\mc{G(X,D)}$ the set of guards induced
  by the following grammar:
  \begin{displaymath}
    g:=g_x \ | \ g_d \ | \ g_1\wedge g_2
  \end{displaymath}
  where $g_x\in\mc{C(X)}$, $g_d$ is a predicate on a set of data variables
  of $\mc{D}$, and $g_1$, $g_2$ are guards over clocks and/or data variables.
\end{definition}

We extend the notion of valuation to data variable in the following manner:
\begin{itemize}
  \item Valuations assign values to data variable (in addition to clocks)
  \item The satisfaction of a valuation to a constraint is straightforwardly 
    extended to data variables
  \item Data variables are insensitive to the progress of time
  \item Update operations are also defined for data variables 
\end{itemize}
We put $R$ to denote the set of update operations, thatis, 
all reset operations over clocks of $\mc{X}$ and 
assignment operations over data variables of $\mc{D}$. 

\begin{definition}[Timed Component with Data]\label{def:tce}
  A timed component with data is a tuple 
  $B_d=(\Loc,\loc_0,\mc{X},\mc{D},\mc{A},\mc{E},\mc{I})$ such that:
  \begin{itemize}
    \item $\Loc$, $\loc_0$, $\mc{X}$, $\mc{A}$ and $\mc{I}$ are defined as
      in Definition~\ref{def:tc},
    \item $\mc{D}$ is a finite set of data variables,
    \item $\mc{E}\subseteq\Loc\times(\mc{A}\times\mc{G(X,D)}\times2^{R})\times\Loc$
    is a finite set of labeled transitions with an action, an extended guard, 
      and a set of clocks to be reset and assignement operations over elements
      of $\mc{D}$. 
  \end{itemize}
\end{definition}

\begin{figure}[!h]
 \centering
  \begin{tikzpicture}[scale=0.8,every node/.style={scale=0.8}]

  \node [accepting, place,label=below:\textcolor{red}{$x\le4$},label={[shift={(-.9,.3)}]$B$}](l0) {$\loc_0$};
  \node [place,right=2cm of l0] (l1) {$\loc_1$};

  \path (l0) edge [bend left] node[above,align=center]{$a$\\$2\le x\le4\wedge k<=5$\\$k:=k+10$}(l1)
        (l1) edge [bend left] node[below,align=center]{$b$\\$x:=0$} (l0)
        (l0) edge [loop left] node[left,align=center]{$c$\\$k>5\wedge1<x\le4$\\$k:=k-3\wedge x:=0$}(l0);
  \node [rounded corners,inner xsep=26mm,inner ysep=15mm,draw, fit=(l0)(l1),xshift=-2.1cm] (rec1) {};
\end{tikzpicture}
  \caption{Example of an Extended Timed Component}
 \label{fig:tcd}
\end{figure}  




\begin{example}
  Component of Figure~\ref{fig:tcd} is an extended timed component with actions $a$, $b$ and $c$.
  The set of data is $\mc{D}=\{k\}$. It is used on the guards of the transitions labeled by
  $a$ and $c$ ($k\le 5$ and $k>5$ respectively). The update operations for both transitions
  are respectively $\{k:k+10\}$ and $\{k:=k-3\wedge x:=0\}$.
\end{example}
\begin{remark}
  Notice that extending a timed component with data will result in a restriction
  of the behavior of the initial timed components since extending guards to data
  variables will only constrain the execution of transitions.
\end{remark}

\begin{definition}[Timed System with Data]
  Given $n$ components $B_{d_i}=(\Loc_i,\loc_{0_i},$\\$\mc{X}_i,\mc{D}_i,\mc{A}_i,
  \mc{E}_i,\mc{I}_i)$ and an interaction set $\gamma$,
  the composition $S_d=\gamma(B_{d_1},\cdots,B_{d_n})$ with respect to 
  the interaction set $\gamma$ is defined by the timed component 
  $(\Loc,\loc_0,\mc{X},\mc{D},\gamma,\mc{E}_{\gamma},\mc{I})$ where:
  \begin{itemize}
    \item $\Loc$, $\loc_0$, $\mc{X}$, $\gamma$ and $\mc{I}$ are defined as
      in Definition~\ref{def:comp}.
    \item $\mc{D}=\mc{D}_1\cup\cdots\cup\mc{D}_n$
    \item $\mc{E}_{\gamma}$ is straightforwardly extended by considering guards
      on data variables and assignment operations on execution of interactions
  \end{itemize}
\end{definition}
The semantics of an extend timed system extends the semantics of Property~\ref{pr:sem_comp}
in the sense that an interaction takes place if the guards on data variables also evaluates
to $\true$ and that interaction execution applies the underlying update operations of the
corresponding components transition.

In what follows, when referring to timed component or timed system we imply by that 
extended timed component or extend timed systems, unless explicitly stated. 

\section{Verification of Timed Systems}\label{sec:2.4}
\subsection{Symbolic Reachability}

The semantics of timed systems as presented in Property~\ref{pr:sem_comp} 
defines states as pairs of locations configurations and clocks
valuations. By considering a continuous time domain ($\realpoz$ here), the 
resulting timed transition system yields an infinity of states.
Consequently, the usual practice is to rely on a symbolic representation~\cite{}
of states to make this state space finite.
A symbolic state is defined by a pair $(\loc,\xi)$ where $\loc$ is a location
and $\xi$ is a zone, a set of clocks valuations defined by clocks constraints
(as defined in~\ref{eq:cc}). Consequently, the set of reachable states of a 
timed component B (system) can be put on the form:
\begin{displaymath}
  Reach(B)=\bigvee_{j\in{J}}\al{\loc_j}\wedge\xi_j
\end{displaymath}

\subsection{Compositional Verification}

Standard verification techniques such as model checking are based on explicit 
exploration and exhaustive enumeration of all the reachable symbolic states 
of a given system. The main issue with this 
method is the combinatorial explosion when considering large scale systems.
Compositional verification have been introduced to cope with state explosion
problem, and thus achieve scalability when verifying large scale systems.
This approach is based on the concept of divide-and-conquer in order to 
break up the verification
problems into smaller subsequent problems. Compositional verification have been 
extensively studied under different manners e.g., assume-guarantee 
reasoning~\cite{}, 
contract-based verification~\cite{}, compositional generation~\cite{}, 
deductive verification
~\cite{}, etc.  
In this thesis, we choose to use a deductive compositional verification 
method that exploits compositionality for analysis of timed systems 
using \emph{invariants}. Invariants are symbolic approximations of the set 
of reachable states of the system as opposed to the exact 
reachability analysis in model-checking. The key principle of this 
approach is the computation of a global invariant as the conjunction of
other invariants (components invariants, interaction invariants, etc.).



%:\emph{(i)} local invariants of 
%the underlying components, \emph{(ii)} an interaction 
%invariant deduced from the synchronizations 
%(interactions) between components, and \emph{(iii)} an additional invariant 
%capturing the 
%global timing aspects and relating clocks of different 
%components.

Let $S=\gamma(B_1,\cdots,B_n)$ be a system composed of $n$ timed components $B_i$ 
synchronizing through the interaction 
set $\gamma$, and let $\psi$ be a property of interest. Assuming that 
$GI(S)$ is the global invariant for this system, 
the verification rule of $\psi$ can be intuitively
written as follows:

\begin{align*}
  &\hspace{1cm}GI(S)\Rightarrow\psi\\
    \cline{1-2}
   &\gamma(B_1,\cdots,B_n)\models\square\psi
\end{align*}
where the notation \enquote{$\gamma(B_1,\cdots,B_n)\models\square\psi$} is to 
be read as \enquote{$\psi$ holds in every reachable state of the
composition $\gamma(B_1,\cdots,B_n)$}.

Usually when verifying timed system, the common practice is to verify the system 
against some \emph{safety property}, that is a property 
describing that \enquote{some bad states are never reached}. This property 
formulation allows to assert that nothing bad happens.
We use satisfiability checking~\cite{} to verify that the global 
invariant of a system 
implies the property of interest. Particularly, for a system S characterized 
by the global invariant $GI(S)$, and 
given a safety property $\psi$, we check the unsatisfiability of 
$GI(S)\wedge\neg\psi$. 



